%!
%! Copyright (C) 2014-2025 Andrea Dal Corso 
%! This file is distributed under the terms of the
%! GNU General Public License. See the file `License'
%! in the root directory of the present distribution,
%! or http://www.gnu.org/copyleft/gpl.txt .
%!
\documentclass[12pt,a4paper,twoside]{report}
\def\version{2.0.3}

\usepackage[T1]{fontenc}
\usepackage{tcolorbox}
\usepackage{bookman}
\usepackage{graphicx}
\usepackage{fancyhdr}
\usepackage[Lenny]{fncychap}
\usepackage{color}
\usepackage{geometry}
\usepackage{amsmath}
\usepackage{amsbsy}
\usepackage{amssymb}
\usepackage{mathtools}

\tcbuselibrary{breakable}
\pagestyle{fancy}
\lhead{Symmetry guide}
\rhead{}
\cfoot{\thepage}

\newgeometry{
      top=3cm,
      bottom=3cm,
      outer=2.25cm,
      inner=2.75cm,
}

\definecolor{web-blue}{rgb}{0,0.5,1.0}
\definecolor{steelblue}{rgb}{0.27,0.5,0.7}
\definecolor{coral}{rgb}{1.0,0.5,0.3}
\definecolor{red}{rgb}{1.0,0,0.0}
\definecolor{green}{rgb}{0.,0.5,0.0}
\definecolor{dark-blue}{rgb}{0.,0.0,0.6}
\definecolor{limegreen}{rgb}{0.19,0.8,0.19}
\definecolor{orange}{rgb}{1.0,0.44,0.0}
\definecolor{violet}{rgb}{0.50,0.33,0.9}
\definecolor{light-yellow}{rgb}{0.94,0.85,0.62}

\tcbset{colback=light-yellow,colframe=dark-blue,breakable}

\def\qe{{\sc Quantum ESPRESSO}}
\def\pwx{\texttt{pw.x}}
\def\phx{\texttt{ph.x}}
\def\configure{\texttt{configure}}
\def\PWscf{\texttt{PWscf}}
\def\PHonon{\texttt{PHonon}}
\def\tpw{\texttt{thermo\_pw}}
\def\make{\texttt{make}}


\begin{document}
\author{Andrea Dal Corso \\ (SISSA - Trieste)}
\date{}

\title{
%  \includegraphics[width=8cm]{thermo_pw.jpg} \\
  \vspace{3truecm}
  % title
  \Huge \color{dark-blue} Symmetries in Quantum ESPRESSO \ \\
   (v.\version)
}

\maketitle

\newpage

\tableofcontents

\newpage

\newpage
{\color{dark-blue}\chapter{Introduction}}
These notes describe how to exploit symmetry in electronic structure codes. They assume only that the reader knows the basic group theory concepts as applied to quantum mechanics and in particular to solids. I discuss some of the formulas implemented in Quantum ESPRESSO with the hope that these concepts might be useful to those who want to further develop the code. The exposition is divided in two parts. In the first part I will use the space group operations of a solid while in the second part I will consider also the magnetic symmetries and discuss how to treat symmetries that might require time reversal. 


\newpage
\section{People}
These notes have been written by Andrea Dal Corso (SISSA, Trieste). 

Disclaimer: These notes contain the concepts that I found useful to understand the symmetry routines of Quantum ESPRESSO, but I am not the author of many of them, so what is written here might be inaccurate or not reflect the original intention of the authors. If you find that something is incorrect or does not reflect what is implemented please e-mail me at dalcorso .at. sissa.it. 
The concepts presented here have benefited from discussions with several people, among them I would like to thank S. de Gironcoli (and his unpublished notes on the symmetry of the phonon code), A. Smogunov, and A. Urru (and the appendices of his PhD thesis). 

Acknowledgments: The writing of these notes
has been supported by MAX ``MAterials design at the eXascale" Centre of Excellence 
for Supercomputing applications (Grant agreement No. 101093374, co-funded by the European High Performance 
Computing joint Undertaking (JU) of the European Union and participating countries).

\newpage
\section{Preliminary definitions}
The equilibrium positions of the atoms in a solid are determined by
a Bravais lattice and by the positions of the atoms inside a unit cell.

The Bravais lattice can be defined by three primitive vectors:
\begin{equation}
{\bf a}_1, {\bf a_2}, {\bf a}_3.
\end{equation}
Each point of the lattice is given by three integer values
$n_1, n_2, n_3$ that can be positive, negative, or null.
We indicate with ${\bf R}_\mu$ the coordinates of the Bravais lattice
points:
\begin{equation}
{\bf R}_\mu = n_1 {\bf a}_1 + n_2 {\bf a}_2 + n_3 {\bf a}_3.
\end{equation}
We indicate with $\boldsymbol{\tau}_s$ the positions of the atoms
inside the unit cell ($s=1,\cdots, N_{at}$).
Each atom $s$ within the
unit cell has a type that we call $\gamma(s)$.
The equilibrium positions of the atoms in the solid are therefore:
\begin{equation}
{\bf R}_I = {\bf R}_\mu + \boldsymbol{\tau}_s,
\end{equation}
and the index $I=\{\mu,s\}$ is a composite index that indicates both the
Bravais lattice and the atom inside the unit cell.

We say that a rototranslation $\{S,{\bf a}\}$, where $S$ is a proper
(or improper) rotation and ${\bf a}$ is a translation, belongs to the solid
space group if for any $I=\{\mu,s\}$ there is an
$\bar I=\{\bar \mu,\bar s\}$ such that:
\begin{equation}
\{S,{\bf a}\} {\bf R}_I = S {\bf R}_\mu + S \boldsymbol{\tau}_s + {\bf a} =
{\bf R}_{\bar \mu} + \boldsymbol{\tau}_{\bar s},
\label{sg}
\end{equation}
and $\gamma(s)=\gamma(\bar s)$.

The operations $\{S_i,{\bf a}_i\}$ for which Eq.~\ref{sg} holds form a group.
We have:
\begin{equation}
\{S_i,{\bf a}_i\} \{S_j,{\bf a}_j\} = \{S_i S_j,S_i {\bf a}_j + {\bf a}_i\}.
\end{equation}
The neutral element of the group is $\{\mathbb{I},{\bf 0}\}$ where
$\mathbb{I}$ is the identity matrix and ${\bf 0}$ is the null translation.
The inverse of an element is
\begin{equation}
\{S_i,{\bf a}_i\}^{-1} = \{S_i^{-1}, -S_i^{-1} {\bf a}_i \}.
\end{equation}
It is useful also to introduce the fractional translations ${\bf f}_i$,
by considering the vector ${\bf R}_{\mu_i}$ closer to ${\bf a}_i$ and
writing the operations of the space group as
\begin{equation}
\{S_i,{\bf a}_i\}= \{S_i,{\bf R}_{\mu_i}+{\bf f}_i\}.
\end{equation}
The set of rotations $\{S_i\}$ form a group called the solid point group. In general, the
set $\{S_i,{\bf 0}\}$ is not a subgroup of the space group.

\section{Coordinates}

\subsection{The rotation matrix}

There are two ways to express the rotation matrices $S$ of the point group. We can consider the three vectors obtained by  application
of $S$ to ${\bf a}_1$, ${\bf a}_2$, and ${\bf a}_3$. They must be Bravais lattice vectors and they can be expressed as
\begin{equation}
{\bf a}'_l = \sum_{m=1}^3 S_{l,m} {\bf a}_m,
\label{scrys}
\end{equation}
where $S_{l,m}$ is a $3\times3$ integer matrix.
Alternatively, we can introduce Cartesian coordinates and write the primitive
vectors as ${\bf a}_{\alpha,i}$ where $\alpha$ indicates a Cartesian coordinate.
We can indicate the Cartesian rotation matrix as $S_{\alpha,\beta}$
and we have:
\begin{equation}
{\bf a}'_{\alpha,l} = \sum_{\beta=1}^3 S_{\alpha,\beta} {\bf a}_{\beta,l}.
\label{scart}
\end{equation}
The matrix $S_{\alpha,\beta}$ is an orthogonal ($S_{\beta,\alpha}=S^{-1}_{\alpha,\beta} $) $3\times 3$ matrix, while 
$S_{l,m}$ in general is not orthogonal. We
can write $S_{\alpha,\beta}$ in terms of $S_{l,m}$ or $S_{l,m}$ in terms
of $S_{\alpha,\beta}$ using the following considerations.

Assuming that ${\bf a}_m$ are in units of $a$ (the lattice constants), we
introduce the primitive reciprocal lattice vector ${\bf b}_1$, ${\bf b}_2$,
and ${\bf b}_3$ (in units of $2\pi \over a$) such that
\begin{equation}
{\bf a}_i \cdot {\bf b}_j = \sum_{\alpha=1}^3 {\bf a}_{\alpha,i} {\bf b}_{\alpha, j} =  \delta_{i,j}.
\label{recips}
\end{equation}
From this equation one derives also that:
\begin{equation}
\sum_{j=1}^3 {\bf b}_{\alpha, j} {\bf a}_{\beta, j} =  \delta_{\alpha,\beta}.
\label{complete}
\end{equation}
Multiplying Eq.~\ref{scrys} by ${\bf b}_{\alpha, j}$, summing on $\alpha$ and
using Eq.~\ref{recips} we have
\begin{equation}
\sum_{\alpha=1}^3 {\bf a}'_{\alpha,l} {\bf b}_{\alpha, j} = S_{l,j},
\end{equation}
and using Eq.~\ref{scart}:
\begin{equation}
S_{l,j} =\sum_{\alpha=1}^3 \sum_{\beta=1}^3 {\bf b}_{\alpha, j} S_{\alpha,\beta} {\bf a}_{\beta,l}.
\label{slj}
\end{equation}
The inverse of this equation can be found by multiplying both terms by
${\bf a}_{\gamma,j}$ and ${\bf b}_{\delta,l}$, summing over $j$ and $l$ and
using Eq.~\ref{complete} twice:
\begin{equation}
\sum_{l,j} {\bf b}_{\delta,l} S_{l,j} {\bf a}_{\gamma,j} = S_{\gamma,\delta}.
\end{equation}

\subsection{Rotation of a vector}
Given the Cartesian coordinates of a vector
${\bf v}$, the rotated Cartesian coordinates are:
\begin{equation}
{\bf v}'_\alpha = \sum_{\beta=1}^3 S_{\alpha,\beta} {\bf v}_\beta.
\end{equation}
When the vector ${\bf v}$ is expressed in crystal coordinates:
\begin{equation}
{\bf v} = \sum_{l=1}^3 {\bf v}_l {\bf a}_l.
\end{equation}
the rotated vector can be written as
\begin{equation}
{\bf v}' = \sum_{i=1}^3 {\bf v}_l {\bf a}'_l.
\end{equation}
and its coordinates can also be written again in the original basis
${\bf a}_i$ using Eq.~\ref{scrys}:
\begin{equation}
{\bf v}' = \sum_{l=1}^3 \sum_{m=1}^3 {\bf v}_l S_{l,m} {\bf a}_m,
\end{equation}
with
\begin{equation}
{\bf v}'_m = \sum_{l=1}^3 {\bf v}_l S_{l,m}.
\label{rotvcry}
\end{equation}
\subsection{Rotation of a vector in reciprocal space}
We can rotate the vectors
${\bf b}_1$, ${\bf b}_2$,
and ${\bf b}_3$. They must be reciprocal lattice vectors so they can be written as linear combination
of ${\bf b}_1$, ${\bf b}_2$,
and ${\bf b}_3$:
\begin{equation}
{\bf b'}_j = \sum_{m=1}^3 U_{j,m} {\bf b}_m,
\end{equation}
where the matrix $U_{j,m}$ is a $3\times 3$ integer matrix.
Since
\begin{equation}
{\bf a}'_i \cdot {\bf b'}_j = \delta_{i,j},
\end{equation}
we must have:
\begin{equation}
\sum_{l=1}^3 \sum_{m=1}^3 S_{i,l} {\bf a}_l \cdot U_{j,m} {\bf b_m} =  \sum_{l=1}^3 S_{i,l} U_{j,l} =
\delta_{i,j}.
\end{equation}
From this equation we find that
\begin{equation}
U_{j,l} = S^{-1}_{l,j},
\end{equation}
where the operator $S^{-1}$ is the inverse of the rotation $S$, belongs to the point group, and is an integer matrix.
The rotated principal reciprocal lattice vectors are:
\begin{equation}
{\bf b'}_j = \sum_{m=1}^3 S^{-1}_{m,j} {\bf b}_m.
\end{equation}
With this equation we can rotate 
a vector ${\bf v}$ written in the basis of
the primitive reciprocal lattice vectors. We have:
\begin{equation}
{\bf v} = \sum_{l=1}^3 {\bf v}_l {\bf b}_l.
\end{equation}
and the rotated vector is
\begin{equation}
{\bf v}' = \sum_{l=1}^3 {\bf v}_l {\bf b}'_l,
\end{equation}
or, in the original reciprocal primitive basis:
\begin{equation}
{\bf v}' = \sum_{l=1}^3 \sum_{m=1}^3 {\bf v}_l S^{-1}_{m,l} {\bf b}_m.
\end{equation}
So the components of the rotated vector are:
\begin{equation}
{\bf v}'_m = \sum_{l=1}^3 {\bf v}_l S^{-1}_{m,l}.
\label{rotreccomp}
\end{equation}

\subsection{Basis change}
When a vector is given in Cartesian coordinates (${\bf v}_{\alpha}$) we can write it in the basis of the primitive Bravais lattice vectors.
We have:
\begin{equation}
{\bf v}= \sum_{l=1}^3 {\bf v}_l {\bf a}_l.
\end{equation}
Multiplying both members of this equation by ${\bf b}_m$
we get
\begin{equation}
{\bf v}_m= {\bf v} \cdot {\bf b}_m=\sum_{\alpha=1}^3
{\bf v}_{\alpha} {\bf b}_{\alpha,m}.
\end{equation}

When a vector is given in Cartesian coordinates
(${\bf v}_{\alpha}$) we can write it in the basis of the primitive reciprocal lattice vectors:
\begin{equation}
{\bf v}= \sum_{l=1}^3 {\bf v}_l {\bf b}_l.
\end{equation}
Multiplying both members of this equation by ${\bf a}_m$
we get
\begin{equation}
{\bf v}_m= {\bf v} \cdot {\bf a}_m=\sum_{\alpha=1}^3
{\bf v}_{\alpha} {\bf a}_{\alpha,m}.
\end{equation}

When a vector is given in the basis of the primitive Bravais lattice vectors its Cartesian coordinates are
\begin{equation}
{\bf v}_\alpha= \sum_{l=1}^3 {\bf v}_l {\bf a}_{\alpha,l}.
\end{equation}

When a vector is given in the basis of the primitive reciprocal lattice vectors its Cartesian coordinates are
\begin{equation}
{\bf v}_\alpha= \sum_{l=1}^3 {\bf v}_l {\bf b}_{\alpha,l}.
\end{equation}

{\color{dark-blue}\chapter{Symmetry in self-consistent codes}}
\color{black}

\section{The Kohn and Sham wavefunctions}

We can consider the Bloch functions, solutions of the Kohn and Sham equations:
\begin{equation}
H_{KS} \psi_{{\bf k}, n}({\bf r})= \varepsilon_{{\bf k}, n}\psi_{{\bf k}, n} ({\bf r}),
\end{equation}
where ${\bf k}$ is a wave vector in the first Brillouin zone an $n$ is a band index.
For each element of the space group $\{S_i,{\bf a}_i\}$ we can associate an operator $O_{\{S_i,{\bf a}_i\}}$ that rotates the Bloch functions. We require that after the rotation the new function $O_{\{S_i,{\bf a}_i\}} \psi_{{\bf k} n} ({\bf r})$ has in ${\bf r}$ the same value that the original function had in the point
${\bf r}'$ that becomes ${\bf r}$ after the rotation.
We have 
\begin{equation}
{\bf r}'= \{S_i,{\bf a}_i\}^{-1} {\bf r}
\end{equation}
therefore
\begin{equation}
O_{\{S_i,{\bf a}_i\}} \psi_{{\bf k}, n} ({\bf r}) =
\psi_{{\bf k}, n} (\{S_i,{\bf a}_i\}^{-1} {\bf r}).
\end{equation}
Writing the Bloch functions in the form
\begin{equation}
\psi_{{\bf k}, n} ({\bf r}) = e^{i{\bf k} \cdot {\bf r}} u_{{\bf k}, n} ({\bf r}),
\end{equation}
where $u_{{\bf k}, n} ({\bf r})$ is a lattice periodic function, we have:
\begin{equation}
O_{\{S_i,{\bf a}_i\}} \psi_{{\bf k}, n} ({\bf r})
= e^{iS_i{\bf k} \cdot {\bf r}}e^{-iS_i{\bf k} \cdot {\bf a}_i} u_{{\bf k}, n} (\{S_i,{\bf a}_i\}^{-1} {\bf r}).
\label{psisik}
\end{equation}
Therefore, $O_{\{S_i,{\bf a}_i\}} \psi_{{\bf k}, n} ({\bf r})$ is a Bloch function with wave-vector $S_i {\bf k}$.
If $\{S_i,{\bf a}_i\}$ belongs to the solid space group, $O_{\{S_i,{\bf a}_i\}}$ commutes with $H_{KS}$
and $O_{\{S_i,{\bf a}_i\}} \psi_{{\bf k}, n} ({\bf r})$
is an eigenstate with
eigenvalue $\varepsilon_{{\bf k}, n}$. We can call it $\psi_{S_i {\bf k},n} ({\bf r})$.

\section{The charge density in real space}

The charge density is calculated as a sum over the
Brillouin zone and the occupied bands:
\begin{equation}
n({\bf r}) = 2 \sum_{{\bf k}, v} |\psi_{{\bf k}, v} ({\bf r}) |^2,
\end{equation}
where the factor two accounts for spin degeneracy.
Usually we use a uniform grid $N_{k_1}\times N_{k_2} \times N_{k_3}$ of ${\bf k}$ of points to perform this sum, and we use a grid such that if we apply a rotation $S_i$ of the point group to a wave-vector ${\bf k}$ we obtain the wave-vector $S_i {\bf k}$ which belongs to the grid. 
In these hypotheses, the set of wave vectors $S_i {\bf k}$ obtained by applying $S_i$ to all points of the grid
coincides with the original grid itself. Therefore we 
can write 
\begin{equation}
n({\bf r}) = {2 \over N_S} \sum_{i=1}^{N_S} \sum_{{\bf k}, v} |\psi_{S_i{\bf k}, v} ({\bf r}) |^2,
\end{equation}
where $N_S$ is the number of symmetries of the point group. Now each ${\bf k}$ in the mesh can be obtained from a ${\bf k}$ in the irreducible Brillouin zone (IBZ) by applying an appropriate rotation $S_l$.
We can write:
\begin{equation}
n({\bf r}) = {2 \over N_S} \sum_{i=1}^{N_S} \sum_{{\bf k} \in IBZ, v} \sum_l |\psi_{S_iS_l{\bf k}, v} ({\bf r}) |^2.
\end{equation}
The sum over ${\bf k}$ is limited to the IBZ and the sum over $l$ is over the $N_{\bf k}$ symmetries needed to obtain all the star of ${\bf k}$ from the 
${\bf k}$ in the IBZ. But now we can exchange the sum over $l$ and the sum over $i$ and notice that the sum over $i$ is over all the elements of the point group and for the rearrangement lemma the elements $\{ S_i S_l\}$ are still all the elements of the point group so the sum over $i$ is independent from $l$. It is convenient to introduce
a weight $w_{\bf k}$ that gives the number of points in the star of ${\bf k}$ so that we can write
\begin{equation}
n({\bf r}) = {2 \over N_S} \sum_{i=1}^{N_S}  \sum_{{\bf k} \in IBZ, v} w_{\bf k}\ |\psi_{S_i{\bf k}, v} ({\bf r}) |^2.
\end{equation}
Now we can use Eq.~\ref{psisik} and write
\begin{equation}
n({\bf r}) = {2 \over N_S} \sum_{i=1}^{N_S} \sum_{{\bf k}\in IBZ , v} w_{\bf k}\ |\psi_{{\bf k}, v} ( \{S_i,{\bf f}_i\}^{-1} {\bf r}) |^2,
\label{chargedensity}
\end{equation}
where we choose the spatial group operation $\{S_i,{\bf f}_i\}$ since Eq.~\ref{chargedensity} does not depend on
${\bf R}_{\mu_i}$.

The charge density is computed in two steps. First one computes the unsymmetrized charge density:
\begin{equation}
\tilde n({\bf r}) = 2 \sum_{{\bf k}\in IBZ , v} w_{\bf k}\ |\psi_{{\bf k}, v} ({\bf r}) |^2,
\label{unsymchden}
\end{equation}
and finally the charge density is symmetrized:
\begin{equation}
n({\bf r}) = {1 \over N_S} \sum_{i=1}^{N_S} \tilde n ( \{S_i,{\bf f}_i\}^{-1} {\bf r}).
\label{symchden}
\end{equation}
Since we are adding on all operations of the group, we obtain the same result applying $\{S_i,{\bf f}_i\}$ in Eq.~\ref{symchden}:
\begin{equation}
n({\bf r}) = {1 \over N_S} \sum_{i=1}^{N_S} \tilde n ( \{S_i,{\bf f}_i\} {\bf r}).
\label{symchden1}
\end{equation}

\section{The charge density in reciprocal space}
Eq.~\ref{symchden1} can be used to find a symmetrization formula in reciprocal space. Taking the Fourier transform at the reciprocal lattice vector ${\bf G}$, we have:
\begin{eqnarray}
n({\bf G}) &=& {1\over N_S} {1\over \Omega}
\sum_{i=1}^{N_S} \int_{\Omega} \tilde n ( \{S_i,{\bf f}_i\} {\bf r}) e^{-i{\bf G} \cdot {\bf r}} d^3r\nonumber \\
&=&{1\over N_S} {1\over \Omega}
\sum_{i=1}^{N_S} \int_{\Omega} \tilde n ({\bf r}) e^{-i{\bf G} \cdot \{S_i,{\bf f}_i\}^{-1}{\bf r}} d^3r \nonumber \\
&=&{1\over N_S} {1\over \Omega}
\sum_{i=1}^{N_S} \int_{\Omega} \tilde n ({\bf r}) e^{-i{\bf G} \cdot S_i^{-1}{\bf r}} e^{i{\bf G} \cdot S_i^{-1}{\bf f}_i} d^3r \nonumber \\
&=&{1\over N_S} {1\over \Omega}
\sum_{i=1}^{N_S} e^{iS_i {\bf G} \cdot {\bf f}_i} \int_{\Omega} \tilde n ({\bf r}) e^{-iS_i {\bf G} \cdot {\bf r}} d^3r \nonumber \\
&=&{1\over N_S}
\sum_{i=1}^{N_S} e^{iS_i {\bf G} \cdot {\bf f}_i}  \tilde n (S_i{\bf G}). 
\label{symng}
\end{eqnarray}
This formula could be applied for any ${\bf G}$ vector but it useful to observe that if we know $n({\bf G})$ we can find
$n(S_i{\bf G})$ with a simple formula.
Since after an operation of the space group $\{S_i,{\bf f}_i\}$ the solid does not change we must have:
\begin{equation}
n({\bf r})=n(\{S_i,{\bf f}_i\}{\bf r}).
\end{equation}
Doing a Fourier transform as shown above, from this formula we get
\begin{equation}
 n({\bf G})= e^{iS_i {\bf G} \cdot {\bf f}_i} n (S_i{\bf G}), 
\end{equation}
or 
\begin{equation}
 n(S_i{\bf G})= e^{-iS_i {\bf G} \cdot {\bf f}_i} n ({\bf G}). 
 \label{ngdist}
\end{equation}

\section{The forces}

Let us now consider a set of displacements of each atom of the solid. We indicate with ${\bf u}_{I}$ the displacement of the atom that in equilibrium is in
${\bf R}_I$. Suppose that we compute the total energy
with the atomic coordinates $\{{\bf R}_I + {\bf u}_I\}$.
If we do an opertion of the solid space group
$\{S_i,{\bf a}_i\}$ the atomic positions becomes
\begin{equation}
\{S_i,{\bf a}_i\} ({\bf R}_I + {\bf u}_I) = 
{\bf R}_{\bar I} + S_i {\bf u}_I.
\end{equation}
So if in the rotated solids we choose the displacements
\begin{equation}
{\bf u}'_{\bar I} =  S_i {\bf u}_I,
\label{rotated_disp}
\end{equation}
the energy cannot change and we can write:
\begin{equation}
E_{tot}\left(\{{\bf R}_I + {\bf u}_I\}\right) = E_{tot}\left (\{\{S_i,{\bf a}_i\} ({\bf R}_I + {\bf u}_I)\}\right )=
E_{tot}\left(\{ {\bf R}_{\bar I} + {\bf u}'_{\bar I}\}\right).
\end{equation}
Computing the derivatives with respect to ${\bf u}_{I,\alpha}$ we have:
\begin{equation}
{d E_{tot} \over d {\bf u}_{\mu,s,\alpha}} = \sum_{\beta=1}^3
{d E_{tot} \over d {\bf u}_{{\bar \mu},{\bar s},\beta}} 
{d {\bf u}_{{\bar \mu},{\bar s},\beta} \over
d {\bf u}_{\mu, s,\alpha}}=\sum_{\beta=1}^3
{d E_{tot} \over d {\bf u}_{{\bar \mu},{\bar s},\beta}} 
S_{i,\beta,\alpha}=\sum_{\beta=1}^3 S^{-1}_{i,\alpha,\beta}
{d E_{tot} \over d {\bf u}_{{\bar \mu},{\bar s},\beta}},
\end{equation}
where we used the fact that $S_{i,\alpha,\beta}$ is an orthogonal matrix.

This relation can be used by adding on all the operations of the point group $S_i$ and dividing by
$N_S$. In the left hand side that does not depend on
$S_i$ we obtain the symmetrised forces:
\begin{equation}
{\bf F}_{s,\alpha} ={1 \over N_S} \sum_{i=1}^{N_S} \sum_{\beta=1}^3 S^{-1}_{i,\alpha,\beta}
{\bf F}_{\bar s,\beta}.
\label{symforce}
\end{equation}
In the right-hand side, we can use the unsymmetrized
forces computed by doing a sum over the IBZ when
adding on ${\bf k}$.

Reasoning in the same way we can write a similar relationship for the second derivative of the energy with respect to two displacements:
\begin{equation}
{d^2 E_{tot} \over d {\bf u}_{\mu,s,\alpha} d {\bf u}_{\nu,s',\beta}} =
\sum_{\gamma=1}^3\sum_{\delta=1}^3  S^{-1}_{i,\alpha,\gamma} S^{-1}_{i,\beta,\delta}
{d^2 E_{tot} \over d {\bf u}_{{\bar \mu},{\bar s},\gamma}d {\bf u}_{{\bar \nu},{\bar s}',\delta} }.
\label{secondder}
\end{equation}

\section{The stress}
The stress is defined as the derivative of the total
energy with respect to strain:
\begin{equation}
\sigma_{\alpha,\beta} = {1\over V} {\partial E_{tot} \over \partial \varepsilon_{\alpha,\beta}}.
\end{equation}
If we rotate the solid with an operation $\{S,{\bf a}\}$ of the solid space group, the total energy does not change if we put a rotated strain. Since the strain is a second rank tensor we must put a strain
\begin{equation}
\varepsilon'_{\alpha,\beta}= \sum_{\gamma,\delta}S_{\alpha,\gamma} S_{\beta,\delta} \varepsilon_{\gamma,\delta}.
\end{equation}
With this strain we can write
\begin{equation}
E_{tot}\left(\{\varepsilon_{\alpha,\beta}\}\right) = 
E_{tot}\left(\{\varepsilon'_{\alpha,\beta}\}\right),
\end{equation}
and doing the derivatives we have:
\begin{equation}
{\partial E_{tot}\left(\{\varepsilon_{\alpha,\beta}\}\right) \over \partial \varepsilon_{\alpha,\beta}} =\sum_{\gamma,\delta=1}^3 {\partial 
E_{tot}\left(\{\varepsilon'_{\alpha,\beta}\}\right)
\over \partial \varepsilon'_{\gamma,\delta}} {\partial \varepsilon'_{\gamma,\delta} \over \partial \varepsilon_{\alpha,\beta}} = 
\sum_{\gamma,\delta=1}^3 S^{-1}_{\alpha,\gamma}
S^{-1}_{\beta,\delta}{\partial 
E_{tot}\left(\{\varepsilon'_{\alpha,\beta}\}\right)
\over \partial \varepsilon'_{\gamma,\delta}}. 
\end{equation}
Equivalently we can write:
\begin{equation}
\sigma_{\alpha,\beta}= 
\sum_{\gamma,\delta=1}^3 S^{-1}_{\alpha,\gamma}
S^{-1}_{\beta,\delta}
\sigma_{\gamma,\delta}.
\end{equation}
The working symmetrization formula is obtained adding on the $N_S$ symmetry operations of the point group
and dividing by $N_S$. We have:
\begin{equation}
\sigma_{\alpha,\beta}= {1 \over N_S} \sum_{i=1}^3
\sum_{\gamma,\delta=1}^3 S^{-1}_{i,\alpha,\gamma}
S^{-1}_{i,\beta,\delta}
\sigma_{\gamma,\delta}.
\label{symstress}
\end{equation}

{\color{dark-blue}\chapter{Symmetry in linear response codes}}
\color{black}

\section{Dynamical matrix}

The dynamical matrix is defined in term of the interatomic force constants by the relationship:
\begin{equation}
D_{s,\alpha,s',\beta}({\bf q}) =
{1 \over N \sqrt{M_s M_s'}} \sum_{\mu} \sum_{\nu}
e^{-i {\bf q} \cdot {\bf R}_\mu}
{d^2 E_{tot} \over d {\bf u}_{\mu,s,\alpha} d {\bf u}_{\nu,s',\beta}}
e^{i {\bf q} \cdot {\bf R}_\nu}.
\label{dyn_mat}
\end{equation}
We can insert a sum over $\bar \mu$ and a sum over
$\bar \nu$ and divide by $N^2$. We have
\begin{equation}
D_{s,\alpha,s',\beta}({\bf q}) =
{1 \over N^3 \sqrt{M_s M_s'}} \sum_{\mu,\bar \mu} \sum_{\nu,\bar \nu}
e^{-i {\bf q} \cdot ({\bf R}_\mu - S^{-1}{\bf R}_{\bar \mu}) }
e^{-i S {\bf q} \cdot {\bf R}_{\bar \mu}}
{d^2 E_{tot} \over d {\bf u}_{\mu,s,\alpha} d {\bf u}_{\nu,s',\beta}}
e^{i S {\bf q} \cdot {\bf R}_{\bar \nu}}
e^{i {\bf q} \cdot ({\bf R}_\nu - S^{-1}{\bf R}_{\bar \nu})}
\end{equation}
and now we can use Eq.\ref{secondder} to write
\begin{eqnarray}
D_{s,\alpha,s',\beta}({\bf q}) =
{1 \over N^3 \sqrt{M_s M_s'}} \sum_{\mu,\bar \mu, \nu,\bar \nu} \sum_{\gamma,\delta=1}^3  S^{-1}_{i,\alpha,\gamma} S^{-1}_{i,\beta,\delta}
e^{-i {\bf q} \cdot ({\bf R}_\mu - S^{-1}{\bf R}_{\bar \mu}) } \times \nonumber \\
e^{-i S{\bf q} \cdot {\bf R}_{\bar \mu}}
{d^2 E_{tot} \over d {\bf u}_{{\bar \mu},{\bar s},\gamma}d {\bf u}_{{\bar \nu},{\bar s}',\delta} }
e^{i S{\bf q} \cdot {\bf R}_{\bar \nu}}
e^{i {\bf q} \cdot ({\bf R}_\nu - S^{-1}{\bf R}_{\bar \nu})}
\end{eqnarray}
or
\begin{equation}
D_{s,\alpha,s',\beta}({\bf q}) =
{1 \over N^2 }\sum_{\mu, \nu} \sum_{\gamma,\delta=1}^3  S^{-1}_{i,\alpha,\gamma} S^{-1}_{i,\beta,\delta}
e^{-i {\bf q} \cdot ({\bf R}_\mu - S^{-1}{\bf R}_{\bar \mu}) }
D_{\bar s,\gamma,\bar s',\delta}(S{\bf q})
e^{i {\bf q} \cdot ({\bf R}_\nu - S^{-1}{\bf R}_{\bar \nu})}.
\end{equation}
If $\{S,{\bf a}\}$ is an operation of the solid space group we have:
\begin{equation}
\{S,{\bf a}\} ({\bf R}_\mu + \boldsymbol{\tau}_{s}) = 
S {\bf R}_\mu + S \boldsymbol{\tau}_{s} +{\bf a}=
{\bf R}_{\bar \mu} + \boldsymbol{\tau}_{\bar s},
\end{equation}
or
\begin{equation}
S {\bf R}_\mu -  {\bf R}_{\bar \mu} = - S \boldsymbol{\tau}_s + \boldsymbol{\tau}_{\bar s} - {\bf a},
\end{equation}
and
\begin{equation}
S {\bf R}_\nu -  {\bf R}_{\bar \nu} = - S \boldsymbol{\tau}_{s'} + \boldsymbol{\tau}_{\bar s'} - {\bf a}.
\end{equation}
Calling
\begin{equation}
{\bf R}^{S}_{\boldsymbol{\tau}_s} = S \boldsymbol{\tau}_{s} - \boldsymbol{\tau}_{\bar s},
\label{rstaus}
\end{equation}
we obtain the equation:
\begin{equation}
D_{s,\alpha,s',\beta}({\bf q}) =
\sum_{\gamma,\delta=1}^3  S^{-1}_{i,\alpha,\gamma} S^{-1}_{i,\beta,\delta}
e^{i S {\bf q} \cdot {\bf R}^{S}_{\boldsymbol{\tau}_s} }
D_{\bar s,\gamma,\bar s',\delta}(S{\bf q})
e^{-i S{\bf q} \cdot {\bf R}^{S}_{\boldsymbol{\tau}_{s'}}}.
\label{symdyn0}
\end{equation}
This expression tell us the equations satisfied by the dynamical matrix elements.
Its inverse can be obtained easily:
\begin{equation}
D_{\bar s,\alpha,\bar s',\beta}(S{\bf q}) =
\sum_{\gamma,\delta=1}^3  S_{\alpha,\gamma} S_{\beta,\delta}
e^{-i {\bf q} \cdot {\bf R}^{S}_{\boldsymbol{\tau}_s} }
D_{s,\gamma,s',\delta}({\bf q})
e^{i {\bf q} \cdot {\bf R}^{S}_{\boldsymbol{\tau}_{s'}}},
\label{dynmat_inv}
\end{equation}
and gives as an equation to find the dynamical matrix at $S{\bf q}$ if we know the dynamical matrix at
${\bf q}$.
Eq.~\ref{symdyn0} can be used in the form:
\begin{equation}
D_{s,\alpha,s',\beta}({\bf q}) ={1\over N_S}
\sum_{i=1}^{N_S}
\sum_{\gamma,\delta=1}^3  S^{-1}_{i,\alpha,\gamma} S^{-1}_{i,\beta,\delta}
e^{i S {\bf q} \cdot {\bf R}^{S}_{\boldsymbol{\tau}_s} }
\tilde D_{\bar s,\gamma,\bar s',\delta}(S{\bf q})
e^{-i S{\bf q} \cdot {\bf R}^{S}_{\boldsymbol{\tau}_{s'}}},
\label{symdyn1}
\end{equation}
to symmetrize a dynamical matrix obtained by doing sums over the ${\bf k}$ points in the IBZ. Usually in this case we reduce the ${\bf k}$ points using only the symmetries of the cogroup of ${\bf q}$ that is only the rotations for which
\begin{equation}
S {\bf q} = {\bf q} + {\bf G}_S
\end{equation}
where ${\bf G}_S$ is a reciprocal lattice vectors.
Eq.~\ref{symdyn1} simplifies as:
\begin{equation}
D_{s,\alpha,s',\beta}({\bf q}) ={1\over N_{S_{\bf q}}}
\sum_{i=1}^{N_{S_{\bf q}}}
\sum_{\gamma,\delta=1}^3  S^{-1}_{i,\alpha,\gamma} S^{-1}_{i,\beta,\delta}
\tilde D_{\bar s,\gamma,\bar s',\delta}({\bf q})
e^{i {\bf q} \cdot \left( {\bf R}^{S}_{\boldsymbol{\tau}_s} - {\bf R}^{S}_{\boldsymbol{\tau}_{s'}} \right) },
\label{symdyn2}
\end{equation}
where we used the fact that the dynamical matrix as a function of ${\bf q}$ is periodic with the periodicity of the reciprocal lattice, and
${\bf R}^{S}_{\boldsymbol{\tau}_s}+{\bf a}$ is a
Bravais lattice vector.

Also Eq.~\ref{dynmat_inv} can be specialized to the symmetries of the cogroup of ${\bf q}$
\begin{equation}
D_{\bar s,\alpha,\bar s',\beta}({\bf q}) =
\sum_{\gamma,\delta=1}^3  S_{i,\alpha,\gamma} S_{i,\beta,\delta}
D_{s,\gamma,s',\delta}({\bf q})
e^{-i {\bf q} \cdot ({\bf R}^{S}_{\boldsymbol{\tau}_s} - {\bf R}^{S}_{\boldsymbol{\tau}_{s'}})},
\label{dynmat_inv_smallq}
\end{equation}
and used to calculate all the matrix elements  
$D_{\bar s,\alpha,\bar s',\beta}({\bf q})$ once
we have calculated 
$D_{s,\alpha,s',\beta}({\bf q})$ with Eq.~\ref{symdyn2}.

\subsection{Eigenvectors of the dynamical matrix}
We have seen (in the section on forces) how the displacements in the rotated system are related to those in the original solid. We can find also the relationship between the eigenvectors of the dynamical matrix in the rotated system. A phonon displacement has
the form:
\begin{equation}
{\bf u}_{\mu,s,\alpha}= {\bf u}_{s,\alpha}({\bf q}) e^{i{\bf q}\cdot {\bf R}_\mu}.
\end{equation}
In the rotated system we must have a similar equation
\begin{equation}
{\bf u}'_{{\bar \mu},{\bar s},\alpha}= {\bf u}'_{{\bar s},\alpha}({\bf q}') e^{i{\bf q}'\cdot {\bf R}_{\bar \mu}}= \sum_{\beta=1}^3 S_{\alpha,\beta} {\bf u}_{\mu,s,\beta} = \sum_{\beta=1}^3 S_{\alpha,\beta} {\bf u}_{s,\beta}({\bf q}) e^{i{\bf q}\cdot {\bf R}_\mu}
\label{u1}.
\end{equation}
We can rewrite this equation in the form
\begin{equation}
{\bf u}'_{{\bar \mu},{\bar s},\alpha}= {\bf u}'_{{\bar s},\alpha}({\bf q}') e^{i{\bf q}'\cdot {\bf R}_{\bar \mu}}= \sum_{\beta=1}^3 S_{\alpha,\beta} {\bf u}_{s,\beta}({\bf q}) 
e^{iS {\bf q}\cdot {\bf R}_{\bar \mu} }
e^{i{\bf q}\cdot ({\bf R}_\mu - S^{-1} {\bf R}_{\bar \mu}) }, 
\label{u1}
\end{equation}
and using the quantity ${\bf R}^S_{\boldsymbol{\tau}_s}$ (see Eq.~\ref{rstaus}), we can write
Eq.~\ref{u1} as 
\begin{equation}
{\bf u}'_{{\bar s},\alpha}({\bf q}') e^{i{\bf q}'\cdot {\bf R}_{\bar \mu}}=\sum_{\beta=1}^3 S_{\alpha,\beta} {\bf u}_{s,\beta}({\bf q}) e^{iS {\bf q}\cdot {\bf R}_{\bar \mu}} e^{-i S {\bf q} \cdot \left({\bf R}^S_{\boldsymbol{\tau}_s} + {\bf a} \right)}.
\end{equation}
This shows that ${\bf q}'=S {\bf q}$ and 
\begin{equation}
{\bf u}'_{{\bar s},\alpha}(S {\bf q}) 
= \sum_{\beta=1}^3 S_{\alpha,\beta} {\bf u}_{s,\beta}({\bf q}) e^{-iS {\bf q} \cdot 
{\bf R}^S_{\boldsymbol{\tau}_s}}
e^{-iS {\bf q} \cdot {\bf a}}.
\end{equation}

\section{The charge density induced by a phonon}

To symmetrize the charge density induced by a phonon perturbation we consider a space-group operation $\{S,{\bf a}\}$
and the charge density in the rotated system. If
we look in the point $\{S,{\bf a}\} {\bf r}$ in the rotated system we should have the same charge as in the 
point ${\bf r}$ in the original system provided that we use Eq.~\ref{rotated_disp} for the displacements.
Therefore
\begin{equation}
n\left({\bf r}, \{{\bf R}_I + {\bf u}_I\}\right)
=n\left(\{S,{\bf a}\} {\bf r}, \{{\bf R}_{\bar I} + {\bf u}_{\bar I}\}\right).
\end{equation}
Deriving with respect to ${\bf u}_{\mu,s,\alpha}$ we get
\begin{eqnarray}
{d n\left({\bf r}, \{{\bf R}_I + {\bf u}_I\}\right)
\over d {\bf u}_{\mu,s,\alpha}}
&=&\sum_{\beta=1}^3 {d n\left(\{S,{\bf a}\} {\bf r}, \{{\bf R}_{\bar I} + {\bf u}_{\bar I}\}\right)
\over d {\bf u}_{\bar \mu,\bar s,\beta}} {d {\bf u}_{\bar \mu,\bar s,\beta} \over  
d {\bf u}_{\mu,s,\alpha} } \nonumber \\
&=& \sum_{\beta=1}^3 S^{-1}_{\alpha,\beta} {d n\left(\{S,{\bf a}\} {\bf r}, \{{\bf R}_{\bar I} + {\bf u}_{\bar I}\}\right)
\over d {\bf u}_{\bar \mu,\bar s,\beta}}.
\label{dnmu}
\end{eqnarray}
The charge density induced by a phonon of wave vector ${\bf q}$ is given by
\begin{equation}
{d n\left({\bf r}, \{{\bf R}_I + {\bf u}_I\}\right)
\over d {\bf u}_{s,\alpha}({\bf q})} =
\sum_\mu {d n\left({\bf r}, \{{\bf R}_I + {\bf u}_I\}\right)
\over d {\bf u}_{\mu,s,\alpha}} e^{i{\bf q}\cdot {\bf R}_\mu},
\label{drhoph}
\end{equation}
and using Eq.~\ref{dnmu} we obtain:
\begin{eqnarray}
{d n\left({\bf r}, \{{\bf R}_I + {\bf u}_I\}\right)
\over d {\bf u}_{s,\alpha}({\bf q})} &=&
{1\over N}\sum_{\mu,\bar \mu} \sum_{\beta=1}^3 S^{-1}_{\alpha,\beta} {d n\left(\{S,{\bf a}\} {\bf r}, \{{\bf R}_{\bar I} + {\bf u}_{\bar I}\}\right)
\over d {\bf u}_{\bar \mu,\bar s,\beta}}e^{iS{\bf q}\cdot {\bf R}_{\bar \mu} } e^{i{\bf q}\cdot ({\bf R}_\mu - S^{-1}{\bf R}_{\bar \mu}) } \nonumber \\
&=& \sum_{\beta=1}^3 S^{-1}_{\alpha,\beta} {d n\left(\{S,{\bf a}\} {\bf r}, \{{\bf R}_{\bar I} + {\bf u}_{\bar I}\}\right)
\over d {\bf u}_{\bar s,\beta} (S{\bf q})} e^{-iS{\bf q}\cdot {\bf R}^S_{{\boldsymbol{\tau}_s}}} e^{-iS{\bf q}\cdot {\bf a}},
\end{eqnarray}
where ${\bf R}^S_{{\boldsymbol{\tau}_s}}$ is defined in Eq.~\ref{rstaus}.

The charge density induced by a phonon can be written as:
\begin{equation}
{d n\left({\bf r}, \{{\bf R}_I + {\bf u}_I\}\right)
\over d {\bf u}_{s,\alpha}({\bf q})} = 
e^{i {\bf q} \cdot {\bf r}} {d \tilde n\left({\bf r}, \{{\bf R}_I + {\bf u}_I\}\right)
\over d {\bf u}_{s,\alpha}({\bf q})},
\end{equation}
where ${d \tilde n\left({\bf r}, \{{\bf R}_I + {\bf u}_I\}\right)
\over d {\bf u}_{s,\alpha}({\bf q})}$ is a lattice periodic function.
We need to symmetrize only the lattice periodic
part of the induced charge density, so we have
\begin{equation}
{d \tilde n\left({\bf r}, \{{\bf R}_I + {\bf u}_I\}\right)
\over d {\bf u}_{s,\alpha}({\bf q})} = e^{-i{\bf q \cdot {\bf r }}} \sum_{\beta=1}^3 S^{-1}_{\alpha,\beta} 
e^{i S {\bf q} \cdot \{S,{\bf a}\} {\bf r }}
{d \tilde n\left(\{S,{\bf a}\} {\bf r}, \{{\bf R}_{\bar I} + {\bf u}_{\bar I}\}\right)
\over d {\bf u}_{\bar s,\beta} (S{\bf q})} e^{-iS{\bf q}\cdot {\bf R}^S_{{\boldsymbol{\tau}_s}}} e^{-iS{\bf q}\cdot {\bf a}},
\label{induced_charge}
\end{equation}

This expression can be simplified if we use it for rotations that belong to the small cogroup of {\bf q}, such that $S{\bf q}={\bf q}+{\bf G}_S$.
Since ${\bf R}^S_{{\boldsymbol{\tau}_s}}+{\bf a}$ is a lattice vector we have
\begin{equation}
{d \tilde n\left({\bf r}, \{{\bf R}_I + {\bf u}_I\}\right)
\over d {\bf u}_{s,\alpha}({\bf q})} = \sum_{\beta=1}^3 S^{-1}_{\alpha,\beta} e^{i S {\bf q} \cdot {\bf a}} {d \tilde n\left(\{S,{\bf a}\} {\bf r}, \{{\bf R}_{\bar I} + {\bf u}_{\bar I}\}\right)
\over d {\bf u}_{\bar s,\beta} (S{\bf q})} e^{-i{\bf q}\cdot {\bf R}^S_{{\boldsymbol{\tau}_s}}} e^{-i{\bf q}\cdot {\bf a}}.
\end{equation}
Now using the fact that 
\begin{equation}
{d \tilde n\left(\{S,{\bf a}\}{\bf r}, \{{\bf R}_I + {\bf u}_I\}\right)
\over d {\bf u}_{\bar s,\beta}({\bf q}+{\bf G}_S)} 
= e^{-i{\bf G}_S \cdot (S {\bf r} + {\bf a})}
{d \tilde n\left(\{S,{\bf a}\}{\bf r}, \{{\bf R}_I + {\bf u}_I\}\right)
\over d {\bf u}_{\bar s,\beta}({\bf q})}.
\end{equation}
we have
\begin{equation}
{d \tilde n\left({\bf r}, \{{\bf R}_I + {\bf u}_I\}\right)
\over d {\bf u}_{s,\alpha}({\bf q})} 
= e^{-iS^{-1} {\bf G}_S \cdot {\bf r}}
\sum_{\beta=1}^3 S^{-1}_{\alpha,\beta} {d \tilde n\left(\{S,{\bf a}\} {\bf r}, \{{\bf R}_{\bar I} + {\bf u}_{\bar I}\}\right)
\over d {\bf u}_{\bar s,\beta} ({\bf q})} e^{-i{\bf q}\cdot {\bf R}^S_{{\boldsymbol{\tau}_s}}}.
\label{symdrho_one}
\end{equation}
This equation can be written also as 
\begin{equation}
{d \tilde n\left({\bf r}, \{{\bf R}_I + {\bf u}_I\}\right)
\over d {\bf u}_{s,\alpha}({\bf q})} 
= e^{i{\bf G}_{S^{-1}} \cdot {\bf r}}
\sum_{\beta=1}^3 S^{-1}_{\alpha,\beta} {d \tilde n\left(\{S,{\bf a}\} {\bf r}, \{{\bf R}_{\bar I} + {\bf u}_{\bar I}\}\right)
\over d {\bf u}_{\bar s,\beta} ({\bf q})} e^{-i{\bf q}\cdot {\bf R}^S_{{\boldsymbol{\tau}_s}}}.
\label{symdrho}
\end{equation}
using the fact that $S^{-1} {\bf G}_S=-{\bf G}_{S^{-1}}$
as can be deduced comparing the two expressions:
\begin{eqnarray}
S{\bf q}&=&{\bf q} + {\bf G}_S \nonumber \\
S^{-1}{\bf q}&=&{\bf q} + {\bf G}_{S^{-1}}
\end{eqnarray}
after applying $S^{-1}$ to the first.

This equation is further transformed before implementation (see the Appendix on irrep).

\section{The charge density induced by an electric field}
Reasoning as above we can say that if $\{S,{\bf a}\}$ is a space group operation the charge density
at the point ${\bf r}$ when there is an electric field ${\bf E}$ is equal to the charge density at the rotated point  $\{S,{\bf a}\}{\bf r}$ when the electric field
is ${\bf E}'=S {\bf E}$. Therefore
\begin{equation}
n\left({\bf r}, {\bf E}\right)
=n\left(\{S,{\bf a}\} {\bf r}, {\bf E}'\right), 
\end{equation}
and deriving with respect to ${\bf E}_\alpha$ we get
\begin{equation}
{d n\left({\bf r}, {\bf E}\right)
\over d {\bf E}_\alpha }=\sum_{\beta=1}^3 S^{-1}_{\alpha,\beta} {d n\left(\{S,{\bf a}\} {\bf r}, {\bf E}'\right)
\over d {\bf E}'_\beta }.
\end{equation}

\section{The dielectric constant}
The dielectric constant can be written as 
\begin{equation}
\epsilon_{\alpha,\beta}=\delta_{\alpha,\beta} -{4 \pi
\over V} {\partial^2 E_{tot} ({\bf E}) \over \partial {\bf E}_\alpha \partial {\bf E}_\beta}.
\end{equation}
For an operation $\{S,{\bf a}\}$ of the solid space group and a rotated electric field to ${\bf E}'=S{\bf E}$, the total energy in the rotated system does not change. We have therefore:
\begin{equation}
{\partial^2 E_{tot} ({\bf E}) \over \partial {\bf E}_\alpha \partial {\bf E}_\beta}=\sum_{\gamma,\delta=1}^3
S^{-1}_{\alpha,\gamma} S^{-1}_{\beta,\delta}
{\partial^2 E_{tot} ({\bf E}') \over \partial {\bf E}'_\gamma \partial {\bf E}'_\delta},
\end{equation}
or equivalently we can write
\begin{equation}
\epsilon_{\alpha,\beta}=\sum_{\gamma,\delta=1}^3
S^{-1}_{\alpha,\gamma} S^{-1}_{\beta,\delta}
\epsilon_{\gamma,\delta}.
\end{equation}
The working symmetrization formula is obtained adding on the $N_S$ symmetry operations of the point group
and dividing by $N_S$. We have:
\begin{equation}
\epsilon_{\alpha,\beta}= {1 \over N_S} \sum_{i=1}^3
\sum_{\gamma,\delta=1}^3 S^{-1}_{i,\alpha,\gamma}
S^{-1}_{i,\beta,\delta}
\epsilon_{\gamma,\delta}.
\label{symepsilon}
\end{equation}

\section{The Born effective charges}
Born effective charges are second derivative of the total energy with respect to phonon displacements and electric field component:
\begin{equation}
qZ^*_{s,\alpha,\beta}= {1\over N} {\partial E_{tot} \over \partial {\bf u}_{s,\alpha} \partial {\bf E}_\beta}.
\end{equation}

So we use the fact that for an operation $\{S,{\bf a}\}$ of the point group of the solid in which the atoms goes in $\{S,{\bf a}\} ({\bf R}_I+{\bf u}_I)$ and the electric field becomes 
${\bf E}'=S{\bf E}$ the total energy should not change.
\begin{equation}
E_{tot}(\{{\bf R}_I+{\bf u}_I\},{\bf E}) =
E_{tot}(\{\{S,{\bf a}\}({\bf R}_I+{\bf u}_I)\},{\bf E}'),
\end{equation}
and performing the derivatives we have:
\begin{equation}
{\partial E_{tot}(\{{\bf R}_I+{\bf u}_I\},{\bf E}) \over  \partial {\bf u}_{s,\alpha} \partial {\bf E}_\beta} = \sum_{\gamma,\delta} S^{-1}_{\alpha,\gamma} S^{-1}_{\beta,\delta} {\partial 
E_{tot}(\{\{S,{\bf a}\}({\bf R}_I+{\bf u}_I)\},{\bf E}')
\over \partial {\bf u}_{\bar s,\gamma} \partial {\bf E}'_\delta},
\end{equation}
or equivalently
\begin{equation}
Z^*_{s,\alpha,\beta}
= \sum_{\gamma,\delta} S^{-1}_{\alpha,\gamma} S^{-1}_{\beta,\delta} 
Z^*_{\bar s,\gamma,\delta}.
\end{equation}

\newpage
{\color{dark-blue}\chapter{Symmetry with spinors}}
In this chapter we present the modifications of the previous considerantions needed to deal with spinor wavefunctions.
Spinors can be used together with fully relativistic pseudopotentials to include the effects of spin-orbit coupling. In this case, we can distinguish calculations performed in presence of time-reversal symmetry (nonmagnetic material) or without time reversal symmetry (magnetic material). In the latter case there might be symmetry operations that belong to the solid space group only when coupled with time reversal. If this happens, half rotations of the point group require time reversal and half do not require it. We indicate with $\{\mathcal{T}S,{\bf a}\}$ an operation of the space group that requires time reversal. 

In the context of spinors it is also useful to define for each symmetry operation (with or without time reversal), the operator $\tilde S$ defined as $\tilde S=S$ if $S$ is a proper rotation or $\tilde S= I S$ if $S$ is an improper rotation. Here $I$ is the inversion operator. Therefore $\tilde S$ is always a proper rotation. The inversion operation is defined as $I {\bf r }=-{\bf r}$.

For each rotation matrix $S$, there is in spin space a $2\times 2$ matrix that belongs to the $SU(2)$ group that we indicate with $U(\tilde S)$ that acts on the spinor part of the wavefunctions. It depends only on $\tilde S$ since inversion does not affect the spin part of the wavefunctions.
A proper rotation is characterized by the versor of the
rotation axis ${\bf n}$ and the rotation angle
$\theta$ and the operator $U(\tilde S)$ can be written as:
\begin{equation}
U(\tilde S)= e^{-i {\theta \over 2} \boldsymbol{\sigma} \cdot {\bf n} },
\end{equation}
where $\boldsymbol{\sigma}$ are the Pauli matrices. 

\section{The Kohn and Sham wavefunctions}

In the noncollinear case, the Kohn and Sham equations are
\begin{equation}
\sum_{\sigma'} H_{KS}^{\sigma,\sigma'}
\psi_{{\bf k},n,\sigma'}({\bf r})=
\epsilon_{{\bf k},n} \psi_{{\bf k},n,\sigma}({\bf r}),
\end{equation}
and the Kohn and Sham Hamiltonian is a $2\times 2$ matrix acting on two component spinors:
\begin{equation}
\Psi_{{\bf k},n}= \left( \begin{array}{c}
\psi_{{\bf k},n,\uparrow}({\bf r})  \\
\psi_{{\bf k},n,\downarrow} ({\bf r})
\end{array}
\right).
\end{equation}
Each spinor component is a Bloch function that
can be written as:
\begin{equation}
\psi_{{\bf k},n,\sigma}({\bf r})=
e^{i{\bf k}\cdot {\bf r}} u_{{\bf k},n,\sigma}({\bf r}),
\end{equation}
where $u_{{\bf k},n,\sigma}({\bf r})$ is lattice periodic.

The operator $\mathcal{T}$ that applies time reversal to these spinors is an antilinear operator that can be written as:
\begin{equation}
\mathcal{T}=i \sigma_y \mathcal{K},
\end{equation}
where $\sigma_y$ is the Pauli matrix:
\begin{equation}
\sigma_y= \left( \begin{array}{cc}
0, & -i  \\
i, & 0  
\end{array}
\right)
\end{equation}
and $\mathcal{K}$ is the complex conjugation operator.
We have therefore:
\begin{equation}
\mathcal{T}\Psi_{{\bf k},n}= \left( \begin{array}{c}
\psi^*_{{\bf k},n,\downarrow}({\bf r})  \\
-\psi^*_{{\bf k},n,\uparrow} ({\bf r})
\end{array}
\right).
\end{equation}
and this is a Bloch function with wave-vector
$-{\bf k}$. If the Hamiltonian is time reversal invariant, it commutes with $\mathcal{T}$ and
$\mathcal{T}\Psi_{{\bf k},n}$ is also an
eigenfunction of the Hamiltonian with the same
eigenvalue $\epsilon_{{\bf k},n}$. We can call
it $\Psi_{-{\bf k},n}$.

For each element of the space group $\{S,{\bf a}\}$ there will be an operator $O_{\{S,{\bf a}\}}$ that acts on the spinor wavefunctions
as:
\begin{equation}
\left(O_{\{S,{\bf a}\}} \Psi_{{\bf k},n}\right)_{\sigma} = \sum_{\sigma'} U(\tilde S)_{\sigma,\sigma'} \psi_{{\bf k},n,\sigma'}(\{S,{\bf a}\}^{-1}{\bf r}). 
\end{equation}

\section{Charge and magnetization density}
When the wavefunctions are spinors, the charge density is:
\begin{equation}
n({\bf r}) = \sum_{{\bf k}, v, \sigma} |\psi_{{\bf k}, v, \sigma} ({\bf r}) |^2,
\end{equation}
One can demonstrate [2] that when we rotate the system with the operation $\{S,{\bf a}\}$ or with the operation $\{\mathcal{T} S,{\bf a}\}$
the charge density satisfies the same relation as in Eq.~\ref{symchden}:
\begin{equation}
n({\bf r}) = {1 \over N_S} \sum_{i=1}^{N_S} \tilde n ( \{S_i,{\bf f}_i\}^{-1} {\bf r}).
\end{equation}
For the magnetization density defined as
\begin{equation}
{\bf m}_\alpha({\bf r}) = \mu_B \sum_{{\bf k}, v, \sigma,\sigma'} \psi_{{\bf k}, v, \sigma} ({\bf r})^* \boldsymbol{\sigma}_{\alpha}^{\sigma,\sigma'}
\psi_{{\bf k}, v, \sigma'} ({\bf r}),
\end{equation}
where $\mu_B$ is the Bohr magneton, one has
\begin{equation}
{\bf m}_\alpha({\bf r}) = {1 \over N_S} \sum_{i=1}^{N_S} (-1)^{T_{S_i}} \tilde S_{i,\alpha,\beta} \tilde {\bf m}_{\beta} ( \{S_i,{\bf f}_i\}^{-1} {\bf r}),
\label{symmag}
\end{equation}
where $\tilde S_i$ has been defined above and 
$T_{S_i}=1$ if the operation needs time reversal and $T_{S_i}=0$ otherwise.

We note that to calculate 
$\tilde {\bf m}_{\beta} ({\bf r})$ we must sum over the ${\bf k}$ points belonging to the irreducible Brillouin zone. This zone is found noting that for operations that require time reversal:
\begin{equation}
\mathcal{T} S{\bf k} = - S{\bf k}.
\end{equation}

\section{Forces and stress}
No change is made in the symmetrization expression for forces and stress due to the
time reversal operator since only the spatial part of the symmetry is used to rotate the atomic displacements and the strain.

\section{Dynamical matrix}
Eq.~\ref{symdyn1} is still valid both for
operations that require time reversal and for
those that do not require it:
\begin{equation}
D_{s,\alpha,s',\beta}({\bf q}) ={1\over N_S}
\sum_{i=1}^{N_S}
\sum_{\gamma,\delta=1}^3  S^{-1}_{i,\alpha,\gamma} S^{-1}_{i,\beta,\delta}
e^{i S_i {\bf q} \cdot {\bf R}^{S_i}_{\boldsymbol{\tau}_s} }
\tilde D_{\bar s,\gamma,\bar s',\delta}(S_i{\bf q})
e^{-i S{\bf q} \cdot {\bf R}^{S_i}_{\boldsymbol{\tau}_{s'}}}.
\label{symdyn1_again}
\end{equation}
We use these relations only for symmetries that belong to the small cogroup of 
${\bf q}$:
\begin{equation}
S{\bf q} = {\bf q} + {\bf G}_S
\end{equation}
if they do not require time reversal, or for which
\begin{equation}
S{\bf q} = -{\bf q} + {\bf G}_S
\end{equation}
if they require time reversal. Introducing the operator $O(S_i)=\mathcal{K}$ if the operation is $\{\mathcal{T}S_i,{\bf a}_i\}$
or $O(S_i)=\mathbb{I}$ if the operation
does not require time reversal, we can write
\begin{equation}
D_{s,\alpha,s',\beta}({\bf q}) ={1\over N_S}
\sum_{i=1}^{N_S}
\sum_{\gamma,\delta=1}^3  S^{-1}_{i,\alpha,\gamma} S^{-1}_{i,\beta,\delta} O(S_i)\left[
e^{i {\bf q} \cdot {\bf R}^{S_i}_{\boldsymbol{\tau}_s} }
\tilde D_{\bar s,\gamma,\bar s',\delta}({\bf q})
e^{-i {\bf q} \cdot {\bf R}^{S_i}_{\boldsymbol{\tau}_{s'}}}\right],
\label{symdyn1_again1}
\end{equation}
where we used the fact that
\begin{equation}
\tilde D_{\bar s,\gamma,\bar s',\delta}(-{\bf q})= \tilde D_{\bar s,\gamma,\bar s',\delta}^*({\bf q}).
\label{dynqmq}
\end{equation}
Finally, in order to apply Eq.~\ref{dynmat_inv} we must recognize that for operations that require time reversal it actually reads: 
\begin{equation}
D^*_{\bar s,\alpha,\bar s',\beta}(S{\bf q}) =
\sum_{\gamma,\delta=1}^3  S_{i,\alpha,\gamma} S_{i,\beta,\delta}
e^{-i {\bf q} \cdot {\bf R}^{S}_{\boldsymbol{\tau}_s} }
D_{s,\gamma,s',\delta}({\bf q})
e^{i {\bf q} \cdot {\bf R}^{S}_{\boldsymbol{\tau}_{s'}}},
\label{dynmat_inv_tr}
\end{equation}
What written here (and in particular  Eq.~\ref{dynqmq}) is valid for the adiabatic dynamical matrix at zero frequency. For magnetic material however a dynamical description (as well as possibly the coupling with magnons) for which Eq.~\ref{dynqmq} does not hold, is sometimes necessary to recover the experimental symmetries of the phonon modes. 

\section{Charge density and magnetization induced by a phonon}
Eq.~\ref{induced_charge} that gives the charge
induced by a phonon after an operation of the
solid space group is still valid also if the operation requires time reversal. We have: 

\begin{equation}
{d \tilde n\left({\bf r}, \{{\bf R}_I + {\bf u}_I\}\right)
\over d {\bf u}_{s,\alpha}({\bf q})} = e^{-i{\bf q \cdot {\bf r }}} \sum_{\beta=1}^3 S^{-1}_{\alpha,\beta} 
e^{i S {\bf q} \cdot \{S,{\bf a}\} {\bf r }}
{d \tilde n\left(\{S,{\bf a}\} {\bf r}, \{{\bf R}_{\bar I} + {\bf u}_{\bar I}\}\right)
\over d {\bf u}_{\bar s,\beta} (S{\bf q})} e^{-iS{\bf q}\cdot {\bf R}^S_{{\boldsymbol{\tau}_s}}} e^{-iS{\bf q}\cdot {\bf a}},
\label{induced_charge_1}
\end{equation}
For the practical implementation we choose to symmetrize only with operations 
$\{S,{\bf a}\}$ for which
\begin{equation}
S {\bf q} = {\bf q} + {\bf G}_S.
\end{equation}
and with operations
$\{\mathcal{T}S,{\bf a}\}$ for which 
\begin{equation}
S {\bf q} = - {\bf q} + {\bf G}_S.
\end{equation}
For the former one obtains the Eq.~\ref{symdrho}, while for the later we have
\begin{equation}
{d \tilde n\left({\bf r}, \{{\bf R}_I + {\bf u}_I\}\right)
\over d {\bf u}_{s,\alpha}({\bf q})} = \sum_{\beta=1}^3 S^{-1}_{\alpha,\beta} e^{i S {\bf q} \cdot {\bf a}} {d \tilde n\left(\{S,{\bf a}\} {\bf r}, \{{\bf R}_{\bar I} + {\bf u}_{\bar I}\}\right)
\over d {\bf u}_{\bar s,\beta} (S{\bf q})} e^{i{\bf q}\cdot {\bf R}^S_{{\boldsymbol{\tau}_s}}} e^{i{\bf q}\cdot {\bf a}}.
\end{equation}
Now using the fact that 
\begin{equation}
{d \tilde n\left(\{S,{\bf a}\}{\bf r}, \{{\bf R}_I + {\bf u}_I\}\right)
\over d {\bf u}_{\bar s,\beta}(-{\bf q}+{\bf G}_S)} 
= e^{-i{\bf G}_S \cdot (S {\bf r} + {\bf a})}
{d \tilde n\left(\{S,{\bf a}\}{\bf r}, \{{\bf R}_I + {\bf u}_I\}\right)
\over d {\bf u}_{\bar s,\beta}(-{\bf q})}.
\end{equation}
we have
\begin{equation}
{d \tilde n\left({\bf r}, \{{\bf R}_I + {\bf u}_I\}\right)
\over d {\bf u}_{s,\alpha}({\bf q})} 
= e^{-iS^{-1} {\bf G}_S \cdot {\bf r}}
\sum_{\beta=1}^3 S^{-1}_{\alpha,\beta} {d \tilde n\left(\{S,{\bf a}\} {\bf r}, \{{\bf R}_{\bar I} + {\bf u}_{\bar I}\}\right)
\over d {\bf u}_{\bar s,\beta} (-{\bf q})} e^{i{\bf q}\cdot {\bf R}^S_{{\boldsymbol{\tau}_s}}}.
\label{symdrho_one_1}
\end{equation}
This equation can be written also as: 
\begin{equation}
{d \tilde n\left({\bf r}, \{{\bf R}_I + {\bf u}_I\}\right)
\over d {\bf u}_{s,\alpha}({\bf q})} 
= e^{-i{\bf G}_{S^{-1}} \cdot {\bf r}}
\sum_{\beta=1}^3 S^{-1}_{\alpha,\beta} {d \tilde n\left(\{S,{\bf a}\} {\bf r}, \{{\bf R}_{\bar I} + {\bf u}_{\bar I}\}\right)
\over d {\bf u}_{\bar s,\beta} (-{\bf q})} e^{i{\bf q}\cdot {\bf R}^S_{{\boldsymbol{\tau}_s}}}.
\label{symdrho1}
\end{equation}
using the fact that $S^{-1} {\bf G}_S={\bf G}_{S^{-1}}$
as can be deduced comparing the two expressions:
\begin{eqnarray}
S{\bf q}&=&-{\bf q} + {\bf G}_S \nonumber \\
S^{-1}{\bf q}&=&-{\bf q} + {\bf G}_{S^{-1}}
\end{eqnarray}
after applying $S^{-1}$ to the first. Note that if the operation requires time reversal, also its inverse requires it.
Eqs.~\ref{symdrho} and \ref{symdrho1} can be summarized with the expression:
\begin{equation}
{d \tilde n\left({\bf r}, \{{\bf R}_I + {\bf u}_I\}\right)
\over d {\bf u}_{s,\alpha}({\bf q})} 
= O(S)\left[ e^{i{\bf G}_{S^{-1}} \cdot {\bf r}}
\sum_{\beta=1}^3 S^{-1}_{\alpha,\beta} {d \tilde n\left(\{S,{\bf a}\} {\bf r}, \{{\bf R}_{\bar I} + {\bf u}_{\bar I}\}\right)
\over d {\bf u}_{\bar s,\beta} ({\bf q})} e^{-i{\bf q}\cdot {\bf R}^S_{{\boldsymbol{\tau}_s}}}\right].
\label{symdrho_final1}
\end{equation}
This equation is further transformed before implementation (see the appendix on \texttt{irrep}).

Reasoning in the same manner, starting from
Eq.~\ref{symmag} for the magnetization density, we obtain the equation
\begin{equation}
{d \tilde {\bf m}_\gamma \left({\bf r}, \{{\bf R}_I + {\bf u}_I\}\right)
\over d {\bf u}_{s,\alpha}({\bf q})} 
= (-1)^{T_{S}} O(S)\left[ e^{i{\bf G}_{S^{-1}} \cdot {\bf r}}
\sum_{\beta,\delta=1}^3 \tilde S^{-1}_{\gamma,\delta} S^{-1}_{\alpha,\beta} {d \tilde {\bf m}_\delta\left(\{S,{\bf a}\} {\bf r}, \{{\bf R}_{\bar I} + {\bf u}_{\bar I}\}\right)
\over d {\bf u}_{\bar s,\beta} ({\bf q})} e^{-i{\bf q}\cdot {\bf R}^S_{{\boldsymbol{\tau}_s}}}\right].
\label{symdmag_final1}
\end{equation}
This equation is further transformed before implementation (see the appendix on \texttt{irrep}).


\section{Dielectric constant and Born effective charges}

No change is made in the symmetrization expression for dielectric constants and Born effective charges due to the
time reversal operator since only the spatial part of the symmetry is used to rotate the displacements and the electric field. 

\newpage
{\color{dark-blue}\chapter{Appendices}}
\color{black}
The main symmetry variables available in Quantum ESPRESSO and their relationship with the symbol used in these notes are:

\begin{itemize}

\item
\texttt{at(3,3)} contains the principal lattice vectors ${\bf a}_{\alpha,i}$ with this order for the two indices (the first is cartesian, the second indicates the vector). 

\item
\texttt{bg(3,3)} contains the principal reciprocal lattice vectors ${\bf b}_{\alpha,i}$ with this order for the two indices (the first is cartesian, the second indicates the vector).

\item 
\texttt{nsym} is $N_S$ the number of symmetry operations of the point group.

\item
\texttt{s(3,3,48)} contains $S_{i,l,m}$. The symmetry index $i$ is the third, while $l,m$ are the first two indices.

\item
\texttt{sr(3,3,48)} contains $S_{i,\alpha,\beta}$, the rotation matrices in Cartesian coordinates.

\item
\texttt{ft(3,48)} contains $-{\bf f}_i$ the fractional translation with the negative sign.
It is in crystal coordinates of the principal lattice vector. The symmetry index $i$ is the second.

\item
\texttt{invs(48)} for each symmetry $i$ gives the index of the inverse of $S_i$.

\item
\texttt{irt(48,nat)} for each atom $s$ and symmetry $i$ gives the index of the atom $\bar s$.

\item 
\texttt{nsymq} is $N_{S_{\bf q}}$ the number of symmetry operations of the point cogroup of the
{\bf q} point.

\item 
\texttt{rtau(3,48,nat)} contains ${\bf R}^{S_i}_{\boldsymbol{\tau}_s}$ (see Eq.~\ref{rstaus}) in Cartesian coordinates.

\item
\texttt{gi(3,48)} The vector ${\bf G}_{S_i^{-1}}$
associated to each symmetry $S_i^{-1}{\bf q}={\bf q}+{\bf G}_{S_i^{-1}}$.

\item 
\texttt{t\_rev(48)} Contains $T_{S_i}$.  For each symmetry operation it is equal to $1$ if the symmetry requires time reversal, zero otherwise (used only in the noncollinear magnetic case).

\item
\texttt{u(3*nat,3*nat)} contains the displacements 
of each ($A^p_{s,\alpha}$). The second index is the
index $p$ of the different modes of the \texttt{irrep}, while the first is the composite index ${s,\alpha}$.

\item 
\texttt{t(npertx,npertx,48,3*nat)} contains the rotation matrices in the basis of the modes $t^S_{q,p}$.
\texttt{npertx} is the maximum number of perturbations in a single \texttt{irrep}. The last index labels the \texttt{irreps}.

\item 
\texttt{tmq(npertx,npertx,3*nat)} contains the rotation matrix $S_m$ that sends $\bf {q}$ in
$-{\bf q}+{\bf G}_{S_m}$ in the basis of the modes $t^{S_m}_{q,p}$.



\end{itemize}

\section{Appendix: Symvector} 
This routine should apply Eq.~\ref{symforce}. It actually calculates 
\begin{equation}
{\bf F}_{s,\alpha} ={1 \over N_S} \sum_{i=1}^{N_S} \sum_{\beta, \delta, \epsilon, l, m} {\bf b}_{\alpha,l} {\bf a}_{\epsilon,l} S^{-1}_{i,\epsilon,\beta} {\bf b}_{\beta,m} {\bf a}_{\delta,m}
{\bf F}_{\bar s,\delta},
\label{symforceapp1}
\end{equation}
obtained introducing two delta functions in Eq.~\ref{symforce}. The sum on $\delta$ in the last two terms gives the unsymmetrized forces in the basis of the
reciprocal primitive vectors ${\bf F}_{{\bar s},m}$ and
using Eq.~\ref{slj} to make the sum over {$\beta$ and $\epsilon$} we can write 
\begin{equation}
{\bf F}_{s,\alpha} ={1 \over N_S} \sum_{i=1}^{N_S} \sum_{l, m} {\bf b}_{\alpha,l} S_{i,l,m}  {\bf F}_{\bar s,m},
\label{symforceapp2}
\end{equation}
which is the expression implemented in this routine.

\section{Appendix: Symmatrix}
This routine symmetrizes $3\times 3$ matrices such as the stress or the dielectric constants. These two quantities can be symmetrized with the same formula
(see Eq.~\ref{symstress} and \ref{symepsilon}).
We write the espression for stress:
\begin{equation}
\sigma_{\alpha,\beta}={1 \over N_S} \sum_{i=1}^{N_S}
\sum_{\gamma, \delta, \epsilon, \eta, \nu, \rho} \sum_{l, m, n, p} {\bf b}_{\alpha,l} {\bf a}_{\epsilon,l} S^{-1}_{i,\epsilon,\eta} {\bf b}_{\eta,m} {\bf a}_{\gamma,m} {\bf b}_{\beta,n} {\bf a}_{\nu,n} S^{-1}_{i,\nu,\rho} {\bf b}_{\rho,p} {\bf a}_{\delta,p} \sigma_{\gamma,\delta}.
\end{equation}
The sum over $\gamma$ and $\delta$ give the stress in the basis of the primitive reciprocal vectors
\begin{equation}
\sigma_{m,p}=\sum_{\gamma, \delta=1}^3
{\bf a}_{\gamma,m} {\bf a}_{\delta,p} \sigma_{\gamma,\delta},
\label{smp}
\end{equation}
and Eq.~\ref{slj} used twice to do the sums over $\epsilon$, $\eta$, $\nu$ and $\rho$ gives
\begin{equation}
\sigma_{\alpha,\beta}={1 \over N_S} \sum_{i=1}^{N_S}
 \sum_{l, m, n, p} {\bf b}_{\alpha,l} {\bf b}_{\beta,n}S_{i,l,m} S_{i,n,p} \sigma_{m,p}.
\end{equation}

\section{Appendix: Symtensor}
With the same logic seen for the two previous routines, 
we find the expression used for symmetrizing the
Born effective charges by this routine
\begin{equation}
Z^*_{s,\alpha,\beta}={1 \over N_S} \sum_{i=1}^{N_S}
 \sum_{l, m, n, p} {\bf b}_{\alpha,l} {\bf b}_{\beta,n}S_{i,l,m} S_{i,n,p} Z^*_{\bar s,m,p},
\end{equation}
where 
\begin{equation}
Z^*_{\bar s,m,p} = \sum_{\gamma, \delta=1}^3 {\bf a}_{\gamma,m} {\bf a}_{\delta,p} Z^*_{\bar s,\gamma,\delta}
\end{equation}
is the Born effective charge written in the basis of the primitive reciprocal lattice vectors.

\section{Appendix: Crys\_to\_cart}
This routine receives a rank-two tensor in Cartesian coordinates and transforms it into a tensor in the basis of the primitive reciprocal vectors. It implements the Eq.~\ref{smp}:
\begin{equation}
\sigma_{m,p}=\sum_{\gamma, \delta=1}^3
{\bf a}_{\gamma,m} {\bf a}_{\delta,p} \sigma_{\gamma,\delta}.
\end{equation}

\section{Appendix: Cart\_to\_crys}
This routine receives a rank-two tensor in the basis of the primitive reciprocal lattice and transforms it into a tensor in Cartesian coordinates. It implements the equation:
\begin{equation}
\sigma_{\alpha,\beta}=\sum_{m,p=1}^3
{\bf b}_{\alpha,m} {\bf b}_{\beta,p} \sigma_{m,p}.
\end{equation}

\section{Appendix: Rotate\_grid\_point}

This routine receives the three integers
\texttt{(i,j,k)} that represent the point:
\begin{equation}
{\bf r}_{i,j,k}= {(i-1) \over N_1} {\bf a}_1 +
{(j-1) \over N_2} {\bf a}_2 +
{(k-1) \over N_3} {\bf a}_3,
\end{equation}
where $N_1$, $N_2$, and $N_3$ are the sizes of the FFT mesh and gives as output the three integers
\texttt{(ri,rj,rk)} that represent the point
${\bf r}'_{ri,rj,rk}=S {\bf r}_{i,j,k}+{\bf f}$ with
\begin{equation}
{\bf r}'_{ri,rj,rk}= {(ri-1) \over N_1} {\bf a}_1 +
{(rj-1) \over N_2} {\bf a}_2 +
{(rk-1) \over N_3} {\bf a}_3.
\end{equation}
Using Eq.~\ref{rotvcry}, we obtain:
\begin{eqnarray}
{(ri-1) \over N_1} &=& S_{1,1} {(i-1) \over N_1}+
S_{2,1} {(j-1) \over N_2} + S_{3,1} {(k-1) \over N_3} + {\bf f}_1, \nonumber \\
{(rj-1) \over N_2} &=& S_{1,2} {(i-1) \over N_1}+
S_{2,2} {(j-1) \over N_2} + S_{3,2} {(k-1) \over N_3} + {\bf f}_2, \nonumber \\
{(rk-1) \over N_3} &=& S_{1,3} {(i-1) \over N_1}+
S_{2,3} {(j-1) \over N_2} + S_{3,3} {(k-1) \over N_3} + {\bf f}_3, 
\end{eqnarray}
which can be rewritten as
\begin{eqnarray}
ri-1 &=& S_{1,1} (i-1) +
{S_{2,1} N_1 \over N_2} (j-1) + {S_{3,1} N_1 \over N_3} (k-1) + {\bf f}_1 N_1, \nonumber \\
rj-1 &=& {S_{1,2} N_2 \over N_1} (i-1)+
S_{2,2} (j-1) + {S_{3,2} N_2 \over N_3}(k-1) + {\bf f}_2 N_2, \nonumber \\
rk-1 &=& {S_{1,3} N_3 \over N_1} (i-1) +
{S_{2,3} N_3 \over N_2}(j-1) + S_{3,3} (k-1)  + {\bf f}_3 N_3. 
\end{eqnarray}
The routine \texttt{scale\_sym\_ops} updates the coefficients of the matrix $S$ which are given in input to this routine. 

\section{Appendix: Sym\_rho\_serial}

This routine first applies Eq.~\ref{symng} to save in \texttt{rhosum} the symmetrized charge density, and then uses Eq.~\ref{ngdist} to distribute the results to all the components
$S_i {\bf G}$. ${\bf G}$ is transformed in the basis of the principal reciprocal lattice 
vectors and it is rotated with Eq.~\ref{rotreccomp}, so actually the rotation
\texttt{ns} is applied not its inverse. \texttt{ft} is transformed in crystal coordinates at the beginning of the routine and the vector $S_i {\bf G}$ is actually \texttt{g\_(:,igs)}. Note also that \texttt{ft} contains $-{\bf f}_i$ so the
applied phase is actually $e^{iS_i{\bf G}\cdot {\bf f}}$ as written in Eq.~\ref{symng}.

\section{Appendix: The irrep}
In the phonon code we do not calculate derivatives of the charge density with respect to 
${\bf u}_{s,\alpha}({\bf q})$, but linear combinations of these derivatives that transform as irreducible representations of the small space group of ${\bf q}$. 

Suppose these linear combinations are given by $3\times N_{at}$ vectors $A^{p}_{s,\alpha}$, where
$1\le p \le 3 \times N_{at}$. These vectors are obtained in the code as eigenvectors of a random matrix which is symmetrized as the dynamical matrix according to Eq.~\ref{symdyn2}. These vectors are orthogonal, form a complete set, and are a basis for irreducible representations of the small co-group of ${\bf q}$.
Orthogonality means that
\begin{equation}
\sum_{s,\alpha} A^{*,p}_{s,\alpha} A^{q}_{s,\alpha}=\delta^{p,q}.
\end{equation}
Completeness means that
\begin{equation}
\sum_{p} A^{p}_{s,\alpha} A^{*,p}_{s',\beta}=\delta_{s,s'} \delta_{\alpha,\beta}.
\end{equation}
Making linear combinations of Eq.~\ref{symdrho}
we get
\begin{eqnarray}
\sum_{s,\alpha} A^p_{s,\alpha} {d \tilde n\left({\bf r}, \{{\bf R}_I + {\bf u}_I\}\right)
\over d {\bf u}_{s,\alpha}({\bf q})} 
&=& e^{i{\bf G}_{S^{-1}} \cdot {\bf r}}
\sum_{s,\alpha}
\sum_{\beta=1}^3 A^p_{s,\alpha} S^{-1}_{\alpha,\beta} {d \tilde n\left(\{S,{\bf a}\} {\bf r}, \{{\bf R}_{\bar I} + {\bf u}_{\bar I}\}\right)
\over d {\bf u}_{\bar s,\beta} ({\bf q})} e^{-i{\bf q}\cdot {\bf R}^S_{{\boldsymbol{\tau}_s}}}\nonumber \\
&=& e^{i{\bf G}_{S^{-1}} \cdot {\bf r}}
\sum_{q,s,\alpha, \bar{\bar s},\gamma,\beta} A^p_{s,\alpha} S^{-1}_{\alpha,\beta} e^{-i{\bf q}\cdot {\bf R}^S_{{\boldsymbol{\tau}_s}}} A^{*,q}_{\bar s,\beta} A^q_{\bar {\bar s},\gamma} {d \tilde n\left(\{S,{\bf a}\} {\bf r}, \{{\bf R}_{\bar I} + {\bf u}_{\bar I}\}\right)
\over d {\bf u}_{\bar {\bar s},\gamma} ({\bf q})}. \nonumber \\
\label{symdrhop}
\end{eqnarray}
Defining the charge induced by the irreducible mode $\lambda^p$ as:
\begin{equation}
{d \tilde n\left({\bf r}, \{{\bf R}_I + {\bf u}_I\}\right)
\over d \lambda^p}=
\sum_{s,\alpha} A^p_{s,\alpha} {d \tilde n\left({\bf r}, \{{\bf R}_I + {\bf u}_I\}\right)
\over d {\bf u}_{s,\alpha}({\bf q})},
\end{equation}
the rotation matrix
\begin{equation}
t_{q,p}^S=
\sum_{s,\alpha}
\sum_{\beta=1}^3 A^p_{s,\alpha} S^{-1}_{\alpha,\beta} e^{-i{\bf q}\cdot {\bf R}^S_{{\boldsymbol{\tau}_s}}} A^{*,q}_{\bar s,\beta} = \sum_{s,\alpha}
\sum_{\beta=1}^3 A^{*,q}_{\bar s,\beta} S_{\beta,\alpha} e^{-i{\bf q}\cdot {\bf R}^S_{{\boldsymbol{\tau}_s}}} A^p_{s,\alpha},
\label{tnotr}
\end{equation}
we have the relationship:
\begin{equation}
{d \tilde n\left({\bf r}, \{{\bf R}_I + {\bf u}_I\}\right) \over d \lambda^p} = e^{i{\bf G}_{S^{-1}} \cdot {\bf r}}
\sum_q t_{q,p}^S {d \tilde n\left(\{S,{\bf a}\} {\bf r}, \{{\bf R}_{\bar I} + {\bf u}_{\bar I}\}\right) \over d \lambda^q}. 
\end{equation}
This equation is implemented by summing on all symmetry operations of the small cogroup of
{\bf q} and dividing by their number:
\begin{equation}
{d \tilde n\left({\bf r}, \{{\bf R}_I + {\bf u}_I\}\right) \over d \lambda^p} = {1\over N_{S_{\bf q}}} \sum_{i=1}^{N_{S_{\bf q}}} e^{i{\bf G}_{S_i^{-1}} \cdot {\bf r}}
\sum_q t_{q,p}^{S_i} {d \tilde n\left(\{S_i,{\bf a}_i\} {\bf r}, \{{\bf R}_{\bar I} + {\bf u}_{\bar I}\}\right) \over d \lambda^q}.
\label{drho_irrep}
\end{equation}

For operations that require the time reversal operators we must start by doing linear combinations of 
Eq.~\ref{symdrho_final1}:
\begin{eqnarray}
{d \tilde n\left({\bf r}, \{{\bf R}_I + {\bf u}_I\}\right)
\over d \lambda^p}&=&
\sum_{s,\alpha} A^p_{s,\alpha}
{d \tilde n\left({\bf r}, \{{\bf R}_I + {\bf u}_I\}\right)
\over d {\bf u}_{s,\alpha}({\bf q})} 
\nonumber \\ 
&=& O(S)\left[ e^{i{\bf G}_{S^{-1}} \cdot {\bf r}} \sum_{q,s,\alpha,\bar{\bar s}, \gamma, \beta} 
A^{*p}_{s,\alpha} S^{-1}_{\alpha,\beta} 
A^{*q}_{\bar s,\beta} e^{-i{\bf q}\cdot {\bf R}^S_{{\boldsymbol{\tau}_s}}}
A^{q}_{\bar {\bar s},\gamma}
{d \tilde n\left(\{S,{\bf a}\} {\bf r}, \{{\bf R}_{\bar I} + {\bf u}_{\bar I}\}\right)
\over d {\bf u}_{\bar{\bar s},\gamma} ({\bf q})} \right].
\label{symdrho_final_irrep1} \nonumber \\
\end{eqnarray}
In this case we define
\begin{equation}
t_{q,p}^S= O(S)\left[
\sum_{s,\alpha}
\sum_{\beta=1}^3 A^p_{s,\alpha} S^{-1}_{\alpha,\beta} e^{i{\bf q}\cdot {\bf R}^S_{{\boldsymbol{\tau}_s}}} A^{q}_{\bar s,\beta}\right] = O(S) \left[\sum_{s,\alpha}
\sum_{\beta=1}^3 A^{q}_{\bar s,\beta} S_{\beta,\alpha} e^{i{\bf q}\cdot {\bf R}^S_{{\boldsymbol{\tau}_s}}} A^p_{s,\alpha}\right],
\label{ttr}
\end{equation}
and we have the relationship:
\begin{equation}
{d \tilde n\left({\bf r}, \{{\bf R}_I + {\bf u}_I\}\right) \over d \lambda^p} = O(S) \left[ e^{i{\bf G}_{S^{-1}} \cdot {\bf r}}
\sum_q t_{q,p}^S {d \tilde n\left(\{S,{\bf a}\} {\bf r}, \{{\bf R}_{\bar I} + {\bf u}_{\bar I}\}\right) \over d \lambda^q}\right]. 
\end{equation}
Eq.~\ref{drho_irrep} can be generalized
in the noncolinear magnetic case as:
\begin{equation}
{d \tilde n\left({\bf r}, \{{\bf R}_I + {\bf u}_I\}\right) \over d \lambda^p} = {1\over N_{S_{\bf q}}} \sum_{i=1}^{N_{S_{\bf q}}} O(S_i)\left[ e^{i{\bf G}_{S_i^{-1}} \cdot {\bf r}}
\sum_q t_{q,p}^{S_i} {d \tilde n\left(\{S_i,{\bf a}_i\} {\bf r}, \{{\bf R}_{\bar I} + {\bf u}_{\bar I}\}\right) \over d \lambda^q} \right].
\label{drho_irrep_1}
\end{equation}
where however we have two different definitions of $t_{q,p}^{S_i}$ Eq.~\ref{tnotr} when time reversal is not needed and Eq.~\ref{ttr} when the symmetry operation needs time reversal.

Similar expressions can be be found for the symmetrization of the magnetization density. We need now to start from Eq.~\ref{symdmag_final1} and calculate the derivative of the magnetization with respect to the $p$ irreducible mode. For operations that do not require time reversal we have:

\begin{eqnarray}
\sum_{s,\alpha} A^p_{s,\alpha} {d \tilde {\bf m}_\gamma \left({\bf r}, \{{\bf R}_I + {\bf u}_I\}\right)
\over d {\bf u}_{s,\alpha}({\bf q})} 
&=& \Bigg[ e^{i{\bf G}_{S^{-1}} \cdot {\bf r}}
\sum_{q,s,\alpha,\bar{\bar s},\beta,\delta,\eta} A^{p}_{s,\alpha} S^{-1}_{\alpha,\beta} A^{*q}_{\bar s,\beta} e^{-i{\bf q}\cdot {\bf R}^S_{{\boldsymbol{\tau}_s}}}
 \nonumber \\
&\times& A^{q}_{\bar {\bar s},\eta}
\tilde S^{-1}_{\gamma,\delta} {d \tilde {\bf m}_\delta\left(\{S,{\bf a}\} {\bf r}, \{{\bf R}_{\bar I} + {\bf u}_{\bar I}\}\right)
\over d {\bf u}_{\bar {\bar s},\eta} ({\bf q})} \Bigg].
\label{symdmag_final1_irrep}
\end{eqnarray}
while for operations that require time reversal we have:
\begin{eqnarray}
\sum_{s,\alpha} A^p_{s,\alpha} {d \tilde {\bf m}_\gamma \left({\bf r}, \{{\bf R}_I + {\bf u}_I\}\right)
\over d {\bf u}_{s,\alpha}({\bf q})} 
&=& (-1)^{T_{S}} O(S) \Bigg[ e^{i{\bf G}_{S^{-1}} \cdot {\bf r}}
\sum_{q,s,\alpha,\bar{\bar s},\beta,\delta,\eta} A^{*p}_{s,\alpha} S^{-1}_{\alpha,\beta} A^{*q}_{\bar s,\beta} e^{-i{\bf q}\cdot {\bf R}^S_{{\boldsymbol{\tau}_s}}}
 \nonumber \\
&\times& A^{q}_{\bar {\bar s},\eta}
\tilde S^{-1}_{\gamma,\delta} {d \tilde {\bf m}_\delta\left(\{S,{\bf a}\} {\bf r}, \{{\bf R}_{\bar I} + {\bf u}_{\bar I}\}\right)
\over d {\bf u}_{\bar {\bar s},\eta} ({\bf q})} \Bigg].
\label{symdmag_final1_irrep_tr}
\end{eqnarray}
With the same definitions of $t^S_{q,p}$ used for the induced density we can summarize these two expressions with 
\begin{equation}
{d \tilde {\bf m}_\gamma \left({\bf r}, \{{\bf R}_I + {\bf u}_I\}\right)
\over d \lambda^p} 
= (-1)^{T_{S}} O(S) \Bigg[ e^{i{\bf G}_{S^{-1}} \cdot {\bf r}}
\sum_{q,\delta} t_{q,p}^S
\tilde S^{-1}_{\gamma,\delta} {d \tilde {\bf m}_\delta\left(\{S,{\bf a}\} {\bf r}, \{{\bf R}_{\bar I} + {\bf u}_{\bar I}\}\right)
\over d \lambda^q} \Bigg].
\label{symdmag_final2_irrep_tr}
\end{equation}

\section{Appendix: Dynamical matrix in the basis of irrep}
Given the eigenvalue equation that gives the phonon frequencies and the vibrational modes:
\begin{equation}
\sum_{s',\beta} D_{s,\alpha,s',\beta}({\bf q})
{\bf u}_{s',\beta}({\bf q})
= \omega^2({\bf q}) {\bf u}_{s,\alpha}({\bf q}),
\end{equation}
we can change the basis using the irrep. Multiplying the equation by $A^{*,p}_{s,\alpha}$ we get
\begin{equation}
\sum_q \sum_{s,\alpha} \sum_{s',\beta} \sum_{s'',\gamma} A^{*,p}_{s,\alpha} D_{s,\alpha,s',\beta}({\bf q})
A^{q}_{s',\beta} A^{*,q}_{s'',\gamma}
{\bf u}_{s'',\gamma}({\bf q})
= \omega^2({\bf q}) \sum_{s,\alpha} A^{*,p}_{s,\alpha}{\bf u}_{s,\alpha}({\bf q}),
\end{equation}
and defining the dynamical matrix in the basis of the irrep:
\begin{equation}
D_{p,q}({\bf q}) =\sum_{s,\alpha} \sum_{s',\beta} A^{*,p}_{s,\alpha} D_{s,\alpha,s',\beta}({\bf q}) A^{q}_{s',\beta},
\label{rotate_pattern_add}
\end{equation}
and the new modes:
\begin{equation}
{\bf u}_{q}({\bf q}) = \sum_{s'',\gamma} A^{*,q}_{s'',\gamma}
{\bf u}_{s'',\gamma}({\bf q}),
\end{equation}
the equation for phonon frequencies can be rewritten as:
\begin{equation}
\sum_q  D_{p,q}({\bf q})
{\bf u}_{q}({\bf q})
= \omega^2({\bf q}){\bf u}_{p}({\bf q}).
\end{equation}

We can also write the equation that transforms the dynamical matrix from the basis of the irrep to the Cartesian basis inverting Eq.~\ref{rotate_pattern_add}:
\begin{equation}
D_{s,\alpha,s',\beta}({\bf q}) =\sum_{p,q}  A^{p}_{s,\alpha} D_{p,q}({\bf q}) A^{*,q}_{s',\beta}.
\label{dyn_pattern_to_cart}
\end{equation}

\section{Appendix: symdynph\_gq}
This routine uses the Eqs.~\ref{symdyn2} and
\ref{dynmat_inv_smallq} to symmetryze the dynamical matrix. We can rewrite Eq.~\ref{symdyn2} in the basis of the primitive reciprocal lattice vectors inserting four
$\delta$ in it:
\begin{eqnarray}
D_{s,\alpha,s',\beta}({\bf q}) &=& {1\over N_{S_{\bf q}}}
\sum_{i=1}^{N_{S_{\bf q}}}
\sum_{m,n,o,p}
\sum_{\gamma,\delta,\eta,\lambda,\nu,\rho}
{\bf b}_{\alpha,m} {\bf a}_{\eta,m}
S_{i,\lambda,\eta} {\bf b}_{\lambda,n} {\bf a}_{\gamma,n}  {\bf b}_{\beta,o} {\bf a}_{\nu,o} S_{i,\rho,\nu} {\bf b}_{\rho,p} {\bf a}_{\delta,p}
\tilde D_{\bar s,\gamma,\bar s',\delta}({\bf q})
e^{i {\bf q} \cdot \left( {\bf R}^{S}_{\boldsymbol{\tau}_s} - {\bf R}^{S}_{\boldsymbol{\tau}_{s'}} \right) }
\nonumber\\
&=& {1\over N_{S_{\bf q}}}
\sum_{i=1}^{N_{S_{\bf q}}}
\sum_{m,n,o,p}
\sum_{\gamma, \delta}
{\bf b}_{\alpha,m} {\bf b}_{\beta,o}
S_{i,m,n} S_{i,o,p} {\bf a}_{\gamma,n} {\bf a}_{\delta,p}
\tilde D_{\bar s,\gamma,\bar s',\delta}({\bf q})
e^{i {\bf q} \cdot \left( {\bf R}^{S}_{\boldsymbol{\tau}_s} - {\bf R}^{S}_{\boldsymbol{\tau}_{s'}} \right)}.
\label{symdyn2_crys}
\end{eqnarray}
The routine receives as input the dynamical matrix in the basis of the primitive reciprocal lattice vectors: 
\begin{equation}
\tilde D_{s,n,s',p}({\bf q})
=\sum_{\gamma, \delta} {\bf a}_{\gamma,n} {\bf a}_{\delta,p}
\tilde D_{s,\gamma,s',\delta}({\bf q}),
\end{equation}
computes 
\begin{equation}
D_{s,m,s',o}({\bf q})
={1\over N_{S_{\bf q}}}
\sum_{i=1}^{N_{S_{\bf q}}}
\sum_{n,p}
S_{i,m,n} S_{i,o,p}
\tilde D_{\bar s,n,\bar s',p}({\bf q})
e^{i {\bf q} \cdot \left( {\bf R}^{S}_{\boldsymbol{\tau}_s} - {\bf R}^{S}_{\boldsymbol{\tau}_{s'}} \right)}. 
\end{equation}
This symmetrized dynamical matrix in the basis
of the primitive reciprocal lattice vectors 
can be assigned to all elements $\bar s, \bar s'$.
To do this we can use Eq.~\ref{dynmat_inv_smallq}
and write it in the basis of the primitive reciprocal lattice vectors:
\begin{eqnarray}
D_{\bar s,\alpha,\bar s',\beta}({\bf q}) &=&
\sum_{m,n,o,p} \sum_{\gamma,\delta,\eta,\lambda,\nu,\rho} {\bf b}_{\alpha,m} {\bf a}_{\eta,m} S^{-1}_{i,\lambda,\eta} {\bf b}_{\lambda,n} {\bf a}_{\gamma,n} {\bf b}_{\beta,o} {\bf a}_{\nu,o}  S^{-1}_{i,\rho,\nu} {\bf b}_{\rho,p} {\bf a}_{\delta,p}
D_{s,\gamma,s',\delta}({\bf q})
e^{-i {\bf q} \cdot ({\bf R}^{S}_{\boldsymbol{\tau}_s} - {\bf R}^{S}_{\boldsymbol{\tau}_{s'}})} \nonumber \\
&=& \sum_{m,n,o,p} \sum_{\gamma,\delta} {\bf b}_{\alpha,m} {\bf b}_{\beta,o}  S^{-1}_{i,m,n}    S^{-1}_{i,o,p} {\bf a}_{\gamma,n} {\bf a}_{\delta,p}
D_{s,\gamma,s',\delta}({\bf q})
e^{-i {\bf q} \cdot ({\bf R}^{S}_{\boldsymbol{\tau}_s} - {\bf R}^{S}_{\boldsymbol{\tau}_{s'}})}, 
\label{dynmat_inv_smallq_rec}
\end{eqnarray}
or
\begin{equation}
D_{\bar s,m,\bar s',o}({\bf q}) =
\sum_{n,p}  S^{-1}_{i,m,n}    S^{-1}_{i,o,p} 
D_{s,n,s',p}({\bf q})
e^{-i {\bf q} \cdot ({\bf R}^{S}_{\boldsymbol{\tau}_s} - {\bf R}^{S}_{\boldsymbol{\tau}_{s'}})}.
\end{equation}
On output the dynamical matrix is still in the basis of the primitive reciprocal lattice vectors and the sum over $m,o$ in Eq.~\ref{dynmat_inv_smallq_rec} that is needed to bring it in Cartesian coordinates is made outside this routine.

\section{Appendix: $S_{m} {\bf q} = -{\bf q} + {\bf G}_m$ }

This option is used only with the collinear version of the code. It is disabled for spinors.

From the definition of the dynamical matrix
(Eq.~\ref{dyn_mat}) we see that
\begin{equation}
D_{s,\alpha,s',\beta}({\bf q}) =
D^*_{s,\alpha,s',\beta}(-{\bf q}).
\end{equation}
If among the symmetry operations of the point group there is $S_m$ such that
\begin{equation}
S_{m} {\bf q} = -{\bf q} + {\bf G}_{S_m}
\label{sminusq}
\end{equation}
we can use Eq.~\ref{symdyn0} to write
\begin{eqnarray}
D_{s,\alpha,s',\beta}({\bf q}) &=&
\sum_{\gamma,\delta=1}^3  S^{-1}_{m,\alpha,\gamma} S^{-1}_{m,\beta,\delta}
e^{-i {\bf q} \cdot {\bf R}^{S}_{\boldsymbol{\tau}_s} }
D_{\bar s,\gamma,\bar s',\delta}(-{\bf q})
e^{i {\bf q} \cdot {\bf R}^{S}_{\boldsymbol{\tau}_{s'}}} \nonumber \\
&=&
\left[ \sum_{\gamma,\delta=1}^3  S^{-1}_{m,\alpha,\gamma} S^{-1}_{m,\beta,\delta}
e^{i {\bf q} \cdot {\bf R}^{S}_{\boldsymbol{\tau}_s} }
D_{\bar s,\gamma,\bar s',\delta}({\bf q})
e^{-i {\bf q} \cdot {\bf R}^{S}_{\boldsymbol{\tau}_{s'}}} 
\right]^*.
\label{symdyn0_mq}
\end{eqnarray}
So if we have a nonsymmetrized dynamical matrix we can use it with the equation
\begin{equation}
D_{s,\alpha,s',\beta}({\bf q})=
{1\over 2}
\left\{D_{s,\alpha,s',\beta}({\bf q}) +
\left[\sum_{\gamma,\delta=1}^3  S^{-1}_{m,\alpha,\gamma} S^{-1}_{m,\beta,\delta}
e^{i {\bf q} \cdot {\bf R}^{S}_{\boldsymbol{\tau}_s} }
D_{\bar s,\gamma,\bar s',\delta}({\bf q})
e^{-i {\bf q} \cdot {\bf R}^{S}_{\boldsymbol{\tau}_{s'}}} 
\right]^* \right\}
\end{equation}
to symmetrize a dynamical matrix in which the
matrix $S_m$ has been used to reduce the {\bf k} points.

A similar expression can be found also for charge density induced by a phonon. From the definition in Eq.~\ref{drhoph} we have that:

\begin{equation}
{d n\left({\bf r}, \{{\bf R}_I + {\bf u}_I\}\right)
\over d {\bf u}_{s,\alpha}(-{\bf q})} =
{d n\left({\bf r}, \{{\bf R}_I + {\bf u}_I\}\right)^*
\over d {\bf u}_{s,\alpha}({\bf q})}
\label{drhoph_def}
\end{equation}
Now for the operation $S_m$ in Eq.~\ref{sminusq}, Eq.~\ref{induced_charge} gives:
\begin{eqnarray}
{d \tilde n\left({\bf r}, \{{\bf R}_I + {\bf u}_I\}\right)
\over d {\bf u}_{s,\alpha}({\bf q})} &=& e^{-i{\bf G}_{S^{-1}_m} \cdot {\bf r }} \sum_{\beta=1}^3 S^{-1}_{m,\alpha,\beta} 
{d \tilde n\left(\{S_m,{\bf a}_m\} {\bf r}, \{{\bf R}_{\bar I} + {\bf u}_{\bar I}\}\right)
\over d {\bf u}_{\bar s,\beta} (-{\bf q})} e^{i{\bf q}\cdot {\bf R}^S_{{\boldsymbol{\tau}_s}}} \nonumber \\
&=& \left[
e^{i{\bf G}_{S^{-1}_m} \cdot {\bf r }} \sum_{\beta=1}^3 S^{-1}_{m,\alpha,\beta} 
{d \tilde n\left(\{S_m,{\bf a}_m\} {\bf r}, \{{\bf R}_{\bar I} + {\bf u}_{\bar I}\}\right)
\over d {\bf u}_{\bar s,\beta} ({\bf q})} e^{-i{\bf q}\cdot {\bf R}^S_{{\boldsymbol{\tau}_s}}}\right]^*
\nonumber \\
\label{induced_charge_minusq}
\end{eqnarray}
Passing now to the basis of \texttt{irrep} we have that
\begin{equation}
{d \tilde n\left({\bf r}, \{{\bf R}_I + {\bf u}_I\}\right)
\over d \lambda_p} =
\left[
e^{i{\bf G}_{S^{-1}_m} \cdot {\bf r }} \sum_{q,s,\alpha,\bar{\bar s},\beta,\gamma} A^{*p}_{s,\alpha}S^{-1}_{m,\alpha,\beta} e^{-i{\bf q}\cdot {\bf R}^S_{{\boldsymbol{\tau}_s}}} A^{*q}_{\bar s,\beta} A^{q}_{\bar {\bar s},\gamma}
{d \tilde n\left(\{S_m,{\bf a}_m\} {\bf r}, \{{\bf R}_{\bar I} + {\bf u}_{\bar I}\}\right)
\over d {\bf u}_{\bar {\bar s},\gamma} ({\bf q})} \right]^*
\end{equation}
Defining:
\begin{equation}
t_{q,p}^{S_m}= \left[
\sum_{s,\alpha}
\sum_{\beta=1}^3 A^p_{s,\alpha} S^{-1}_{m,\alpha,\beta} e^{i{\bf q}\cdot {\bf R}^{S_m}_{{\boldsymbol{\tau}_s}}} A^{q}_{\bar s,\beta}\right]^* = \left[\sum_{s,\alpha}
\sum_{\beta=1}^3 A^{q}_{\bar s,\beta} S_{m,\beta,\alpha} e^{i{\bf q}\cdot {\bf R}^{S_m}_{{\boldsymbol{\tau}_s}}} A^p_{s,\alpha}\right]^*,
\label{tmq}
\end{equation}
we obtain the equation:
\begin{equation}
{d \tilde n\left({\bf r}, \{{\bf R}_I + {\bf u}_I\}\right)
\over d \lambda_p} =
\left[
e^{i{\bf G}_{S^{-1}_m} \cdot {\bf r }} \sum_{q} 
t^{S_m}_{q,p}
{d \tilde n\left(\{S_m,{\bf a}_m\} {\bf r}, \{{\bf R}_{\bar I} + {\bf u}_{\bar I}\}\right)
\over d \lambda_q} \right]^*,
\end{equation}
that can be used in the symmetrization expression:
\begin{equation}
{d \tilde n\left({\bf r}, \{{\bf R}_I + {\bf u}_I\}\right)
\over d \lambda_p} ={1\over 2}\left\{
{d \tilde n\left({\bf r}, \{{\bf R}_I + {\bf u}_I\}\right)
\over d \lambda_p}+
\left[
e^{i{\bf G}_{S^{-1}_m} \cdot {\bf r }} \sum_{q} 
t^{S_m}_{q,p}
{d \tilde n\left(\{S_m,{\bf a}_m\} {\bf r}, \{{\bf R}_{\bar I} + {\bf u}_{\bar I}\}\right)
\over d \lambda_q} \right]^* \right\},
\end{equation}

\section{Appendix: set\_irr\_sym}
This routine implements the three Eqs.~\ref{ttr},~\ref{tnotr}, and \ref{tmq}. In all cases the matrix $S_{\beta,\alpha}$ is applied in 
the crystal basis. We show explicitly only the
case of $t^S_{q,p}$. We have that using twice Eq.~\ref{complete} we obtain:
\begin{eqnarray}
t_{q,p}^S&=&
\sum_{s,\alpha,\beta, \delta, \eta, j, l}
 A^{*,q}_{\bar s,\beta} {\bf b}_{\delta, j} {\bf a}_{\beta, j} S_{\delta,\alpha} e^{-i{\bf q}\cdot {\bf R}^S_{{\boldsymbol{\tau}_s}}} {\bf b}_{\eta, l} {\bf a}_{\alpha, l}  A^p_{s,\eta}\nonumber \\ &=&
 \sum_{s,\alpha,\beta, \delta, \eta, j, l}
 A^{*,q}_{\bar s,\beta}  {\bf a}_{\beta, j} {\bf b}_{\delta, j} S_{\delta,\alpha} {\bf a}_{\alpha, l} e^{-i{\bf q}\cdot {\bf R}^S_{{\boldsymbol{\tau}_s}}} {\bf b}_{\eta, l} A^p_{s,\eta} \nonumber \\ &=&
 \sum_{s,\beta,j,l}
 A^{*,q}_{\bar s,\beta}  {\bf a}_{\beta, j}    e^{-i{\bf q}\cdot {\bf R}^S_{{\boldsymbol{\tau}_s}}}  S_{l,j} A^p_{s,l},
\label{tnotr_crys}
\end{eqnarray}
where
\begin{equation}
A^p_{s,l} = \sum_\eta {\bf b}_{\eta, l} A^p_{s,\eta}
\end{equation}
\newpage
is the mode in the basis of the primitive lattice vectors.

\section{Appendix: rotate\_and\_add\_dyn}

This routine implements Eq.~\ref{dynmat_inv} adding four
$\delta$ functions:
\begin{eqnarray}
D_{\bar s,\alpha,\bar s',\beta}(S{\bf q}) &=&
\sum_{p,q,r,t,\gamma,\delta,\epsilon,\eta,\nu,\rho}  {\bf b}_{\alpha,p} {\bf a}_{\eta,p}S_{\eta,\nu}{\bf b}_{\nu,q} {\bf a}_{\gamma,q} {\bf b}_{\beta,r} {\bf a}_{\rho,r} S_{ \rho,\epsilon} {\bf b}_{\epsilon,t} {\bf a}_{\delta,t}
D_{s,\gamma,s',\delta}({\bf q})
e^{-i {\bf q} \cdot \left({\bf R}^{S}_{\boldsymbol{\tau}_s} - {\bf R}^{S}_{\boldsymbol{\tau}_{s'}}\right)}
\nonumber \\ &=&
\sum_{p,q,r,t,\gamma,\delta,\epsilon,\eta,\nu,\rho}  {\bf b}_{\alpha,p}{\bf b}_{\beta,r}  {\bf a}_{\eta,p}S^{-1}_{\nu,\eta}{\bf b}_{\nu,q}  {\bf a}_{\rho,r} S^{-1}_{ \epsilon,\rho} {\bf b}_{\epsilon,t} {\bf a}_{\gamma,q} {\bf a}_{\delta,t}
D_{s,\gamma,s',\delta}({\bf q})
e^{-i {\bf q} \cdot \left({\bf R}^{S}_{\boldsymbol{\tau}_s} - {\bf R}^{S}_{\boldsymbol{\tau}_{s'}}\right)}
\nonumber \\
&=&
\sum_{p,q,r,t}  {\bf b}_{\alpha,p}{\bf b}_{\beta,r}  S^{-1}_{p,q} S^{-1}_{r,t} 
D_{s,q,s',t}({\bf q})
e^{-i {\bf q} \cdot \left({\bf R}^{S}_{\boldsymbol{\tau}_s} - {\bf R}^{S}_{\boldsymbol{\tau}_{s'}}\right)}
,
\label{dynmat_inv_cryst}
\end{eqnarray}
Where the dynamical matrix in the basis of the primitive lattice vectors is given as input to the routine:
\begin{equation}
D_{s,q,s',t}({\bf q})= \sum_{\gamma,\delta}
 {\bf a}_{\gamma,q} {\bf a}_{\delta,t}
D_{s,\gamma,s',\delta}({\bf q})
\end{equation}
and the output of the routine is the dynamical matrix
at the rotated ${\bf q}$
in the same basis
\begin{equation}
D_{\bar s,p,\bar s',r}(S{\bf q})=
\sum_{q,t}  S^{-1}_{p,q} S^{-1}_{r,t} 
D_{s,q,s',t}({\bf q})
e^{-i {\bf q} \cdot \left({\bf R}^{S}_{\boldsymbol{\tau}_s} - {\bf R}^{S}_{\boldsymbol{\tau}_{s'}}\right)}
\end{equation}
The operations needed to pass to the Cartesian basis 
are made outside the routine if needed:
\begin{equation}
D_{s,\alpha,s',\beta}(S{\bf q})
=\sum_{p,r}  {\bf b}_{\alpha,p}{\bf b}_{\beta,r}
D_{s,p,s',r}(S{\bf q})
\end{equation}

{\color{dark-blue}\chapter{Bibliography}}
\color{black}

\begin{enumerate}

\item
M. Tinkham, `Group theory and quantum mechanics', Dover Publications, New York,
(1992). 

\item
A. Urru, PhD Thesis (SISSA 2020), Appendix E.

\item 
A. A. Maradunin, S. H. Vosko, `Symmetry properties of the normal vibrations of a crystal', Rev. Mod. Phys.
{\bf 40}, 1 (1968).
\end{enumerate}

\end{document}
