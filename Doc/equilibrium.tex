%!
%! Copyright (C) 2014-2025 Andrea Dal Corso 
%! This file is distributed under the terms of the
%! GNU General Public License. See the file `License'
%! in the root directory of the present distribution,
%! or http://www.gnu.org/copyleft/gpl.txt .
%!
\documentclass[12pt,a4paper,twoside]{report}
\def\version{2.2.0}

\usepackage[T1]{fontenc}
\usepackage{tcolorbox}
\usepackage{bookman}
\usepackage{html}
\usepackage{graphicx}
\usepackage{fancyhdr}
\usepackage[Lenny]{fncychap}
\usepackage{color}
\usepackage{geometry}
\usepackage{amsmath}
\usepackage{mathtools}

\tcbuselibrary{breakable}
\pagestyle{fancy}
\lhead{Equilibrium}
\rhead{}
\cfoot{\thepage}

\newgeometry{
      top=3cm,
      bottom=3cm,
      outer=2.25cm,
      inner=2.75cm,
}

\definecolor{web-blue}{rgb}{0,0.5,1.0}
\definecolor{steelblue}{rgb}{0.27,0.5,0.7}
\definecolor{coral}{rgb}{1.0,0.5,0.3}
\definecolor{red}{rgb}{1.0,0,0.0}
\definecolor{green}{rgb}{0.,0.5,0.0}
\definecolor{dark-blue}{rgb}{0.,0.0,0.6}
\definecolor{limegreen}{rgb}{0.19,0.8,0.19}
\definecolor{orange}{rgb}{1.0,0.44,0.0}
\definecolor{violet}{rgb}{0.50,0.33,0.9}
\definecolor{light-yellow}{rgb}{0.94,0.85,0.62}

\tcbset{colback=light-yellow,colframe=dark-blue,breakable}

\def\qe{{\sc Quantum ESPRESSO}}
\def\pwx{\texttt{pw.x}}
\def\phx{\texttt{ph.x}}
\def\configure{\texttt{configure}}
\def\PWscf{\texttt{PWscf}}
\def\PHonon{\texttt{PHonon}}
\def\tpw{{\sc Thermo\_pw}}
\def\make{\texttt{make}}




\def\hplanck{6.62607015\times 10^{-34}}
\def\hbarf{1.0545718176462\times 10^{-34}}
\def\cspeed{2.99792458\times 10^{8}}
\def\e{1.602176634\times 10^{-19}}
\def\rydberg{1.0973731568160\times 10^{7}}
\def\alphaf{7.2973525693\times 10^{-3}}
\def\amu{1.66053906660\times 10^{-27}}

\def\me{9.1093837015\times 10^{-31}}
\def\abohr{5.29177210903\times 10^{-11}}
\def\muzero{1.25663706212\times 10^{-6}}
\def\epsilonzero{8.8541878128\times 10^{-12}}
\def\ehartree{4.3597447222072\times 10^{-18}}
\def\bohrmag{9.2740100783\times 10^{-24}}

\def\barl{5.29177210903\times 10^{-11}}
\def\barm{9.1093837015\times 10^{-31}}
\def\bart{2.4188843265857\times 10^{-17}}
\def\barnu{4.1341373335182\times 10^{16}}
\def\barv{2.18769126364\times 10^{6}}
\def\bara{9.0442161272\times 10^{22}}
\def\barp{1.99285191410\times 10^{-24}}
\def\baram{1.0545718176462\times 10^{-34}}
\def\barf{8.2387234982\times 10^{-8}}
\def\baru{4.3597447222072\times 10^{-18}}
\def\barw{1.8023783420686\times 10^{-1}}
\def\barpr{2.9421015696\times 10^{13}}
\def\bari{6.623618237510\times 10^{-3}}
\def\barc{1.602176634\times 10^{-19}}
\def\barrho{1.08120238456\times 10^{12}}
\def\barcur{2.36533701094\times 10^{18}}
\def\bare{5.14220674763\times 10^{11}}
\def\barphi{2.7211386245988\times 10^{1}}
\def\barcap{5.887890530517\times 10^{-21}}
\def\bardip{8.4783536255\times 10^{-30}}
\def\barpolar{5.7214766229\times 10^{1}}
\def\bard{4.5530064316}
\def\barohm{4.1082359022277\times 10^{3}}
\def\barb{2.35051756758\times 10^{5}}
\def\barav{1.24384033059\times 10^{-5}}
\def\barwb{6.5821195695091\times 10^{-16}}
\def\bary{9.937347433815\times 10^{-14}}
\def\barmu{1.85480201567\times 10^{-23}}
\def\barmag{1.25168244230\times 10^{8}}
\def\barh{9.9605723936\times 10^{6}}

\def\cmtom{1.0\times 10^{-2}}
\def\gtokg{1.0\times 10^{-3}}
\def\ptop{1.0\times 10^{-5}}
\def\ltol{1.0\times 10^{-7}}
\def\ftof{1.0\times 10^{-5}}
\def\utou{1.0\times 10^{-7}}
\def\prtopr{1.0\times 10^{-1}}
\def\chtoch{3.33564095107\times 10^{-10}}
\def\kappaa{1.000000000274\times 10^{-1}}
\def\kappa{2.99792458082\times 10^{9}}
\def\kappadiecic{1.00000000027}
\def\itoi{3.33564095107\times 10^{-10}}
\def\rhotorho{3.33564095107\times 10^{-4}}
\def\curtocur{3.33564095107\times 10^{-6}}
\def\etoe{2.99792458082\times 10^{4}}
\def\phitophi{2.99792458082\times 10^{2}}
\def\captocap{1.11265005544\times 10^{-12}}
\def\diptodip{3.33564095107\times 10^{-12}}
\def\polartopolar{3.33564095107\times 10^{-6}}
\def\dtod{2.65441872871\times 10^{-7}}
\def\ohmtoohm{8.9875517923\times 10^{11}}
\def\btob{1.000000000274\times 10^{-4}}
\def\avtoav{1.000000000274\times 10^{-6}}
\def\wbtowb{1.000000000274\times 10^{-8}}
\def\ytoy{8.9875517923\times 10^{11}}
\def\mutomu{9.99999999726\times 10^{-4}}
\def\magtomag{9.99999999726\times 10^{2}}
\def\htoh{7.95774715242\times 10^{1}}
\def\toverg{9.99999999726\times 10^{3}}

\def\barlcgs{5.29177210903\times 10^{-9}}
\def\barmcgs{9.1093837015\times 10^{-28}}
\def\barvcgs{2.18769126364\times 10^{8}}
\def\baracgs{9.0442161272\times 10^{24}}
\def\barpcgs{1.99285191410\times 10^{-19}}
\def\baramcgs{1.0545718176462\times 10^{-27}}
\def\barfcgs{8.2387234982\times 10^{-3}}
\def\barucgs{4.3597447222072\times 10^{-11}}
\def\barwcgs{1.8023783420686\times 10^{6}}
\def\barprcgs{2.9421015696\times 10^{14}}
\def\baricgs{1.98571079282\times 10^{7}}
\def\barccgs{4.80320471388\times 10^{-10}}
\def\barrhocgs{3.2413632055\times 10^{15}}
\def\barcurcgs{7.0911019670\times 10^{23}}
\def\barecgs{1.71525554062\times 10^{7}}
\def\barphicgs{9.07674142975\times 10^{-2}}
\def\barcapcgs{5.29177210903\times 10^{-9}}
\def\bardipcgs{2.54174647389\times 10^{-18}}
\def\barpolarcgs{1.71525554062\times 10^{7}}
\def\bardcgs{1.71525554062\times 10^{7}}
\def\barohmcgs{4.57102890439\times 10^{-9}}
\def\barbcgs{2.35051756693\times 10^{9}}
\def\baravcgs{1.24384033025\times 10^{1}}
\def\barwbcgs{6.58211956771\times 10^{-8}}
\def\barycgs{1.10567901732\times 10^{-25}}
\def\barmucgs{1.85480201618\times 10^{-20}}
\def\barmagcgs{1.25168244264\times 10^{5}}
\def\barhcgs{1.25168244264\times 10^{5}}

\def\barmry{1.82187674030\times 10^{-30}}
\def\bartry{4.8377686531714\times 10^{-17}}
\def\barnury{2.0670686667591\times 10^{16}}
\def\barvry{1.09384563182\times 10^{6}}
\def\barary{4.52210806362\times 10^{22}}
\def\barfry{4.11936174912\times 10^{-8}}
\def\barury{2.1798723611036\times 10^{-18}}
\def\barwry{4.505945855171\times 10^{-2}}
\def\barprry{1.47105078482\times 10^{13}}
\def\bariry{2.3418026858671\times 10^{-3}}
\def\barcry{1.1329099625600\times 10^{-19}}
\def\barrhory{7.6452553796\times 10^{11}}
\def\barcurry{8.36272920114\times 10^{17}}
\def\barery{3.63608926151\times 10^{11}}
\def\barphiry{1.9241355740025\times 10^{1}}
\def\bardipry{5.99510134192\times 10^{-30}}
\def\barpolarry{4.0456949184\times 10^{1}}
\def\bardry{3.21946172254}
\def\barohmry{8.2164718044553\times 10^{3}}
\def\barbry{3.3241338227\times 10^{5}}
\def\baravry{1.75905586495\times 10^{-5}}
\def\barwbry{9.3085227643611\times 10^{-16}}
\def\baryry{3.9749389735260\times 10^{-13}}
\def\barmury{6.5577154152\times 10^{-24}}
\def\barmagry{4.42536571420\times 10^{7}}
\def\barhry{3.52159414202\times 10^{6}}

\def\barbg{1.71525554109\times 10^{3}}
\def\baravg{9.0767414322\times 10^{-8}}
\def\barwbg{4.80320471520\times 10^{-18}}
\def\barmug{2.5417464732\times 10^{-21}}
\def\barmagg{1.71525554016\times 10^{10}}
\def\barhg{1.36495698941\times 10^{9}}

\def\cspeedau{1.37035999084\times 10^{2}}
\def\amuau{1.8228884862\times 10^{3}}
\def\ryev{1.3605693122994\times 10^{1}}

\def\barifc{1.55689310282\times 10^{3}}
\def\bardmc{3.50506782501\times 10^{-13}}
\def\baralpha{4.57102890439\times 10^{-7}}
\def\baralphap{3.05882401116\times 10^{-10}}

\def\zmtozm{9.99999999726\times 10^{-2}}
\def\alphatoalpha{3.33564095198\times 10^{-9}}
\def\alphaptoalphap{4.19169004390\times 10^{-8}}


\begin{document} 

\author{Andrea Dal Corso \\ (SISSA - Trieste)}
\date{}

\title{
  \includegraphics[width=8cm]{thermo_pw.jpg} \\
  \vspace{3truecm}
  % title
  \Huge \color{dark-blue} {\sc Thermo\_pw}: Equilibrium material properties \\ (v.\version)
}

\maketitle

\newpage

\tableofcontents

\newpage

{\color{dark-blue}\chapter{Introduction}}
\color{black}

These notes discuss the units of the equilibrium material properties
using the same color conventions of the file \texttt{units.pdf}.\\
The notes are part of the \tpw\ package. The complete package is
available at \texttt{https://github.com/dalcorso/thermo\_pw}.


\newpage
{\color{coral}\section{People}}
\color{black}

These notes have been written by Andrea Dal Corso (SISSA - Trieste). \\
Disclaimer: I am not an expert of units. 
These notes reflect what I think about units.
If you think that some formula is wrong, that I misunderstood something, or 
that something can be calculated more simply, please let me know, I would 
like to learn more. 
You can contact me by e-mail: \texttt{dalcorso@sissa.it}. 

\newpage
{\color{coral}\section{Overview}}
\color{black}
The units used in electronic structure codes have been
discussed in a separate file called \texttt{units.pdf}. In this guide
we extend the previous discussion to equilibrium material properties.

As the previous manual these notes are organized in Sections, one for 
each physical quantity and we use the same conventions for the text color.

We start from the expression of the magneto-electric enthalpy 
written as a function of the atomic displacement 
${\bf u}_{s,\alpha}$, the strain $\epsilon_{\alpha,\beta}$, the electric
field ${\bf E}_\alpha$, and the magnetic field strength ${\bf H}_\alpha$
(here $s$ indicates a sublattice and $\alpha$ and $\beta$ are cartesian
coordinates indices). We consider only displacements in which the
atoms move in the same way in each unit cell.
The Taylor expansion of this function up to second order gives
\begin{align}
\mathcal{F}(\{{\bf u}_{s,\alpha}\}, \{\epsilon_{\alpha,\beta}\},
\{ {\bf E}_\alpha\}, \{{\bf H}_\alpha\})=\mathcal{F}_0 &+
\sum_{s,\alpha} {\partial \mathcal{F} \over \partial 
{\bf u}_{s,\alpha}} {\bf u}_{s,\alpha}+ \sum_{\alpha,\beta}
{\partial \mathcal{F} \over \partial 
\epsilon_{\alpha,\beta}} \epsilon_{\alpha,\beta} \nonumber \\ 
&+\sum_{\alpha}
{\partial \mathcal{F} \over \partial 
{\bf E}_\alpha} {\bf E}_\alpha + 
\sum_{\alpha}
{\partial \mathcal{F} \over \partial 
{\bf H}_\alpha} {\bf H}_\alpha \nonumber \\
&+{1\over 2}
\sum_{s,\alpha,s',\beta} {\partial^2 \mathcal{F} \over \partial 
{\bf u}_{s,\alpha} \partial {\bf u}_{s',\beta}} {\bf u}_{s,\alpha}
{\bf u}_{s',\beta}\nonumber \\
&+ {1\over 2}\sum_{\alpha,\beta,\gamma,\delta} 
{\partial^2 \mathcal{F} \over \partial 
\epsilon_{\alpha,\beta} \partial \epsilon_{\gamma,\delta}} 
\epsilon_{\alpha,\beta}
\epsilon_{\gamma,\delta} \nonumber \\
&+ {1\over 2} \sum_{\alpha,\beta} 
{\partial^2 \mathcal{F} \over \partial 
{\bf E}_{\alpha} \partial {\bf E}_{\beta}} 
{\bf E}_{\alpha}
{\bf E}_{\beta} \nonumber \\
&+ {1\over 2} \sum_{\alpha,\beta} 
{\partial^2 \mathcal{F} \over \partial 
{\bf H}_{\alpha} \partial {\bf H}_{\beta}} 
{\bf H}_{\alpha}
{\bf H}_{\beta} \nonumber \\
&+ \sum_{s,\alpha,\beta,\gamma} {\partial^2 \mathcal{F} \over \partial 
{\bf u}_{s,\alpha} \partial \epsilon_{\beta,\gamma}} {\bf u}_{s,\alpha}
\epsilon_{\beta,\gamma}\nonumber \\
&+ \sum_{s,\alpha,\beta} {\partial^2 \mathcal{F} \over \partial 
{\bf u}_{s,\alpha} \partial {\bf E}_{\beta}} {\bf u}_{s,\alpha}
{\bf E}_{\beta} \nonumber \\
&+ \sum_{s,\alpha,\beta} {\partial^2 \mathcal{F} \over \partial 
{\bf u}_{s,\alpha} \partial {\bf H}_{\beta}} {\bf u}_{s,\alpha}
{\bf H}_{\beta} \nonumber \\
&+ \sum_{\alpha,\beta,\gamma} {\partial^2 \mathcal{F} \over \partial 
\epsilon_{\alpha,\beta} \partial {\bf E}_{\gamma}} \epsilon_{\alpha,\beta}
{\bf E}_{\gamma} \nonumber \\
&+ \sum_{\alpha,\beta,\gamma} {\partial^2 \mathcal{F} \over \partial 
\epsilon_{\alpha,\beta} \partial {\bf H}_{\gamma}} \epsilon_{\alpha,\beta}
{\bf H}_{\gamma} \nonumber \\
&+ \sum_{\alpha,\beta} {\partial^2 \mathcal{F} \over \partial 
{\bf E}_{\alpha} \partial {\bf H}_{\beta}} {\bf E}_{\alpha}
{\bf H}_{\beta},
\end{align}
where all the derivatives are computed for ${\bf u}_{s,\alpha}=0$,
$\epsilon_{\alpha,\beta}=0$, ${\bf E}_\alpha=0$, and ${\bf H}_\alpha=0$.
\newpage
This is the microscopic magneto-electric enthalpy of the solid. This 
expression is valid in all systems of units. It can be compared with 
the expression of the 
phenomenological magneto-electric enthalpy. The form of this phenomenological
expansion depends on the conventions adopted in each system. We give here the
form of the expansion and discuss in the following each term separately.
In the SI units we have:
\begin{align}
\mathcal{F}(\{{\bf u}_{s,\alpha}\}, \{\epsilon_{\alpha,\beta}\},
\{ {\bf E}_\alpha\}, \{{\bf H}_\alpha\})=\mathcal{F}_0 &-
\sum_{s,\alpha} {\bf f}^{(0)}_{s,\alpha} {\bf u}_{s,\alpha}+ 
V \sum_{\alpha,\beta}
\sigma^{(0)}_{\alpha,\beta} \epsilon_{\alpha,\beta} \nonumber \\ 
&- V \sum_{\alpha} {\bf P}^{(0)}_\alpha {\bf E}_\alpha - 
\mu_0 V \sum_{\alpha}
{\bf M}^{(0)}_\alpha {\bf H}_\alpha \nonumber \\
&+ {1\over 2}
\sum_{s,\alpha,s',\beta} C_{s,\alpha,s',\beta} {\bf u}_{s,\alpha}
{\bf u}_{s',\beta}\nonumber \\
&+ {V\over 2}\sum_{\alpha,\beta,\gamma,\delta} 
C_{\alpha,\beta,\gamma,\delta}  
\epsilon_{\alpha,\beta}
\epsilon_{\gamma,\delta} \nonumber \\
&- {1\over 2} \epsilon_0 V \sum_{\alpha,\beta} 
\chi^e_{\alpha,\beta}
{\bf E}_{\alpha}
{\bf E}_{\beta} \nonumber \\
&- {1\over 2} \mu_0 V \sum_{\alpha,\beta} 
\chi^m_{\alpha,\beta}
{\bf H}_{\alpha}
{\bf H}_{\beta} \nonumber \\
&- \sum_{s,\alpha,\beta,\gamma}
\Lambda_{s,\alpha,\beta,\gamma} {\bf u}_{s,\alpha}
\epsilon_{\beta,\gamma}\nonumber \\
&- e \sum_{s,\alpha,\beta} Z^*_{s,\alpha,\beta} 
{\bf u}_{s,\alpha} {\bf E}_{\beta} \nonumber \\
&- \mu_0 \sum_{s,\alpha,\beta} 
Z^m_{s,\alpha,\beta} {\bf u}_{s,\alpha}{\bf H}_{\beta} \nonumber \\
&- V \sum_{\alpha,\beta,\gamma} e_{\alpha,\beta,\gamma} 
\epsilon_{\alpha,\beta} {\bf E}_{\gamma} \nonumber \\
&- \mu_0 V \sum_{\alpha,\beta,\gamma}  h_{\alpha,\beta,\gamma} 
\epsilon_{\alpha,\beta} {\bf H}_{\gamma} \nonumber \\
&- V \sum_{\alpha,\beta} \alpha_{\alpha,\beta} {\bf E}_{\alpha}
{\bf H}_{\beta}.
\end{align}

\newpage
{\color{web-blue} In a.u. the microscopic magneto-electric enthalpy
is written as (see below for the derivation):
\begin{align}
\mathcal{F}(\{{\bf u}_{s,\alpha}\}, \{\epsilon_{\alpha,\beta}\},
\{ {\bf E}_\alpha\}, \{{\bf H}_\alpha\})=\mathcal{F}_0 &-
\sum_{s,\alpha} {\bf f}^{(0)}_{s,\alpha} {\bf u}_{s,\alpha}
+ V \sum_{\alpha,\beta}
\sigma_{\alpha,\beta}^{(0)} \epsilon_{\alpha,\beta} \nonumber \\ 
&- V \sum_{\alpha} {\bf P}^{(0)}_\alpha {\bf E}_\alpha - 
\alpha^2 V \sum_{\alpha}
{\bf M}^{(0)}_\alpha {\bf H}_\alpha \nonumber \\
&+ {1\over 2}
\sum_{s,\alpha,s',\beta} C_{s,\alpha,s',\beta} {\bf u}_{s,\alpha}
{\bf u}_{s',\beta}\nonumber \\
&+ {V\over 2}\sum_{\alpha,\beta,\gamma,\delta} 
C_{\alpha,\beta,\gamma,\delta}  
\epsilon_{\alpha,\beta}
\epsilon_{\gamma,\delta} \nonumber \\
&- {V\over 2} \sum_{\alpha,\beta} 
\chi^e_{\alpha,\beta}
{\bf E}_{\alpha}
{\bf E}_{\beta} \nonumber \\
&- {\alpha^2 V \over 2}  \sum_{\alpha,\beta} 
\chi^m_{\alpha,\beta}
{\bf H}_{\alpha}
{\bf H}_{\beta} \nonumber \\
&- \sum_{s,\alpha,\beta,\gamma}
\Lambda_{s,\alpha,\beta,\gamma} {\bf u}_{s,\alpha}
\epsilon_{\beta,\gamma}\nonumber \\
&- \sum_{s,\alpha,\beta} Z^*_{s,\alpha,\beta} 
{\bf u}_{s,\alpha} {\bf E}_{\beta} \nonumber \\
&- \alpha^2 \sum_{s,\alpha,\beta} 
Z^m_{s,\alpha,\beta} {\bf u}_{s,\alpha}{\bf H}_{\beta} \nonumber \\
&- V \sum_{\alpha,\beta,\gamma} e_{\alpha,\beta,\gamma} 
\epsilon_{\alpha,\beta} {\bf E}_{\gamma} \nonumber \\
&- \alpha^2 V \sum_{\alpha,\beta,\gamma}  h_{\alpha,\beta,\gamma} 
\epsilon_{\alpha,\beta} {\bf H}_{\gamma} \nonumber \\
&- {V \over 4 \pi} \sum_{\alpha,\beta} 
\alpha_{\alpha,\beta} {\bf E}_{\alpha}
{\bf H}_{\beta}.
\end{align}
}

\newpage
{\color{orange} In c.g.s.-Gaussian units the microscopic
magneto-electric enthalpy is written as (see below for the derivation):
\begin{align}
\mathcal{F}(\{{\bf u}_{s,\alpha}\}, \{\epsilon_{\alpha,\beta}\},
\{ {\bf E}_\alpha\}, \{{\bf H}_\alpha\})=\mathcal{F}_0 &-
\sum_{s,\alpha} {\bf f}^{(0)}_{s,\alpha} {\bf u}_{s,\alpha}+ 
V \sum_{\alpha,\beta}
\sigma^{(0)}_{\alpha,\beta} \epsilon_{\alpha,\beta} \nonumber \\ 
&- V \sum_{\alpha} {\bf P}^{(0)}_\alpha {\bf E}_\alpha - 
V \sum_{\alpha}
{\bf M}^{(0)}_\alpha {\bf H}_\alpha \nonumber \\
&+ {1\over 2}
\sum_{s,\alpha,s',\beta} C_{s,\alpha,s',\beta} {\bf u}_{s,\alpha}
{\bf u}_{s',\beta}\nonumber \\
&+ {V\over 2}\sum_{\alpha,\beta,\gamma,\delta} 
C_{\alpha,\beta,\gamma,\delta}  
\epsilon_{\alpha,\beta}
\epsilon_{\gamma,\delta} \nonumber \\
&- {V\over 2} \sum_{\alpha,\beta} 
\chi^e_{\alpha,\beta}
{\bf E}_{\alpha}
{\bf E}_{\beta} \nonumber \\
&- {V \over 2}  \sum_{\alpha,\beta} 
\chi^m_{\alpha,\beta}
{\bf H}_{\alpha}
{\bf H}_{\beta} \nonumber \\
&- \sum_{s,\alpha,\beta,\gamma}
\Lambda_{s,\alpha,\beta,\gamma} {\bf u}_{s,\alpha}
\epsilon_{\beta,\gamma}\nonumber \\
&- e\sum_{s,\alpha,\beta} Z^*_{s,\alpha,\beta} 
{\bf u}_{s,\alpha} {\bf E}_{\beta} \nonumber \\
&- \sum_{s,\alpha,\beta} 
Z^m_{s,\alpha,\beta} {\bf u}_{s,\alpha}{\bf H}_{\beta} \nonumber \\
&- V \sum_{\alpha,\beta,\gamma} e_{\alpha,\beta,\gamma} 
\epsilon_{\alpha,\beta} {\bf E}_{\gamma} \nonumber \\
&- V \sum_{\alpha,\beta,\gamma}  h_{\alpha,\beta,\gamma} 
\epsilon_{\alpha,\beta} {\bf H}_{\gamma} \nonumber \\
&- {V \over 4 \pi} \sum_{\alpha,\beta} 
\alpha_{\alpha,\beta} {\bf E}_{\alpha}
{\bf H}_{\beta}.
\end{align}
}

\newpage
From this microscopic magneto-electric enthalpy one can compute the forces
acting on the atoms in presence of strain, electric or magnetic fields. 
Minimizing this
enthalpy one obtains the equilibrium atomic positions that can be used
to obtain the macroscopic magneto-electric enthalpy. This procedure
links the microscopic (frozen-ions) material properties to the
macroscopic ones (relaxed-ions) as we discuss in the following. 

Temperature can be included in this scheme using the appropriate
thermodynamic potential at each temperature instead of the macroscopic
magneto-electric enthalpy calculated at $T=0$. 

The magneto-electric enthalpy that we have discussed is not unique. One could 
choose different independent 
variables and define the respective expansion coefficients. In particular 
we can use $\sigma_{\alpha,\beta}$ instead of $\epsilon_{\alpha,\beta}$, 
${\bf D}_\alpha$ instead of ${\bf E}_\alpha$, and ${\bf B}_\alpha$ instead of
${\bf H}_\alpha$. 
The material properties deriving from the magneto-electric
enthalpy discussed here are presented in the
main text while in the Appendices A and B we consider  
other choices for the independent variables.
In particular, while theoretically it is easy to study a strained solid,
experimentally it is simpler to control the applied stress. In Appendix A 
we discuss the consequences of this choice. 
While experimentally one controls the magnetic field intensity
${\bf H}$, the magnetic flux density ${\bf B}$ enters in the quantum
mechanical equations used in electronic structure calculations. 
We discuss in Appendix B the relationship between quantities 
calculated at fixed ${\bf H}$ or at fixed ${\bf B}$. 

Finally we note that the enthalpy used here does not include the contribution 
of the electric and magnetic fields that would exist in absence of the 
material and a different enthalpy is obtained by including them. 
We discuss this point in Appendix C and use an enthalpy that includes
the contributions of these fields in the thermodynamic equations.

\newpage

{\color{dark-blue}\chapter{Physical quantities}}
\color{black}

{\color{coral}\section{Forces}}
\color{black}

The forces acting on the atoms are given by:
\begin{equation}
{\bf f}^{(0)}_{s,\alpha}=-{\partial \mathcal{F} \over \partial {\bf u}_{s,\alpha}}
\label{force}
\end{equation}
We have already discussed the unit of forces. It is ${\rm N}={\rm J}/{\rm m}$ in agreement
with the fact that ${\bf u}_{s,\alpha}$ is in ${\rm m}$ and $\mathcal{F}$ in ${\rm J}$.
\\

{\color{web-blue} In a.u. the unit of force is 
${\rm \bar f}= {E_h \over a_B}$ so ${{\rm \bar U} \over {\rm \bar l}\cdot {\rm \bar f}}=1$ and
Eq.~\ref{force} is unchanged.
}
\\

{\color{orange} In the c.g.s. system the unit of force 
is ${\rm \bar f_{cgs}}={\rm dyne}=10^{-5} {\rm N}$ and since ${{\rm \bar U_{cgs}} \over 
{\rm \bar l_{cgs}} \cdot {\rm \bar f_{cgs}}}=1$ Eq.~\ref{force} is unchanged.
\\
}

\newpage
{\color{coral}\section{Stress}}
\color{black}

The stress present in a crystal is given by:
\begin{equation}
{\sigma}^{(0)}_{\alpha,\beta}={1\over V} 
{\partial \mathcal{F} \over \partial \epsilon_{\alpha,\beta}},
\label{stress}
\end{equation}
where $V$ is the volume of the crystal.
We have already discussed the unit of stress in the SI system. It is
in ${\rm Pa}={\rm N}/{\rm m}^2={\rm J}/{\rm m}^3$ in agreement with the 
fact that $\epsilon_{\alpha,\beta}$ is dimensionless.
\\

{\color{web-blue} In a.u. the unit of stress is 
${\rm \bar \sigma}= {E_h \over a_B^3}$ so ${{\rm \bar U} \over {\rm \bar l}^3 \cdot {\rm \bar \sigma}}=1$ and
Eq.~\ref{stress} is unchanged.
}
\\

{\color{orange} In the c.g.s. system the unit of stress
is ${\rm \bar \sigma_{cgs}}={\rm Ba}=10^{-1} {\rm Pa}$ and since ${{\rm \bar U_{cgs}} \over 
{\rm \bar l_{cgs}}^3 \cdot {\rm \bar \sigma_{cgs}}}=1$ Eq.~\ref{stress} is unchanged.
\\
}

\newpage
{\color{coral}\section{Polarization}}
\color{black}

The polarization present in a crystal is given by:
\begin{equation}
{\bf P}^{(0)}_{\alpha}=-{1\over V} 
{\partial \mathcal{F} \over \partial {\bf E}_{\alpha}},
\label{polarization}
\end{equation}
where $V$ is the volume of the crystal.
We have already discussed the unit of polarization in the SI system. It is
in ${\rm C}/{\rm m}^2={{\rm J}\cdot {\rm C} \over {\rm N}\cdot {\rm m}^3}$ in agreement with the fact that the
electric field is in ${\rm N}/{\rm C}$ and $\mathcal{F}$ is in ${\rm J}$.
\\

{\color{web-blue} In a.u. the unit of polarization is 
${\rm \bar P}= {e \over a_B^2}$ so ${{\rm \bar U} \over {\rm \bar l}^3 
\cdot {\rm \bar E} \cdot {\rm \bar P}}=1$ and
Eq.~\ref{polarization} is unchanged.
}
\\

{\color{orange} In c.g.s.-Gaussian system the unit of polarization is 
${\rm \bar P_{cgs}}= {\rm statC}/{\rm cm}^2$ so ${{\rm \bar U_{cgs}} \over 
{\rm \bar l_{cgs}}^3 \cdot
{\rm \bar E_{cgs}}\cdot {\rm \bar P_{cgs}}}=1$ and
Eq.~\ref{polarization} is unchanged.
}

\newpage
{\color{coral}\section{Magnetization}}
\color{black}

The magnetization present in a crystal is given by:
\begin{equation}
{\bf M}^{(0)}_{\alpha}=-{1\over \mu_0 V} 
{\partial \mathcal{F} \over \partial {\bf H}_{\alpha}},
\label{magnetization}
\end{equation}
where $V$ is the volume of the crystal.
We have already discussed the unit of magnetization in the SI system. It is
in ${\rm A}/{\rm m}={{\rm J}\cdot {\rm A} \cdot {\rm m} \over {\rm N}\cdot {\rm m}^3}$ in agreement with the fact 
that the magnetic field strength is in ${\rm A}/{\rm m}$, $\mu_0$ is in ${\rm N}/{\rm A}^2$, $V$ is
in ${\rm m}^3$, and $\mathcal{F}$ is in ${\rm J}$.
\\

{\color{web-blue} In a.u. the unit of magnetization is 
${\rm \bar M}= {{\rm \bar I} \over {\rm \bar l}}$ and ${\rm \bar H} ={{\rm \bar M} \over 4 \pi}$ so 
${{\rm \bar U} \over \mu_0 {\rm \bar l}^3 \cdot {\rm \bar H}\cdot {\rm \bar M}}={1\over \alpha^2}$ and
the definition of the magnetization is:
\begin{equation}
{\bf M}^{(0)}_{\alpha}=-{1\over \alpha^2 V} 
{\partial \mathcal{F} \over \partial {\bf H}_{\alpha}}.
\label{magnetizationau}
\end{equation}
}
\\

{\color{orange} In the c.g.s-Gaussian system the unit of magnetization is 
${\rm \bar M_{cgs}}= {1\over 10^{-2} \mathcal{K}_A} {\rm A}/{\rm m}$ so 
${{\rm \bar U_{cgs}} \over \mu_0 {\rm \bar l_{cgs}}^3 
\cdot {\rm \bar H_{cgs}}\cdot {\bar M_{cgs}}}=1$ and the definition of the 
magnetization is:
\begin{equation}
{\bf M}^{(0)}_{\alpha}=-{1\over V} 
{\partial \mathcal{F} \over \partial {\bf H}_{\alpha}}.
\label{magnetizationcgs}
\end{equation}
}


\newpage
{\color{coral}\section{Interatomic force constants}}
\color{black}

The interatomic force constants are defined as:
\begin{equation}
C_{s,\alpha,s',\beta}={\partial^2 \mathcal{F} \over \partial 
{\bf u}_{s,\alpha} \partial {\bf u}_{s',\beta}}.
\label{ifc}
\end{equation}
Their unit is ${\rm J}/{\rm m}^2$.
\\

{\color{web-blue} In a.u. the unit of the interatomic force constants is 
${{\rm \bar U} \over {\rm \bar l}^2}={E_h\over a_B^2}$ and Eq.~\ref{ifc} is 
unchanged.
The conversion factor with the SI unit is:
\begin{equation}
{{\rm \bar U} \over {\rm \bar l}^2}={\baru\ {\rm J} \over (\abohr\ {\rm m})^2}=\barifc\ {\rm J}/{\rm m}^2.
\end{equation}
}
\\

{\color{orange} In c.g.s. units the unit of the interatomic 
force constants is 
${{\rm \bar U_{cgs}} \over {\rm \bar l_{cgs}}^2}={\rm erg}/{\rm cm}^2$ and Eq.~\ref{ifc} is 
unchanged.
The conversion factor with the SI unit is:
\begin{equation}
{\rm erg}/{\rm cm}^2= 10^{-3} {\rm J}/{\rm m}^2.
\end{equation}
}

\newpage
{\color{coral}\section{Elastic constants}}
\color{black}

The frozen-ions elastic constants are defined as:
\begin{equation}
C_{\alpha,\beta,\gamma,\delta}= {1\over V} 
{\partial^2 \mathcal{F} \over \partial 
\epsilon_{\alpha,\beta} \partial \epsilon_{\gamma,\delta}}={\partial
\sigma_{\alpha,\beta}\over \partial \epsilon_{\gamma,\delta}}.
\label{elastcons}
\end{equation}
Their unit is ${\rm Pa}={\rm J}/{\rm m}^3$. Here $V$ is the volume of the 
unperturbed crystal.
\\

{\color{web-blue} In a.u. the unit of the elastic constants is 
${{\rm \bar U} \over {\rm \bar l}^3}={E_h\over a_B^3}$ and 
Eq.~\ref{elastcons} is 
unchanged. This is the same unit of the pressure or stress.
}
\\

{\color{orange} In the c.g.s. system the unit of the elastic constants
is ${\rm \bar \sigma_{cgs}}={\rm Ba}=10^{-1} {\rm Pa}$ and since ${{\rm \bar U_{cgs}} \over 
{\rm \bar l_{cgs}}^3\cdot {\rm \bar \sigma_{cgs}}}=1$ Eq.~\ref{elastcons} is unchanged.
\\
}

\newpage
{\color{coral}\section{Electric susceptibility}}
\color{black}

The electric susceptibility of the crystal is defined as:
\begin{equation}
\chi^e_{\alpha,\beta}=-{1\over \epsilon_0 V} 
{\partial^2 \mathcal{F} \over \partial 
{\bf E}_{\alpha} \partial {\bf E}_{\beta}}= {1\over \epsilon_0}
{\partial {\bf P}_\alpha \over \partial {\bf E}_{\beta}},
\end{equation}
and is a dimensionless quantity since ${1\over V} 
{\partial \mathcal{F} \over \partial 
{\bf E}_{\alpha}}$ and $\epsilon_0 {\bf E}$ have both the unit of
the polarization.\\
The relative dielectric constant is $\epsilon_{r,\alpha,\beta}=
{1\over \epsilon_0} {\partial {\bf D}_{\alpha} \over \partial {\bf E}_{\beta}}=
\delta_{\alpha,\beta} 
+{1\over \epsilon_0} {\partial {\bf P}_\alpha \over \partial {\bf E}_\beta} = 
\delta_{\alpha,\beta} + 
\chi^e_{\alpha,\beta}$.
\\

{\color{web-blue} The electric susceptibility is dimensionless but
in a.u. we have
${{\rm \bar U} \over \epsilon_0 {\rm \bar l}^3 \cdot {\rm \bar E}^2}=4\pi$. 
The susceptibility is usually defined as:
\begin{equation}
\chi^{e,a.u.}_{\alpha,\beta}=-{1\over V} 
{\partial^2 \mathcal{F} \over \partial 
{\bf E}_{\alpha} \partial {\bf E}_{\beta}}= 
{\partial {\bf P}_\alpha \over \partial {\bf E}_\beta}, 
\end{equation}
so, even if $\chi^e_{\alpha,\beta}$ is dimensionless,  
the numerical value of $\chi^{e,a.u.}_{\alpha,\beta}$ differs from the
SI value and we have
$\chi^{e,a.u.}_{\alpha,\beta} = {\chi^e_{\alpha,\beta} \over 4 \pi}$. \\
The relative dielectric constant is $\epsilon_{r,\alpha,\beta}=
{\partial {\bf D}_{\alpha} \over \partial {\bf E}_{\beta}}=
\delta_{\alpha,\beta} 
+4 \pi {\partial {\bf P}_\alpha \over \partial {\bf E}_\beta} = 
\delta_{\alpha,\beta} + 
4\pi \chi^{e,a.u.}_{\alpha,\beta}$.
}
\\

{\color{orange} The electric susceptibility is dimensionless but
in c.g.s.-Gaussian units we have
${{\rm \bar U_{cgs}} \over \epsilon_0 {\rm \bar l_{cgs}}^3 \cdot 
{\rm \bar E_{cgs}}^2}=4\pi$. 
The electric susceptibility is usually defined as:
\begin{equation}
\chi^{e,cgs}_{\alpha,\beta}=-{1\over V} 
{\partial^2 \mathcal{F} \over \partial 
{\bf E}_{\alpha} \partial {\bf E}_{\beta}}=
{\partial {\bf P}_\alpha \over \partial {\bf E}_\beta}, 
\end{equation}
so, even if $\chi^e_{\alpha,\beta}$ is dimensionless,  
the numerical value of $\chi^{e,cgs}_{\alpha,\beta}$ differs from the
SI value and we have
$\chi^{e,cgs}_{\alpha,\beta} = {\chi^e_{\alpha,\beta} \over 4 \pi}$. \\
The relative dielectric constant is $\epsilon_{r,\alpha,\beta}=
{\partial {\bf D}_{\alpha} \over \partial {\bf E}_{\beta}}=
\delta_{\alpha,\beta} 
+4 \pi {\partial {\bf P}_\alpha \over \partial {\bf E}_\beta} = 
\delta_{\alpha,\beta} + 
4\pi \chi^{e,cgs}_{\alpha,\beta}$.
}

\newpage
{\color{coral}\section{Magnetic susceptibility}}
\color{black}

The magnetic susceptibility of the crystal is defined as
\begin{equation}
\chi^m_{\alpha,\beta}=-{1\over \mu_0 V} 
{\partial^2 \mathcal{F} \over \partial 
{\bf H}_{\alpha} \partial {\bf H}_{\beta}}=
{\partial {\bf M}_{\alpha} \over \partial {\bf H}_{\beta}},
\end{equation}
and is a dimensionless quantity since ${1\over \mu_0 V}
{\partial \mathcal{F} \over \partial 
{\bf H}_{\alpha}}$ and the magnetic field strength have
both the unit of a magnetization.\\
The relative permeability is $\mu_{r,\alpha,\beta}=
{1\over \mu_0} {\partial {\bf B}_{\alpha} \over \partial {\bf H}_\beta}=
\delta_{\alpha,\beta} 
+{\partial {\bf M}_\alpha \over \partial {\bf H}_\beta} = 
\delta_{\alpha,\beta} + 
\chi^m_{\alpha,\beta}$.
\\

{\color{web-blue} The magnetic susceptibility is dimensionless but
in a.u. we have
${{\rm \bar U} \over \mu_0 {\rm \bar l}^3\cdot {\rm \bar H}^2}={4\pi \over \alpha^2}$.
Since the magnetic susceptibility is defined as:
\begin{equation}
\chi^{m,a.u.}_{\alpha,\beta}=-{1 \over \alpha^2 V} 
{\partial^2 \mathcal{F} \over \partial 
{\bf H}_{\alpha} \partial {\bf H}_{\beta}}
={\partial {\bf M}_{\alpha} \over \partial {\bf H}_{\beta}},
\end{equation}
even if $\chi^m_{\alpha,\beta}$ is dimensionless,
the numerical value of $\chi^{m,a.u.}_{\alpha,\beta}$ differs from the
SI value and we have
$\chi^{m,a.u.}_{\alpha,\beta} = {\chi^m_{\alpha,\beta} \over 4 \pi}$. \\
The relative permeability is $\mu_{r,\alpha,\beta}=
{1\over \alpha^2}{\partial {\bf B}_{\alpha} \over \partial {\bf H}_\beta}=
\delta_{\alpha,\beta} 
+4 \pi {\partial {\bf M}_\alpha \over \partial {\bf H}_\beta} = 
\delta_{\alpha,\beta} + 
4 \pi \chi^{m,a.u.}_{\alpha,\beta}$.
}
\\

{\color{orange} The magnetic susceptibility is dimensionless but
in c.g.s.-Gaussian units we have
${{\rm \bar U_{cgs}} \over \mu_0 {\rm \bar l_{cgs}}^3 \cdot {\rm \bar H_{cgs}}^2}=
{4\pi}$. Since the magnetic susceptibility is defined as:
\begin{equation}
\chi^{m,cgs}_{\alpha,\beta}=-{1 \over V} 
{\partial^2 \mathcal{F} \over \partial 
{\bf H}_{\alpha} \partial {\bf H}_{\beta}}=
{\partial {\bf M}_{\alpha} \over \partial {\bf H}_{\beta}},
\end{equation}
even if $\chi^m_{\alpha,\beta}$ is dimensionless,
the numerical value of $\chi^{m,cgs}_{\alpha,\beta}$ differs from the
SI value and we have:
$\chi^{m,cgs}_{\alpha,\beta} = {\chi^m_{\alpha,\beta} \over 4 \pi}$. \\
The relative permeability is $\mu_{r,\alpha,\beta}=
{\partial {\bf B}_{\alpha} \over \partial {\bf H}_\beta}=
\delta_{\alpha,\beta} 
+4 \pi {\partial {\bf M}_\alpha \over \partial {\bf H}_\beta} = 
\delta_{\alpha,\beta} + 
4 \pi \chi^{m,cgs}_{\alpha,\beta}$.
}

\newpage
{\color{coral}\section{Internal strain parameters}}
\color{black}

The internal strain parameters are defined as:
\begin{equation}
\Lambda_{s,\alpha,\beta,\gamma}=-{\partial^2 \mathcal{F} \over \partial 
{\bf u}_{s,\alpha} \partial \epsilon_{\beta,\gamma}}
={\partial {\bf f}_{s,\alpha} \over \partial \epsilon_{\beta,\gamma}}
=-V {\partial \sigma_{\beta,\gamma}\over \partial {\bf u}_{s,\alpha}}.
\label{isp}
\end{equation}
Its unit is ${\rm J}/{\rm m}={\rm N}$.
\\

{\color{web-blue} In a.u. the unit of the internal strain parameter is
${{\rm \bar U} \over {\rm \bar l}}={E_h \over a_B}$, so it has the same unit as the force
and Eq.~\ref{isp} is unchanged.
}
\\

{\color{orange} In c.g.s. units the unit of the internal strain parameter is
${{\rm \bar U_{cgs}} \over {\rm \bar l_{cgs}}}={\rm erg}/{\rm cm}={\rm dyne}$, so it has the same unit 
as the force and Eq.~\ref{isp} is unchanged.
}

\newpage
{\color{coral}\section{Born effective charges}}
\color{black}

The Born effective charges are defined as:
\begin{equation}
Z^*_{s,\alpha,\beta}=-{1\over e} {\partial^2 \mathcal{F} \over \partial 
{\bf u}_{s,\alpha} \partial {\bf E}_{\beta}}={V\over e} {
\partial {\bf P}_{\beta} \over \partial 
{\bf u}_{s,\alpha}}={1\over e} {\partial {\bf f}_{s,\alpha} 
\over \partial {\bf E}_{\beta}}.
\end{equation}
They are dimensionless quantities since the unit of 
${\partial^2 \mathcal{F} \over \partial 
{\bf u}_{s,\alpha} \partial {\bf E}_{\beta}}$ is ${{\rm J} \cdot {\rm C} \over {\rm m}\cdot {\rm N}}=
{\rm C}$.
\\

{\color{web-blue} The Born effective charges are dimensionless 
and since 
${{\rm \bar U} \over e {\rm \bar l}\cdot {\rm \bar E}}=1$ we can write:
\begin{equation}
Z^*_{s,\alpha,\beta}=-{\partial^2 \mathcal{F} \over \partial 
{\bf u}_{s,\alpha} \partial {\bf E}_{\beta}} 
=V {
\partial {\bf P}_{\beta} \over \partial 
{\bf u}_{s,\alpha}}={\partial {\bf f}_{s,\alpha} 
\over \partial {\bf E}_{\beta}}.
\end{equation}
}
\\

{\color{orange} The Born effective charges are dimensionless
and since 
${{\rm \bar U_{cgs}} \over {\rm \bar C_{cgs}}\cdot {\rm \bar l_{cgs}}
\cdot {\rm \bar E_{cgs}}}=1$ we can write:
\begin{equation}
Z^*_{s,\alpha,\beta}=-{1\over e}{\partial^2 \mathcal{F} \over \partial 
{\bf u}_{s,\alpha} \partial {\bf E}_{\beta}}={V\over e} {
\partial {\bf P}_{\beta} \over \partial 
{\bf u}_{s,\alpha}}={1\over e} {\partial {\bf f}_{s,\alpha} 
\over \partial {\bf E}_{\beta}}.
\end{equation}
}

\newpage
{\color{coral}\section{Dynamical magnetic charges}}
\color{black}

The dynamical magnetic charges are defined as:
\begin{equation}
Z^m_{s,\alpha,\beta}= -{1\over \mu_0} {\partial^2 \mathcal{F} \over \partial 
{\bf u}_{s,\alpha} \partial {\bf H}_{\beta}}=V {\partial {\bf M}_{\beta}
\over \partial {\bf u}_{s,\alpha}}= {1\over \mu_0}{\partial {\bf f}_{s,\alpha} 
\over \partial {\bf H}_{\beta}}. 
\end{equation}
They have unit of $A\cdot m$ since the unit of 
${\partial^2 \mathcal{F} \over \partial 
{\bf u}_{s,\alpha} \partial {\bf H}_{\beta}}$ is ${{\rm J} \cdot {\rm m} \over {\rm m}\cdot {\rm A}}=
{{\rm N}\cdot {\rm m} \over {\rm A}}$ and the unit of $\mu_0$ is ${\rm N}/{\rm A}^2$.
\\

{\color{web-blue} In a.u. the unit of the dynamical magnetic charges is
${\rm \bar Z^m} = {\rm \bar I}\cdot {\rm \bar l} = {\rm \bar M}\cdot {\rm \bar l}^2$ and since 
${{\rm \bar U} \over \mu_0 {\rm \bar l}^2 \cdot {\rm \bar H}\cdot {\rm \bar I}}={1\over \alpha^2}$ we can write:
\begin{equation}
Z^m_{s,\alpha,\beta}=-{1\over \alpha^2} {\partial^2 \mathcal{F} \over \partial 
{\bf u}_{s,\alpha} \partial {\bf H}_{\beta}}=
V {\partial {\bf M}_{\beta} \over \partial {\bf u}_{s,\alpha}}=
{1\over \alpha^2}{\partial {\bf f}_{s,\alpha} \over \partial {\bf H}_{\beta}}.
\end{equation}
The conversion factor with the SI unit is:
\begin{equation}
{\rm \bar Z^m} = {{\rm \bar C}\cdot {\rm \bar l} \over {\rm \bar t}} = {e \hbar \over m_e a_B}={2 \mu_B \over a_B} = {e \over \hbar} E_h a_B=\bardmc {\rm A}\cdot {\rm m}.
\end{equation}
}
\\

{\color{orange} In c.g.s.-Gaussian units the unit of the dynamical magnetic 
charges depend on the equation used for its definition.
Requiring that
\begin{equation}
Z^m_{s,\alpha,\beta}= V 
{\partial {\bf M}_{\beta} \over \partial {\bf u}_{s,\alpha}},
\end{equation}
then ${\rm \bar Z^m_{cgs}} = {\rm \bar M_{cgs}}\cdot {\rm \bar l_{cgs}}^2$ 
and since 
${{\rm \bar U_{cgs}} \over \mu_0 {\rm \bar l_{cgs}}^3\cdot {\rm \bar H_{cgs}}
\cdot {\rm \bar M_{cgs}}}=1
$ we can write:
\begin{equation}
Z^m_{s,\alpha,\beta}=-{\partial^2 \mathcal{F} \over \partial 
{\bf u}_{s,\alpha} \partial {\bf H}_{\beta}}= V 
{\partial {\bf M}_{\beta} \over \partial {\bf u}_{s,\alpha}}=
{\partial {\bf f}_{s,\alpha} \over \partial {\bf H}_{\beta}}.
\end{equation}
The conversion factor with the SI unit is:
\begin{equation}
{\rm \bar Z^m_{cgs}} = {\rm \bar l_{cgs}}^2\cdot {\rm \bar M_{cgs}}={10^{-4} \over 10^{-2} 
\mathcal{K}_A}
{\rm A}\cdot {\rm m}=\zmtozm\ {\rm A}\cdot {\rm m}.
\end{equation}
}


\newpage
{\color{coral}\section{Piezoelectric tensor}}
\color{black}

The frozen-ions piezoelectric tensor is defined as:
\begin{equation}
e_{\alpha,\beta,\gamma}=-{1\over V} {\partial^2 \mathcal{F} \over \partial 
\epsilon_{\alpha,\beta} \partial {\bf E}_{\gamma}}={\partial 
{\bf P}_{\gamma} \over \partial \epsilon_{\alpha,\beta}}=
-{\partial \sigma_{\alpha,\beta} \over
\partial {\bf E}_{\gamma}}.
\label{piezo}
\end{equation}
Its unit is ${{\rm J}\cdot {\rm C} \over {\rm N} \cdot {\rm m}^3}= {{\rm C}\over {\rm m}^2}$.
\\

{\color{web-blue} In a.u. the unit of the piezoelectric tensor is
${{\rm \bar C} \over {\rm \bar l}^2}={e\over a_B^2}$ so it has the same unit of the
polarization. Since  
${{\rm \bar U} \over {\rm \bar l}\cdot {\rm \bar E}\cdot {\rm \bar C}}=1$ the piezoelectric tensor is
defined by Eq.~\ref{piezo} also in a.u..
}
\\

{\color{orange} In c.g.s.-Gaussian units the unit of the piezoelectric
tensor is
${{\rm \bar C_{cgs}} \over {\rm \bar l_{cgs}}^2}$ and since 
${{\rm \bar U_{cgs}} \over {\rm \bar l_{cgs}}\cdot {\rm \bar E_{cgs}}
\cdot {\rm \bar C_{cgs}}}=1$ 
the piezoelectric tensor is
defined by Eq.~\ref{piezo} also in c.g.s.-Gaussian units.
}

\newpage
{\color{coral}\section{Piezomagnetic tensor}}
\color{black}

The frozen-ions piezomagnetic tensor is defined as:
\begin{equation}
h_{\alpha,\beta,\gamma}=-{1\over \mu_0 V} 
{\partial^2 \mathcal{F} \over \partial 
\epsilon_{\alpha,\beta} \partial {\bf H}_{\gamma}}=
{\partial {\bf M}_{\gamma} \over \partial 
\epsilon_{\alpha,\beta}}=-{1\over \mu_0} 
{\partial \sigma_{\alpha,\beta} \over \partial {\bf H}_{\gamma}}.
\end{equation}
Its unit is ${{\rm J} \cdot {\rm A}^2 \cdot {\rm m}\over {\rm N} \cdot {\rm m}^3 \cdot {\rm A}}= {{\rm A}\over {\rm m}}$.
\\

{\color{web-blue} In a.u. the unit of the piezomagnetic tensor is
${\rm \bar h}={{\rm \bar I} \over {\rm \bar l}}$ and since 
${{\rm \bar U} \over \mu_0 {\rm \bar l}^2 \cdot {\rm \bar H} \cdot {\rm \bar I}}={1\over \alpha^2}$ we can write:
\begin{equation}
h_{\alpha,\beta,\gamma}=-{1\over \alpha^2 V} 
{\partial^2 \mathcal{F} \over \partial 
\epsilon_{\alpha,\beta} \partial {\bf H}_{\gamma}}=
{\partial {\bf M}_{\gamma} \over \partial 
\epsilon_{\alpha,\beta}}=-{1\over \alpha^2} 
{\partial \sigma_{\alpha,\beta} \over \partial {\bf H}_{\gamma}}.
\end{equation}
It has the same unit of the magnetization.
}
\\

{\color{orange} In c.g.s.-Gaussian units the unit of the piezomagnetic 
tensor depends on the equation used for its definition.
Requiring that
\begin{equation}
h_{\alpha,\beta,\gamma}={\partial {\bf M}_{\gamma} \over \partial 
\epsilon_{\alpha,\beta}},
\end{equation}
we obtain
${\rm \bar h_{cgs}}={\rm \bar M_{cgs}}$ and since 
${{\rm \bar U_{cgs}} \over \mu_0 {\rm \bar l_{cgs}}^3\cdot {\rm \bar H_{cgs}}
\cdot {\rm \bar M_{cgs}}}=
1$ we can write:
\begin{equation}
h_{\alpha,\beta,\gamma}=-{1\over V} 
{\partial^2 \mathcal{F} \over \partial 
\epsilon_{\alpha,\beta} \partial {\bf H}_{\gamma}}=
{\partial {\bf M}_{\gamma} \over \partial 
\epsilon_{\alpha,\beta}}=-
{\partial \sigma_{\alpha,\beta} \over \partial {\bf H}_{\gamma}}.
\end{equation}
It has the same unit of the magnetization.
}
\\

\newpage
{\color{coral}\section{Magnetoelectric tensor}}
\color{black}

The frozen-ions magnetoelectric tensor is defined as
\begin{equation}
\alpha_{\alpha,\beta}=-{1\over V} {\partial^2 \mathcal{F} \over \partial 
{\bf E}_{\alpha} \partial {\bf H}_{\beta}}=
\mu_0 {\partial {\bf M}_{\beta} \over \partial {\bf E}_{\alpha}}
={\partial {\bf P}_{\alpha} \over \partial {\bf H}_{\beta}}. 
\end{equation}
Its unit is ${{\rm J} \cdot {\rm C} \cdot {\rm m} \over {\rm m}^3 \cdot {\rm N} \cdot {\rm A}}={{\rm s} \over {\rm m}}=
{{\rm V} \cdot {\rm s} \over {\rm V}\cdot {\rm m}}= {{\rm Wb} \over {\rm V}\cdot {\rm m}} = {{\rm T} \cdot {\rm m} \over {\rm V}}$.
\\

{\color{web-blue} 

In a.u. we can choose to define the magnetoelectric tensor as (convention I):
\begin{equation}
\alpha_{\alpha,\beta}=-{4 \pi\over V} {\partial^2 \mathcal{F} \over \partial 
{\bf E}_{\alpha} \partial {\bf H}_{\beta}}= 4 \pi \alpha^2 {\partial {\bf M}_{\beta} \over \partial {\bf E}_{\alpha}}= 4 \pi
{\partial {\bf P}_{\alpha} \over \partial {\bf H}_{\beta}}, 
\end{equation}
then we have: $\bar \alpha=
{{\rm \bar U} \over 4 \pi {\rm \bar l}^3\cdot {\rm \bar E}\cdot {\rm \bar H}}={\mu_0 {\rm \bar M} \over 4 \pi {\rm \bar E}}$.
Since
${{\rm \bar U} \over {\rm \bar l}^3 \cdot {\rm \bar E} \cdot {\rm \bar H}}=
4 \pi {{\rm \bar t} \over {\rm \bar l} }$, 
the conversion factor with the SI unit is:
\begin{equation}
{\rm \bar \alpha} = {{\rm \bar t} \over {\rm \bar l}} = {1\over {\rm \bar v}}=\baralpha\ {{\rm s} \over {\rm m}}.
\end{equation}
\\
Alternatively we can define $\bar \alpha'=4 \pi \alpha^2 {{\rm \bar t} \over {\rm \bar l}}$ 
(convention II) and we have:
\begin{equation}
\alpha'_{\alpha,\beta}=-{1 \over \alpha^2 V} 
{\partial^2 \mathcal{F} \over \partial {\bf E}_{\alpha} 
\partial {\bf H}_{\beta}}= {\partial {\bf M}_{\beta} \over 
\partial {\bf E}_{\alpha}}= {1\over \alpha^2}
{\partial {\bf P}_{\alpha} \over \partial {\bf H}_{\beta}}. 
\end{equation}
The conversion factor to the SI unit is: 
\begin{equation}
\bar \alpha'= 4 \pi \alpha^2 \bar \alpha = {4 \pi \alpha^2\over {\rm \bar v}}
={4 \pi \hbar \over m_e c^2 a_B}
=\baralphap\ {{\rm s} \over {\rm m}}.
\end{equation}
}
\\

{\color{orange} 
In c.g.s.-Gaussian unit we can choose to define the magnetoelectric 
tensor as (convention I): 
\begin{equation}
\alpha_{\alpha,\beta}=-{4 \pi\over V} {\partial^2 \mathcal{F} \over \partial 
{\bf E}_{\alpha} \partial {\bf H}_{\beta}}= 4 \pi {\partial {\bf M}_{\beta} 
\over \partial {\bf E}_{\alpha}}= 4 \pi
{\partial {\bf P}_{\alpha} \over \partial {\bf H}_{\beta}},
\end{equation}
then we have: ${\rm \bar \alpha_{cgs}}=
{{\rm \bar U_{cgs}} \over 4 \pi {\rm \bar l_{cgs}}^3\cdot {\rm \bar E_{cgs}} 
\cdot {\rm \bar H_{cgs}}}={\mu_0 {\rm \bar M_{cgs}} \over 4 \pi {\rm \bar E_{cgs}}}$ 
and since ${{\rm \bar U_{cgs}} \over {\rm \bar l_{cgs}}^3 \cdot {\rm \bar E_{cgs}}\cdot {\rm \bar H_{cgs}}} ={4 \pi \over c}$ we have:
\begin{equation}
{\rm \bar \alpha_{cgs}}={1 \over c} = \alphatoalpha\ {{\rm s} \over {\rm m}}.
\end{equation}
\\
Alternatively we can define (convention II):
\begin{equation}
\alpha'_{\alpha,\beta}=-{1\over V} {\partial^2 \mathcal{F} \over \partial 
{\bf E}_{\alpha} \partial {\bf H}_{\beta}}={\partial {\bf M}_{\beta} \over \partial 
{\bf E}_{\alpha}}={\partial {\bf P}_{\alpha} \over \partial 
{\bf H}_{\beta}}. 
\end{equation}
The conversion factor to the SI unit in this case is 
${\rm \bar \alpha'_{cgs}}={{\rm \bar U_{cgs}} \over {\rm \bar l_{cgs}}^3 
\cdot {\rm \bar E_{cgs}} \cdot 
{\rm \bar H_{cgs}}}={\mu_0 {\rm \bar M_{cgs}} \over {\rm \bar E_{cgs}}}$ or:
\begin{equation}
{\rm \bar \alpha'_{cgs}}={4 \pi \over c}=\alphaptoalphap\ {{\rm s}\over 
{\rm m}}.
\end{equation}
}

\newpage
{\color{dark-blue}\chapter{Equations of motion}}
\color{black}

The first derivatives of the magneto-electric enthalpy give the following equations:
\begin{align}
{\bf f}_{s,\alpha}&= {\bf f}^{(0)}_{s,\alpha}
-\sum_{s',\beta} C_{s,\alpha,s',\beta} {\bf u}_{s',\beta}
+\sum_{\gamma,\delta}
\Lambda_{s,\alpha,\gamma,\delta} 
\epsilon_{\gamma,\delta} +
e \sum_{\gamma} Z^*_{s,\alpha,\gamma} {\bf E}_{\gamma}
+\mu_0 \sum_{\gamma} 
Z^m_{s,\alpha,\gamma} {\bf H}_{\gamma}, \\
\sigma_{\lambda,\mu}&= \sigma_{\lambda,\mu}^{(0)}
-{1\over V} \sum_{s,\alpha}
\Lambda_{s,\alpha,\lambda,\mu} {\bf u}_{s,\alpha}
+\sum_{\gamma,\delta} C_{\lambda,\mu,\gamma,\delta}  
\epsilon_{\gamma,\delta} -
\sum_{\gamma} e_{\lambda,\mu,\gamma} 
 {\bf E}_{\gamma}
-\mu_0 \sum_{\gamma}  h_{\lambda,\mu,\gamma} 
{\bf H}_{\gamma}, \\
{\bf P}_{\lambda}&= {\bf P}^{(0)}_{\lambda}
+{e\over V} \sum_{s,\alpha} Z^*_{s,\alpha,\lambda} 
{\bf u}_{s,\alpha} +
\sum_{\gamma,\delta} e_{\gamma,\delta,\lambda} 
\epsilon_{\gamma,\delta} +
\epsilon_0 \sum_{\gamma} 
\chi^e_{\lambda,\gamma}
{\bf E}_{\gamma} +
\sum_{\gamma} \alpha_{\lambda,\gamma} 
{\bf H}_{\gamma}, \\
{\bf M}_{\lambda}&= {\bf M}^{(0)}_{\lambda}
+{1\over V}\sum_{s,\alpha} 
Z^m_{s,\alpha,\lambda} {\bf u}_{s,\alpha} +
\sum_{\gamma,\delta}  h_{\gamma,\delta,\lambda} 
\epsilon_{\gamma,\delta} 
+{1\over \mu_0} \sum_{\gamma} \alpha_{\gamma,\lambda} {\bf E}_{\gamma}
+\sum_{\gamma} 
\chi^m_{\lambda,\gamma}
{\bf H}_{\gamma}.
\end{align}
We can write the equation of motion for ${\bf u}_{s,\alpha}$, 
\begin{equation}
M_s {d^2 {\bf u}_{s,\alpha} \over d t^2} = {\bf f}_{s,\alpha}
\end{equation}
and searching the solution in the form ${\bf u}_{s,\alpha}(t)=
{\bf u}_{s,\alpha} e^{i\omega t}$ we obtain:
\begin{equation}
\sum_{s',\beta} \left(C_{s,\alpha,s',\beta} - 
M_s \omega^2 \delta_{ss'}\delta_{\alpha\beta}\right)
{\bf u}_{s',\beta}= {\bf f}^{(0)}_{s,\alpha} + \sum_{\gamma,\delta} 
\Lambda_{s,\alpha,\gamma,\delta} \epsilon_{\gamma,\delta}
+ e \sum_{\gamma} Z^*_{s,\alpha,\gamma} {\bf E}_{\gamma} 
+ \mu_0 \sum_{\gamma} Z^m_{s,\alpha,\gamma} 
{\bf H}_{\gamma},
\end{equation}
where we can assume the same temporal dependence for 
${\bf f}^{(0)}_{s,\alpha}$, $\epsilon_{\gamma,\delta}$,
${\bf E}_{\gamma}$,
${\bf H}_{\gamma}$ since we are interested only in the limit 
$\omega \rightarrow 0$.
It is useful to solve first the eivenvalue problem:
\begin{equation}
\sum_{s',\beta} \left(C_{s,\alpha,s',\beta} - 
M_s \omega_j^2 \delta_{ss'}\delta_{\alpha\beta}\right){\bf e}^j_{s',\beta}=0
\end{equation}
and to define the matrix 
\begin{equation}
C^{-1}_{s,\alpha,s',\beta}(\omega) = \sideset{}{'}\sum_j {{\bf e}^j_{s,\alpha} 
{\bf e}^j_{s',\beta}
\over \sqrt{M_s M_{s'}}(\omega^2_j - \omega^2)}.
\end{equation}
The apostrophe on the sum means that we remove the three acoustic modes 
from the sum in order to avoid the divergence in the
$\lim_{\omega\rightarrow 0}$. 
It is simple to show that
\begin{equation}
\lim_{\omega\rightarrow 0} \sum_{s,\alpha} C_{s'',\gamma,s,\alpha} C^{-1}_{s,\alpha,s',\beta}(\omega)
=\delta_{s'',s'} \delta_{\gamma,\beta}
\end{equation}
so we have the displacement:
\begin{eqnarray}
{\bf u}_{s,\alpha}&=&\sum_{s',\beta} C^{-1}_{s,\alpha,s',\beta}
{\bf f}^{(0)}_{s',\beta} + \sum_{s',\beta,\gamma,\delta} 
C^{-1}_{s,\alpha,s',\beta}
\Lambda_{s',\beta,\gamma,\delta} \epsilon_{\gamma,\delta}
+ e \sum_{s',\beta,\gamma} C^{-1}_{s,\alpha,s',\beta}
Z^*_{s',\beta,\gamma} {\bf E}_{\gamma} \nonumber \\
&+& \mu_0 \sum_{s',\beta,\gamma} C^{-1}_{s,\alpha,s',\beta} 
Z^m_{s',\beta,\gamma} 
{\bf H}_{\gamma},
\end{eqnarray}
and inserting this expression in the other equations we can write:
\begin{align}
\sigma_{\lambda,\mu}&= \tilde \sigma_{\lambda,\mu}^{(0)}
+\sum_{\gamma,\delta} \tilde C_{\lambda,\mu,\gamma,\delta}  
\epsilon_{\gamma,\delta} -
\sum_{\gamma} \tilde e_{\lambda,\mu,\gamma} 
 {\bf E}_{\gamma}
-\mu_0 \sum_{\gamma}  \tilde h_{\lambda,\mu,\gamma} 
{\bf H}_{\gamma}, \\
{\bf P}_{\lambda}&= \tilde {\bf P}^{(0)}_{\lambda}
+\sum_{\gamma,\delta} \tilde e_{\gamma,\delta,\lambda} 
\epsilon_{\gamma,\delta} +
\epsilon_0 \sum_{\gamma} 
\tilde \chi^e_{\lambda,\gamma}
{\bf E}_{\gamma} +
\sum_{\gamma} \tilde \alpha_{\lambda,\gamma} 
{\bf H}_{\gamma}, \\
{\bf M}_{\lambda}&= \tilde {\bf M}^{(0)}_{\lambda}
+\sum_{\gamma,\delta}  \tilde h_{\gamma,\delta,\lambda} 
\epsilon_{\gamma,\delta} 
+{1\over \mu_0} \sum_{\gamma} \tilde \alpha_{\gamma,\lambda} {\bf E}_{\gamma}
+\sum_{\gamma} 
\tilde \chi^m_{\lambda,\gamma}
{\bf H}_{\gamma},
\end{align}
where 
\begin{align}
\tilde \sigma^{(0)}_{\lambda,\mu}&= \sigma^{(0)}_{\lambda,\mu}
-{1\over V} \sum_{s,\alpha,s',\beta} \Lambda_{s,\alpha,\lambda,\mu}
C^{-1}_{s,\alpha,s',\beta} {\bf f}^{(0)}_{s',\beta}, \\
\tilde {\bf P}^{(0)}_\lambda&= {\bf P}^{(0)}_\lambda +{e\over V}
\sum_{s,\alpha,s',\beta} Z^*_{s,\alpha,\lambda} C^{-1}_{s,\alpha,s',\beta}
{\bf f}^{(0)}_{s',\beta}, \\
\tilde {\bf M}^{(0)}_\lambda&= {\bf M}^{(0)}_\lambda +{1\over V}
\sum_{s,\alpha,s',\beta} Z^m_{s,\alpha,\lambda} C^{-1}_{s,\alpha,s',\beta}
{\bf f}^{(0)}_{s',\beta},
\end{align}
are the stress, polarization, and magnetization at the equilibrium position,
while
\begin{align}
\tilde C_{\lambda,\mu,\gamma,\delta}&= C_{\lambda,\mu,\gamma,\delta}
-{1\over V}\sum_{s,\alpha,s',\beta} \Lambda_{s,\alpha,\lambda,\mu}
C^{-1}_{s,\alpha,s',\beta}  \Lambda_{s',\beta,\gamma,\delta},\\
\tilde \chi^e_{\lambda,\gamma}&= \chi^e_{\lambda,\gamma} 
+{e^2 \over \epsilon_0 V} \sum_{s,\alpha,s',\beta} Z^*_{s,\alpha,\lambda}
C^{-1}_{s,\alpha,s',\beta}  Z^*_{s',\beta,\gamma},\\
\tilde \chi^m_{\lambda,\gamma}&= \chi^m_{\lambda,\gamma} 
+{\mu_0 \over V} \sum_{s,\alpha,s',\beta} Z^m_{s,\alpha,\lambda}
C^{-1}_{s,\alpha,s',\beta}  Z^m_{s',\beta,\gamma}, \\
\tilde e_{\lambda,\mu,\gamma}&= e_{\lambda,\mu,\gamma}
+{e\over V}\sum_{s,\alpha,s',\beta} \Lambda_{s,\alpha,\lambda,\mu}
C^{-1}_{s,\alpha,s',\beta} Z^*_{s',\beta,\gamma}
,\\
\tilde h_{\lambda,\mu,\gamma}&= h_{\lambda,\mu,\gamma}
+{1\over V}\sum_{s,\alpha,s',\beta} \Lambda_{s,\alpha,\lambda,\mu}
C^{-1}_{s,\alpha,s',\beta} Z^m_{s',\beta,\gamma},\\
\tilde \alpha_{\lambda,\gamma}&= \alpha_{\lambda,\gamma}
+{\mu_0 e \over V}\sum_{s,\alpha,s',\beta} Z^*_{s,\alpha,\lambda}
C^{-1}_{s,\alpha,s',\beta}  Z^m_{s',\beta,\gamma}
\end{align}
are the macroscopic tensors.
\\

{\color{web-blue}
The first derivatives of the magneto-electric enthalpy give the following equations:
\begin{align}
{\bf f}_{s,\alpha}&= {\bf f}^{(0)}_{s,\alpha}
-\sum_{s',\beta} C_{s,\alpha,s',\beta} {\bf u}_{s',\beta}
+\sum_{\gamma,\delta}
\Lambda_{s,\alpha,\gamma,\delta} 
\epsilon_{\gamma,\delta} +
\sum_{\gamma} Z^*_{s,\alpha,\gamma} {\bf E}_{\gamma}
+\alpha^2 \sum_{\gamma} 
Z^m_{s,\alpha,\gamma} {\bf H}_{\gamma}, \\
\sigma_{\lambda,\mu}&= \sigma_{\lambda,\mu}^{(0)}
-{1\over V} \sum_{s,\alpha}
\Lambda_{s,\alpha,\lambda,\mu} {\bf u}_{s,\alpha}
+\sum_{\gamma,\delta} C_{\lambda,\mu,\gamma,\delta}  
\epsilon_{\gamma,\delta} -
\sum_{\gamma} e_{\lambda,\mu,\gamma} 
 {\bf E}_{\gamma}
-\alpha^2 \sum_{\gamma}  h_{\lambda,\mu,\gamma} 
{\bf H}_{\gamma}, \\
{\bf P}_{\lambda}&= {\bf P}^{(0)}_{\lambda}
+{1\over V} \sum_{s,\alpha} Z^*_{s,\alpha,\lambda} 
{\bf u}_{s,\alpha} +
\sum_{\gamma,\delta} e_{\gamma,\delta,\lambda} 
\epsilon_{\gamma,\delta} +
\sum_{\gamma} 
\chi^e_{\lambda,\gamma}
{\bf E}_{\gamma} +
{1\over 4 \pi}\sum_{\gamma} \alpha_{\lambda,\gamma} 
{\bf H}_{\gamma}, \\
{\bf M}_{\lambda}&= {\bf M}^{(0)}_{\lambda}
+{1\over V}\sum_{s,\alpha} 
Z^m_{s,\alpha,\lambda} {\bf u}_{s,\alpha} +
\sum_{\gamma,\delta}  h_{\gamma,\delta,\lambda} 
\epsilon_{\gamma,\delta} 
+{1\over 4 \pi \alpha^2} 
\sum_{\gamma} \alpha_{\gamma,\lambda} {\bf E}_{\gamma}
+\sum_{\gamma} 
\chi^m_{\lambda,\gamma}
{\bf H}_{\gamma}.
\end{align}
We can write the equation of motion for ${\bf u}_{s,\alpha}$,
\begin{equation}
M_s {d^2 {\bf u}_{s,\alpha} \over d t^2} = {\bf f}_{s,\alpha}
\end{equation}
and searching the solution in the form ${\bf u}_{s,\alpha}(t)=
{\bf u}_{s,\alpha} e^{i\omega t}$ we obtain:
\begin{equation}
\sum_{s',\beta} \left(C_{s,\alpha,s',\beta} - 
M_s \omega^2 \delta_{ss'}\delta_{\alpha\beta}\right)
{\bf u}_{s',\beta}= {\bf f}^{(0)}_{s,\alpha} + \sum_{\gamma,\delta} 
\Lambda_{s,\alpha,\gamma,\delta} \epsilon_{\gamma,\delta}
+ \sum_{\gamma} Z^*_{s,\alpha,\gamma} {\bf E}_{\gamma} 
+ \alpha^2 \sum_{\gamma} Z^m_{s,\alpha,\gamma} 
{\bf H}_{\gamma},
\end{equation}
where we can assume the same temporal dependence for
${\bf f}^{(0)}_{s,\alpha}$, $\epsilon_{\gamma,\delta}$,
${\bf E}_{\gamma}$,
${\bf H}_{\gamma}$ since we are interested only in the limit
$\omega \rightarrow 0$.
It is useful to solve first the eivenvalue problem:
\begin{equation}
\sum_{s',\beta} \left(C_{s,\alpha,s',\beta} - 
M_s \omega_j^2 \delta_{ss'}\delta_{\alpha\beta}\right){\bf e}^j_{s',\beta}=0
\end{equation}
and to define the matrix
\begin{equation}
C^{-1}_{s,\alpha,s',\beta}(\omega) = \sideset{}{'}\sum_j {{\bf e}^j_{s,\alpha} {\bf e}^j_{s',\beta}
\over \sqrt{M_s M_{s'}}(\omega^2_j - \omega^2)}.
\end{equation}
The apostrophe on the sum means that we remove the three acoustic modes
from the sum in order to avoid the divergence in the
$\lim_{\omega\rightarrow 0}$.
It is simple to show that
\begin{equation}
\lim_{\omega\rightarrow 0} \sum_{s,\alpha} C_{s'',\gamma,s,\alpha} C^{-1}_{s,\alpha,s',\beta}(\omega)
=\delta_{s'',s'} \delta_{\gamma,\beta}
\end{equation}
so we have the displacement:
\begin{eqnarray}
{\bf u}_{s,\alpha}&=&\sum_{s',\beta} C^{-1}_{s,\alpha,s',\beta}
{\bf f}^{(0)}_{s',\beta} + \sum_{s',\beta,\gamma,\delta} 
C^{-1}_{s,\alpha,s',\beta}
\Lambda_{s',\beta,\gamma,\delta} \epsilon_{\gamma,\delta}
+ \sum_{s',\beta,\gamma} C^{-1}_{s,\alpha,s',\beta}
Z^*_{s',\beta,\gamma} {\bf E}_{\gamma} \nonumber \\
&+& \alpha^2 \sum_{s',\beta,\gamma} C^{-1}_{s,\alpha,s',\beta} 
Z^m_{s',\beta,\gamma} 
{\bf H}_{\gamma},
\end{eqnarray}
and inserting this expression in the other equations we can write:
\begin{align}
\sigma_{\lambda,\mu}&= \tilde \sigma_{\lambda,\mu}^{(0)}
+\sum_{\gamma,\delta} \tilde C_{\lambda,\mu,\gamma,\delta}  
\epsilon_{\gamma,\delta} -
\sum_{\gamma} \tilde e_{\lambda,\mu,\gamma} 
 {\bf E}_{\gamma}
-\alpha^2 \sum_{\gamma}  \tilde h_{\lambda,\mu,\gamma} 
{\bf H}_{\gamma}, \\
{\bf P}_{\lambda}&= \tilde {\bf P}^{(0)}_{\lambda}
+\sum_{\gamma,\delta} \tilde e_{\gamma,\delta,\lambda} 
\epsilon_{\gamma,\delta} +
\sum_{\gamma} 
\tilde \chi^e_{\lambda,\gamma}
{\bf E}_{\gamma} +
{1\over 4 \pi}\sum_{\gamma} \tilde \alpha_{\lambda,\gamma} 
{\bf H}_{\gamma}, \\
{\bf M}_{\lambda}&= \tilde {\bf M}^{(0)}_{\lambda}
+\sum_{\gamma,\delta}  \tilde h_{\gamma,\delta,\lambda} 
\epsilon_{\gamma,\delta} 
+{1\over 4 \pi \alpha^2} \sum_{\gamma} \tilde \alpha_{\gamma,\lambda} 
{\bf E}_{\gamma}
+\sum_{\gamma} 
\tilde \chi^m_{\lambda,\gamma}
{\bf H}_{\gamma},
\end{align}
where 
\begin{align}
\tilde \sigma^{(0)}_{\lambda,\mu}&= \sigma^{(0)}_{\lambda,\mu}
-{1\over V} \sum_{s,\alpha,s',\beta} \Lambda_{s,\alpha,\lambda,\mu}
C^{-1}_{s,\alpha,s',\beta} {\bf f}^{(0)}_{s',\beta}, \\
\tilde {\bf P}^{(0)}_\lambda&= {\bf P}^{(0)}_\lambda +{1\over V}
\sum_{s,\alpha,s',\beta} Z^*_{s,\alpha,\lambda} C^{-1}_{s,\alpha,s',\beta}
{\bf f}^{(0)}_{s',\beta}, \\
\tilde {\bf M}^{(0)}_\lambda&= {\bf M}^{(0)}_\lambda +{1\over V}
\sum_{s,\alpha,s',\beta} Z^m_{s,\alpha,\lambda} C^{-1}_{s,\alpha,s',\beta}
{\bf f}^{(0)}_{s',\beta},
\end{align}
are the stress, polarization, and magnetization at the equilibrium position,
while
\begin{align}
\tilde C_{\lambda,\mu,\gamma,\delta}&= C_{\lambda,\mu,\gamma,\delta}
-{1\over V}\sum_{s,\alpha,s',\beta} \Lambda_{s,\alpha,\lambda,\mu}
C^{-1}_{s,\alpha,s',\beta}  \Lambda_{s',\beta,\gamma,\delta},\\
\tilde \chi^e_{\lambda,\gamma}&= \chi^e_{\lambda,\gamma} 
+{1 \over V} \sum_{s,\alpha,s',\beta} Z^*_{s,\alpha,\lambda}
C^{-1}_{s,\alpha,s',\beta}  Z^*_{s',\beta,\gamma},\\
\tilde \chi^m_{\lambda,\gamma}&= \chi^m_{\lambda,\gamma} 
+{\alpha^2 \over V} \sum_{s,\alpha,s',\beta} Z^m_{s,\alpha,\lambda}
C^{-1}_{s,\alpha,s',\beta}  Z^m_{s',\beta,\gamma}, \\
\tilde e_{\lambda,\mu,\gamma}&= e_{\lambda,\mu,\gamma}
+{1\over V}\sum_{s,\alpha,s',\beta} \Lambda_{s,\alpha,\lambda,\mu}
C^{-1}_{s,\alpha,s',\beta} Z^*_{s',\beta,\gamma},\\
\tilde h_{\lambda,\mu,\gamma}&= h_{\lambda,\mu,\gamma}
+{1\over V}\sum_{s,\alpha,s',\beta} \Lambda_{s,\alpha,\lambda,\mu}
C^{-1}_{s,\alpha,s',\beta}  Z^m_{s',\beta,\gamma},\\
\tilde \alpha_{\lambda,\gamma}&= \alpha_{\lambda,\gamma}
+{4 \pi \alpha^2 \over V}\sum_{s,\alpha,s',\beta} Z^*_{s,\alpha,\lambda}
C^{-1}_{s,\alpha,s',\beta}  Z^m_{s',\beta,\gamma}
\end{align}
are the macroscopic tensors.
}
\\

{\color{orange}
The first derivatives of the magneto-electric enthalpy give the following equations:
\begin{align}
{\bf f}_{s,\alpha}&= {\bf f}^{(0)}_{s,\alpha}
-\sum_{s',\beta} C_{s,\alpha,s',\beta} {\bf u}_{s',\beta}
+\sum_{\gamma,\delta}
\Lambda_{s,\alpha,\gamma,\delta} 
\epsilon_{\gamma,\delta} +
e \sum_{\gamma} Z^*_{s,\alpha,\gamma} {\bf E}_{\gamma}
+\sum_{\gamma} 
Z^m_{s,\alpha,\gamma} {\bf H}_{\gamma}, \\
\sigma_{\lambda,\mu}&=  \sigma_{\lambda,\mu}^{(0)}
-{1\over V} \sum_{s,\alpha}
\Lambda_{s,\alpha,\lambda,\mu} {\bf u}_{s,\alpha}
+\sum_{\gamma,\delta} C_{\lambda,\mu,\gamma,\delta}  
\epsilon_{\gamma,\delta} -
\sum_{\gamma} e_{\lambda,\mu,\gamma} 
 {\bf E}_{\gamma}
-\sum_{\gamma}  h_{\lambda,\mu,\gamma} 
{\bf H}_{\gamma}, \\
{\bf P}_{\lambda}&= {\bf P}^{(0)}_{\lambda}
+{e\over V} \sum_{s,\alpha} Z^*_{s,\alpha,\lambda} 
{\bf u}_{s,\alpha} +
\sum_{\gamma,\delta} e_{\gamma,\delta,\lambda} 
\epsilon_{\gamma,\delta} +
\sum_{\gamma} 
\chi^e_{\lambda,\gamma}
{\bf E}_{\gamma} +
{1\over 4 \pi} \sum_{\gamma} \alpha_{\lambda,\gamma} 
{\bf H}_{\gamma}, \\
{\bf M}_{\lambda}&= {\bf M}^{(0)}_{\lambda}
+{1\over V}\sum_{s,\alpha} 
Z^m_{s,\alpha,\lambda} {\bf u}_{s,\alpha} +
\sum_{\gamma,\delta}  h_{\gamma,\delta,\lambda} 
\epsilon_{\gamma,\delta} 
+{1\over 4 \pi} \sum_{\gamma} \alpha_{\gamma,\lambda} {\bf E}_{\gamma}
+\sum_{\gamma} 
\chi^m_{\lambda,\gamma}
{\bf H}_{\gamma}.
\end{align}
We can write the equation of motion for ${\bf u}_{s,\alpha}$,
\begin{equation}
M_s {d^2 {\bf u}_{s,\alpha} \over d t^2} = {\bf f}_{s,\alpha}
\end{equation}
and searching the solution in the form ${\bf u}_{s,\alpha}(t)=
{\bf u}_{s,\alpha} e^{i\omega t}$ we obtain:
\begin{equation}
\sum_{s',\beta} \left(C_{s,\alpha,s',\beta} - 
M_s \omega^2 \delta_{ss'}\delta_{\alpha\beta}\right)
{\bf u}_{s',\beta}= {\bf f}^{(0)}_{s,\alpha} + \sum_{\gamma,\delta} 
\Lambda_{s,\alpha,\gamma,\delta} \epsilon_{\gamma,\delta}
+ e \sum_{\gamma} Z^*_{s,\alpha,\gamma} {\bf E}_{\gamma} 
+ \sum_{\gamma} Z^m_{s,\alpha,\gamma} 
{\bf H}_{\gamma},
\end{equation}
where we can assume the same temporal dependence for
${\bf f}^{(0)}_{s,\alpha}$, $\epsilon_{\gamma,\delta}$,
${\bf E}_{\gamma}$,
${\bf H}_{\gamma}$ since we are interested only in the limit
$\omega \rightarrow 0$.
It is useful to solve first the eivenvalue problem:
\begin{equation}
\sum_{s',\beta} \left(C_{s,\alpha,s',\beta} - 
M_s \omega_j^2 \delta_{ss'}\delta_{\alpha\beta}\right){\bf e}^j_{s',\beta}=0
\end{equation}
and to define the matrix
\begin{equation}
C^{-1}_{s,\alpha,s',\beta}(\omega) = \sideset{}{'}\sum_j {{\bf e}^j_{s,\alpha} {\bf e}^j_{s',\beta}
\over \sqrt{M_s M_{s'}}(\omega^2_j - \omega^2)}.
\end{equation}
The apostrophe on the sum means that we remove the three acoustic modes
from the sum in order to avoid the divergence in the
$\lim_{\omega\rightarrow 0}$.
It is simple to show that
\begin{equation}
\lim_{\omega\rightarrow 0} \sum_{s,\alpha} C_{s'',\gamma,s,\alpha} C^{-1}_{s,\alpha,s',\beta}(\omega)
=\delta_{s'',s'} \delta_{\gamma,\beta}
\end{equation}
so we have the displacement:
\begin{eqnarray}
{\bf u}_{s,\alpha}&=&\sum_{s',\beta} C^{-1}_{s,\alpha,s',\beta}
{\bf f}^{(0)}_{s',\beta} + \sum_{s',\beta,\gamma,\delta} 
C^{-1}_{s,\alpha,s',\beta}
\Lambda_{s',\beta,\gamma,\delta} \epsilon_{\gamma,\delta}
+ e \sum_{s',\beta,\gamma} C^{-1}_{s,\alpha,s',\beta}
Z^*_{s',\beta,\gamma} {\bf E}_{\gamma} \nonumber \\
&+& \sum_{s',\beta,\gamma} C^{-1}_{s,\alpha,s',\beta} Z^m_{s',\beta,\gamma} 
{\bf H}_{\gamma},
\end{eqnarray}
and inserting this expression in the other equations we can write:
\begin{align}
\sigma_{\lambda,\mu}&= \tilde \sigma_{\lambda,\mu}^{(0)}
+\sum_{\gamma,\delta} \tilde C_{\lambda,\mu,\gamma,\delta}  
\epsilon_{\gamma,\delta} -
\sum_{\gamma} \tilde e_{\lambda,\mu,\gamma} 
 {\bf E}_{\gamma}
-\sum_{\gamma}  \tilde h_{\lambda,\mu,\gamma} 
{\bf H}_{\gamma}, \\
{\bf P}_{\lambda}&= \tilde {\bf P}^{(0)}_{\lambda}
+\sum_{\gamma,\delta} \tilde e_{\gamma,\delta,\lambda} 
\epsilon_{\gamma,\delta} +
\sum_{\gamma} 
\tilde \chi^e_{\lambda,\gamma}
{\bf E}_{\gamma} +
{1\over 4 \pi}\sum_{\gamma} \tilde \alpha_{\lambda,\gamma} 
{\bf H}_{\gamma}, \\
{\bf M}_{\lambda}&= \tilde {\bf M}^{(0)}_{\lambda}
+\sum_{\gamma,\delta}  \tilde h_{\gamma,\delta,\lambda} 
\epsilon_{\gamma,\delta} 
+{1\over 4 \pi} \sum_{\gamma} \tilde \alpha_{\gamma,\lambda} {\bf E}_{\gamma}
+\sum_{\gamma} 
\tilde \chi^m_{\lambda,\gamma}
{\bf H}_{\gamma},
\end{align}
where 
\begin{align}
\tilde \sigma^{(0)}_{\lambda,\mu}&= \sigma^{(0)}_{\lambda,\mu}
-{1\over V} \sum_{s,\alpha,s',\beta} \Lambda_{s,\alpha,\lambda,\mu}
C^{-1}_{s,\alpha,s',\beta} {\bf f}^{(0)}_{s',\beta}, \\
\tilde {\bf P}^{(0)}_\lambda&= {\bf P}^{(0)}_\lambda +{e\over V}
\sum_{s,\alpha,s',\beta} Z^*_{s,\alpha,\lambda} C^{-1}_{s,\alpha,s',\beta}
{\bf f}^{(0)}_{s',\beta}, \\
\tilde {\bf M}^{(0)}_\lambda&= {\bf M}^{(0)}_\lambda +{1\over V}
\sum_{s,\alpha,s',\beta} Z^m_{s,\alpha,\lambda} C^{-1}_{s,\alpha,s',\beta}
{\bf f}^{(0)}_{s',\beta},
\end{align}
are the stress, polarization, and magnetization at the equilibrium position,
while
\begin{align}
\tilde C_{\lambda,\mu,\gamma,\delta}&= C_{\lambda,\mu,\gamma,\delta}
-{1\over V}\sum_{s,\alpha,s',\beta} \Lambda_{s,\alpha,\lambda,\mu}
C^{-1}_{s,\alpha,s',\beta}  \Lambda_{s',\beta,\gamma,\delta},\\
\tilde \chi^e_{\lambda,\gamma}&= \chi^e_{\lambda,\gamma} 
+{e^2 \over V} \sum_{s,\alpha,s',\beta} Z^*_{s,\alpha,\lambda}
C^{-1}_{s,\alpha,s',\beta}  Z^*_{s',\beta,\gamma},\\
\tilde \chi^m_{\lambda,\gamma}&= \chi^m_{\lambda,\gamma} 
+{1 \over V} \sum_{s,\alpha,s',\beta} Z^m_{s,\alpha,\lambda}
C^{-1}_{s,\alpha,s',\beta}  Z^m_{s',\beta,\gamma}, \\
\tilde e_{\lambda,\mu,\gamma}&= e_{\lambda,\mu,\gamma}
+{e\over V}\sum_{s,\alpha,s',\beta} 
\Lambda_{s,\alpha,\lambda,\mu} C^{-1}_{s,\alpha,s',\beta} 
Z^*_{s',\beta,\gamma},\\
\tilde h_{\lambda,\mu,\gamma}&= h_{\lambda,\mu,\gamma}
+{1\over V}\sum_{s,\alpha,s',\beta} \Lambda_{s,\alpha,\lambda,\mu}
C^{-1}_{s,\alpha,s',\beta} Z^m_{s',\beta,\gamma},\\
\tilde \alpha_{\lambda,\gamma}&= \alpha_{\lambda,\gamma}
+{4 \pi e \over V}\sum_{s,\alpha,s',\beta} Z^*_{s,\alpha,\lambda}
C^{-1}_{s,\alpha,s',\beta}  Z^m_{s',\beta,\gamma}
\end{align}
are the macroscopic tensors.
}
\\
\newpage

{\color{dark-blue}\chapter{Thermodynamics}}
\color{black}

The two principles of thermodynamics can be summarized by the equation:
\begin{equation}
dU = -dW + T dS,
\end{equation}
that gives the change of the internal energy of the system $dU$ when
it does a work $dW$ and we give reversibly the heat $T dS$, where $T$ 
is the temperature and $dS$ is the change of entropy.
We can consider the work done by the system when the strain
$\epsilon_{\alpha,\beta}$ change:
\begin{equation}
dW_1 =-V \sum_{\alpha,\beta} \sigma_{\alpha,\beta} d \epsilon_{\alpha,\beta}. 
\end{equation}
The work done by an insulating solid when the electric field is 
${\bf E}_\alpha$, the magnetic field intensity is ${\bf H}_\alpha$, 
and $d {\bf D}_\alpha$ are the electric displacement changes and 
$d {\bf B}_\alpha$ are the magnetic flux density changes, can be derived
from the Maxwell equations as: 
\begin{equation}
dW_2 = -V \sum_\alpha {\bf E}_\alpha d {\bf D}_\alpha 
-V \sum_{\alpha,\beta} {\bf H}_{\alpha} d {\bf B}_\alpha.
\end{equation}
The minus sign comes from the fact that we are writing the work done by the
solid. The change of internal energy is:
\begin{equation}
dU = V \sum_{\alpha,\beta} \sigma_{\alpha,\beta} d \epsilon_{\alpha,\beta}
+ V \sum_\alpha {\bf E}_\alpha d {\bf D}_\alpha + 
V \sum_{\alpha,\beta} {\bf H}_{\alpha} d {\bf B}_\alpha + T dS.
\end{equation}
It is convenient to take the temperature $T$,
the electric field ${\bf E}_\alpha$ and the magnetic field intensity
${\bf H}_\alpha$ as independent variables.
This can be done by introducing the generalized Helmholtz free energy 
by writing
$F=U- V \sum_{\alpha} {\bf E}_\alpha {\bf D}_\alpha -
V \sum_{\alpha} {\bf H}_\alpha {\bf B}_\alpha - TS $, we have then
\begin{equation}
dF = V \sum_{\alpha,\beta} \sigma_{\alpha,\beta} d \epsilon_{\alpha,\beta}
- V \sum_\alpha {\bf D}_\alpha d {\bf E}_\alpha - 
V \sum_{\alpha,\beta} {\bf B}_{\alpha} d {\bf H}_\alpha - S dT.
\end{equation}
From the previous equations we obtain:
\begin{align}
S&=-{\partial F \over \partial T}, \\
\sigma_{\alpha,\beta}&= {1\over V}
{\partial F \over \partial \epsilon_{\alpha,\beta}}, \\
{\bf D}_{\alpha}&= -{1\over V}
{\partial F \over \partial {\bf E}_{\alpha}}, \\
{\bf B}_{\alpha}&= -{1\over V}
{\partial F \over \partial {\bf H}_{\alpha}}.
\end{align}
Assuming that these quantities are functions of the independent variables
$T$, $\epsilon_{\alpha,\beta}$, ${\bf E}_\alpha$, and ${\bf H}_\alpha$ we
obtain the temperature dependent macroscopic equations:
\begin{align}
dS&= {\partial S \over \partial T} dT+\sum_{\alpha,\beta}
{\partial S \over \partial \epsilon_{\alpha,\beta}}d\epsilon_{\alpha,\beta}
+\sum_\gamma{\partial S \over \partial {\bf E}_{\gamma}}d{\bf E}_{\gamma}+ 
\sum_\delta{\partial S \over \partial {\bf H}_{\delta}}d{\bf H}_{\delta}, \\
d\sigma_{\alpha,\beta}&=
{\partial \sigma_{\alpha,\beta} \over \partial T} dT+\sum_{\gamma,\delta}
{\partial \sigma_{\alpha,\beta} \over \partial 
\epsilon_{\gamma,\delta}}d\epsilon_{\gamma,\delta}
+\sum_\gamma{\partial \sigma_{\alpha,\beta} \over \partial 
{\bf E}_{\gamma}}d{\bf E}_{\gamma}+ 
\sum_\delta{\partial \sigma_{\alpha,\beta} \over \partial 
{\bf H}_{\delta}}d{\bf H}_{\delta}, \\
d{\bf D}_\gamma&=
{\partial  {\bf D}_\gamma \over \partial T} dT+\sum_{\alpha,\beta}
{\partial {\bf D}_\gamma \over \partial \epsilon_{\alpha,\beta}}
d\epsilon_{\alpha,\beta}
+\sum_\alpha{\partial {\bf D}_\gamma \over \partial {\bf E}_{\alpha}}d{\bf E}_{\alpha}+ 
\sum_\delta{\partial {\bf D}_\gamma \over \partial {\bf H}_{\delta}}d{\bf H}_{\delta}, \\
d{\bf B}_\delta&=
{\partial {\bf B}_\delta \over \partial T} dT+\sum_{\alpha,\beta}
{\partial {\bf B}_\delta \over \partial \epsilon_{\alpha,\beta}}
d\epsilon_{\alpha,\beta}
+\sum_\gamma{\partial {\bf B}_\delta \over \partial {\bf E}_{\gamma}}
d{\bf E}_{\gamma}+ 
\sum_\alpha{\partial {\bf B}_\delta \over \partial {\bf H}_{\alpha}}
d{\bf H}_{\alpha}, 
\end{align}
These equations allow the definition of the isothermal material properties 
at finite temperature. In addition to generalizing to finite temperatures
the macroscopic quantities introduced so far, they provide other material 
properties not yet analyzed:

\subsection{\ \ Heat capacity at constant strain}
The heat capacity at constant strain is defined as
\begin{equation}
C_\epsilon= - T {\partial^2 F \over \partial T^2} = 
T {\partial S \over \partial T},
\end{equation}
and has the same unit as $S$: ${\rm J}\over {\rm K}$. Usually one considers the heat
capacity of a mole of substance and gives $C_\epsilon$ in 
${{\rm J}\over {\rm K} \cdot {\rm mol}}$.

\subsection{\ \ Thermal stresses}
The thermal stress is defined as:
\begin{equation}
\tilde b_{\alpha,\beta} = 
-{1\over V} {\partial^2 F \over \partial T
\partial \epsilon_{\alpha,\beta}}=
-{\partial \sigma_{\alpha,\beta} \over \partial T}=
{1\over V} {\partial S \over \partial {\bf \epsilon}_{\alpha,\beta}},
\end{equation}
its unit is ${{\rm N} \over {\rm m}^2 \cdot {\rm K}}$.

\subsection{\ \ Pyroelectric and electro-caloric tensors}
The pyroelectric and the electro-caloric tensors are defined as:
\begin{equation}
\tilde {\bf p}_\gamma =
-{1\over V} {\partial^2 F \over \partial T
\partial {\bf E}_\gamma}=
{\partial {\bf D}_\gamma \over \partial T}=
{1\over V} {\partial S \over \partial {\bf E}_{\gamma}},
\end{equation}
their unit is ${{\rm C} \over {\rm m}^2 \cdot {\rm K}}$. Note that one has also
${\partial {\bf D}_\gamma \over \partial T}=
{\partial {\bf P}_\gamma \over \partial T}$. 

\subsection{\ \ Pyromagnetic and magneto-caloric tensors}
The pyromagnetic and the magneto-caloric tensors are defined as:
\begin{equation}
\tilde {\bf q}_\gamma = 
-{1\over V} {\partial^2 F \over \partial T
\partial {\bf H}_\gamma}=
{\partial {\bf B}_\gamma \over \partial T}=
{1\over V} {\partial S \over \partial {\bf H}_{\gamma}},
\end{equation}
their unit is ${{\rm T} \over {\rm K}}$. Note that one has also
${\partial {\bf B}_\gamma \over \partial T}= \mu_0
{\partial {\bf M}_\gamma \over \partial T}$.

\subsection{\ \ Macroscopic equations at finite temperature}

The other derivatives are just finite temperature generalizations of the
previously discussed quantities. We can rewrite the macroscopic equations as:
\begin{align}
dS&= {C_\epsilon\over T} dT+ V \sum_{\alpha,\beta} \tilde b_{\alpha,\beta} 
d\epsilon_{\alpha,\beta}
+ V \sum_\gamma \tilde {\bf p}_\gamma d{\bf E}_{\gamma} +
V \sum_\delta \tilde {\bf q}_\delta d{\bf H}_{\delta}, \\
d\sigma_{\alpha,\beta}&= -\tilde b_{\alpha,\beta} dT + 
\sum_{\gamma,\delta} \tilde C^T_{\alpha,\beta,\gamma,\delta} 
d \epsilon_{\gamma,\delta} - \sum_\gamma \tilde e^T_{\alpha,\beta,\gamma} 
d{\bf E}_{\gamma} -\mu_0 
\sum_\delta \tilde h^T_{\alpha,\beta,\delta} d{\bf H}_{\delta}, \\
d{\bf D}_\gamma&=
\tilde {\bf p}_\gamma dT + \sum_{\alpha,\beta} \tilde 
e^T_{\alpha,\beta,\gamma} d\epsilon_{\alpha,\beta}
+\epsilon_0 \sum_\alpha \tilde \epsilon^T_{r,\gamma,\alpha} 
d{\bf E}_{\alpha} + 
\sum_\delta \tilde \alpha^T_{\gamma,\delta} d{\bf H}_{\delta}, 
\label{dsi} \\
d{\bf B}_\delta&=
\tilde {\bf q}_\delta dT+\mu_0 \sum_{\alpha,\beta}
\tilde h^T_{\alpha,\beta,\gamma}
d\epsilon_{\alpha,\beta}
+\sum_\gamma \tilde \alpha^T_{\gamma,\delta}
d{\bf E}_{\gamma}+ \mu_0
\sum_\alpha \tilde \mu^{T}_{r,\delta,\alpha}
d{\bf H}_{\alpha}, \label{bsi}
\end{align}
where the index $T$ indicates isothermal quantities and 
the tilde $\tilde{ }$ indicates that the independent variables are
$T$, $\epsilon_{\alpha,\beta}$, ${\bf E}_\alpha$, and ${\bf H}_\alpha$. 
Moreover we used the fact
that 
${\partial {\bf D}_\gamma \over \partial T} =
{\partial {\bf P}_\gamma \over \partial T}$,
${\partial {\bf D}_\gamma \over \partial \epsilon_{\alpha,\beta}}=
{\partial {\bf P}_\gamma \over \partial \epsilon_{\alpha,\beta}}$,
${\partial {\bf D}_\gamma \over \partial {\bf E}_{\alpha}}= \epsilon_0 
\delta_{\gamma,\alpha} +
{\partial {\bf P}_\gamma \over \partial {\bf E}_{\alpha}}$, 
${\partial {\bf D}_\gamma \over \partial {\bf H}_{\delta}}= 
{\partial {\bf P}_\gamma \over \partial {\bf H}_{\delta}}$, 
${\partial {\bf B}_\gamma \over \partial T} = \mu_0
{\partial {\bf M}_\gamma \over \partial T}$,
${\partial {\bf B}_\gamma \over \partial \epsilon_{\alpha,\beta}}= \mu_0
{\partial {\bf M}_\gamma \over \partial \epsilon_{\alpha,\beta}}$,
${\partial {\bf B}_\gamma \over \partial {\bf E}_{\alpha}}= \mu_0
{\partial {\bf M}_\gamma \over \partial {\bf E}_{\alpha}}$, and
${\partial {\bf B}_\gamma \over \partial {\bf H}_{\delta}}= \mu_0
(\delta_{\gamma,\delta} +
{\partial {\bf M}_\gamma \over \partial {\bf H}_{\delta}})$. 

Using these Equations Eqs.~\ref{dsi} and \ref{bsi} can be written
equivalently as:
\begin{align}
d{\bf P}_\gamma&=
\tilde {\bf p}_\gamma dT + \sum_{\alpha,\beta} \tilde 
e^T_{\alpha,\beta,\gamma} d\epsilon_{\alpha,\beta}
+\epsilon_0 \sum_\alpha \tilde \chi^{e,T}_{\gamma,\alpha} 
d{\bf E}_{\alpha} + 
\sum_\delta \tilde \alpha^T_{\gamma,\delta} d{\bf H}_{\delta}, \\
d{\bf M}_\delta&=
{1\over \mu_0}\tilde {\bf q}_\delta dT+ \sum_{\alpha,\beta}
\tilde h^T_{\alpha,\beta,\gamma}
d\epsilon_{\alpha,\beta}
+{1\over \mu_0}\sum_\gamma \tilde \alpha^T_{\gamma,\delta}
d{\bf E}_{\gamma}+ 
\sum_\alpha \tilde \chi^{m,T}_{\delta,\alpha}
d{\bf H}_{\alpha}, 
\end{align}

\newpage
{\color{web-blue}
In a.u. the two principles of thermodynamics can be summarized by the equation:
\begin{equation}
dU = -dW + T dS,
\end{equation}
that gives the change of the internal energy of the system $dU$ when
it does a work $dW$ and we give reversibly the heat $T dS$, where $T$ 
is the temperature and $dS$ is the change of entropy.
We can consider the work done by the system when the strain
$\epsilon_{\alpha,\beta}$ change:
\begin{equation}
dW_1 =-V \sum_{\alpha,\beta} \sigma_{\alpha,\beta} d \epsilon_{\alpha,\beta}. 
\end{equation}
The work done by an insulating solid when the electric field is 
${\bf E}_\alpha$, the magnetic field intensity is ${\bf H}_\alpha$, 
and $d {\bf D}_\alpha$ are the electric displacement changes and 
$d {\bf B}_\alpha$ are the magnetic flux density changes, can be derived
from the Maxwell equations as: 
\begin{equation}
dW_2 = -{V \over 4 \pi} \sum_\alpha {\bf E}_\alpha d {\bf D}_\alpha 
-{V \over 4 \pi} \sum_{\alpha,\beta} {\bf H}_{\alpha} d {\bf B}_\alpha.
\end{equation}
The minus sign comes from the fact that we are writing the work done by the
solid. The change of internal energy is:
\begin{equation}
dU = V \sum_{\alpha,\beta} \sigma_{\alpha,\beta} d \epsilon_{\alpha,\beta}
+ {V\over 4 \pi} \sum_\alpha {\bf E}_\alpha d {\bf D}_\alpha + 
{V \over 4 \pi} \sum_{\alpha,\beta} {\bf H}_{\alpha} d {\bf B}_\alpha + T dS.
\end{equation}
It is convenient to take the temperature $T$,
the electric field ${\bf E}_\alpha$ and the magnetic field intensity
${\bf H}_\alpha$ as independent variables.
This can be done by introducing the generalized Helmholtz free energy 
by writing
$F=U-{V \over 4 \pi}\sum_{\alpha} {\bf E}_\alpha {\bf D}_\alpha -
{V\over 4 \pi} \sum_{\alpha} {\bf H}_\alpha {\bf B}_\alpha - TS $, 
we have then
\begin{equation}
dF = V \sum_{\alpha,\beta} \sigma_{\alpha,\beta} d \epsilon_{\alpha,\beta}
- {V\over 4 \pi} \sum_\alpha {\bf D}_\alpha d {\bf E}_\alpha - 
{V \over 4 \pi} \sum_{\alpha,\beta} {\bf B}_{\alpha} d {\bf H}_\alpha - S dT.
\end{equation}
From the previous equations we obtain:
\begin{align}
S&= -{\partial F \over \partial T}, \\
\sigma_{\alpha,\beta}&= {1\over V}
{\partial F \over \partial \epsilon_{\alpha,\beta}}, \\
{\bf D}_{\alpha}&= -{4 \pi \over V}
{\partial F \over \partial {\bf E}_{\alpha}}, \\
{\bf B}_{\alpha}&= -{4 \pi \over V}
{\partial F \over \partial {\bf H}_{\alpha}}.
\end{align}
Assuming that these quantities are functions of the independent variables
$T$, $\epsilon_{\alpha,\beta}$, ${\bf E}_\alpha$, and ${\bf H}_\alpha$ we
obtain the temperature dependent macroscopic equations:
\begin{align}
dS&= {\partial S \over \partial T} dT+\sum_{\alpha,\beta}
{\partial S \over \partial \epsilon_{\alpha,\beta}}d\epsilon_{\alpha,\beta}
+\sum_\gamma{\partial S \over \partial {\bf E}_{\gamma}}d{\bf E}_{\gamma}+ 
\sum_\delta{\partial S \over \partial {\bf H}_{\delta}}d{\bf H}_{\delta}, \\
d\sigma_{\alpha,\beta}&=
{\partial \sigma_{\alpha,\beta} \over \partial T} dT+\sum_{\gamma,\delta}
{\partial \sigma_{\alpha,\beta} \over \partial 
\epsilon_{\gamma,\delta}}d\epsilon_{\gamma,\delta}
+\sum_\gamma{\partial \sigma_{\alpha,\beta} \over \partial 
{\bf E}_{\gamma}}d{\bf E}_{\gamma}+ 
\sum_\delta{\partial \sigma_{\alpha,\beta} \over \partial 
{\bf H}_{\delta}}d{\bf H}_{\delta}, \\
d{\bf D}_\gamma&=
{\partial  {\bf D}_\gamma \over \partial T} dT+\sum_{\alpha,\beta}
{\partial {\bf D}_\gamma \over \partial \epsilon_{\alpha,\beta}}
d\epsilon_{\alpha,\beta}
+\sum_\alpha{\partial {\bf D}_\gamma \over \partial {\bf E}_{\alpha}}d{\bf E}_{\alpha}+ 
\sum_\delta{\partial {\bf D}_\gamma \over \partial {\bf H}_{\delta}}d{\bf H}_{\delta}, \\
d{\bf B}_\delta&=
{\partial {\bf B}_\delta \over \partial T} dT+\sum_{\alpha,\beta}
{\partial {\bf B}_\delta \over \partial \epsilon_{\alpha,\beta}}
d\epsilon_{\alpha,\beta}
+\sum_\gamma{\partial {\bf B}_\delta \over \partial {\bf E}_{\gamma}}
d{\bf E}_{\gamma}+ 
\sum_\alpha{\partial {\bf B}_\delta \over \partial {\bf H}_{\alpha}}
d{\bf H}_{\alpha}, 
\end{align}
These equations allow the definition of the isothermal material properties 
at finite temperature. In addition to generalizing to finite temperatures
the macroscopic quantities introduced so far, they provide other material 
properties not yet analyzed:

\subsection{\color{web-blue}\ \ Heat capacity at constant strain}
The heat capacity at constant strain is defined as
\begin{equation}
C_\epsilon= - T {\partial^2 F \over \partial T^2} = 
T {\partial S \over \partial T},
\end{equation}
and has the same unit as $S$: ${\rm \bar U}\over {\rm K}$. Usually 
one considers the heat
capacity of a mole of substance and gives $C_\epsilon$ in 
${{\rm \bar U}\over {\rm K} \cdot {\rm mol}}$.

\subsection{\color{web-blue}\ \ Thermal stresses}
The thermal stress is defined as:
\begin{equation}
\tilde b_{\alpha,\beta} = 
-{1\over V} {\partial^2 F \over \partial T
\partial \epsilon_{\alpha,\beta}}=
-{\partial \sigma_{\alpha,\beta} \over \partial T}=
{1\over V} {\partial S \over \partial {\bf \epsilon}_{\alpha,\beta}},
\end{equation}
its unit is ${E_h \over a_B^3 \cdot K}$.

\subsection{\color{web-blue}\ \ Pyroelectric and electro-caloric tensors}
The pyroelectric and the electro-caloric tensors are defined as:
\begin{equation}
\tilde {\bf p}_\gamma =
-{1\over V} {\partial^2 F \over \partial T
\partial {\bf E}_\gamma}=
{1\over 4 \pi} {\partial {\bf D}_\gamma \over \partial T}=
{1\over V} {\partial S \over \partial {\bf E}_{\gamma}},
\end{equation}
their unit is ${e \over a_B^2 \cdot K}$. Note that one has also
${\partial {\bf D}_\gamma \over \partial T}=
4 \pi {\partial {\bf P}_\gamma \over \partial T}$. 

\subsection{\color{web-blue}\ \ Pyromagnetic and magneto-caloric tensors}
The pyromagnetic and the magneto-caloric tensors are defined as:
\begin{equation}
\tilde {\bf q}_\gamma = 
-{1\over V} {\partial^2 F \over \partial T
\partial {\bf H}_\gamma}=
{1\over 4 \pi} {\partial {\bf B}_\gamma \over \partial T}=
{1\over V} {\partial S \over \partial {\bf H}_{\gamma}},
\end{equation}
their unit is ${4 \pi \hbar \over a_B^2 \cdot e \cdot K}$. 
Note that one has also
${\partial {\bf B}_\gamma \over \partial T}= 4 \pi \alpha^2
{\partial {\bf M}_\gamma \over \partial T}$.

\subsection{\color{web-blue}\ \ Macroscopic equations at finite temperature}
The other derivatives are just finite temperature generalizations of the
previously discussed quantities. We can rewrite the macroscopic equations as:
\begin{align}
dS&= {C_\epsilon\over T} dT+ V \sum_{\alpha,\beta} \tilde b_{\alpha,\beta} 
d\epsilon_{\alpha,\beta}
+ V \sum_\gamma \tilde {\bf p}_\gamma d{\bf E}_{\gamma} + 
V \sum_\delta \tilde {\bf q}_\delta d{\bf H}_{\delta}, \\
d\sigma_{\alpha,\beta}&= -\tilde b_{\alpha,\beta} dT + 
\sum_{\gamma,\delta} \tilde C^T_{\alpha,\beta,\gamma,\delta} 
d \epsilon_{\gamma,\delta} - \sum_\gamma \tilde e^T_{\alpha,\beta,\gamma} 
d{\bf E}_{\gamma} -\alpha^2
\sum_\delta \tilde h^T_{\alpha,\beta,\delta} d{\bf H}_{\delta}, \\
d{\bf D}_\gamma&=
4 \pi \tilde {\bf p}_\gamma dT + 4 \pi \sum_{\alpha,\beta} \tilde 
e^T_{\alpha,\beta,\gamma} d\epsilon_{\alpha,\beta}
+\sum_\alpha \tilde \epsilon^T_{r,\gamma,\alpha} 
d{\bf E}_{\alpha} + 
\sum_\delta \tilde \alpha^T_{\gamma,\delta} d{\bf H}_{\delta}, 
\label{dau} \\
d{\bf B}_\delta&=
4 \pi \tilde {\bf q}_\delta dT+ 4 \pi \alpha^2 \sum_{\alpha,\beta}
\tilde h^T_{\alpha,\beta,\gamma}
d\epsilon_{\alpha,\beta}
+\sum_\gamma \tilde \alpha^T_{\delta,\gamma}
d{\bf E}_{\gamma}+ \alpha^2
\sum_\alpha \tilde \mu^{T}_{r,\delta,\alpha}
d{\bf H}_{\alpha}, \label{bau} 
\end{align}
where the index $T$ indicates isothermal quantities and 
the tilde $\tilde{ }$ indicates that the independent variables are
$T$, $\epsilon_{\alpha,\beta}$, ${\bf E}_\alpha$, and ${\bf H}_\alpha$. 
Moreover we used the fact
that 
${\partial {\bf D}_\gamma \over \partial T} =
4 \pi {\partial {\bf P}_\gamma \over \partial T}$,
${\partial {\bf D}_\gamma \over \partial \epsilon_{\alpha,\beta}}=
4 \pi {\partial {\bf P}_\gamma \over \partial \epsilon_{\alpha,\beta}}$,
${\partial {\bf D}_\gamma \over \partial {\bf E}_{\alpha}}=  
\delta_{\gamma,\alpha} + 4 \pi 
{\partial {\bf P}_\gamma \over \partial {\bf E}_{\alpha}}$, 
${\partial {\bf D}_\gamma \over \partial {\bf H}_{\delta}}= 
4 \pi {\partial {\bf P}_\gamma \over \partial {\bf H}_{\delta}}$, 
${\partial {\bf B}_\gamma \over \partial T} = 4 \pi \alpha^2
{\partial {\bf M}_\gamma \over \partial T}$,
${\partial {\bf B}_\gamma \over \partial \epsilon_{\alpha,\beta}}= 4 \pi 
\alpha^2 {\partial {\bf M}_\gamma \over \partial \epsilon_{\alpha,\beta}}$,
${\partial {\bf B}_\gamma \over \partial {\bf E}_{\alpha}}= 4 \pi \alpha^2
{\partial {\bf M}_\gamma \over \partial {\bf E}_{\alpha}}$, and
${\partial {\bf B}_\gamma \over \partial {\bf H}_{\delta}}= \alpha^2
(\delta_{\gamma,\delta} + 4 \pi
{\partial {\bf M}_\gamma \over \partial {\bf H}_{\delta}})$. 

Using these Equations Eqs.~\ref{dau} and \ref{bau} can be written
equivalently as:
\begin{align}
d{\bf P}_\gamma&=
\tilde {\bf p}_\gamma dT + \sum_{\alpha,\beta} \tilde 
e^T_{\alpha,\beta,\gamma} d\epsilon_{\alpha,\beta}
+ \sum_\alpha \tilde \chi^{e,T}_{\gamma,\alpha} 
d{\bf E}_{\alpha} + {1\over 4 \pi}
\sum_\delta \tilde \alpha^T_{\gamma,\delta} d{\bf H}_{\delta}, \\
d{\bf M}_\delta&=
{1\over \alpha^2}\tilde {\bf q}_\delta dT+ \sum_{\alpha,\beta}
\tilde h^T_{\alpha,\beta,\gamma}
d\epsilon_{\alpha,\beta}
+{1\over 4 \pi \alpha^2} \sum_\gamma \tilde \alpha^T_{\gamma,\delta}
d{\bf E}_{\gamma}+ 
\sum_\alpha \tilde \chi^{m,T}_{\delta,\alpha}
d{\bf H}_{\alpha}, 
\end{align}
}

\newpage
{\color{orange}
The two principles of thermodynamics can be summarized by the equation:
\begin{equation}
dU = -dW + T dS,
\end{equation}
that gives the change of the internal energy of the system $dU$ when
it does a work $dW$ and we give reversibly the heat $T dS$, where $T$ 
is the temperature and $dS$ is the change of entropy.
We can consider the work done by the system when the strain
$\epsilon_{\alpha,\beta}$ change:
\begin{equation}
dW_1 =-V \sum_{\alpha,\beta} \sigma_{\alpha,\beta} d \epsilon_{\alpha,\beta}. 
\end{equation}
The work done by an insulating solid when the electric field is 
${\bf E}_\alpha$, the magnetic field intensity is ${\bf H}_\alpha$, 
and $d {\bf D}_\alpha$ are the electric displacement changes and 
$d {\bf B}_\alpha$ are the magnetic flux density changes, can be derived
from the Maxwell equations as: 
\begin{equation}
dW_2 = -{V \over 4 \pi} \sum_\alpha {\bf E}_\alpha d {\bf D}_\alpha 
-{V \over 4 \pi} \sum_{\alpha,\beta} {\bf H}_{\alpha} d {\bf B}_\alpha.
\end{equation}
The minus sign comes from the fact that we are writing the work done by the
solid. The change of internal energy is:
\begin{equation}
dU = V \sum_{\alpha,\beta} \sigma_{\alpha,\beta} d \epsilon_{\alpha,\beta}
+ {V \over 4 \pi} \sum_\alpha {\bf E}_\alpha d {\bf D}_\alpha + 
{V \over 4 \pi} \sum_{\alpha,\beta} {\bf H}_{\alpha} d {\bf B}_\alpha + T dS.
\end{equation}
It is convenient to take the temperature $T$,
the electric field ${\bf E}_\alpha$ and the magnetic field intensity
${\bf H}_\alpha$ as independent variables.
This can be done by introducing the generalized Helmholtz free energy 
by writing
$F=U- {V \over 4 \pi} \sum_{\alpha} {\bf E}_\alpha {\bf D}_\alpha -
{V \over 4 \pi} \sum_{\alpha} {\bf H}_\alpha {\bf B}_\alpha - TS $, we 
have then
\begin{equation}
dF = V \sum_{\alpha,\beta} \sigma_{\alpha,\beta} d \epsilon_{\alpha,\beta}
- {V \over 4 \pi} \sum_\alpha {\bf D}_\alpha d {\bf E}_\alpha - 
{V \over 4 \pi} \sum_{\alpha,\beta} {\bf B}_{\alpha} d {\bf H}_\alpha - S dT.
\end{equation}
From the previous equations we obtain:
\begin{align}
S&=-{\partial F \over \partial T}, \\
\sigma_{\alpha,\beta}&= {1\over V}
{\partial F \over \partial \epsilon_{\alpha,\beta}}, \\
{\bf D}_{\alpha}&= -{4 \pi \over V}
{\partial F \over \partial {\bf E}_{\alpha}}, \\
{\bf B}_{\alpha}&= -{4 \pi \over V}
{\partial F \over \partial {\bf H}_{\alpha}}.
\end{align}
Assuming that these quantities are functions of the independent variables
$T$, $\epsilon_{\alpha,\beta}$, ${\bf E}_\alpha$, and ${\bf H}_\alpha$ we
obtain the temperature dependent macroscopic equations:
\begin{align}
dS&= {\partial S \over \partial T} dT+\sum_{\alpha,\beta}
{\partial S \over \partial \epsilon_{\alpha,\beta}}d\epsilon_{\alpha,\beta}
+\sum_\gamma{\partial S \over \partial {\bf E}_{\gamma}}d{\bf E}_{\gamma}+ 
\sum_\delta{\partial S \over \partial {\bf H}_{\delta}}d{\bf H}_{\delta}, \\
d\sigma_{\alpha,\beta}&=
{\partial \sigma_{\alpha,\beta} \over \partial T} dT+\sum_{\gamma,\delta}
{\partial \sigma_{\alpha,\beta} \over \partial 
\epsilon_{\gamma,\delta}}d\epsilon_{\gamma,\delta}
+\sum_\gamma{\partial \sigma_{\alpha,\beta} \over \partial 
{\bf E}_{\gamma}}d{\bf E}_{\gamma}+ 
\sum_\delta{\partial \sigma_{\alpha,\beta} \over \partial 
{\bf H}_{\delta}}d{\bf H}_{\delta}, \\
d{\bf D}_\gamma&=
{\partial  {\bf D}_\gamma \over \partial T} dT+\sum_{\alpha,\beta}
{\partial {\bf D}_\gamma \over \partial \epsilon_{\alpha,\beta}}
d\epsilon_{\alpha,\beta}
+\sum_\alpha{\partial {\bf D}_\gamma \over \partial {\bf E}_{\alpha}}d{\bf E}_{\alpha}+ 
\sum_\delta{\partial {\bf D}_\gamma \over \partial {\bf H}_{\delta}}d{\bf H}_{\delta}, \\
d{\bf B}_\delta&=
{\partial {\bf B}_\delta \over \partial T} dT+\sum_{\alpha,\beta}
{\partial {\bf B}_\delta \over \partial \epsilon_{\alpha,\beta}}
d\epsilon_{\alpha,\beta}
+\sum_\gamma{\partial {\bf B}_\delta \over \partial {\bf E}_{\gamma}}
d{\bf E}_{\gamma}+ 
\sum_\alpha{\partial {\bf B}_\delta \over \partial {\bf H}_{\alpha}}
d{\bf H}_{\alpha}, 
\end{align}
These equations allow the definition of the isothermal material properties 
at finite temperature. In addition to generalizing to finite temperatures
the macroscopic quantities introduced so far, they provide other material 
properties not yet analyzed:

\subsection{\color{orange}\ \ Heat capacity at constant strain}
The heat capacity at constant strain is defined as
\begin{equation}
C_\epsilon= - T {\partial^2 F \over \partial T^2} = 
T {\partial S \over \partial T},
\end{equation}
and has the same unit as $S$: ${\rm erg}\over {\rm K}$. Usually one 
considers the heat
capacity of a mole of substance and gives $C_\epsilon$ in 
${{\rm erg}\over {\rm K} \cdot {\rm mol}}$.

\subsection{\color{orange}\ \ Thermal stresses}
The thermal stress is defined as:
\begin{equation}
\tilde b_{\alpha,\beta} = 
-{1\over V} {\partial^2 F \over \partial T
\partial \epsilon_{\alpha,\beta}}=
-{\partial \sigma_{\alpha,\beta} \over \partial T}=
{1\over V} {\partial S \over \partial {\bf \epsilon}_{\alpha,\beta}},
\end{equation}
its unit is ${{\rm dyne} \over {\rm cm}^2 \cdot {\rm K}}$.

\subsection{\color{orange}\ \ Pyroelectric and electro-caloric tensors}
The pyroelectric and the electro-caloric tensors are defined as:
\begin{equation}
\tilde {\bf p}_\gamma =
-{1\over V} {\partial^2 F \over \partial T
\partial {\bf E}_\gamma}=
{1\over 4 \pi} {\partial {\bf D}_\gamma \over \partial T}=
{1\over V} {\partial S \over \partial {\bf E}_{\gamma}},
\end{equation}
their unit is ${{\rm statC} \over {\rm cm}^2 \cdot {\rm K}}$. Note that one has also
${\partial {\bf D}_\gamma \over \partial T}=4 \pi
{\partial {\bf P}_\gamma \over \partial T}$. 

\subsection{\color{orange}\ \ Pyromagnetic and magneto-caloric tensors}
The pyromagnetic and the magneto-caloric tensors are defined as:
\begin{equation}
\tilde {\bf q}_\gamma = 
-{1\over V} {\partial^2 F \over \partial T
\partial {\bf H}_\gamma}=
{1 \over 4 \pi} {\partial {\bf B}_\gamma \over \partial T}=
{1\over V} {\partial S \over \partial {\bf H}_{\gamma}},
\end{equation}
their unit is ${4 \pi G \over K}$. Note that one has also
${\partial {\bf B}_\gamma \over \partial T}= 4 \pi
{\partial {\bf M}_\gamma \over \partial T}$.

\subsection{\color{orange}\ \ Macroscopic equations at finite temperature}
The other derivatives are just finite temperature generalizations of the
previously discussed quantities. We can rewrite the macroscopic equations as:
\begin{align}
dS&= {C_\epsilon\over T} dT+ V \sum_{\alpha,\beta} \tilde b_{\alpha,\beta} 
d\epsilon_{\alpha,\beta}
+ V \sum_\gamma \tilde {\bf p}_\gamma d{\bf E}_{\gamma} +
V \sum_\delta \tilde {\bf q}_\delta d{\bf H}_{\delta}, \\
d\sigma_{\alpha,\beta}&= -\tilde b_{\alpha,\beta} dT + 
\sum_{\gamma,\delta} \tilde C^T_{\alpha,\beta,\gamma,\delta} 
d \epsilon_{\gamma,\delta} - \sum_\gamma \tilde e^T_{\alpha,\beta,\gamma} 
d{\bf E}_{\gamma} -
\sum_\delta \tilde h^T_{\alpha,\beta,\delta} d{\bf H}_{\delta}, \\
d{\bf D}_\gamma&=
4 \pi \tilde {\bf p}_\gamma dT + 4 \pi \sum_{\alpha,\beta} \tilde 
e^T_{\alpha,\beta,\gamma} d\epsilon_{\alpha,\beta}
+ \sum_\alpha \tilde \epsilon^T_{r,\gamma,\alpha} 
d{\bf E}_{\alpha} + 
\sum_\delta \tilde \alpha^T_{\gamma,\delta} d{\bf H}_{\delta}, 
\label{dcgs} \\
d{\bf B}_\delta&=
4 \pi \tilde {\bf q}_\delta dT+ 4 \pi \sum_{\alpha,\beta}
\tilde h^T_{\alpha,\beta,\gamma}
d\epsilon_{\alpha,\beta}
+ \sum_\gamma \tilde \alpha^T_{\gamma,\delta}
d{\bf E}_{\gamma}+ 
\sum_\alpha \tilde \mu^{T}_{r,\delta,\alpha}
d{\bf H}_{\alpha}, \label{bcgs}
\end{align}
where the index $T$ indicates isothermal quantities and 
the tilde $\tilde{ }$ indicates that the independent variables are
$T$, $\epsilon_{\alpha,\beta}$, ${\bf E}_\alpha$, and ${\bf H}_\alpha$. 
Moreover we used the fact that 
${\partial {\bf D}_\gamma \over \partial T} =
4 \pi {\partial {\bf P}_\gamma \over \partial T}$,
${\partial {\bf D}_\gamma \over \partial \epsilon_{\alpha,\beta}}=
4 \pi {\partial {\bf P}_\gamma \over \partial \epsilon_{\alpha,\beta}}$,
${\partial {\bf D}_\gamma \over \partial {\bf E}_{\alpha}}=  
\delta_{\gamma,\alpha} + 4 \pi 
{\partial {\bf P}_\gamma \over \partial {\bf E}_{\alpha}}$, 
${\partial {\bf D}_\gamma \over \partial {\bf H}_{\delta}}= 
4 \pi {\partial {\bf P}_\gamma \over \partial {\bf H}_{\delta}}$, 
${\partial {\bf B}_\gamma \over \partial T} = 4 \pi
{\partial {\bf M}_\gamma \over \partial T}$,
${\partial {\bf B}_\gamma \over \partial \epsilon_{\alpha,\beta}}= 4 \pi
{\partial {\bf M}_\gamma \over \partial \epsilon_{\alpha,\beta}}$,
${\partial {\bf B}_\gamma \over \partial {\bf E}_{\alpha}}= 4 \pi
{\partial {\bf M}_\gamma \over \partial {\bf E}_{\alpha}}$, and
${\partial {\bf B}_\gamma \over \partial {\bf H}_{\delta}}= 
\delta_{\gamma,\delta} + 4 \pi
{\partial {\bf M}_\gamma \over \partial {\bf H}_{\delta}}$. 

Using these Equations Eqs.~\ref{dcgs} and \ref{bcgs} can be written
equivalently as:
\begin{align}
d{\bf P}_\gamma&=
\tilde {\bf p}_\gamma dT + \sum_{\alpha,\beta} \tilde 
e^T_{\alpha,\beta,\gamma} d\epsilon_{\alpha,\beta}
+ \sum_\alpha \tilde \chi^{e,T}_{\gamma,\alpha} 
d{\bf E}_{\alpha} + {1\over 4 \pi}
\sum_\delta \tilde \alpha^T_{\gamma,\delta} d{\bf H}_{\delta}, \\
d{\bf M}_\delta&= \tilde {\bf q}_\delta dT+ \sum_{\alpha,\beta}
\tilde h^T_{\alpha,\beta,\gamma}
d\epsilon_{\alpha,\beta}
+{1\over 4 \pi} \sum_\gamma \tilde \alpha^T_{\gamma,\delta}
d{\bf E}_{\gamma}+ 
\sum_\alpha \tilde \chi^{m,T}_{\delta,\alpha}
d{\bf H}_{\alpha}, 
\end{align}
}

\newpage
{\color{dark-blue}\chapter{Bibliography}}
\color{black}

\begin{enumerate}

\item
[1.] J.D. Jackson, Classical electrodynamics, Third edition,
J. Wiley and Sons (1999).

\item
[2.] M. Ye and D. Vanderbilt, Phys. Rev. B {\bf 89}, 064301 (2014).

\item[3.] J.F. Nye, Physical properties of crystals, Oxford university 
press (1985).
\end{enumerate}


\appendix
{\color{dark-blue}\chapter{Stress as an independent variable}}
\color{black}

While in calculations it is easy to control the strain applied to
a given crystal, experimentally it is simpler to apply a given stress
on a solid. It is therefore common for instance to define the 
piezoelectric (piezomagnetic) tensor from the polarization 
(magnetization) induced by a stress. In order to define it one 
can introduce a free-energy depending on $\sigma_{\alpha,\beta}$,
${\bf E}_\alpha$, and ${\bf H}_\alpha$. Alternatively one can use the
equations of motion to write the strain as a function of stress and
to insert its espression in the equation for the polarization
(magnetization). We obtain:
\begin{align}
\epsilon_{\gamma,\delta} &= 
- \sum_{\alpha,\beta} 
C^{-1}_{\gamma,\delta,\alpha,\beta} \sigma^{(0)}_{\alpha,\beta} 
+ \sum_{\alpha,\beta} 
C^{-1}_{\gamma,\delta,\alpha,\beta} \sigma_{\alpha,\beta} 
+ \sum_{\alpha,\beta,\mu} 
C^{-1}_{\gamma,\delta,\alpha,\beta} \tilde e_{\alpha,\beta,\mu} {\bf E}_\mu 
\nonumber \\
&+ \mu_0 \sum_{\alpha,\beta,\mu} 
C^{-1}_{\gamma,\delta,\alpha,\beta} \tilde h_{\alpha,\beta,\mu} {\bf H}_\mu 
\\
{\bf P}_{\lambda}&= \hat {\bf P}^{(0)}_{\lambda}
+\sum_{\alpha,\beta} \hat d_{\alpha,\beta,\lambda} 
\sigma_{\alpha,\beta} +
\epsilon_0 \sum_{\mu} 
\hat \chi^e_{\lambda,\mu}
{\bf E}_{\mu} +
\sum_{\mu} \hat \alpha_{\lambda,\mu} 
{\bf H}_{\mu}, \\
{\bf M}_{\lambda}&= \hat {\bf M}^{(0)}_{\lambda}
+\sum_{\alpha,\beta} \hat h_{\alpha,\beta,\lambda} 
\sigma_{\alpha,\beta} +
{1\over \mu_0} \sum_{\mu} \hat \alpha_{\lambda,\mu} 
{\bf E}_{\mu}+
\sum_{\mu} 
\hat \chi^m_{\lambda,\mu}
{\bf H}_{\mu}
\end{align} 
where
\begin{align}
\hat {\bf P}^{(0)}_{\lambda}&= \tilde {\bf P}^{(0)}_{\lambda}
-\sum_{\alpha,\beta,\gamma,\delta} \tilde e_{\alpha,\beta,\lambda}
\tilde C^{-1}_{\alpha,\beta,\gamma,\delta} 
\tilde \sigma^{(0)}_{\gamma,\delta}, \\
\hat {\bf M}^{(0)}_{\lambda}&= \tilde {\bf M}^{(0)}_{\lambda}
-\sum_{\alpha,\beta,\gamma,\delta} \tilde h_{\alpha,\beta,\lambda}
\tilde C^{-1}_{\alpha,\beta,\gamma,\delta} 
\tilde \sigma^{(0)}_{\gamma,\delta}, \\
\hat d_{\alpha,\beta,\lambda}&= \sum_{\gamma,\delta} 
\tilde e_{\gamma,\delta,\lambda} 
\tilde C^{-1}_{\gamma,\delta,\alpha,\beta}, \\
\hat h_{\alpha,\beta,\lambda}&= \sum_{\gamma,\delta} 
\tilde h_{\gamma,\delta,\lambda} 
\tilde C^{-1}_{\gamma,\delta,\alpha,\beta}, \\
\hat \chi^e_{\lambda,\mu}&= \tilde \chi^e_{\lambda,\mu}
+{1\over \epsilon_0} 
\sum_{\alpha,\beta,\gamma,\delta}\tilde e_{\gamma,\delta,\lambda}
\tilde C^{-1}_{\gamma,\delta,\alpha,\beta} \tilde e_{\alpha,\beta,\mu}, \\
\hat \chi^m_{\lambda,\mu}&= \tilde \chi^m_{\lambda,\mu}
+{\mu_0}\sum_{\alpha,\beta,\gamma,\delta}\tilde h_{\gamma,\delta,\lambda}
\tilde C^{-1}_{\gamma,\delta,\alpha,\beta} \tilde h_{\alpha,\beta,\mu}, \\
\hat \alpha_{\lambda,\mu}&= \tilde \alpha_{\lambda,\mu}
+\mu_0 \sum_{\alpha,\beta,\gamma,\delta}\tilde e_{\gamma,\delta,\lambda}
\tilde C^{-1}_{\gamma,\delta,\alpha,\beta} \tilde h_{\alpha,\beta,\mu},
\end{align}
where $\hat \chi^e_{\lambda,\mu}$, $\hat \chi^m_{\lambda,\mu}$ and
$\hat \alpha_{\lambda,\mu}$ are the dielectric susceptibility, the magnetic
susceptibility, 
and the magneto-electric tensor at constant stress, while 
$\hat d_{\alpha,\beta,\lambda}$ ($\hat h_{\alpha,\beta,\lambda}$)
is the piezoelectric (piezomagnetic) tensor that
gives the polarization (magnetization) induced by a stress. 

\subsection{\ \ Thermodynamic equations}
Similarly for the thermodynamic equations we can introduce the Gibbs
free energy as:
\begin{equation}
G=F - V \sum_{\alpha,\beta} \sigma_{\alpha,\beta} \epsilon_{\alpha,\beta},
\end{equation}
and to write:
\begin{equation}
dG=SdT - V \sum_{\alpha,\beta} \epsilon_{\alpha,\beta} d \sigma_{\alpha,\beta}
-V \sum_{\alpha} {\bf D}_\alpha d {\bf E}_\alpha
-V \sum_{\alpha} {\bf B}_\alpha d {\bf H}_\alpha
\end{equation}
We have therefore:
\begin{align}
S&= -{\partial G \over \partial T}, \\
\epsilon_{\alpha,\beta}&= -{1\over V}
{\partial G \over \partial \sigma_{\alpha,\beta}}, \\
{\bf D}_{\alpha}&= -{1\over V}
{\partial G \over \partial {\bf E}_{\alpha}}, \\
{\bf B}_{\alpha}&= -{1\over V}
{\partial G \over \partial {\bf H}_{\alpha}}.
\end{align}
Assuming that these quantities are functions of the independent variables
$T$, $\sigma_{\alpha,\beta}$, ${\bf E}_\alpha$, and ${\bf H}_\alpha$ we
obtain the temperature dependent macroscopic equations:
\begin{align}
dS&= {\partial S \over \partial T} dT+\sum_{\alpha,\beta}
{\partial S \over \partial \sigma_{\alpha,\beta}}d\sigma_{\alpha,\beta}
+\sum_\gamma{\partial S \over \partial {\bf E}_{\gamma}}d{\bf E}_{\gamma}+ 
\sum_\delta{\partial S \over \partial {\bf H}_{\delta}}d{\bf H}_{\delta}, \\
d\epsilon_{\alpha,\beta}&=
{\partial \epsilon_{\alpha,\beta} \over \partial T} dT+\sum_{\gamma,\delta}
{\partial \epsilon_{\alpha,\beta} \over \partial 
\sigma_{\gamma,\delta}}d\sigma_{\gamma,\delta}
+\sum_\gamma{\partial \epsilon_{\alpha,\beta} \over \partial 
{\bf E}_{\gamma}}d{\bf E}_{\gamma}+ 
\sum_\delta{\partial \epsilon_{\alpha,\beta} \over \partial 
{\bf H}_{\delta}}d{\bf H}_{\delta}, \\
d{\bf D}_\gamma&=
{\partial  {\bf D}_\gamma \over \partial T} dT+\sum_{\alpha,\beta}
{\partial {\bf D}_\gamma \over \partial \sigma_{\alpha,\beta}}
d\sigma_{\alpha,\beta}
+\sum_\alpha{\partial {\bf D}_\gamma \over \partial {\bf E}_{\alpha}}d{\bf E}_{\alpha}+ 
\sum_\delta{\partial {\bf D}_\gamma \over \partial {\bf H}_{\delta}}d{\bf H}_{\delta}, \\
d{\bf B}_\delta&=
{\partial {\bf B}_\delta \over \partial T} dT+\sum_{\alpha,\beta}
{\partial {\bf B}_\delta \over \partial \sigma_{\alpha,\beta}}
d\sigma_{\alpha,\beta}
+\sum_\gamma{\partial {\bf B}_\delta \over \partial {\bf E}_{\gamma}}
d{\bf E}_{\gamma}+ 
\sum_\alpha{\partial {\bf B}_\delta \over \partial {\bf H}_{\alpha}}
d{\bf H}_{\alpha}. 
\end{align}

This expansion introduces other material properties not yet defined:
\subsection{\ \ Heat capacity at constant stress}
The heat capacity at constant stress is defined as
\begin{equation}
C_\sigma= - T {\partial^2 G \over \partial T^2} = 
T {\partial S \over \partial T},
\end{equation}
and has the same unit as $S$: ${\rm J}\over {\rm K}$. Usually one considers the heat
capacity of a mole of substance and gives $C_\sigma$ in 
${{\rm J} \over {\rm K} \cdot {\rm mol}}$.

\subsection{\ \ Thermal expansion and piezocaloric effect}
The thermal expansion coefficient or the piezocaloric effects are
described by the tensor:
\begin{equation}
\gamma_{\alpha,\beta}=-{1\over V} {\partial^2 G \over \partial T 
\partial \sigma_{\alpha,\beta}}={\partial \epsilon_{\alpha,\beta} \over 
\partial T}= {1\over V}
{\partial S \over \partial \sigma_{\alpha,\beta}},
\end{equation}
its unit is $1/{\rm K}$.

\subsection{\ \ Direct and converse piezo-electric effects}
The direct and converse piezo-electric effects are described by the 
tensor:
\begin{equation}
d_{\alpha,\beta,\gamma}=-{1\over V} {\partial^2 G \over
\partial \sigma_{\alpha,\beta} \partial {\bf E}_\gamma}=
{\partial \epsilon_{\alpha,\beta} \over \partial {\bf E}_\gamma}= 
{\partial {\bf D}_\gamma \over \partial \sigma_{\alpha,\beta}},
\end{equation}
its unit is ${\rm C}/{\rm N}$. Note also that 
${\partial {\bf D}_\gamma \over \partial \sigma_{\alpha,\beta}}
={\partial {\bf P}_\gamma \over \partial \sigma_{\alpha,\beta}}$.

\subsection{\ \ Direct and converse piezo-magnetic effects}
The direct and converse piezo-magnetic effects are described by the 
tensor:
\begin{equation}
\hat h_{\alpha,\beta,\gamma}=-{1\over \mu_0 V} {\partial^2 G \over
\partial \sigma_{\alpha,\beta} \partial {\bf H}_\gamma}= {1\over \mu_0}
{\partial \epsilon_{\alpha,\beta} \over \partial {\bf H}_\gamma}= 
{1 \over \mu_0}{\partial {\bf B}_\gamma \over \partial \sigma_{\alpha,\beta}},
\end{equation}
its unit is ${{\rm A}\cdot {\rm m} \over {\rm N}} = {1\over T}$. Note also that 
${\partial {\bf B}_\gamma \over \partial \sigma_{\alpha,\beta}}
=\mu_0 {\partial {\bf M}_\gamma \over \partial \sigma_{\alpha,\beta}}$.
The piezomagnetic tensor 
($q_{\alpha,\beta,\gamma}$) is sometimes defined
from the relationship between magnetization intensity ${\bf I}$
and stress. Since ${\bf I}=\mu_0 {\bf M}$ we have
$q_{\alpha,\beta,\gamma}=\mu_0 \hat h_{\alpha,\beta,\gamma}$. The
unit of $q_{\alpha,\beta,\gamma}$ is ${{\rm N}\over {\rm A}^2}{{\rm A} \cdot {\rm m}\over {\rm N}}
={{\rm m}\over {\rm A}}={{\rm Wb}\over {\rm N}}$. 


\subsection{\ \ Macroscopic equations at finite temperature}
The other derivatives are just finite temperature generalization of the
previously discussed quantities. We can rewrite the macroscopic equations as:
\begin{align}
dS&= {C_\sigma\over T} dT+ V \sum_{\alpha,\beta} \hat \gamma_{\alpha,\beta} 
d\sigma_{\alpha,\beta}
+ V \sum_\gamma \hat {\bf p}_\gamma d{\bf E}_{\gamma} + 
V \sum_\delta \hat {\bf q}_\delta d{\bf H}_{\delta}, \\
d\epsilon_{\alpha,\beta}&= \hat \gamma_{\alpha,\beta} dT + 
\sum_{\gamma,\delta} \hat C^{-1 T}_{\alpha,\beta,\gamma,\delta} 
d \sigma_{\gamma,\delta} + \sum_\gamma \hat d^T_{\alpha,\beta,\gamma} 
d{\bf E}_{\gamma} +\mu_0 
\sum_\delta \hat h^T_{\alpha,\beta,\delta} d{\bf H}_{\delta}, \\
d{\bf D}_\gamma&=
\hat {\bf p}_\gamma dT + \sum_{\alpha,\beta} \hat 
d^T_{\alpha,\beta,\gamma} d\sigma_{\alpha,\beta}
+\epsilon_0 \sum_\alpha \hat \epsilon^T_{r,\gamma,\alpha} d{\bf E}_{\alpha} + 
\sum_\delta \hat \alpha^T_{\gamma,\delta} d{\bf H}_{\delta}, \\
d{\bf B}_\delta&=
\hat {\bf q}_\delta dT+ \mu_0 \sum_{\alpha,\beta}
\hat h^T_{\alpha,\beta,\gamma}
d\sigma_{\alpha,\beta}
+\sum_\gamma \hat \alpha^T_{\gamma,\delta}
d{\bf E}_{\gamma}+ \mu_0
\sum_\alpha \hat \mu^T_{r,\delta,\alpha}
d{\bf H}_{\alpha}, 
\end{align}
where the index $T$ indicates isothermal quantities and the hat $\hat{}$
indicates that the independent variables are $T$, $\sigma_{\alpha,\beta}$, 
${\bf E}_\alpha$, and ${\bf H}_\alpha$. 

\newpage
{\color{web-blue}
While in calculations it is easy to control the strain applied to
a given crystal, experimentally it is simpler to apply a given stress
on a solid. It is therefore common for instance to define the 
piezoelectric (piezomagnetic) tensor from the polarization 
(magnetization) induced by a stress. In order to define it one 
can introduce a free-energy depending on $\sigma_{\alpha,\beta}$,
${\bf E}_\alpha$, and ${\bf H}_\alpha$. Alternatively one can use the
equations of motion to write the strain as a function of stress and
to insert its espression in the equation for the polarization
(magnetization). We obtain:
\begin{align}
\epsilon_{\gamma,\delta} &=
- \sum_{\alpha,\beta} 
C^{-1}_{\gamma,\delta,\alpha,\beta} \sigma^{(0)}_{\alpha,\beta} 
+ \sum_{\alpha,\beta} 
C^{-1}_{\gamma,\delta,\alpha,\beta} \sigma_{\alpha,\beta} 
+ \sum_{\alpha,\beta,\mu} 
C^{-1}_{\gamma,\delta,\alpha,\beta} \tilde e_{\alpha,\beta,\mu} {\bf E}_\mu 
\nonumber \\
&+ \alpha^2 \sum_{\alpha,\beta,\mu} 
C^{-1}_{\gamma,\delta,\alpha,\beta} \tilde h_{\alpha,\beta,\mu} {\bf H}_\mu 
\\
{\bf P}_{\lambda}&= \hat {\bf P}^{(0)}_{\lambda}
+\sum_{\alpha,\beta} \hat d_{\alpha,\beta,\lambda} 
\sigma_{\alpha,\beta} +
\sum_{\mu} 
\hat \chi^e_{\lambda,\mu}
{\bf E}_{\mu} +
{1\over 4 \pi} \sum_{\mu} \hat \alpha_{\lambda,\mu} 
{\bf H}_{\mu}, \\
{\bf M}_{\lambda}&= \hat {\bf M}^{(0)}_{\lambda}
+\sum_{\alpha,\beta} \hat h_{\alpha,\beta,\lambda} 
\sigma_{\alpha,\beta} +
{1\over 4 \pi \alpha^2} \sum_{\mu} \hat \alpha_{\lambda,\mu} 
{\bf E}_{\mu}+
\sum_{\mu} 
\hat \chi^m_{\lambda,\mu}
{\bf H}_{\mu}
\end{align} 
where
\begin{align}
\hat {\bf P}^{(0)}_{\lambda}&= \tilde {\bf P}^{(0)}_{\lambda}
-\sum_{\alpha,\beta,\gamma,\delta} \tilde e_{\alpha,\beta,\lambda}
\tilde C^{-1}_{\alpha,\beta,\gamma,\delta} 
\tilde \sigma^{(0)}_{\gamma,\delta}, \\
\hat {\bf M}^{(0)}_{\lambda}&= \tilde {\bf M}^{(0)}_{\lambda}
-\sum_{\alpha,\beta,\gamma,\delta} \tilde h_{\alpha,\beta,\lambda}
\tilde C^{-1}_{\alpha,\beta,\gamma,\delta} 
\tilde \sigma^{(0)}_{\gamma,\delta}, \\
\hat d_{\alpha,\beta,\lambda}&= \sum_{\gamma,\delta} 
\tilde e_{\gamma,\delta,\lambda} 
\tilde C^{-1}_{\gamma,\delta,\alpha,\beta}, \\
\hat h_{\alpha,\beta,\lambda}&= \sum_{\gamma,\delta} 
\tilde h_{\gamma,\delta,\lambda} 
\tilde C^{-1}_{\gamma,\delta,\alpha,\beta}, \\
\hat \chi^e_{\lambda,\mu}&= \tilde \chi^e_{\lambda,\mu}
+ 
\sum_{\alpha,\beta,\gamma,\delta}\tilde e_{\gamma,\delta,\lambda}
\tilde C^{-1}_{\gamma,\delta,\alpha,\beta} \tilde e_{\alpha,\beta,\mu}, \\
\hat \chi^m_{\lambda,\mu}&= \tilde \chi^m_{\lambda,\mu}
+\alpha^2 \sum_{\alpha,\beta,\gamma,\delta}\tilde h_{\gamma,\delta,\lambda}
\tilde C^{-1}_{\gamma,\delta,\alpha,\beta} \tilde h_{\alpha,\beta,\mu}, \\
\hat \alpha_{\lambda,\mu}&= \tilde \alpha_{\lambda,\mu}
+4 \pi \alpha^2 
\sum_{\alpha,\beta,\gamma,\delta}\tilde e_{\gamma,\delta,\lambda}
\tilde C^{-1}_{\gamma,\delta,\alpha,\beta} \tilde h_{\alpha,\beta,\mu},
\end{align}
where $\hat \chi^e_{\lambda,\mu}$, $\hat \chi^m_{\lambda,\mu}$ and
$\hat \alpha_{\lambda,\mu}$ are the dielectric susceptibility, the magnetic
susceptibility, 
and the magneto-electric tensor at constant stress, while 
$\hat d_{\alpha,\beta,\lambda}$ ($\hat h_{\alpha,\beta,\lambda}$)
is the piezoelectric (piezomagnetic) tensor that
gives the polarization (magnetization) induced by a stress. 

\subsection{\color{web-blue}\ \ Thermodynamic equations}
Similarly for the thermodynamic equations we can introduce the Gibbs
free energy as:
\begin{equation}
G=F - V \sum_{\alpha,\beta} \sigma_{\alpha,\beta} \epsilon_{\alpha,\beta},
\end{equation}
and to write:
\begin{equation}
dG=SdT - V \sum_{\alpha,\beta} \epsilon_{\alpha,\beta} d \sigma_{\alpha,\beta}
-{V \over 4 \pi} \sum_{\alpha} {\bf D}_\alpha d {\bf E}_\alpha
-{V \over 4 \pi} \sum_{\alpha} {\bf B}_\alpha d {\bf H}_\alpha
\end{equation}
We have therefore:
\begin{align}
S&=-{\partial G \over \partial T}, \\
\epsilon_{\alpha,\beta}&= -{1\over V}
{\partial G \over \partial \sigma_{\alpha,\beta}}, \\
{\bf D}_{\alpha}&= -{4 \pi \over V}
{\partial G \over \partial {\bf E}_{\alpha}}, \\
{\bf B}_{\alpha}&= -{4 \pi \over V}
{\partial G \over \partial {\bf H}_{\alpha}}.
\end{align}
Assuming that these quantities are functions of the independent variables
$T$, $\sigma_{\alpha,\beta}$, ${\bf E}_\alpha$, and ${\bf H}_\alpha$ we
obtain the temperature dependent macroscopic equations:
\begin{align}
dS&= {\partial S \over \partial T} dT+\sum_{\alpha,\beta}
{\partial S \over \partial \sigma_{\alpha,\beta}}d\sigma_{\alpha,\beta}
+\sum_\gamma{\partial S \over \partial {\bf E}_{\gamma}}d{\bf E}_{\gamma}+ 
\sum_\delta{\partial S \over \partial {\bf H}_{\delta}}d{\bf H}_{\delta}, \\
d\epsilon_{\alpha,\beta}&= 
{\partial \epsilon_{\alpha,\beta} \over \partial T} dT+\sum_{\gamma,\delta}
{\partial \epsilon_{\alpha,\beta} \over \partial 
\sigma_{\gamma,\delta}}d\sigma_{\gamma,\delta}
+\sum_\gamma{\partial \epsilon_{\alpha,\beta} \over \partial 
{\bf E}_{\gamma}}d{\bf E}_{\gamma}+ 
\sum_\delta{\partial \epsilon_{\alpha,\beta} \over \partial 
{\bf H}_{\delta}}d{\bf H}_{\delta}, \\
d{\bf D}_\gamma&= 
{\partial  {\bf D}_\gamma \over \partial T} dT+\sum_{\alpha,\beta}
{\partial {\bf D}_\gamma \over \partial \sigma_{\alpha,\beta}}
d\sigma_{\alpha,\beta}
+\sum_\alpha{\partial {\bf D}_\gamma \over \partial {\bf E}_{\alpha}}d{\bf E}_{\alpha}+ 
\sum_\delta{\partial {\bf D}_\gamma \over \partial {\bf H}_{\delta}}d{\bf H}_{\delta}, \\
d{\bf B}_\delta&= 
{\partial {\bf B}_\delta \over \partial T} dT+\sum_{\alpha,\beta}
{\partial {\bf B}_\delta \over \partial \sigma_{\alpha,\beta}}
d\sigma_{\alpha,\beta}
+\sum_\gamma{\partial {\bf B}_\delta \over \partial {\bf E}_{\gamma}}
d{\bf E}_{\gamma}+ 
\sum_\alpha{\partial {\bf B}_\delta \over \partial {\bf H}_{\alpha}}
d{\bf H}_{\alpha}. 
\end{align}

This expansion introduces other material properties not yet defined:
\subsection{\color{web-blue}\ \ Heat capacity at constant stress}
The heat capacity at constant stress is defined as
\begin{equation}
C_\sigma= - T {\partial^2 G \over \partial T^2} = 
T {\partial S \over \partial T},
\end{equation}
and has the same unit as $S$: ${\rm \bar U} \over {\rm K}$. Usually one considers the heat
capacity of a mole of substance and gives $C_\sigma$ in 
${{\rm \bar U} \over {\rm K} \cdot {\rm mol}}$.

\subsection{\color{web-blue}\ \ Thermal expansion and piezocaloric effect}
The thermal expansion coefficient or the piezocaloric effects are
described by the tensor:
\begin{equation}
\gamma_{\alpha,\beta}=-{1\over V} {\partial^2 G \over \partial T 
\partial \sigma_{\alpha,\beta}}={\partial \epsilon_{\alpha,\beta} \over 
\partial T}= {1\over V}
{\partial S \over \partial \sigma_{\alpha,\beta}},
\end{equation}
its unit is $1/{\rm K}$.

\subsection{\color{web-blue}\ \ Direct and converse piezo-electric effects}
The direct and converse piezo-electric effects are described by the 
tensor:
\begin{equation}
d_{\alpha,\beta,\gamma}=-{1\over V} {\partial^2 G \over
\partial \sigma_{\alpha,\beta} \partial {\bf E}_\gamma}=
{\partial \epsilon_{\alpha,\beta} \over \partial {\bf E}_\gamma}= 
{1\over 4 \pi} {\partial {\bf D}_\gamma \over \partial \sigma_{\alpha,\beta}},
\end{equation}
its unit is $e a_B/E_h$. Note also that 
${\partial {\bf D}_\gamma \over \partial \sigma_{\alpha,\beta}}
=4 \pi {\partial {\bf P}_\gamma \over \partial \sigma_{\alpha,\beta}}$.

\subsection{\color{web-blue}\ \ Direct and converse piezo-magnetic effects}
The direct and converse piezo-magnetic effects are described by the 
tensor:
\begin{equation}
\hat h_{\alpha,\beta,\gamma}=-{1\over V} {\partial^2 G \over
\partial \sigma_{\alpha,\beta} \partial {\bf H}_\gamma}= 
{\partial \epsilon_{\alpha,\beta} \over \partial {\bf H}_\gamma}= 
{1 \over 4 \pi}{\partial {\bf B}_\gamma \over \partial \sigma_{\alpha,\beta}},
\end{equation}
its unit is ${1 \over \alpha^2 \bar B}$. Note also that 
${\partial {\bf B}_\gamma \over \partial \sigma_{\alpha,\beta}}
=4\pi \alpha^2 {\partial {\bf M}_\gamma \over \partial \sigma_{\alpha,\beta}}$.

\subsection{\color{web-blue}\ \ Macroscopic equations at finite temperature}

The other derivatives are just finite temperature generalization of the
previously discussed quantities. We can rewrite the macroscopic equations as:
\begin{align}
dS&= {C_\sigma\over T} dT+ V \sum_{\alpha,\beta} \hat \gamma_{\alpha,\beta} 
d\sigma_{\alpha,\beta}
+ V \sum_\gamma \hat {\bf p}_\gamma d{\bf E}_{\gamma} + 
V \sum_\delta \hat {\bf q}_\delta d{\bf H}_{\delta}, \\
d\epsilon_{\alpha,\beta}&= \hat \gamma_{\alpha,\beta} dT + 
\sum_{\gamma,\delta} \hat C^{-1 T}_{\alpha,\beta,\gamma,\delta} 
d \sigma_{\gamma,\delta} + \sum_\gamma \hat d^T_{\alpha,\beta,\gamma} 
d{\bf E}_{\gamma} +
\sum_\delta \hat h^T_{\alpha,\beta,\delta} d{\bf H}_{\delta}, \\
d{\bf D}_\gamma&= 
4 \pi \hat {\bf p}_\gamma dT + 4 \pi \sum_{\alpha,\beta} \hat 
d^T_{\alpha,\beta,\gamma} d\sigma_{\alpha,\beta}
+ \sum_\alpha \hat \epsilon^T_{r,\gamma,\alpha} d{\bf E}_{\alpha} + 
\sum_\delta \hat \alpha^T_{\gamma,\delta} d{\bf H}_{\delta}, \\
d{\bf B}_\delta&= 
4 \pi \hat {\bf q}_\delta dT+ 4 \pi \sum_{\alpha,\beta}
\hat h^T_{\alpha,\beta,\gamma}
d\sigma_{\alpha,\beta}
+\sum_\gamma \hat \alpha^T_{\gamma,\delta}
d{\bf E}_{\gamma}+ \alpha^2
\sum_\alpha \hat \mu^T_{r,\delta,\alpha}
d{\bf H}_{\alpha}, 
\end{align}
where the index $T$ indicates isothermal quantities and the hat $\hat{}$
indicates that the independent variables are $T$, $\sigma_{\alpha,\beta}$, 
${\bf E}_\alpha$, and ${\bf H}_\alpha$.  
}

\newpage
{\color{orange}
While in calculations it is easy to control the strain applied to
a given crystal, experimentally it is simpler to apply a given stress
on a solid. It is therefore common for instance to define the 
piezoelectric (piezomagnetic) tensor from the polarization 
(magnetization) induced by a stress. In order to define it one 
can introduce a free-energy depending on $\sigma_{\alpha,\beta}$,
${\bf E}_\alpha$, and ${\bf H}_\alpha$. Alternatively one can use the
equations of motion to write the strain as a function of stress and
to insert its espression in the equation for the polarization
(magnetization). We obtain:
\begin{align}
\epsilon_{\gamma,\delta} &=
- \sum_{\alpha,\beta} 
C^{-1}_{\gamma,\delta,\alpha,\beta} \sigma^{(0)}_{\alpha,\beta} 
+ \sum_{\alpha,\beta} 
C^{-1}_{\gamma,\delta,\alpha,\beta} \sigma_{\alpha,\beta} 
+ \sum_{\alpha,\beta,\mu} 
C^{-1}_{\gamma,\delta,\alpha,\beta} \tilde e_{\alpha,\beta,\mu} {\bf E}_\mu 
\nonumber \\
&+ \sum_{\alpha,\beta,\mu} 
C^{-1}_{\gamma,\delta,\alpha,\beta} \tilde h_{\alpha,\beta,\mu} {\bf H}_\mu 
\\
{\bf P}_{\lambda}&= \hat {\bf P}^{(0)}_{\lambda}
+\sum_{\alpha,\beta} \hat d_{\alpha,\beta,\lambda} 
\sigma_{\alpha,\beta} +
\sum_{\mu} 
\hat \chi^e_{\lambda,\mu}
{\bf E}_{\mu} +
{1\over 4 \pi}\sum_{\mu} \hat \alpha_{\lambda,\mu} 
{\bf H}_{\mu}, \\
{\bf M}_{\lambda}&= \hat {\bf M}^{(0)}_{\lambda}
+\sum_{\alpha,\beta} \hat h_{\alpha,\beta,\lambda} 
\sigma_{\alpha,\beta} +
{1\over 4 \pi} \sum_{\mu} \hat \alpha_{\lambda,\mu} 
{\bf E}_{\mu}+
\sum_{\mu} 
\hat \chi^m_{\lambda,\mu}
{\bf H}_{\mu}
\end{align} 
where
\begin{align}
\hat {\bf P}^{(0)}_{\lambda}&= \tilde {\bf P}^{(0)}_{\lambda}
-\sum_{\alpha,\beta,\gamma,\delta} \tilde e_{\alpha,\beta,\lambda}
\tilde C^{-1}_{\alpha,\beta,\gamma,\delta} 
\tilde \sigma^{(0)}_{\gamma,\delta}, \\
\hat {\bf M}^{(0)}_{\lambda}&= \tilde {\bf M}^{(0)}_{\lambda}
-\sum_{\alpha,\beta,\gamma,\delta} \tilde h_{\alpha,\beta,\lambda}
\tilde C^{-1}_{\alpha,\beta,\gamma,\delta} 
\tilde \sigma^{(0)}_{\gamma,\delta}, \\
\hat d_{\alpha,\beta,\lambda}&= \sum_{\gamma,\delta} 
\tilde e_{\gamma,\delta,\lambda} 
\tilde C^{-1}_{\gamma,\delta,\alpha,\beta}, \\
\hat h_{\alpha,\beta,\lambda}&= \sum_{\gamma,\delta} 
\tilde h_{\gamma,\delta,\lambda} 
\tilde C^{-1}_{\gamma,\delta,\alpha,\beta}, \\
\hat \chi^e_{\lambda,\mu}&= \tilde \chi^e_{\lambda,\mu}
+
\sum_{\alpha,\beta,\gamma,\delta}\tilde e_{\gamma,\delta,\lambda}
\tilde C^{-1}_{\gamma,\delta,\alpha,\beta} \tilde e_{\alpha,\beta,\mu}, \\
\hat \chi^m_{\lambda,\mu}&= \tilde \chi^m_{\lambda,\mu}
+\sum_{\alpha,\beta,\gamma,\delta}\tilde h_{\gamma,\delta,\lambda}
\tilde C^{-1}_{\gamma,\delta,\alpha,\beta} \tilde h_{\alpha,\beta,\mu}, \\
\hat \alpha_{\lambda,\mu}&= \tilde \alpha_{\lambda,\mu}
+4 \pi \sum_{\alpha,\beta,\gamma,\delta}\tilde e_{\gamma,\delta,\lambda}
\tilde C^{-1}_{\gamma,\delta,\alpha,\beta} \tilde h_{\alpha,\beta,\mu},
\end{align}
where $\hat \chi^e_{\lambda,\mu}$, $\hat \chi^m_{\lambda,\mu}$ and
$\hat \alpha_{\lambda,\mu}$ are the dielectric susceptibility, the magnetic
susceptibility, 
and the magneto-electric tensor at constant stress, while 
$\hat d_{\alpha,\beta,\lambda}$ ($\hat h_{\alpha,\beta,\lambda}$)
is the piezoelectric (piezomagnetic) tensor that
gives the polarization (magnetization) induced by a stress. 

\subsection{\color{orange}\ \ Thermodynamic equations}
Similarly for the thermodynamic equations we can introduce the Gibbs
free energy as:
\begin{equation}
G=F - V \sum_{\alpha,\beta} \sigma_{\alpha,\beta} \epsilon_{\alpha,\beta},
\end{equation}
and to write:
\begin{equation}
dG=SdT - V \sum_{\alpha,\beta} \epsilon_{\alpha,\beta} d \sigma_{\alpha,\beta}
-{V \over 4 \pi} \sum_{\alpha} {\bf D}_\alpha d {\bf E}_\alpha
-{V \over 4 \pi} \sum_{\alpha} {\bf B}_\alpha d {\bf H}_\alpha
\end{equation}
We have therefore:
\begin{align}
S&= -{\partial G \over \partial T}, \\
\epsilon_{\alpha,\beta}&= -{1\over V}
{\partial G \over \partial \sigma_{\alpha,\beta}}, \\
{\bf D}_{\alpha}&= -{4 \pi \over V}
{\partial G \over \partial {\bf E}_{\alpha}}, \\
{\bf B}_{\alpha}&= -{4 \pi \over V}
{\partial G \over \partial {\bf H}_{\alpha}}.
\end{align}
Assuming that these quantities are functions of the independent variables
$T$, $\sigma_{\alpha,\beta}$, ${\bf E}_\alpha$, and ${\bf H}_\alpha$ we
obtain the temperature dependent macroscopic equations:
\begin{align}
dS&= {\partial S \over \partial T} dT+\sum_{\alpha,\beta}
{\partial S \over \partial \sigma_{\alpha,\beta}}d\sigma_{\alpha,\beta}
+\sum_\gamma{\partial S \over \partial {\bf E}_{\gamma}}d{\bf E}_{\gamma}+ 
\sum_\delta{\partial S \over \partial {\bf H}_{\delta}}d{\bf H}_{\delta}, \\
d\epsilon_{\alpha,\beta}&=
{\partial \epsilon_{\alpha,\beta} \over \partial T} dT+\sum_{\gamma,\delta}
{\partial \epsilon_{\alpha,\beta} \over \partial 
\sigma_{\gamma,\delta}}d\sigma_{\gamma,\delta}
+\sum_\gamma{\partial \epsilon_{\alpha,\beta} \over \partial 
{\bf E}_{\gamma}}d{\bf E}_{\gamma}+ 
\sum_\delta{\partial \epsilon_{\alpha,\beta} \over \partial 
{\bf H}_{\delta}}d{\bf H}_{\delta}, \\
d{\bf D}_\gamma&= 
{\partial  {\bf D}_\gamma \over \partial T} dT+\sum_{\alpha,\beta}
{\partial {\bf D}_\gamma \over \partial \sigma_{\alpha,\beta}}
d\sigma_{\alpha,\beta}
+\sum_\alpha{\partial {\bf D}_\gamma \over \partial {\bf E}_{\alpha}}d{\bf E}_{\alpha}+ 
\sum_\delta{\partial {\bf D}_\gamma \over \partial {\bf H}_{\delta}}d{\bf H}_{\delta}, \\
d{\bf B}_\delta&= 
{\partial {\bf B}_\delta \over \partial T} dT+\sum_{\alpha,\beta}
{\partial {\bf B}_\delta \over \partial \sigma_{\alpha,\beta}}
d\sigma_{\alpha,\beta}
+\sum_\gamma{\partial {\bf B}_\delta \over \partial {\bf E}_{\gamma}}
d{\bf E}_{\gamma}+ 
\sum_\alpha{\partial {\bf B}_\delta \over \partial {\bf H}_{\alpha}}
d{\bf H}_{\alpha}. 
\end{align}

This expansion introduces other material properties not yet defined:
\subsection{\color{orange}\ \ Heat capacity at constant stress}
The heat capacity at constant stress is defined as
\begin{equation}
C_\sigma= - T {\partial^2 G \over \partial T^2} = 
T {\partial S \over \partial T},
\end{equation}
and has the same unit as $S$: ${\rm erg}\over {\rm K}$. Usually one considers the heat
capacity of a mole of substance and gives $C_\sigma$ in 
${{\rm erg} \over {\rm K} \cdot {\rm mol}}$.

\subsection{\color{orange}\ \ Thermal expansion and piezocaloric effect}
The thermal expansion coefficient or the piezocaloric effects are
described by the tensor:
\begin{equation}
\gamma_{\alpha,\beta}=-{1\over V} {\partial^2 G \over \partial T 
\partial \sigma_{\alpha,\beta}}={\partial \epsilon_{\alpha,\beta} \over 
\partial T}= {1\over V}
{\partial S \over \partial \sigma_{\alpha,\beta}},
\end{equation}
its unit is $1/{\rm K}$.

\subsection{\color{orange}\ \ Direct and converse piezo-electric effects}
The direct and converse piezo-electric effects are described by the 
tensor:
\begin{equation}
d_{\alpha,\beta,\gamma}=-{1\over V} {\partial^2 G \over
\partial \sigma_{\alpha,\beta} \partial {\bf E}_\gamma}=
{\partial \epsilon_{\alpha,\beta} \over \partial {\bf E}_\gamma}= 
{1\over 4\pi}{\partial {\bf D}_\gamma \over \partial \sigma_{\alpha,\beta}},
\end{equation}
its unit is ${\rm statC}/{\rm dyne}$. Note also that 
${\partial {\bf D}_\gamma \over \partial \sigma_{\alpha,\beta}}
=4\pi {\partial {\bf P}_\gamma \over \partial \sigma_{\alpha,\beta}}$.

\subsection{\color{orange}\ \ Direct and converse piezo-magnetic effects}
The direct and converse piezo-magnetic effects are described by the 
tensor:
\begin{equation}
\hat h_{\alpha,\beta,\gamma}=-{1\over V} {\partial^2 G \over
\partial \sigma_{\alpha,\beta} \partial {\bf H}_\gamma}= 
{\partial \epsilon_{\alpha,\beta} \over \partial {\bf H}_\gamma}= 
{1 \over 4 \pi}{\partial {\bf B}_\gamma \over \partial \sigma_{\alpha,\beta}},
\end{equation}
its unit is ${1\over {\rm G}}$. Note also that 
${\partial {\bf B}_\gamma \over \partial \sigma_{\alpha,\beta}}
=4 \pi {\partial {\bf M}_\gamma \over \partial \sigma_{\alpha,\beta}}$.

\subsection{\color{orange}\ \ Macroscopic equations at finite temperature}
The other derivatives are just finite temperature generalization of the
previously discussed quantities. We can rewrite the macroscopic equations as:
\begin{align}
dS&= {C_\sigma\over T} dT+ V \sum_{\alpha,\beta} \hat \gamma_{\alpha,\beta} 
d\sigma_{\alpha,\beta}
+ V \sum_\gamma \hat {\bf p}_\gamma d{\bf E}_{\gamma} + 
V \sum_\delta \hat {\bf q}_\delta d{\bf H}_{\delta}, \\
d\epsilon_{\alpha,\beta}&=  \hat \gamma_{\alpha,\beta} dT + 
\sum_{\gamma,\delta} \hat C^{-1 T}_{\alpha,\beta,\gamma,\delta} 
d \sigma_{\gamma,\delta} + \sum_\gamma \hat d^T_{\alpha,\beta,\gamma} 
d{\bf E}_{\gamma} +
\sum_\delta \hat h^T_{\alpha,\beta,\delta} d{\bf H}_{\delta}, \\
d{\bf D}_\gamma&= 
4 \pi \hat {\bf p}_\gamma dT + 4 \pi \sum_{\alpha,\beta} \hat 
d^T_{\alpha,\beta,\gamma} d\sigma_{\alpha,\beta} +
\sum_\alpha \hat \epsilon^T_{r,\gamma,\alpha} d{\bf E}_{\alpha} + 
\sum_\delta \hat \alpha^T_{\gamma,\delta} d{\bf H}_{\delta}, \\
d{\bf B}_\delta&= 
4 \pi \hat {\bf q}_\delta dT + 4 \pi \sum_{\alpha,\beta}
\hat h^T_{\alpha,\beta,\gamma}
d\sigma_{\alpha,\beta}
+\sum_\gamma \hat \alpha^T_{\delta,\gamma}
d{\bf E}_{\gamma}+ 
\sum_\alpha \hat \mu^T_{r,\delta,\alpha}
d{\bf H}_{\alpha}, 
\end{align}
where the index $T$ indicates isothermal quantities and the hat $\hat{}$
indicates that the independent variables are $T$, $\sigma_{\alpha,\beta}$, 
${\bf E}_\alpha$, and ${\bf H}_\alpha$.  
}

\newpage
{\color{dark-blue}\chapter{{\bf B} as an independent variable}}
\color{black}

In density functional theory the electric field ${\bf E}$ and the
magnetic field ${\bf B}$ are the two quantities that enter in the Kohn
and Sham equations. In this case the total energy functional is 
actually the free enthalpy that depends on $\epsilon$, ${\bf E}$, and 
${\bf B}$. 
Since in the equations of motion we have ${\bf M}$ and not ${\bf B}$ 
to proceed as in the previous case we use the general equation
\begin{equation}
{\bf B}_\alpha= \mu_0({\bf M}_\alpha+{\bf H}_\alpha)
\end{equation}
and obtain ${\bf H}_\alpha$ as a function of ${\bf B}$.
We have:
\begin{equation}
{\bf H}_\delta = - \sum_\nu \tilde \mu^{-1}_{r,\delta,\nu} {\bf M}^{(0)}_\nu
- \sum_{\alpha,\beta,\nu} \tilde \mu^{-1}_{r,\delta,\nu} \tilde h_{\alpha,\beta,\nu}
\epsilon_{\alpha,\beta} - {1\over \mu_0} \sum_{\alpha,\nu}
\tilde \mu^{-1}_{r,\delta,\nu} \tilde \alpha_{\alpha,\nu} {\bf E}_\alpha
+ {1\over \mu_0} \sum_\nu \tilde \mu^{-1}_{r,\delta,\nu} {\bf B}_\nu, 
\end{equation}
where $\tilde \mu_{r,\delta,\nu}$ is the relative magnetic
permeability.
Inserting this expression in the equations of motion we obtain:
\begin{align}
\sigma_{\lambda,\mu}&=  \check \sigma_{\lambda,\mu}^{(0)}
+\sum_{\gamma,\delta} \check C_{\lambda,\mu,\gamma,\delta}  
\epsilon_{\gamma,\delta} -
\sum_{\gamma} \check e_{\lambda,\mu,\gamma} 
 {\bf E}_{\gamma}
-\sum_{\gamma,\nu}  \tilde h_{\lambda,\mu,\gamma} 
\tilde \mu^{-1}_{r,\gamma,\nu} {\bf B}_{\nu}, \\
{\bf P}_{\lambda}&= \check {\bf P}^{(0)}_{\lambda}
+\sum_{\alpha,\beta} \check e_{\alpha,\beta,\lambda} 
\epsilon_{\alpha,\beta} +
\sum_{\beta} 
\check \chi^e_{\lambda,\beta}
{\bf E}_{\beta} +
{1\over 4 \pi \mu_0}\sum_{\delta,\nu} \tilde \alpha_{\lambda,\delta} 
\tilde \mu^{-1}_{r,\delta,\nu}  
{\bf B}_{\nu}, 
\end{align}
one finds the relationship between quantities calculated at constant 
${\bf B}$ and those calculated at constant ${\bf H}$. We have:
\begin{align}
\check C_{\lambda,\mu,\gamma,\delta} &=
\tilde C_{\lambda,\mu,\gamma,\delta} + \mu_0 \sum_{\alpha,\beta,\nu}
\tilde h_{\lambda,\mu,\alpha} \tilde \mu^{-1}_{r,\alpha,\nu} 
\tilde h_{\gamma,\delta,\nu}, \\
\check e_{\lambda,\mu,\gamma}&= \tilde e_{\lambda,\mu,\gamma}
-\sum_{\alpha,\beta} \tilde h_{\lambda,\mu,\alpha} 
\tilde \mu^{-1}_{r,\alpha,\nu}
\tilde \alpha_{\gamma,\nu}, \\
\check \chi^e_{\lambda,\mu}&= \tilde \chi^e_{\lambda,\mu}
-{1\over \epsilon_0 \mu_0} \sum_{\delta,\nu} \tilde \alpha_{\lambda,\delta} 
\tilde \mu^{-1}_{r,\delta,\nu} \tilde \alpha_{\mu,\nu}. 
\end{align}
In addition we have the relationships:
\begin{align}
\check \sigma^{(0)}_{\lambda,\mu} &= \tilde \sigma^{(0)}_{\lambda,\mu}
+ \mu_0 \sum_{\gamma,\nu} \tilde h_{\lambda,\mu,\gamma} 
\tilde \mu^{-1}_{r,\gamma,\nu} {\bf M}^{(0)}_\nu,  \\
\check {\bf P}^{(0)}_{\lambda} &= \tilde {\bf P}^{(0)}_{\lambda}
- \sum_{\beta,\nu} \tilde \alpha_{\lambda,\beta} 
\tilde \mu^{-1}_{r,\beta,\nu} {\bf M}^{(0)}_\nu. 
\end{align}
We can derive similar equations from the microscopic equations of motion.
The dependence of ${\bf H}_\delta$ on ${\bf B}$ is:
\begin{align}
{\bf H}_\delta = 
&- \sum_\nu \mu^{-1}_{r,\delta,\nu} {\bf M}^{(0)}_\nu 
-{1\over V} \sum_{s,\alpha,\nu} \mu^{-1}_{r,\delta,\nu} Z^m_{s,\alpha,\nu}
{\bf u}_{s,\alpha} \nonumber\\
&- \sum_{\alpha,\beta,\nu} \mu^{-1}_{r,\delta,\nu} h_{\alpha,\beta,\nu}
\epsilon_{\alpha,\beta} - {1\over \mu_0} \sum_{\alpha,\nu}
\mu^{-1}_{r,\delta,\nu} \alpha_{\alpha,\nu} {\bf E}_\alpha
+ {1\over \mu_0} \sum_\nu \mu^{-1}_{r,\delta,\nu} {\bf B}_\nu, 
\end{align}
where $\tilde \mu_{r,\delta,\nu}$ is the frozen-ions relative magnetic
permeability.
Inserting this equation in the other equations of motion gives:
\begin{align}
{\bf f}_{s,\alpha}&= \breve{\bf f}^{(0)}_{s,\alpha}
-\sum_{s',\beta} \breve C_{s,\alpha,s',\beta} {\bf u}_{s',\beta}
+\sum_{\beta,\gamma}
\breve \Lambda_{s,\alpha,\beta,\gamma} 
\epsilon_{\beta,\gamma} +
e \sum_{\beta} \breve Z^*_{s,\alpha,\beta} {\bf E}_{\beta}
+\sum_{\beta} 
\breve Z^m_{s,\alpha,\beta} {\bf B}_{\beta}, \\
\sigma_{\lambda,\mu}&= \breve  \sigma_{\lambda,\mu}^{(0)}
-{1\over V} \sum_{s,\alpha}
\breve \Lambda_{s,\alpha,\lambda,\mu} {\bf u}_{s,\alpha}
+\sum_{\gamma,\delta} \breve C_{\lambda,\mu,\gamma,\delta}  
\epsilon_{\gamma,\delta} -
\sum_{\gamma} \breve e_{\lambda,\mu,\gamma} 
 {\bf E}_{\gamma}
-\sum_{\gamma}  \breve h_{\lambda,\mu,\gamma} 
{\bf B}_{\gamma}, \\
{\bf P}_{\lambda}&=  \breve {\bf P}^{(0)}_{\lambda}
+{e\over V} \sum_{s,\alpha} \breve Z^*_{s,\alpha,\lambda} 
{\bf u}_{s,\alpha} +
\sum_{\alpha,\beta} \breve e_{\alpha,\beta,\lambda} 
\epsilon_{\alpha,\beta} +
\epsilon_0 \sum_{\beta} 
\breve \chi^e_{\lambda,\beta}
{\bf E}_{\beta} +
\sum_{\beta} \breve \alpha_{\lambda,\beta} 
{\bf B}_{\beta}, 
\end{align}
where
\begin{align}
\breve {\bf f}^{(0)}_{s,\alpha}&= {\bf f}^{(0)}_{s,\alpha} - \mu_0 
\sum_{\gamma,\nu}
Z^m_{s,\alpha,\gamma} \mu^{-1}_{r,\gamma,\nu} {\bf M}^{(0)}_\nu, \\
\breve \sigma^{(0)}_{\lambda,\mu}&= \sigma^{(0)}_{\lambda,\mu}
+\mu_0 \sum_{\gamma,\nu} h_{\lambda,\mu,\gamma} \mu^{-1}_{r,\gamma,\nu} 
{\bf M}^{(0)}_\nu, \\
\breve {\bf P}^{(0)}_\lambda&= {\bf P}^{(0)}_\lambda -
\sum_{\gamma,\nu} \alpha_{\lambda,\gamma} \mu^{-1}_{r,\gamma,\nu} 
{\bf M}^{(0)}_\nu, \\
\end{align}
and
\begin{align}
\breve C_{s,\alpha,s',\beta} &= C_{s,\alpha,s',\beta}+
{\mu_0\over V}  \sum_{\delta,\lambda} Z^m_{s,\alpha,\delta}
\mu^{-1}_{r,\delta,\lambda}Z^m_{s',\beta,\lambda},\\
\breve \Lambda_{s,\alpha,\beta,\gamma} &= \Lambda_{s,\alpha,\beta,\gamma}
- \mu_0 \sum_{\delta,\lambda} Z^m_{s,\alpha,\delta}
\mu^{-1}_{r,\delta,\lambda} h_{\beta,\gamma,\lambda}, \\
\breve Z^*_{s,\alpha,\beta} &= Z^*_{s,\alpha,\beta}
-{1\over e} \sum_{\delta,\lambda} Z^m_{s,\alpha,\delta} 
\mu^{-1}_{r,\delta,\lambda} \alpha_{\beta,\lambda}, \\
\breve Z^m_{s,\alpha,\beta} &= \sum_{\delta} Z^m_{s,\alpha,\delta} 
\mu^{-1}_{r,\delta,\beta}, 
\end{align}
are the microscopic tensors, while
\begin{align}
\breve C_{\lambda,\mu,\gamma,\delta} &=
C_{\lambda,\mu,\gamma,\delta} + \mu_0 \sum_{\alpha,\beta,\nu}
h_{\lambda,\mu,\alpha} \mu^{-1}_{r,\alpha,\nu} 
h_{\gamma,\delta,\nu}, \\
\breve e_{\lambda,\mu,\gamma}&= e_{\lambda,\mu,\gamma}
-\sum_{\alpha,\beta} h_{\lambda,\mu,\alpha} \mu^{-1}_{r,\alpha,\nu}
\alpha_{\gamma,\nu}, \\
\breve h_{\lambda,\mu,\gamma}&= \sum_\delta h_{\lambda,\mu,\delta}
\mu^{-1}_{r,\delta,\gamma} \\
\breve \chi^e_{\lambda,\mu}&= \chi^e_{\lambda,\mu}
-{1\over \epsilon_0 \mu_0} \sum_{\delta,\nu} \alpha_{\lambda,\delta} 
\mu^{-1}_{r,\delta,\nu} \alpha_{\mu,\nu}, \\
\breve \alpha_{\lambda,\rho} &= \sum_\beta \alpha_{\lambda,\beta}
\mu^{-1}_{r,\beta,\rho}
\end{align}
are the frozen-ions macroscopic tensors.
\\
\newpage

{\color{web-blue}
In a.u. the relationship between ${\bf M}$ and ${\bf B}$ is:
\begin{equation}
{\bf B}_\alpha= \alpha^2 ({\bf H}_\alpha+4 \pi {\bf M}_\alpha)
\end{equation}
and we have:
\begin{equation}
{\bf H}_\delta = - 4 \pi \sum_\nu \tilde \mu^{-1}_{r,\delta,\nu} 
{\bf M}^{(0)}_\nu
- 4 \pi \sum_{\alpha,\beta,\nu} \tilde \mu^{-1}_{r,\delta,\nu} 
\tilde h_{\alpha,\beta,\nu}
\epsilon_{\alpha,\beta} - {1\over \alpha^2} \sum_{\alpha,\nu}
\tilde \mu^{-1}_{r,\delta,\nu} \tilde \alpha_{\alpha,\nu} {\bf E}_\alpha
+ {1\over \alpha^2} \sum_\nu \tilde \mu^{-1}_{r,\delta,\nu} {\bf B}_\nu, 
\end{equation}
where $\tilde \mu_{r,\delta,\nu}$ is the relative magnetic
permeability.
Inserting this expression in the equations of motion we obtain:
\begin{align}
\sigma_{\lambda,\mu}&= \check \sigma_{\lambda,\mu}^{(0)}
+\sum_{\gamma,\delta} \check C_{\lambda,\mu,\gamma,\delta}  
\epsilon_{\gamma,\delta} -
\sum_{\gamma} \check e_{\lambda,\mu,\gamma} 
 {\bf E}_{\gamma}
-\sum_{\gamma,\nu}  \tilde h_{\lambda,\mu,\gamma} 
\tilde \mu^{-1}_{r,\gamma,\nu} {\bf B}_{\nu}, \\
{\bf P}_{\lambda}&= \check {\bf P}^{(0)}_{\lambda}
+\sum_{\alpha,\beta} \check e_{\alpha,\beta,\lambda} 
\epsilon_{\alpha,\beta} +
\sum_{\beta} 
\check \chi^e_{\lambda,\beta}
{\bf E}_{\beta} +
{1\over 4 \pi \mu_0}\sum_{\delta,\nu} \tilde \alpha_{\lambda,\delta} 
\tilde \mu^{-1}_{r,\delta,\nu}  
{\bf B}_{\nu}, 
\end{align}
one finds the relationship between quantities calculated at constant 
${\bf B}$ and those calculated at constant ${\bf H}$. We have:
\begin{align}
\check C_{\lambda,\mu,\gamma,\delta} &=
\tilde C_{\lambda,\mu,\gamma,\delta} + 4 \pi \alpha^2 \sum_{\alpha,\beta,\nu}
\tilde h_{\lambda,\mu,\alpha} \tilde \mu^{-1}_{r,\alpha,\nu} 
\tilde h_{\gamma,\delta,\nu}, \\
\check e_{\lambda,\mu,\gamma}&= \tilde e_{\lambda,\mu,\gamma}
- \sum_{\alpha,\beta} \tilde h_{\lambda,\mu,\alpha} 
\tilde \mu^{-1}_{r,\alpha,\nu}
\tilde \alpha_{\gamma,\nu}, \\
\check \chi^e_{\lambda,\mu}&= \tilde \chi^e_{\lambda,\mu}
-{1\over 4 \pi \alpha^2} \sum_{\delta,\nu} \tilde \alpha_{\lambda,\delta} 
\tilde \mu^{-1}_{r,\delta,\nu} \tilde \alpha_{\mu,\nu}. 
\end{align}
In addition we have the relationships:
\begin{align}
\check \sigma^{(0)}_{\lambda,\mu} &= \tilde \sigma^{(0)}_{\lambda,\mu}
+ 4 \pi \alpha^2 \sum_{\gamma,\nu} \tilde h_{\lambda,\mu,\gamma} 
\tilde \mu^{-1}_{r,\gamma,\nu} {\bf M}^{(0)}_\nu,  \\
\check {\bf P}^{(0)}_{\lambda} &= \tilde {\bf P}^{(0)}_{\lambda}
- \sum_{\beta,\nu} \tilde \alpha_{\lambda,\beta} 
\tilde \mu^{-1}_{r,\beta,\nu} {\bf M}^{(0)}_\nu. 
\end{align}
We can derive similar equations from the microscopic equations of motion.
The dependence of ${\bf H}_\delta$ on ${\bf B}$ is:
\begin{align}
{\bf H}_\delta = 
&- 4 \pi \sum_\nu \mu^{-1}_{r,\delta,\nu} {\bf M}^{(0)}_\nu 
-{4 \pi \over V} \sum_{s,\alpha,\nu} \mu^{-1}_{r,\delta,\nu} Z^m_{s,\alpha,\nu}
{\bf u}_{s,\alpha} \nonumber\\
&- 4 \pi \sum_{\alpha,\beta,\nu} \mu^{-1}_{r,\delta,\nu} h_{\alpha,\beta,\nu}
\epsilon_{\alpha,\beta} - {1\over \alpha^2} \sum_{\alpha,\nu}
\mu^{-1}_{r,\delta,\nu} \alpha_{\alpha,\nu} {\bf E}_\alpha
+ {1\over \alpha^2} \sum_\nu \mu^{-1}_{r,\delta,\nu} {\bf B}_\nu, 
\end{align}
where $\mu^{-1}_{r,\delta,\nu}$ is the frozen-ions relative magnetic
permeability.
Inserting this equation in the other equations of motion gives:
\begin{align}
{\bf f}_{s,\alpha}&= \breve{\bf f}^{(0)}_{s,\alpha}
-\sum_{s',\beta} \breve C_{s,\alpha,s',\beta} {\bf u}_{s',\beta}
+\sum_{\beta,\gamma}
\breve \Lambda_{s,\alpha,\beta,\gamma} 
\epsilon_{\beta,\gamma} +
\sum_{\beta} \breve Z^*_{s,\alpha,\beta} {\bf E}_{\beta}
+\sum_{\beta} 
\breve Z^m_{s,\alpha,\beta} {\bf B}_{\beta}, \\
\sigma_{\lambda,\mu}&= \breve  \sigma_{\lambda,\mu}^{(0)}
-{1\over V} \sum_{s,\alpha}
\breve \Lambda_{s,\alpha,\lambda,\mu} {\bf u}_{s,\alpha}
+\sum_{\gamma,\delta} \breve C_{\lambda,\mu,\gamma,\delta}  
\epsilon_{\gamma,\delta} -
\sum_{\gamma} \breve e_{\lambda,\mu,\gamma} 
 {\bf E}_{\gamma}
-\sum_{\gamma}  \breve h_{\lambda,\mu,\gamma} 
{\bf B}_{\gamma}, \\
{\bf P}_{\lambda}&= \breve {\bf P}^{(0)}_{\lambda}
+{1\over V} \sum_{s,\alpha} \breve Z^*_{s,\alpha,\lambda} 
{\bf u}_{s,\alpha} +
\sum_{\alpha,\beta} \breve e_{\alpha,\beta,\lambda} 
\epsilon_{\alpha,\beta} +
\epsilon_0 \sum_{\beta} 
\breve \chi^e_{\lambda,\beta}
{\bf E}_{\beta} +
\sum_{\beta} \breve \alpha_{\lambda,\beta} 
{\bf B}_{\beta}, 
\end{align}
where
\begin{align}
\breve {\bf f}^{(0)}_{s,\alpha}&= {\bf f}^{(0)}_{s,\alpha} - 4 \pi
\alpha^2
\sum_{\gamma,\nu}
Z^m_{s,\alpha,\gamma} \mu^{-1}_{r,\gamma,\nu} {\bf M}^{(0)}_\nu, \\
\breve \sigma^{(0)}_{\lambda,\mu}&= \sigma^{(0)}_{\lambda,\mu}
+4 \pi \alpha^2 \sum_{\gamma,\nu} h_{\lambda,\mu,\gamma} \mu^{-1}_{r,\gamma,\nu} 
{\bf M}^{(0)}_\nu, \\
\breve {\bf P}^{(0)}_\lambda&= {\bf P}^{(0)}_\lambda -
\sum_{\gamma,\nu} \alpha_{\lambda,\gamma} \mu^{-1}_{r,\gamma,\nu} 
{\bf M}^{(0)}_\nu, 
\end{align}
and
\begin{align}
\breve C_{s,\alpha,s',\beta} &= C_{s,\alpha,s',\beta}+
{4 \pi \alpha^2 \over V}  \sum_{\delta,\lambda} Z^m_{s,\alpha,\delta}
\mu^{-1}_{r,\delta,\lambda}Z^m_{s',\beta,\lambda},\\
\breve \Lambda_{s,\alpha,\beta,\gamma} &= \Lambda_{s,\alpha,\beta,\gamma}
- 4 \pi \alpha^2 \sum_{\delta,\lambda} Z^m_{s,\alpha,\delta}
\mu^{-1}_{r,\delta,\lambda} h_{\beta,\gamma,\lambda}, \\
\breve Z^*_{s,\alpha,\beta} &= Z^*_{s,\alpha,\beta}
-\sum_{\delta,\lambda} Z^m_{s,\alpha,\delta} 
\mu^{-1}_{r,\delta,\lambda} \alpha_{\beta,\lambda}, \\
\breve Z^m_{s,\alpha,\beta} &= \sum_{\delta} Z^m_{s,\alpha,\delta} 
\mu^{-1}_{r,\delta,\beta}, 
\end{align}
are the microscopic tensors, while
\begin{align}
\breve C_{\lambda,\mu,\gamma,\delta} &=
C_{\lambda,\mu,\gamma,\delta} + 4 \pi \alpha^2 \sum_{\alpha,\beta,\nu}
h_{\lambda,\mu,\alpha} \mu^{-1}_{r,\alpha,\nu} 
h_{\gamma,\delta,\nu}, \\
\breve e_{\lambda,\mu,\gamma}&= e_{\lambda,\mu,\gamma}
-\sum_{\alpha,\beta} h_{\lambda,\mu,\alpha} \mu^{-1}_{r,\alpha,\nu}
\alpha_{\gamma,\nu}, \\
\breve h_{\lambda,\mu,\gamma}&= \sum_\delta h_{\lambda,\mu,\delta}
\mu^{-1}_{r,\delta,\gamma} \\
\breve \chi^e_{\lambda,\mu}&= \chi^e_{\lambda,\mu}
-{1\over 4 \pi \alpha^2} \sum_{\delta,\nu} \alpha_{\lambda,\delta} 
\mu^{-1}_{r,\delta,\nu} \alpha_{\mu,\nu}, \\
\breve \alpha_{\lambda,\rho} &= {1\over 4 \pi \alpha^2}
\sum_\beta \alpha_{\lambda,\beta}
\mu^{-1}_{r,\beta,\rho}
\end{align}
are the frozen-ions macroscopic tensors.
}
\\
\newpage
{\color{orange}
In c.g.s.-Gaussian units the relationship between 
${\bf M}$ and ${\bf B}$ is:
\begin{equation}
{\bf B}_\alpha= {\bf H}_\alpha+ 4 \pi {\bf M}_\alpha
\end{equation}
and we have:
\begin{equation}
{\bf H}_\delta = - 4 \pi \sum_\nu \tilde \mu^{-1}_{r,\delta,\nu} 
{\bf M}^{(0)}_\nu
- 4 \pi \sum_{\alpha,\beta,\nu} \tilde \mu^{-1}_{r,\delta,\nu} 
\tilde h_{\alpha,\beta,\nu}
\epsilon_{\alpha,\beta} - \sum_{\alpha,\nu}
\tilde \mu^{-1}_{r,\delta,\nu} \tilde \alpha_{\alpha,\nu} {\bf E}_\alpha
+ \sum_\nu \tilde \mu^{-1}_{r,\delta,\nu} {\bf B}_\nu, 
\end{equation}
where $\tilde \mu^{-1}_{r,\delta,\nu}$ is the relative magnetic
permeability.
Inserting this expression in the equations of motion we obtain:
\begin{align}
\sigma_{\lambda,\mu}&= \check \sigma_{\lambda,\mu}^{(0)}
+\sum_{\gamma,\delta} \check C_{\lambda,\mu,\gamma,\delta}  
\epsilon_{\gamma,\delta} -
\sum_{\gamma} \check e_{\lambda,\mu,\gamma} 
 {\bf E}_{\gamma}
-\sum_{\gamma,\nu}  \tilde h_{\lambda,\mu,\gamma} 
\tilde \mu^{-1}_{r,\gamma,\nu} {\bf B}_{\nu}, \\
{\bf P}_{\lambda}&= \check {\bf P}^{(0)}_{\lambda}
+\sum_{\alpha,\beta} \check e_{\alpha,\beta,\lambda} 
\epsilon_{\alpha,\beta} +
\sum_{\beta} 
\check \chi^e_{\lambda,\beta}
{\bf E}_{\beta} +
{1\over 4 \pi} \sum_{\delta,\nu} \tilde \alpha_{\lambda,\delta} 
\tilde \mu^{-1}_{r,\delta,\nu}  
{\bf B}_{\nu}, 
\end{align}
one finds the relationship between quantities calculated at constant 
${\bf B}$ and those calculated at constant ${\bf H}$. We have:
\begin{align}
\check C_{\lambda,\mu,\gamma,\delta} &=
\tilde C_{\lambda,\mu,\gamma,\delta} + 4 \pi \sum_{\alpha,\beta,\nu}
\tilde h_{\lambda,\mu,\alpha} \tilde \mu^{-1}_{r,\alpha,\nu} 
\tilde h_{\gamma,\delta,\nu}, \\
\check e_{\lambda,\mu,\gamma}&= \tilde e_{\lambda,\mu,\gamma}
-\sum_{\alpha,\beta} \tilde h_{\lambda,\mu,\alpha} 
\tilde \mu^{-1}_{r,\alpha,\nu}
\tilde \alpha_{\gamma,\nu}, \\
\check \chi^e_{\lambda,\mu}&= \tilde \chi^e_{\lambda,\mu}
-{1\over 4 \pi} \sum_{\delta,\nu} \tilde \alpha_{\lambda,\delta} 
\tilde \mu^{-1}_{r,\delta,\nu} \tilde \alpha_{\mu,\nu}. 
\end{align}
In addition we have the relationships:
\begin{align}
\check \sigma^{(0)}_{\lambda,\mu} &= \tilde \sigma^{(0)}_{\lambda,\mu}
+ 4 \pi \sum_{\gamma,\nu} \tilde h_{\lambda,\mu,\gamma} 
\tilde \mu^{-1}_{r,\gamma,\nu} {\bf M}^{(0)}_\nu,  \\
\check {\bf P}^{(0)}_{\lambda} &= \tilde {\bf P}^{(0)}_{\lambda}
- \sum_{\beta,\nu} \tilde \alpha_{\lambda,\beta} 
\tilde \mu^{-1}_{r,\beta,\nu} {\bf M}^{(0)}_\nu. 
\end{align}
We can derive similar equations from the microscopic equations of motion.
The dependence of ${\bf H}_\delta$ on ${\bf B}$ is:
\begin{align}
{\bf H}_\delta = 
&- 4 \pi \sum_\nu \mu^{-1}_{r,\delta,\nu} {\bf M}^{(0)}_\nu 
-{4 \pi \over V} \sum_{s,\alpha,\nu} \mu^{-1}_{r,\delta,\nu} Z^m_{s,\alpha,\nu}
{\bf u}_{s,\alpha} \nonumber\\
&- 4 \pi \sum_{\alpha,\beta,\nu} \mu^{-1}_{r,\delta,\nu} h_{\alpha,\beta,\nu}
\epsilon_{\alpha,\beta} - \sum_{\alpha,\nu}
\mu^{-1}_{r,\delta,\nu} \alpha_{\alpha,\nu} {\bf E}_\alpha
+ \sum_\nu \mu^{-1}_{r,\delta,\nu} {\bf B}_\nu, 
\end{align}
where $\mu^{-1}_{r,\delta,\nu}$ is the frozen-ions relative magnetic
permeability.
Inserting this equation in the other equations of motion gives:
\begin{align}
{\bf f}_{s,\alpha}&= \breve{\bf f}^{(0)}_{s,\alpha}
-\sum_{s',\beta} \breve C_{s,\alpha,s',\beta} {\bf u}_{s',\beta}
+\sum_{\beta,\gamma}
\breve \Lambda_{s,\alpha,\beta,\gamma} 
\epsilon_{\beta,\gamma} +
e \sum_{\beta} \breve Z^*_{s,\alpha,\beta} {\bf E}_{\beta}
+\sum_{\beta} 
\breve Z^m_{s,\alpha,\beta} {\bf B}_{\beta}, \\
\sigma_{\lambda,\mu}&= \breve  \sigma_{\lambda,\mu}^{(0)}
-{1\over V} \sum_{s,\alpha}
\breve \Lambda_{s,\alpha,\lambda,\mu} {\bf u}_{s,\alpha}
+\sum_{\gamma,\delta} \breve C_{\lambda,\mu,\gamma,\delta}  
\epsilon_{\gamma,\delta} -
\sum_{\gamma} \breve e_{\lambda,\mu,\gamma} 
 {\bf E}_{\gamma}
-\sum_{\gamma}  \breve h_{\lambda,\mu,\gamma} 
{\bf B}_{\gamma}, \\
{\bf P}_{\lambda}&= \breve {\bf P}^{(0)}_{\lambda}
+{e\over V} \sum_{s,\alpha} \breve Z^*_{s,\alpha,\lambda} 
{\bf u}_{s,\alpha} +
\sum_{\alpha,\beta} \breve e_{\alpha,\beta,\lambda} 
\epsilon_{\alpha,\beta} +
\sum_{\beta} 
\breve \chi^e_{\lambda,\beta}
{\bf E}_{\beta} +
{1\over 4 \pi} \sum_{\beta} \breve \alpha_{\lambda,\beta} 
{\bf B}_{\beta}, 
\end{align}
where
\begin{align}
\breve {\bf f}^{(0)}_{s,\alpha}&= {\bf f}^{(0)}_{s,\alpha} - 4 \pi
\sum_{\gamma,\nu}
Z^m_{s,\alpha,\gamma} \mu^{-1}_{r,\gamma,\nu} {\bf M}^{(0)}_\nu, \\
\breve \sigma^{(0)}_{\lambda,\mu}&= \sigma^{(0)}_{\lambda,\mu}
+ 4 \pi \sum_{\gamma,\nu} h_{\lambda,\mu,\gamma} \mu^{-1}_{r,\gamma,\nu} 
{\bf M}^{(0)}_\nu, \\
\breve {\bf P}^{(0)}_\lambda&= {\bf P}^{(0)}_\lambda -
\sum_{\gamma,\nu} \alpha_{\lambda,\gamma} \mu^{-1}_{r,\gamma,\nu} 
{\bf M}^{(0)}_\nu, \\
\end{align}
and
\begin{align}
\breve C_{s,\alpha,s',\beta} &= C_{s,\alpha,s',\beta}+
{4 \pi \over V}  \sum_{\delta,\lambda} Z^m_{s,\alpha,\delta}
\mu^{-1}_{r,\delta,\lambda}Z^m_{s',\beta,\lambda},\\
\breve \Lambda_{s,\alpha,\beta,\gamma} &= \Lambda_{s,\alpha,\beta,\gamma}
- 4 \pi \sum_{\delta,\lambda} Z^m_{s,\alpha,\delta}
\mu^{-1}_{r,\delta,\lambda} h_{\beta,\gamma,\lambda}, \\
\breve Z^*_{s,\alpha,\beta} &= Z^*_{s,\alpha,\beta}
-{1\over e} \sum_{\delta,\lambda} Z^m_{s,\alpha,\delta} 
\mu^{-1}_{r,\delta,\lambda} \alpha_{\beta,\lambda}, \\
\breve Z^m_{s,\alpha,\beta} &= \sum_{\delta} Z^m_{s,\alpha,\delta} 
\mu^{-1}_{r,\delta,\beta}, 
\end{align}
are the microscopic tensors, while
\begin{align}
\breve C_{\lambda,\mu,\gamma,\delta} &=
C_{\lambda,\mu,\gamma,\delta} + 4 \pi \sum_{\alpha,\beta,\nu}
h_{\lambda,\mu,\alpha} \mu^{-1}_{r,\alpha,\nu} 
h_{\gamma,\delta,\nu}, \\
\breve e_{\lambda,\mu,\gamma}&= e_{\lambda,\mu,\gamma}
-\sum_{\alpha,\beta} h_{\lambda,\mu,\alpha} \mu^{-1}_{r,\alpha,\nu}
\alpha_{\gamma,\nu}, \\
\breve h_{\lambda,\mu,\gamma}&= \sum_\delta h_{\lambda,\mu,\delta}
\mu^{-1}_{r,\delta,\gamma} \\
\breve \chi^e_{\lambda,\mu}&= \chi^e_{\lambda,\mu}
-{1\over 4 \pi} \sum_{\delta,\nu} \alpha_{\lambda,\delta} 
\mu^{-1}_{r,\delta,\nu} \alpha_{\mu,\nu}, \\
\breve \alpha_{\lambda,\rho} &= \sum_\beta \alpha_{\lambda,\beta}
\mu^{-1}_{r,\beta,\rho}
\end{align}
are the frozen-ions macroscopic tensors.
}

\newpage
{\color{dark-blue}\chapter{Free-energy of free-fields}}
\color{black}

The free energy of the free fields is written as
\begin{equation}
\mathcal{F}_{free}=-{\epsilon_0 V \over 2} \sum_{\alpha,\beta} 
\delta_{\alpha,\beta} {\bf E}_\alpha {\bf E}_\beta-
{\mu_0 V \over 2} \sum_{\alpha,\beta} 
\delta_{\alpha,\beta} {\bf H}_\alpha {\bf H}_\beta.
\end{equation}
Therefore expanding $\mathcal{\tilde F} = \mathcal{F} + \mathcal{F}_{free}$
instead of $\mathcal{F}$ we would obtain four different terms:
\begin{align}
\mathcal{\tilde F}(\{{\bf u}_{s,\alpha}\}, \{\epsilon_{\alpha,\beta}\},
\{ {\bf E}_\alpha\}, \{{\bf H}_\alpha\})&= \cdots -
V \sum_{\alpha} {\bf D}^{(0)}_\alpha {\bf E}_\alpha - 
V \sum_{\alpha} {\bf B}^{(0)}_\alpha {\bf H}_\alpha \nonumber \\
&- {1\over 2} \epsilon_0 V \sum_{\alpha,\beta} 
\epsilon_{r,\alpha,\beta}
{\bf E}_{\alpha}
{\bf E}_{\beta} \nonumber \\
&- {1\over 2} \mu_0 V \sum_{\alpha,\beta} 
\mu_{r,\alpha,\beta}
{\bf H}_{\alpha}
{\bf H}_{\beta} \nonumber \\
&+ \cdots,
\end{align}
where the dots indicate the same terms written above.
Moreover we have:
\begin{equation}
{\bf D}_\alpha = -{1\over V} {\partial \mathcal{\tilde F} \over 
\partial {\bf E}_\alpha},
\end{equation}
and
\begin{equation}
{\bf B}_\alpha = -{1\over V} {\partial \mathcal{\tilde F} \over 
\partial {\bf H}_\alpha}.
\end{equation}
\\

{\color{web-blue} In a.u. the free energy of the 
free fields is written as
\begin{equation}
\mathcal{F}_{free}=-{V \over 8 \pi} \sum_{\alpha,\beta} 
\delta_{\alpha,\beta} {\bf E}_\alpha {\bf E}_\beta-
{\alpha^2 V \over 8 \pi} \sum_{\alpha,\beta} 
\delta_{\alpha,\beta} {\bf H}_\alpha {\bf H}_\beta.
\end{equation}
Therefore expanding $\mathcal{\tilde F} = \mathcal{F} + \mathcal{F}_{free}$
instead of $\mathcal{F}$ we would obtain four different terms:
\begin{align}
\mathcal{\tilde F}(\{{\bf u}_{s,\alpha}\}, \{\epsilon_{\alpha,\beta}\},
\{ {\bf E}_\alpha\}, \{{\bf H}_\alpha\})&= \cdots -
{V \over 4 \pi} \sum_{\alpha} {\bf D}^{(0)}_\alpha {\bf E}_\alpha - 
{V \over 4 \pi} \sum_{\alpha} {\bf B}^{(0)}_\alpha {\bf H}_\alpha \nonumber \\
&- {V\over 8 \pi} \sum_{\alpha,\beta} 
\epsilon_{r,\alpha,\beta}
{\bf E}_{\alpha}
{\bf E}_{\beta} \nonumber \\
&- {\alpha^2 V\over 8 \pi} \sum_{\alpha,\beta} 
\mu_{r,\alpha,\beta}
{\bf H}_{\alpha}
{\bf H}_{\beta} \nonumber \\
&+ \cdots,
\end{align}
where the dots indicate the same terms written above.
Moreover we have:
\begin{equation}
{\bf D}_\alpha = -{4\pi\over V} {\partial \mathcal{\tilde F} \over 
\partial {\bf E}_\alpha},
\end{equation}
and
\begin{equation}
{\bf B}_\alpha = -{4\pi\over V} {\partial \mathcal{\tilde F} \over 
\partial {\bf H}_\alpha}.
\end{equation}
}
\\

{\color{orange} In c.g.s.-Gaussian units the free energy of the 
free fields is written as
\begin{equation}
\mathcal{F}_{free}=-{V \over 8 \pi} \sum_{\alpha,\beta} 
\delta_{\alpha,\beta} {\bf E}_\alpha {\bf E}_\beta-
{V \over 8 \pi} \sum_{\alpha,\beta} 
\delta_{\alpha,\beta} {\bf H}_\alpha {\bf H}_\beta.
\end{equation}
Therefore expanding $\mathcal{\tilde F} = \mathcal{F} + \mathcal{F}_{free}$
instead of $\mathcal{F}$ we would obtain four different terms:
\begin{align}
\mathcal{\tilde F}(\{{\bf u}_{s,\alpha}\}, \{\epsilon_{\alpha,\beta}\},
\{ {\bf E}_\alpha\}, \{{\bf H}_\alpha\})&= \cdots -
{V\over 4 \pi} \sum_{\alpha} {\bf D}^{(0)}_\alpha {\bf E}_\alpha - 
{V \over 4 \pi} \sum_{\alpha} {\bf B}^{(0)}_\alpha {\bf H}_\alpha \nonumber \\
&- {V\over 8 \pi} \sum_{\alpha,\beta} 
\epsilon_{r,\alpha,\beta}
{\bf E}_{\alpha}
{\bf E}_{\beta} \nonumber \\
&- {V\over 8 \pi} \sum_{\alpha,\beta} 
\mu_{r,\alpha,\beta}
{\bf H}_{\alpha}
{\bf H}_{\beta} \nonumber \\
&+ \cdots,
\end{align}
where the dots indicate the same terms written above.
Moreover we have:
\begin{equation}
{\bf D}_\alpha = -{4 \pi \over V} {\partial \mathcal{\tilde F} \over 
\partial {\bf E}_\alpha},
\end{equation}
and
\begin{equation}
{\bf B}_\alpha = -{4 \pi \over V} {\partial \mathcal{\tilde F} \over 
\partial {\bf H}_\alpha}.
\end{equation}
}

%\newpage
%{\color{dark-blue}\chapter{Gaussian atomic units}}
%
%{\color{steelblue}
%Written in Gaussian a.u. the phenomenological expression
%of the magneto-electric enthalpy coincides with the one written in 
%c.g.s.-Gaussian units.
%This expression can be found by notincing that with respect to the
%a.u. discussed in the text, Gaussian a.u. change the definition of the 
%following quantities:
%\\
%
%\noindent {\bf Magnetization:\\}
%Since ${\bar U_G \over \bar l_G^3 \mu_0 \bar H_G \bar M_G}=1$ we have:
%\begin{equation}
%{\bf M}^{(0)}_{\alpha}=-{1\over V} 
%{\partial \mathcal{F} \over \partial {\bf H}_{\alpha}}.
%\label{magnetization}
%\end{equation}
%\\
%
%\noindent {\bf Magnetic susceptibility:\\}
%Since ${\bar U_G \over \mu_0 \bar l_G^3 \bar H_G^2}=4\pi$ we have:
%\begin{equation}
%\chi^{m,a.u.}_{\alpha,\beta}=-{1\over V} 
%{\partial^2 \mathcal{F} \over \partial 
%{\bf H}_{\alpha} \partial {\bf H}_{\beta}}. 
%\end{equation}
%and $\chi^{m,a.u.}_{\alpha,\beta}= {\chi^{m}_{\alpha,\beta}\over 4 \pi}$.
%\\
%
%\noindent {\bf Dynamical magnetic charges:\\}
%In this case $\bar Z^m_G={\bar Z^m\over \alpha}$ and since
%${\alpha \bar U_G \over \mu_0 \bar l_G^2 \bar H_G \bar I_G}=1$ we have:
%\begin{equation}
%Z^m_{s,\alpha,\beta}=-{\partial^2 \mathcal{F} \over \partial 
%{\bf u}_{s,\alpha} \partial {\bf H}_{\beta}}.
%\end{equation}
%\\
%
%\noindent {\bf Piezomagnetic tensor:\\}
%In this case $\bar h_G={\bar h \over \alpha}$ because the piezomagnetic
%tensor and the magnetization have the same unit. Since 
%${\alpha \bar U_G \over \mu_0 \bar l_G^2 \bar H_G \bar I_G}=1$ 
%we have:
%\begin{equation}
%h_{\alpha,\beta,\gamma}=-{1\over V} 
%{\partial^2 \mathcal{F} \over \partial 
%\epsilon_{\alpha,\beta} \partial {\bf H}_{\gamma}}.
%\end{equation}
%\\
%
%\noindent {\bf Magnetoelectric tensor:\\}
%It is convenient to choose $\bar \alpha_G={\bar \alpha \over \alpha}$ so
%that ${\bar U_G \over \bar l_G^2 \bar E_G \bar H_G \bar t}={4 \pi}$ 
%and we can write:
%\begin{equation}
%\alpha_{\alpha,\beta}=-{4 \pi \over V} 
%{\partial^2 \mathcal{F} \over \partial 
%{\bf E}_{\alpha} \partial {\bf H}_{\beta}}. 
%\end{equation}
%}
%

\end{document}

