%!
%! Copyright (C) 2014-2023 Andrea Dal Corso 
%! This file is distributed under the terms of the
%! GNU General Public License. See the file `License'
%! in the root directory of the present distribution,
%! or http://www.gnu.org/copyleft/gpl.txt .
%!
\documentclass[12pt,a4paper,twoside]{report}
\def\version{1.9.0}

\usepackage[T1]{fontenc}
\usepackage{tcolorbox}
\usepackage{bookman}
\usepackage{html}
\usepackage{graphicx}
\usepackage{fancyhdr}
\usepackage[Lenny]{fncychap}
\usepackage{color}
\usepackage{geometry}

\tcbuselibrary{breakable}
\pagestyle{fancy}
\lhead{User's guide}
\rhead{}
\cfoot{\thepage}

\newgeometry{
      top=3cm,
      bottom=3cm,
      outer=2.25cm,
      inner=2.75cm,
}

\definecolor{web-blue}{rgb}{0,0.5,1.0}
\definecolor{coral}{rgb}{1.0,0.5,0.3}
\definecolor{red}{rgb}{1.0,0,0.0}
\definecolor{green}{rgb}{0.,1.0,0.0}
\definecolor{dark-blue}{rgb}{0.,0.0,0.6}
\definecolor{light-pink}{rgb}{1.0,1.0,0.8}
%\tcbset{colback=light-pink,colframe=dark-blue,oversize=\linewidth/15,breakable}
\tcbset{colback=light-pink,colframe=dark-blue,breakable}

\def\qe{{\sc Quantum ESPRESSO}}
\def\pwx{\texttt{pw.x}}
\def\phx{\texttt{ph.x}}
\def\configure{\texttt{configure}}
\def\PWscf{\texttt{PWscf}}
\def\PHonon{\texttt{PHonon}}
\def\thermo{{\sc Thermo}\_{\sc pw}}
\def\make{\texttt{make}}

\begin{document} 

\author{Andrea Dal Corso (SISSA - Trieste)}
\date{}

\title{
  \includegraphics[width=8cm]{thermo_pw.jpg} \\
  \vspace{3truecm}
  % title
  \Huge \color{dark-blue} {\sc Thermo}\_{\sc pw} User's Guide \\
  (v.\version)
}

\maketitle

\newpage

\tableofcontents

\newpage

{\color{dark-blue}\chapter{Introduction}}
\color{black}
This guide covers the installation and usage of the \thermo\ package. 
It assumes some familiarity with \qe. 
For this please consult the web site: \texttt{http://www.quantum-espresso.org}.

\thermo\ computes material properties.
At low level, it calls \qe\ routines and, at high level, it has pre-processing
tools to reduce the information provided by the user and
post-processing tools that use the output of \qe\ to produce plots of material 
properties directly comparable with experiment. 

\thermo\ has the following directory structure, contained in a directory 
\texttt{thermo\_pw/} that should be put in the root directory of \qe:

\begin{tabular}{ll}
\texttt{Doc/}      & : contains this user's guide and other documentation\\
\texttt{examples/} & : some examples \\
\texttt{examples\_qe/} & : \qe\ examples run using 
\thermo \\
\texttt{inputs/}   & : a collection of useful inputs \\
\texttt{pseudo\_test/} & : a collection of inputs to test a pseudopotential library\\
\texttt{space\_groups/} & : a collection of structures for many space groups \\
\texttt{lib/}      & : source files for modules used by \thermo\ \\
\texttt{qe/}       & : routines of \qe\ that require some \\
                   &   change \\
\texttt{fft}       & : some fft routines that can run on the GPU. \\
\texttt{lapack}    & : some lapack routines that can run on the GPU. \\
\texttt{src/}      & : source files for \thermo\ \\
\texttt{tools/}    & : source files for auxiliary tools \\
\texttt{tools\_input/}  & : examples of inputs for the auxiliary tools \\
\end{tabular}\\

The \thermo\ package can calculate the following quantities:
\begin{itemize}
\item Plot of the Brillouin zone (the structure can be seen by reading the
input of \thermo\ by the \texttt{XCrySDen} program).

\item Plot of the X-rays powder diffraction pattern of the input crystal.

\item Total energy at fixed geometry.

\item Total energy as a function of the kinetic energy cut-off.

\item Total energy as a function of {\bf k}-points and smearing.

\item Electronic band structure at fixed geometry.

\item Electronic density of states at fixed geometry. Electronic thermodynamic
properties: energy, free energy, entropy, and heat capacity. 

\item Electronic heat capacity as a function of temperature (for metals only).

\item Complex dielectric constant as a function of the complex
frequency $\omega$ at fixed geometry. Complex index of refraction for
all systems except monoclinic and triclinic. Reflectivity at normal
incidence and adsorption coefficient for cubic solids. 

\item Inverse dielectric constant at a given wavevector {\bf q} as a function 
of the complex frequency $\omega$ at fixed geometry.

\item Phonon frequencies at fixed geometry.

\item Phonon dispersions at fixed geometry and harmonic
thermodynamic properties: vibrational energy, vibrational free energy,
vibrational entropy, and constant volume heat capacity as a function of
temperature. Atomic Debye-Waller factors as a function of temperature.

\item Frozen ions and relaxed ions elastic constants at fixed geometry.

\item Relaxed ions temperature dependent elastic constants at fixed 
unperturbed geometry.

\item Fit of the total energy as a function of the lattice parameters with
a quadratic or quartic polynomial and determination of equilibrium lattice 
parameters. Murnaghan or (third or fourth order) Birch-Murnaghan fit. 
Enthalpy as a function of pressure. Crystal
parameters and volume as a function of pressure. 

\item Electronic band structure at the minimum of the total energy.

\item Electronic density of states at the minimum of the total energy.
Electronic thermodynamic properties.  

\item Complex dielectric constant as a function of the complex
frequency $\omega$ at the minimum of the total energy. Complex index 
of refraction for all systems except monoclinic and triclinic. Reflectivity 
at normal incidence and adsorption coefficient for cubic solids.

\item Inverse dielectric constant at a given wavevector {\bf q} as a function 
of the complex frequency $\omega$ at the minimum of the total energy.

\item Phonon frequencies at the minimum of the total energy.

\item Phonon dispersions and harmonic thermodynamic quantities
at the minimum of the total energy.

\item Frozen ions and relaxed ions elastic constants at the minimum of the total
energy.

\item Anharmonic properties within the quasi-harmonic approximation: 
lattice parameters, thermal expansion tensor, volume, volume thermal 
expansion, and constant strain heat capacity as a function of temperature; 
phonon frequencies and mode Gr\"uneisen parameters interpolated at a given
geometry or at the equilibrium geometry at a given temperature
(limited to cubic, tetragonal, orthorhombic, and hexagonal systems).
Bulk modulus and pressure derivative of the bulk modulus, isobaric heat 
capacity, isoentropic bulk modulus, and average Gr\"uneisen parameter as 
a function of temperature (limited to cubic systems).
Minimum Helmholtz (or Gibbs at finite pressures) free energy 
as a function of temperature.

\item Isothermal and isoentropic elastic constants and 
elastic compliances as a function of temperature within the ``quasi-static'' 
approximation.

\item Isothermal and isoentropic elastic constants and 
elastic compliances as a function of temperature within the 
``quasi-harmonic'' approximation.

\item Surface band structure identification and plot of the projected bulk
band structure.

\end{itemize}

\thermo\ can run on both serial and parallel machines using all 
the parallellization options of \qe. Moreover, \thermo\ can run using 
several images.
When possible, the image parallelization is used in an asynchronous way.
One image takes the role of master and distributes the work 
to all the images that carry it out independently. Presently 
the total energies of several geometries for the determination of the 
equilibrium geometry are calculated in parallel when
there are several images. Stresses or total energies at different strained 
geometries needed for the calculation of the elastic constants are 
calculated in parallel. 
The phonon calculations are carried out in parallel, each image doing one 
irreducible representation of one {\bf q} point. For frequency dependent 
calculation, each frequency, or group of frequencies, can be calculated 
in parallel by different images.
The phonon dispersions of several geometries needed
for the quasi-harmonic calculation of the thermodynamic properties or
of the elastic constants can be calculated in parallel (one geometry at 
a time or all geometries together).

\newpage
{\color{coral}\section{People}}
\color{black}
The \thermo\ code is designed, written, and maintained by Andrea Dal Corso 
(SISSA - Trieste). It is an open source code distributed, as is, within the GPL
license.  

I would like to thank all the people that contributed with comments, requests
of improvements, and bug reports. Some people also found bug corrections or
wrote new routines and shared their improvements with me. In particular 
I would like to mention M. Palumbo, O. Motornyi, A. Urru, C. Malica, 
X. Gong, and B. Thakur.

\newpage
{\color{dark-blue}\chapter{Installing, Compiling, and Running}}

{\color{coral}\section{Installing}}
\color{black}
\begin{table}
\begin{center}
\begin{tabular}{lll}
\hline
\hline
\thermo\ & \qe & release date \\
\hline
1.9.0 & 7.2 & 26/05/2023 \\
1.8.1 & 7.2 & 05/05/2023 \\
1.8.0 & 7.1 & 26/04/2023 \\
1.7.1 & 7.1 & 05/07/2022 \\
1.7.0 & 7.0 & 05/07/2022 \\
1.6.1 & 7.0 & 10/01/2022 \\
1.6.0 & 6.8 & 27/12/2021 \\
1.5.1 & 6.8 & 22/07/2021 \\
1.5.0 & 6.7 & 19/07/2021 \\
1.4.1 & 6.7 & 29/12/2020 \\
1.4.0 & 6.6 & 22/12/2020 \\
1.3.2 & 6.6 & 17/8/2020 \\
1.3.1 & 6.6 & 13/8/2020 \\
1.3.0 & 6.5 & 12/8/2020 \\
1.2.1 & 6.5 & 23/1/2020 \\
1.2.0 & 6.4.1 & 28/12/2019 \\
1.1.1 & 6.4.1 & 16/04/2019 \\
1.1.0 & 6.4 & 16/04/2019 \\
1.0.9 & 6.3 & 6/3/2019 \\
1.0.0 & 6.3 & 17/7/2018 \\
0.9.9 & 6.2.1 & 5/7/2018 \\
0.9.0 & 6.2.1 & 20/12/2017 \\
0.8.0 & 6.2 & 24/10/2017 \\
0.8.0-beta & 6.2-beta & 31/8/2017\\
0.7.9 & 6.1 & 06/7/2017 \\
0.7.0 & 6.1 & 18/3/2017 \\
0.6.0 & 6.0.0 & 5/10/2016 \\
0.5.0 & 5.4.0 & 26/4/2016 \\
0.4.0 & 5.3.0 & 23/1/2016 \\
0.3.0 & 5.2.1, 5.2.0 & 23/6/2015 \\
0.2.0 & 5.1.2 & 13/3/2015 \\
0.1.0 & 5.1.1 & 28/11/2014 \\
\hline
\hline
\end{tabular}
\caption{Compatibility between the versions of 
\thermo\ and of \texttt{QE}.}
\end{center}
\end{table}

The \thermo\ package is tightly bound to \qe. It cannot be compiled without
it. To download and compile \qe, please 
refer to the general User's Guide, available in the file \texttt{Doc/user\_guide.pdf}
in the root \qe\ directory, or on the web site 
\begin{center}
{\texttt{http://www.quantum-espresso.org}}.
\end{center}
The main distribution page of \texttt{Thermo\_pw} is 
\begin{center}
\texttt{https://dalcorso.github.io/thermo\_pw/} 
\end{center}
where you can download one of the \texttt{.tar.gz} files.
Please match carefully \thermo\ and \qe\ versions
as illustrated in Table 1. 
For the versions of \qe\ not listed here, there is no \thermo\ 
package. The source files of \thermo\ can be obtained by
unpacking the \texttt{.tar.gz} file in the root \qe\ directory, for
instance with the command \texttt{tar -xzvf thermo\_pw.1.9.0.tar.gz}.\\
You can also download the \texttt{git} version of 
\thermo\ as described at
the web page: 
\texttt{https://dal} \texttt{corso.github.io/thermo\_pw/thermo\_pw\_help.html}.
The \texttt{git} version of \thermo\ contains the most recent
features and bug fixes but it might work only with the \texttt{git} 
version of \qe\ and its use for production is not recommended. \\
Please read the web page: 
\begin{center}
\texttt{http://dalcorso.github.io/thermo\_pw/thermo\_pw\_help.html} 
\end{center}
for updated information about the compatibility between the \texttt{git} 
version of \thermo\ and \qe. 
This web page contains also information on critical bugs found in the 
\thermo\ package and should be consulted before using 
\thermo.

\newpage
{\color{coral}\section{Compiling}}
\color{black}

In order to compile \thermo, the main \texttt{Makefile} and the files
\texttt{install/} \texttt{makedeps.sh} and \texttt{install/plugins\_makefile}
of \qe\ must be changed. This is done by giving the command 
\texttt{make join\_qe} inside the \texttt{thermo\_pw} directory. 
This command exchanges
the \texttt{thermo\_pw} files with those of the \qe\ package.

Typing \texttt{make thermo\_pw} inside the main \qe\ directory, 
or \texttt{make} 
inside the \texttt{thermo\_pw}\ directory, produces the executable
\texttt{thermo\_pw/} \texttt{src/thermo\_pw.x} that appears in the 
\qe\ \texttt{bin/} directory. A few other tool codes are produced as well
and linked in the \qe\ \texttt{bin/} directory.

Starting from version 1.5.1 \texttt{thermo\_pw}\ can be compiled also
with \texttt{cmake}. The command \texttt{make join\_qe} substitutes
the \texttt{CMakeLists.txt} file of QE with the one contained
in \texttt{thermo\_pw}. After writing \texttt{make join\_qe} follow the 
QE instructions to compile using \texttt{cmake}. Two new commands are
available: \texttt{make thermo\_pw} produces \texttt{thermo\_pw.x}  
and \texttt{make tpw\_tools} produces the tools codes.

\thermo\ has been written on a PC with the Linux operating system using a
\texttt{gfortran} compiler and \texttt{openMPI} parallelization. It has
been run in parallel on a Linux cluster with several hundreds processors.
It has not been tested with other combinations of computer/operating system, 
but it is supposed to run on the same systems where \qe\ runs. If
you have a machine in which you can compile and run \qe\ but not 
\texttt{thermo\_pw}, please report the problem.

\newpage
{\color{coral}\section{Searching help and reporting bugs}}
\color{black}
For problems installing \texttt{thermo\_pw} or for help
with some of its features, please subscribe to the \texttt{thermo\_pw-forum}
mailing list 
(\texttt{https://lists.quantum-} \texttt{espresso.org/mailman/listinfo/}
\texttt{thermo\_pw-forum}). 
Requests of new features are also welcome. If you think you have
found a bug 
in \texttt{thermo\_pw}, you can report it to the mailing list or 
write me: \texttt{dalcorso.at.sissa.it}.

\newpage
{\color{coral}\section{Uninstalling}}
\color{black}

In order to remove \thermo, give the command \texttt{make leave\_qe} in the
\texttt{thermo\_pw} directory. Then just remove the directory. Note that 
the command \texttt{make leave\_qe} is needed to restore the original \qe\ files.

\newpage
{\color{coral}\section{Running \thermo}}
\color{black}

In order to run \thermo, you need an input for \texttt{pw.x},  
a file called \texttt{thermo\_} \texttt{control}, and an input for 
\texttt{ph.x}
(which must be called \texttt{ph\_control}) if required by the task.
 These files must be in your working
directory.
The input of \texttt{pw.x} can have any name and is given as input to
the \thermo\ code. It is better not to specify an \texttt{outdir} 
directory in the \texttt{ph.x} input. Specifying an \texttt{outdir}
directory is not forbidden, but for some tasks \texttt{thermo\_pw.x} 
might add a geometry number to \texttt{outdir} and the
\texttt{outdir} written in the \texttt{ph.x} input must be consistent.

A typical command for running \thermo\ is:

%\begin{tcolorbox}
\begin{verbatim}
mpirun -n np thermo_pw.x -ni ni ... < input_pw > output_thermo_pw
\end{verbatim}
%\end{tcolorbox}

where \texttt{np} is the number of processors and \texttt{ni} is the number 
of images. The dots indicate the other \qe\ parallelization options that
you can find in its manual.

Note that it is very easy to waste resources using too many 
images. Unused images wait for the working images to complete
their tasks wasting cpu-time in an endless loop. 
Some options 
do not use the image feature, so you have to know how the calculation 
is divided and the number of images must not
be larger than the number of tasks (below I give this number for each option).
If you have doubts on this point use one image 
(\texttt{ni=1}).

The outputs of the \thermo\ code are one or more \texttt{postscript} or
\texttt{pdf} files 
with plots of the material properties. \thermo\ produces also files with 
the data of 
the plot and scripts for the \texttt{gnuplot} program. 
Usually, the user does not need to modify these files, but they allow 
the improvement of the figures when needed.
The plot of the Brillouin zone (BZ) is made with the help of the 
\texttt{asymptote} code. \texttt{Thermo\_pw} produces a script 
for the \texttt{asymptote} code and can also run it to produce the \texttt{pdf}
file of the BZ.  

\newpage

{\color{dark-blue}\chapter{Input variables}}
\color{black}

The \texttt{pw.x} and \texttt{ph.x} input files are described in the \qe\ documentation.
In this section we discuss only the creation of the file
\texttt{thermo\_} \texttt{control}. This file contains a namelist:  

%\begin{tcolorbox}
\begin{verbatim}
&INPUT_THERMO
  what=' ',
  ...
 /
\end{verbatim}
%\end{tcolorbox}

The \texttt{what} variable controls the sequence of calculations made
by \thermo. For each possible value of \texttt{what}, we discuss briefly the
input variables that control the output plots. Usually
default values of the input variables are sufficient to carry out  
the basic \thermo\ tasks and you are not supposed to set any variable except
\texttt{what}, but in some cases these input variables give more control
on the calculation and on its accuracy.

\thermo\ writes on files the data to plot and a script to plot
these data. The output \texttt{postscript} or \texttt{pdf} files are produced 
by invoking the 
\texttt{gnuplot} program. Usually any modern Linux 
distribution provides a package to install this code, or has it already 
installed. You can also download the package from 
\texttt{http://www.gnuplot.info/}. \\
The following input variables control the use of the \texttt{gnuplot} code:

%\begin{tcolorbox}
\begin{verbatim}
lgnuplot    : if .TRUE. gnuplot is called by the 
              program and the postscript or pdf files are 
              immediately available. Otherwise give the 
              command gnuplot gnuplot_files/*.
              Default: logical .TRUE.
gnuplot_command  : the command used to call gnuplot.
              Default: character(len=*) 'gnuplot'
flgnuplot   : initial part of the name of the files where 
              gnuplot scripts are written.
              Default: character(len=*) 'gnuplot.tmp'
flext       : extension of the output files. Presently .ps 
              and .pdf are supported for postscript or pdf 
              output. The latter is available only if gnuplot 
              supports the pdfcairo terminal.
              Default: character(len=*) '.ps'
\end{verbatim}
%\end{tcolorbox}

If your system has not \texttt{gnuplot}
you can disable the production of the \texttt{postscript} or 
\texttt{pdf} files and use other
graphical tools to plot the output data. 

\newpage
{\color{coral}\section{Temperature and pressure}}
\color{black}
Several quantities in \thermo\ can be calculated as a function of temperature
at zero pressure or at selected pressures.
Moreover the equilibrium geometry can be searched at fixed pressure 
minimizing the enthalpy instead of the energy.
In some cases it is also possible to plot some quantities as a function 
of pressure at zero temperature or at selected temperatures.
There is therefore the necessity to specify uniform meshes of temperatures
(pressures) or to choose selected temperatures (pressures). 
The meshes are specified giving the minimum and maximum temperatures 
(pressures) and the interval between temperature (pressure) points.
Selected temperatures (pressures) are instead defined giving their number 
and their values. The code will choose these temperatures (pressures) on 
the uniform mesh taking the point closest to the specified temperature 
(or pressure).
For the options where these features are active, the values of temperature
and pressure are controlled by the following variables:

%\begin{tcolorbox}
\begin{verbatim}
tmin       : minimum temperature for the mesh of temperatures.
             Default: real 1 K
tmax       : maximum temperature for the mesh of temperatures.
             Default: real 800 K
deltat     : interval between two temperatures. Be careful 
             with this value because it is used also to 
             compute temperature derivatives numerically. 
             Too small or too large values could give 
             inaccurate anharmonic properties.
             Default: real 3 K
ntemp      : number of temperatures.
             Default: integer determined from previous data
pressure   : The external pressure. The crystal parameters are
             calculated minimizing the enthalpy at this 
             pressure. Given in kbar units. 
             Default: real 0.0 kbar
pmin       : minimum pressure for the mesh of pressures.
             Default: real -50.0 kbar
pmax       : maximum pressure for the mesh of pressures.
             Default: real 100.0 kbar
deltap     : Interval between two pressures in the mesh of pressures.
             Default: real 1.0 kbar
ntemp_plot : number of temperatures in the plots where the 
             temperature is a parameter. When 0 these plots are 
             not produced.
             Default: integer 0
temp_plot(ntemp_plot)  : A real array of dimension ntemp_plot with 
             the values of the temperature.
             No Default value, must be given (in K) when 
             ntemp_plot is not zero. 
npress_plot : number of pressures in the plots where the 
             pressure is a parameter. When 0 these plots are 
             not produced.
             Default: integer 0
press_plot(npress_plot) : A real array of dimension npress_plot 
             with the values of the pressure.
             No Default value, must be given (in kbar) when 
             npress_plot is not zero.
nvol_plot : number of volumes in the plots where the volume is a 
            parameter. When 0 these plots are not produced.
            Default: integer 0
ivol_plot(nvol_plot) : an integer array with the volumes to 
            consider in the plots where the volume is a parameter. 
            The chosen volumes can be only among one of the ngeo(1) 
            geometries.
            No Default value, must be given when nvol_plot 
            is not zero.
\end{verbatim}
%\end{tcolorbox}

Note that when you fix the external pressure, the geometries chosen to 
fit the enthalpy must be about the minimum geometry at that pressure.
Similarly, when using pressure ranges, the number of simulated geometries 
must be large enough to cover the selected range of pressures, otherwise 
the plotted quantities might be inaccurate.

\newpage
{\color{coral}\section{Coordinates and structure}}
\color{black}
The \texttt{thermo\_pw} code requires the Bravais lattice of the solid. 
Moreover for computing some quantities it assumes that the direct lattice 
vectors are those provided by the routine \texttt{latgen.f90} of the \qe\ 
distribution. 
For this reason it is not recommended to use \texttt{ibrav=0} in the
\texttt{pw.x} input. The preferred method is to give the value of 
\texttt{ibrav} and use the primitive vectors provided by \qe.
It is also possible to specify the \texttt{space\_group} number and 
the coordinates of the nonequivalent atoms. When the 
\texttt{pw.x} input contains the \texttt{ibrav=0} option, 
\texttt{thermo\_pw} writes on output
the values of \texttt{ibrav}, \texttt{celldm}, and of the atomic coordinates
that should be used in the input of \texttt{pw.x} to simulate the same 
solid and stops. There are however two input variables of \texttt{thermo\_pw} 
that can modify this behavior:

%\begin{tcolorbox}
\begin{verbatim}
continue_zero_ibrav : when ibrav=0 in the input of pw.x and 
             this variable is set to .TRUE. thermo_pw runs 
             with ibrav=0 (not recommended except when you 
             deal with a supercell). When this variable is 
             .FALSE. and ibrav=0 the behavior depends on 
             find_ibrav.
             Default: logical .FALSE.

find_ibrav : This variable is active only when continue_zero_
             ibrav=.FALSE.. When this variable is set to .TRUE. 
             and the input of pw.x has ibrav=0, thermo_pw finds 
             the values of ibrav, celldm, and of the atomic 
             positions that produce the same crystal and 
             continue the calculation. The geometry used by 
             thermo_pw might be rotated with respect to the 
             input and have different primitive vectors. When 
             this variable is .FALSE. the code stops after 
             writing in output ibrav, celldm, and the atomic 
             positions. These variables can be copied in the 
             pw.x input. Note that the automatic identification 
             of the lattice does not work for supercells.
             Default: logical .FALSE.
\end{verbatim}
%\end{tcolorbox}

\newpage
{\color{coral}\section{what='scf'}}
\color{black}
With this option the code computes only the total energy. This is a single
calculation as if running \texttt{pw.x} with the given input.
No other input variable is necessary.
An example for this option can be found in \texttt{example01}. \\
Number of tasks for this option: \texttt{1}.

\newpage
{\color{coral}\section{what='scf\_ke'}}
\color{black}
With this option the code makes several self-consistent calculations, 
in parallel on several images, varying the kinetic energy cut-off for 
the wavefunctions and for the charge density. 
In the input of \texttt{pw.x} one specifies the minimum values for these two 
cut-offs. These values are then increased in fixed intervals controlled by the 
following variables. The energy is then plotted as a function of the 
wavefunctions kinetic energy cut-off, a different curve for each value of 
the charge density cut-off. \\
The variables that control this option are:

%\begin{tcolorbox}
\begin{verbatim}
nke        : number of kinetic energies tested for the 
             wavefunctions cut-off.
             Default: integer 5
deltake    : delta of wavefunctions kinetic energy cut-off 
             in Ry (can be either positive or negative).
             Default: real 10 Ry
nkeden     : number of kinetic energies tested for the 
             charge density cut-off.
             Default: integer 1
deltakeden : delta of charge density kinetic energy 
             cut-off in Ry (can be either positive or 
             negative).
             Default: real 100 Ry.
flkeconv   : name of the file where the data with the 
             total energy as a function of the kinetic 
             energy is written.
             Default: character(len=*) 'output_keconv.dat'
flpskeconv : name of the postscript file with the plot 
             of the total energy as a function of the 
             kinetic energy cut-off.
             Default: character(len=*) 'output_keconv'
\end{verbatim}
%\end{tcolorbox}

An example for this option can be found in \texttt{example10}. \\
Number of tasks for this option: \texttt{nke * nkedens}.

\newpage
{\color{coral}\section{what='scf\_nk'}}
\color{black}
With this option the code makes several self-consistent calculations, 
in parallel on several images, varying the size of the {\bf k}-point grid, 
and optionally for metals the smearing parameter \texttt{degauss}. In the 
input of \texttt{pw.x} the minimum value of these parameters is given and 
these values are increased in fixed intervals controlled by the following 
variables. On output the energy is plotted as a function of the mesh size, 
one curve for each smearing parameter.\\
The variables that control this option are:

%\begin{tcolorbox}
\begin{verbatim}
nnk        : the number of different values of nk to test.
             Default: integer 5
deltank(3) : the interval between nk values. All three values 
             of nk1, nk2, and nk3 are updated simultaneously.
             Default: integer 2  2  2
nsigma     : the number of smearing intervals.
             Default: integer 1 
deltasigma : the distance between different smearing values
             (can be either positive or negative).
             Default: 0.005 Ry
flnkconv   : file where the data with the k point convergence 
             is written.
             Default: character(len=*) 'output_nkconv.dat'
flpsnkconv : name of the postscript file with the k points 
             convergence plot.
             Default: character(len=*) 'output_nkconv'
\end{verbatim}
%\end{tcolorbox}

An example for this option can be found in \texttt{example11}. \\
Number of tasks for this option: \texttt{nnk * nsigma}.

\newpage
{\color{coral}\section{what='scf\_bands'}}
\color{black}
With this option the code makes a self-consistent calculation followed 
by a band structure calculation. This option is not parallelized over
images and should be used with one image. The output of the band structure 
calculation is further processed in order to produce a plot of the band 
structure.
The zero of the energy is the highest valence band 
of the first {\bf k} point in insulators and the Fermi energy in metals. \\
The energy bands plot can be modified by the following variables:

%\begin{tcolorbox}
\begin{verbatim}
emin_input : minimum energy for the band dispersion plot (in eV).
             Default: real minimum of the bands
emax_input : maximum energy for the band dispersion plot (in eV).
             Default: real maximum of the bands
nbnd_bands : the number of bands in the band calculation.
             Default: integer 2*nbnd, where nbnd is the number 
             of bands given in pw.x input or calculated by pw.x.
only_bands_plot: if the files with the bands and the represen-
             tations are already on files, this option allows to 
             change the parameters of the plot (such as the 
             maximum or minimum energy) and do another plot 
             without additional calculation. If the files are
             missing and this variable is .TRUE. an error 
             occurs. Note that using this option you cannot 
             change the path.
             Default: logical .FALSE.
lsym       : if .TRUE. does the symmetry analysis of the bands.
             Default: .TRUE.
enhance_plot: if .TRUE. writes on the band plot the point 
             group labels, and colors with different background 
             colors lines at the zone border.
             Default: .FALSE.
long_path  : if .TRUE. plots the bands in all the Brillouin 
             zone path. Otherwise makes a faster calculation 
             on a short path. The short path is indicated also 
             for two-dimensional layers perpendicular to the 
             z direction.
             Default: .TRUE.
old_path   : if .TRUE. use an alternative path, usually more 
             similar to the one used in experimental papers 
             (available only for a few lattices).
             Default: .FALSE.
path_fact  : A factor that multiply the number of points along 
             each line of the default path. Note that this is 
             a real number so you can also decrease the default 
             number of points along each line.
             Default: real 1.0
filband    : file where the bands are written in the QE format.
             Default: character(len=*) 'output_band.dat'
flpband    : file(s) where the bands are written in gnuplot 
             format.
             Default: character(len=*) 'output_pband.dat'
flpsband   : postscript file with the electronic band structure.
             Default: character(len=*) 'output_band'
\end{verbatim}
%\end{tcolorbox}

Number of tasks for this option: \texttt{1}.

By default, the bands are plotted along a fixed path in the Brillouin
zone, but the user can modify this behavior giving the path at the end of 
the \texttt{INPUT\_THERMO} namelist with the same format used for 
the \texttt{pw.x} input. The automatic path generation is not available 
for base-centered monoclinic and for triclinic Bravais lattices. For these
lattices the path must be given explicitly.
The following variables control the path:

%\begin{tcolorbox}
\begin{verbatim}
q_in_band_form   : only the first and last point of each k 
                 path are given. The weight of each k point 
                 is an integer, the number of points in the 
                 line that starts at this k point.
                 Default: logical .TRUE.
q_in_cryst_coord : the k - points are given in crystal 
                 coordinates. For centered lattices the 
                 crystal coordinates refer to the primitive 
                 cell (not the conventional one). Same 
                 convention as in QE.
                 Default: logical .FALSE.
point_label_type : the label definition (see the BZ manual).
                 Default: SC
q2d              : the q points define a rectangle in 
                 reciprocal space. See the QE guide for 
                 more details.
                 Default: logical .FALSE.
is_a_path        : if .TRUE. the q points are in a path in 
                 reciprocal space. This is usually the case 
                 except when q2d=.TRUE. or when the input 
                 points are in an arbitrary order. Set this 
                 to .FALSE. only if you want to skip the plot 
                 of the bands.
                 Default: logical .TRUE.
\end{verbatim}
%\end{tcolorbox}

Note that the path is not given in the input of \texttt{pw.x} that
contain instead the information to generate the mesh of {\bf k} points 
for the self-consistent calculation. 
An example for this option can be found in \texttt{example02}.
If you give explicitly the path, be careful with options that require
geometry changes (see below). Only automatic paths, or path given through 
letter labels are easily recalculated. The other paths could turn out 
to be correct only for one geometry.

It is also possible to separate the self-consistent and the band calculation,
by running first \texttt{thermo\_pw.x} using \texttt{what='scf'} and then
running, on the same directory, \texttt{thermo\_pw.x} using 
\texttt{what='scf\_bands'}. The same input can be used in the two 
calculations, only the \texttt{thermo\_control} file need to be changed. 
The number or processors and pools can be changed in the same cases in
which this is possible in \texttt{pw.x}.
You cannot however run twice \texttt{thermo\_pw.x} on the same 
directory using \texttt{what='scf\_bands'} and two different paths, you
must use two different working directories.

\newpage
{\color{coral}\section{what='scf\_2d\_bands'}}
\color{black}
With this option the code makes a self-consistent calculation followed
by a band structure calculation as with the option 
\texttt{what='scf\_bands'}, but it assumes 
that the cell contains a slab with surfaces perpendicular to the $z$ 
direction. Therefore the two-dimensional Bravais lattice of the surface 
is identified and the default path is chosen on the two-dimensional 
Brillouin zone. 
There are three options: Plot of the projected band structure (PBS); 
plot of the bands of the slab; plot of the bands of the slab above the
projected band structure (the Fermi energies are aligned). In the first 
case the code computes several paths
of {\bf k}-points parallel to the surface (at different k$_z$) and does 
not plot the individual bands but selects the energy regions in which 
there are bulk states. The second case is similar to a standard band plot. 
The default path contains only {\bf k}-points parallel to the 
surface (with k$_z=0$). The third case assumes that the projected band 
structure has been already calculated and the information to plot it 
can be found on the file \texttt{flpbs}. For the rest it is 
similar to case two. For each direction, bands belonging to different 
irreducible representations of the point co-group of {\bf k} can be plotted 
in the same panel or on different panels.\\
This option is controlled by the following variables:

%\begin{tcolorbox}
\begin{verbatim}
lprojpbs: When .TRUE. the projected band structure (PBS) is 
     calculated if nkz > 1 otherwise it is read from file. 
     Usually this variable is .TRUE.. Set it to .FALSE. if 
     you do not want to see the PBS, or if you want to see 
     the bands of a bulk projected on the surface Brillouin 
     zone without the PBS.
     Default: logical .TRUE. (forced to .FALSE. if what is 
     not 'scf_2d_bands')
nkz: The number of k_z values used for the PBS plot. 
     If lprojpbs is .FALSE. a plot of the bulk bands projected 
     on the surface Brillouin zone is produced.
     Default: integer 4 (forced to 1 if what is not 
     'scf_2d_bands').
gap_thr: minimum size (in eV) of the gaps in the PBS.
     Default: real 0.1 eV
sym_divide: When .TRUE. the bands belonging to different 
     irreducible representations are plotted in different 
     panels. This option can be controlled by variables 
     specified in the path (see below).
     Default: logical .FALSE.
identify_sur: When .TRUE. the surface bands are searched 
     and identified on the surface band structure. 
     Default: logical .FALSE.
dump_states: If .TRUE. and identify_sur is .TRUE. dump 
     on the file 'dump/state_k_#' the planar averages of 
     the density (and in the noncollinear case also of 
     the magnetization density) of each state. One file for 
     each k point is produced and # is the number of 
     the k points. (Use with a small number of k points 
     or it might create quite large files).
     Default: logical .FALSE.
sur_layers: The number of surface layers on which we add 
     the charge density of each state to check if it is 
     a surface state.
     Default: integer 2
sur_thr: the threshold (in percentage) of the charge 
     density that must be on the surface layers to 
     identify a state as a surface state.
     Default: calculated from the actual charge density 
     values of the states.
sp_min : minimum distance between layers. Two atoms 
     form different layers only if their distance along 
     z is larger than this number. Should be smaller 
     than the interplanar distance (in a.u.) written by 
     the tool gener_3d_slab.
     Default: real 2.0 a.u.
subtract_vacuum: if .TRUE. the charge density of each 
     state on vacuum is subtracted (to remove the vacuum 
     states that are confused with surface states)
     Default: .TRUE.
force_bands: when .TRUE. the bands are plotted in any case.
     Used to plot the bulk bands on top of the PBS, 
     mainly for debugging.
     Default: logical .FALSE.
only_bands_plot: if the files with the bands, the 
     representations, the pbs and the projections are already 
     on files, this option allows to change the parameters of 
     the plot (such as the maximum energy or sur_thr) and
     do another plot without any additional calculation. If the 
     files are missing and this variable is .TRUE. an error 
     occurs.
     Default: logical .FALSE.
flpbs: the name of the file that contains the information on 
     the projected band structure.
     Default: character(len=*) 'output_pbs'
flprojlayer: the name of the file that contains the information 
     of the projection of the charge density of each state on 
     each layer. Calculated only when identify_sur is .TRUE..
     Default: character(len=*) 'output_projlayer'
\end{verbatim}

%\end{tcolorbox}

The bands and the gnuplot scripts are saved on the same files that would
be used with the option \texttt{what='scf\_bands'}. \\
Number of tasks for this option: \texttt{1}. Image parallelization is
not useful with this option. 

By default the symmetry separation is not carried out. The code plots the
bands of the slab on the same panel with a different color for each 
representation as in the bulk band structure plot (color refer to the 
representations of the slab point co-group of {\bf k}). 
In order to plot in different panels the different representations
the user can specify \texttt{sym\_divide=} \texttt{.TRUE.}.
By default this option is disabled and its use is rather tricky.
In order to use it you must indicate explicitly the
path on the two dimensional Brillouin zone using the option
\texttt{q\_in\_band\_form=.TRUE.}. Close to the
starting point of a given line you indicate the number of 
representations for that line ($0$ means all representations) and which ones.
For instance for a $(111)$ surface of an fcc metal in the direction 
$\bar \Gamma-\bar M$ you may
want to plot separately the states even or odd with respect to the mirror
plane perpendicular to the surface that contains the $\bar \Gamma-\bar M$
line.
In order to do so you can specify the path as follows:

%\begin{tcolorbox}
\begin{verbatim}
5
gG   30  0
K    30  0
M    30  1  1
gG   30  1  2
M     1  0
\end{verbatim}
%\end{tcolorbox}

The representations to plot are indicated by their numbers 
(in this case $1$ or $2$).
The number of the representation and the point co-group of each 
{\bf k}-point can be found in the output of \texttt{thermo\_pw}.
These representation numbers refer to the point co-group of each 
{\bf k}-point in the slab when you plot the slab band structures and 
to the point co-group of each {\bf k}-point in the bulk when you plot 
a PBS. Some particular values of $k_z$, such as $k_z=0$
might have a point co-group in the bulk different from the 
point co-group of a point with a generic $k_z$ but in this
case the representations are transformed into those of the smaller
group using the group-subgroup relationships and
the symmetry descent of the irreducible representations
(only when \texttt{sym\_divide=.TRUE.}). The representations of the
smaller point co-group have to be used in the PBS input.

In general, the point co-group of a {\bf k}-point 
${\bf k}=({\bf k}_\parallel, 
k_z)$ with component ${\bf k}_\parallel$ parallel to the surface 
and a generic $k_z$ in the bulk is different 
from the point co-group of a {\bf k}-point ${\bf k}=({\bf k}_\parallel, 0)$ 
in the slab. Moreover, experimentally one cannot consider symmetries
of the slab that exchange the two surfaces, and therefore the 
point co-group a {\bf k}-point ${\bf k}=({\bf k}_\parallel, 0)$ on the surface
is a subgroup of the point co-group of ${\bf k}=({\bf k}_\parallel, 0)$
on the slab. The point co-group ${\bf k}=({\bf k}_\parallel, k_z)$ 
in the bulk does not contain operations that exchange $k_z$ 
with $-k_z$
but it might be larger than the point co-group of 
${\bf k}=({\bf k}_\parallel, 0)$ on the surface because
it might contain symmetries of the bulk that require fractional translations
perpendicular to the surface that are not symmetries neither of the slab nor of
the surface.

The point co-group of a given {\bf k}-point ${\bf k}=({\bf k}_\parallel, 0)$
on the surface can be found by removing from the corresponding slab 
point co-group the
operations that exchange the two surfaces. It is also the group formed
from the intersection of the point co-group of 
${\bf k}=({\bf k}_\parallel, 0)$ in the slab and of the point 
${\bf k}=({\bf k}_\parallel, k_z)$ in the bulk.
It is the user responsibility to specify the same number of panels
for the PBS and for the slab calculation and to assure that the
representations plotted in each panel correspond to each other.
Returning to the example of the $(111)$ surface of an fcc, in the direction
$\bar \Gamma-\bar K$ the slab has $C_2$ symmetry about the $x$-axis, a symmetry
that the surface has not. Therefore you can plot with two different colors
the bands that belong to the $A$ or $B$ representations of the slab,
(states even or odd with respect to a $180^\circ$ rotation about
the $x$ axis, an operation that exchanges the two surfaces)
but you cannot separate the PBS into even or odd states with respect to 
the $C_2$ symmetry. You might specify two different panels with the
$A$ or $B$ bands in each, but the PBS in the two panels will be the same. 
On the contrary, for a {\bf k}-point along the $\bar \Gamma-\bar M$ direction, 
the point co-group has the $C_s$ symmetry both for the slab and for
the surface, so you can separate both the PBS 
and the surface states in two different panels.

There is no input variable to control or change the colors or style of the 
plot. To change the defaults you can modify directly the \texttt{gnuplot} 
script, it is written in such a way that a change of a few variables can
control the entire plot.

\newpage
{\color{coral}\section{what='scf\_dos'}}
\color{black}
With this option the code makes a self-consistent calculation followed
by a band structure calculation on a uniform mesh of {\bf k}-points and
computes and plots the electronic density of states. \\
This option does not use the image parallelization and should be used with
one image. \\
This option is controlled by the following variables:

%\begin{tcolorbox}
\begin{verbatim} 
deltae        : energy interval for electron dos plot (in Ry).
                Default: real 0.01 Ry.
ndose         : number of energy points in the dos plot.
                Default: determined from previous data
nk1_d, nk2_d, nk3_d : thick mesh for dos calculation.
                Default: integer 16, 16, 16
k1_d, k2_d, k3_d : the shift of the k point mesh.
                Default: integer 1, 1, 1
sigmae         : the smearing used for dos calculation (in eV).
                If 0.0 uses the degauss of the electronic 
                structure calculation in metals and 0.01 Ry 
                in insulators.
                Default: real 0.0 
legauss        : When .TRUE. computes the electronic dos using 
                a gaussian smearing. When .false. uses the same 
                smearing of the electronic structure calculation 
                in metals or gaussian smearing in insulators.
                Default: logical .false.
fleldos        : name of the file that contains the electron dos 
                data.
                Default: character output_eldos.dat
flpseldos      : name of the postscript file that contains the 
                electron dos picture.
                Default: character output_eldos
fleltherm      : name of the file that contains the electron 
                thermodynamic data.
                Default: character output_eltherm.dat
flpseltherm    : name of the postscript file that contains the 
                plot of the electron thermodynamic quantities.
                Default: character 'output_eltherm'
\end{verbatim}
%\end{tcolorbox}

The minimum and maximum energy, as well as the number of bands,
are specified as with the option \texttt{what='scf\_bands'}.
However with the present option no energy shift is applied to the bands
and the minimum and maximum energies refer to the unshifted eigenvalues.
Note that after a calculation with \texttt{what='scf\_dos'} you can
run the tool code \texttt{epsilon\_tpw.x} to evaluate the frequency
dependent dielectric constant (for insulators only). \\
With this option, in the metallic case, 
the code computes the electronic thermodynamic quantities of
a gas of independent electrons whose energy levels give the calculated 
density of states and produces a postscript file
with the electronic excitation energy, free energy, entropy,
and constant strain heat capacity as a function of temperature. 
The zero of the electron energy is the energy at the smallest temperature 
required in input when it is lower than $4$ K or $4$ K. \\

Number of tasks for this option: 1.

\newpage
{\color{coral}\section{what='plot\_bz'}}
\color{black}
With this option the code writes a script to make a plot
of the Brillouin zone (BZ) and of the path (the default one or the 
one given in input). 
The script must be read by the \texttt{asymptote} code, available at 
\texttt{http://asymptote.sourceforge.} \texttt{net/}. In many Linux 
distributions
this code is available as a separate package, but it is not installed by
default. \\
The following variables control the plot:

%\begin{tcolorbox}
\begin{verbatim}
lasymptote  : if .TRUE. asymptote is called by the program and 
              the pdf file with the plot of the BZ is produced.
              Default: logical .FALSE.
flasy       : initial part of the name of the file where the 
              asymptote script is written and of the name of the 
              pdf file.
              Default: character(len=*) 'asy_tmp'
asymptote_command  : the command that invokes asymptote and 
              produces the pdf file of the BZ.
              Default: character(len=*) 'asy -f pdf -noprc 
              flasy.asy'
npx         : used only in the monoclinic cell, this parameter 
              is needed to determine the shape of the Brillouin 
              zone. The default value is usually large enough, 
              but for particular shapes of the monoclinic 
              Brillouin zone it could be small. If the code 
              stops with an error asking to increase npx, double 
              it until the error disappears.
              Default: integer 8
\end{verbatim}
%\end{tcolorbox}


Starting from version \texttt{1.5.0} \thermo\ writes also a script to
plot the BZ using the \texttt{freeCAD} software. In order to plot
the BZ you need to download the code (from the site
\texttt{https://www.freecadweb.org/}, version \texttt{0.18} or higher)
and, after opening it, choose 
\texttt{Macro/Macros} to execute the \texttt{freeCAD} macro
that you find in the working directory (the default name is 
\texttt{tpw\_freecad.} \texttt{FCMacro}). 
In the \texttt{Doc} directory there is a file called \texttt{tpw\_bz.svg} 
that contains the template for the PDF file with the BZ plot. This file
should be copied in the directory where you run the \texttt{freeCAD} 
executable before running the script.
In the same directory there is a file called \texttt{brillouin\_view.cam} 
in the \texttt{Doc} directory that can be used with the option 
\texttt{View/Freeze Display} to obtain the standard view of the Brillouin zone.
This feature is still experimental.\\
The following variables control the plot:

%\begin{tcolorbox}
\begin{verbatim}
fcfact      : factor used to convert from the 2\pi/a units used 
              for the Brillouin zone plot to the millimeters 
              units needed to make a plot in freecad. 
              Default: real 100.0
fc_red,     
fc_green,
fc_blue     : color of the BZ in the rgb format (between 0.0 
              and 1.0). By default the BZ will be yellow.
              Default: real 1.0, 1.0, 0.0
fc_transparency : transparency of the BZ between 0 (opaque) 
              and 100 (transparent).
              Default: integer 15.
freecadfile : name of the file that contains the freecad 
              script to plot the Brillouin zone.
              Default: character(len=*) 'tpw_freecad'
\end{verbatim}
%\end{tcolorbox}

The structure of the solid can be seen using the \texttt{XCrySDen} code
reading the input file of \texttt{pw.x}. You can find the code at
\texttt{http://www.xcrysden.org/}. \thermo\ produces also a file in the
\texttt{xsf} format called \texttt{prefix.xsf}, where the variable 
\texttt{prefix}
is given in the input of \texttt{pw.x}. This can be useful when
the nonequivalent atomic positions and the space group are given in the input
of \texttt{pw.x}. To see an \texttt{xsf} file, give the command
\texttt{xcrysden --xsf file.xsf}.

With this option the code produces also a file with the X-ray
powder diffraction intensities for the solid. A plot shows the 
scattering angles and the relative intensity of each peak. Note 
that this plot is made using 
a superposition of atomic charges, not the self-consistent charge.
By setting the flag \texttt{lformf=.TRUE.} the atomic form factors of all 
the atomic types used to calculate the intensities are plotted. 
By setting the flag \texttt{lxrdp=.TRUE.} the intensities plot is done also 
after the cell optimization and after a self-consistent calculation 
for the options that support it.
The variables that control these plots are:

%\begin{tcolorbox}
\begin{verbatim}
lambda      : The X-ray wavelength (in A) used to calculate the 
              scattering angles.
              Default: Cu alpha line 1.541838 A if lambda_element 
              is empty
lambda_elem : The anode element, used to set the X-ray 
              wavelength. Supported elements 'Cr', 'Fe', 'Co', 
              'Cu', 'Mo'. NB: lambda must be zero to use 
              lambda_elem, otherwise the value of lambda given 
              in input is used.
              Default: character(len=2) ' ' 
flxrdp      : name of the file where the scattering angles and 
              intensities are written.
              Default: character 'output_xrdp.dat'
flpsxrdp    : name of the postscript file with the X-ray 
              diffraction spectrum.
              Default: character 'output_xrdp'
lxrdp       : if .TRUE. compute the xrdp also after the cell 
              optimization with all the options mur_lc_... 
              with the uniformly strained atomic positions 
              and after the scf calculation if supported by 
              the option. 
              Default: logical .FALSE.
lformf      : if .TRUE. plot also the form factor of each 
              atom type present in the solid. Note that the 
              atom type is recognized from the atom name in 
              the thermo_pw input. The name must coincide 
              with the symbols in the periodic table. (Cu, H, 
              Li, Li1, ... are correct, CU, LI, H1 ...  are 
              wrong).
              Default: logical .FALSE.
smin        : minimum value of s used in the atomic form 
              factor plot.
              Default: real 0.0
smax        : maximum value of s used in the atomic form 
              factor plot.
              Default: real 1.0
nspoint     : number of points in which the atomic form 
              factor is calculated.
              Default: integer 200
lcm         : when .TRUE. the code uses the Cromer-Mann 
              coefficients form the International Tables of 
              Crystallography to compute the atomic form 
              factors, otherwise uses the Doyle-Turner or 
              Smith-Burge parameters.
              Default: logical .FALSE.
flformf     : name of the file in which the atomic form 
              factor is written. The code adds a number to 
              each file name and creates a file per atom type.
              Default: character 'output_formf.dat'
flpsformf   : name of the postscript file with the atomic 
              form factor. The code adds a number to each 
              file name and creates a file per atom type.
              Default: character 'output_formf'
\end{verbatim}
%\end{tcolorbox}

\newpage
{\color{coral}\section{what='scf\_ph'}}
\color{black}
With this option the code makes a self-consistent calculation followed
by a phonon calculation. The phonon calculation is controlled by the file
\texttt{ph\_control} and can be at a single {\bf q} point or on a mesh of 
{\bf q} points. 
The different representations are calculated in parallel when several images 
are available. No other input variable is necessary. The outputs of this 
calculation are the dynamical matrices files. \\
\texttt{thermo\_pw} adds to the \texttt{ph.x} code the ability to
compute the complex dielectric constant tensor of insulators as a function 
of a complex frequency for the study of optical properties within 
time-dependent density functional perturbation theory (TD-DFPT). 
The code produces also the complex index of refraction for all
systems except monoclinic and triclinic. For cubic solids it makes
also a plot of the reflectivity for normal incidence and of the
adsorption coefficient.
As a default the TD-DFPT algorithm uses the Sternheimer equation and 
a self-consistent loop, but it is also possible to use a Lanczos chain.
The option is activated in the \texttt{ph.x} input by setting 
\texttt{epsil=.TRUE.} and \texttt{fpol=.TRUE.}, but at variance with
the \texttt{ph.x} code, the frequencies must be specified as complex numbers.
The following additional variables can be put in the input of the \texttt{ph.x}
code, to select the frequency range and the number of frequencies to compute:

%\begin{tcolorbox}
\begin{verbatim}
freq_line : if this variable is .TRUE., after the FREQUENCY 
          keyword the code expects the number of frequency 
          points and the starting and final frequencies. 
          If .FALSE. the number of frequencies and a list of 
          frequencies are given. The frequencies are complex 
          numbers and are given with a real and an imaginary 
          part (in Ry), without parenthesis.
          Default: .FALSE.
delta_freq : When freq_line is .TRUE. instead of giving the 
          last frequency of the line one can give the distance 
          between two frequency points delta_freq as a complex 
          number. The last point of the line is calculated using 
          the number of frequencies nfs and the first frequency. 
          When delta_freq is not zero the last frequency is not 
          used and can be omitted.
          Default: complex, (0.0, 0.0).
start_freq : Number of the initial frequency calculated in the 
          job in the sequence of frequencies.
          Default: integer  1
last_freq : Number of the final frequency calculated in the job 
          in the sequence of frequencies.
          Default: integer nfs (total number of frequencies)
lfreq_ev  : If .TRUE. the units of the frequencies are eV 
          instead of the default Ry units.
          Default: logical .FALSE.
linear_im_freq: This option is used only when freq_line=.TRUE. 
          When linear_freq_im is .TRUE., the imaginary part 
          of each frequency is calculated as eta * freq where 
          eta is the imaginary part of the first frequency 
          on the frequency line.
          Default: logical .FALSE.
llanczos: When this flag is .TRUE. at finite frequencies a 
          Lanczos algorithm is used to solve the linear 
          system. Can be very fast but might require much 
          more memory than the standard algorithm. Presently 
          it is incompatible with images.
          Default: .FALSE.
lanczos_steps: steps of the Lanczos chain.
          Default: interger 2000
lanczos_steps_ext:  steps of the extrapolated Lanczos chain
          Default: integer 10000
lanczos_restart_steps: number of steps between saving of the 
          Lanczos status. If 0 the status is saved only at the 
          end of the run. Use recover=.TRUE. to resume an 
          interrupted Lanczos chain or to increase the 
          number of steps.
          Default: integer 0
extrapolation : extrapolation type. Presently only 'no' or 
          'average' are available. In the first case no 
          extrapolation is applied, in the second the average 
          of the beta and gamma is used.
          Default: character 'average'
pseudo_hermitian : when .TRUE. a pseudo-hermitian algorithm is 
          used to make the Lanczos steps. Should be twice 
          faster than the default non hermitian algorithm.
          Default: .TRUE.
only_spectrum : Computes only the spectrum assuming that 
          the Lanczos chain coefficients are in a file. It 
          gives error if the number of requested Lanczos 
          steps is larger than those available on file.
          Default: logical .FALSE.
lcg:      When this flag is .TRUE. a global conjugate 
          gradient algorithm is used to compute the 
          dielectric constant and the phonon frequencies. 
          It will not require mixing, but will use more 
          memory than the standard algorithm (for insulators 
          only). It is not available for the frequency 
          dependent case.
          Default: .FALSE.
\end{verbatim}
%\end{tcolorbox}

When a non zero wave-vector {\bf q} is specified in the input of the phonon
code, the previous options produce the inverse of the dielectric 
constant as a function of the frequency at the wave-vector {\bf q}
(this option can be used both for insulators and metals). \\
Additional variables can be specified in the \thermo\ input to control
where the frequency dependent dielectric constant is written and plotted
and how the work is divided among images:

%\begin{tcolorbox}
\begin{verbatim}
flepsilon : beginning of the name of the file where the 
          frequency dependent dielectric constant is 
          written at finite q (the code adds the 
          extensions _re and _im).
          Default: character(len=*) 'epsilon'
flpsepsilon : name of the postscript file where the frequency 
          dependent dielectric constant is plotted at finite q.
          Default: character(len=*) 'output_epsilon'
floptical : beginning of the name of the file where the 
          frequency dependent dielectric constant and the 
          complex index of refraction are written (the code 
          adds the extensions _xx, _yy, and _zz for the solids 
          that need to distinguish the different directions).
          Default: character(len=*) 'optical'
flpsoptical : name of the postscript file where the frequency 
          dependent dielectric constant, the complex index of 
          refraction and for cubic solid also the reflectivity 
          and the absorption coefficient are plotted.
          Default: character(len=*) 'output_optical'
omega_group : number of frequencies calculated together by 
          each image. This variable is used only with images.
          Default: integer 1.
\end{verbatim}
%\end{tcolorbox}

An example for this option can be found in \texttt{example03},
\texttt{example16}, \texttt{example17}, \texttt{example20}, and 
\texttt{example21}. \\
Number of tasks for this option: for a phonon calculation the
number of parallelizable tasks of the 
phonon code (smaller but of the order of the number of {\bf q} points times 
$3 N_{at}$, where $N_{at}$ is the number of atoms in the unit cell), 
for a dielectric constant calculation using Sternheimer equation
\texttt{nfs/omega\_group}, number of frequencies divided by the number of 
frequencies in each group, for a dielectric constant calculation
using Lanczos $1$ (images not allowed). \\

It is also possible to separate the self-consistent and the phonon calculation,
by running first \texttt{thermo\_pw.x} using \texttt{what='scf'} and then
running, on the same directory, \texttt{thermo\_pw.x} using 
\texttt{what='scf\_ph'}. The same input can be used in the two 
calculations, only
the \texttt{thermo\_control} file need to be changed.
The number or processors/pools/images can be changed in the same cases in
which this is possible in Quantum ESPRESSO.

Using images in a phonon calculation with the master/slave approach
has an overhead because each image must recalculate the initialization
and the band structure at each task, or check if the bands are 
already on disk, calculated previously by the same image. On some systems 
with slow disks it could be faster to recalculate the bands instead of 
reading them from disk. It is also possible to use the image breaking
suggested by the \texttt{ph.x} code that keeps, as much as possible, 
on the same image the tasks that require the same initialization without
recomputing it.
The input variables that control this part of the calculation are:

%\begin{tcolorbox}
\begin{verbatim}
force_band_calculation : if .TRUE. the bands are never read 
              from disk but recalculated when needed.
              Default: logical .FALSE.
use_ph_images : if .TRUE. each image makes a set of tasks so 
              as to minimize the number of band calculations 
              and phonon initialization.
              Default: logical .FALSE. if nimage>1 .TRUE. 
              nimage=1.
sym_for_diago : When .TRUE. use symmetry to calculate the 
              bands and the unperturbed wavefunctions instead 
              of diagonalizing the Hamiltonian.
              Default: logical .FALSE.
\end{verbatim}
%\end{tcolorbox}

\newpage
{\color{coral}\section{what='scf\_disp'}}
\color{black}
With this option the code makes a self-consistent calculation followed by
a phonon dispersion calculation at a fixed geometry. The geometry is given in the 
input of \texttt{pw.x}. The dynamical matrices are used to calculate 
the interatomic force constants that are then used 
to calculate the dynamical matrices and hence
the phonon frequencies along a path in the Brillouin zone and on
a much thicker mesh of {\bf q} points. The path can be generated automatically 
or given in input as in a band structure calculation 
(see above \texttt{what='scf\_bands'}). The uniform mesh of {\bf q} points 
is controlled by \texttt{thermo\_} \texttt{control}. The code uses the phonon
frequencies calculated on the thick mesh of {\bf q} points to get 
the phonon density of states using a smearing approach. The density of 
states is used to calculate the harmonic thermodynamic properties: 
vibrational energy, free energy, entropy, and 
constant strain heat capacity.
The same thermodynamic quantities are calculated also by direct 
integration over the Brillouin zone and compared in the plots.
When \texttt{with\_eigen=.TRUE.} the atomic B-factors are
calculated as a function of temperature by the generalized vibrational 
density of states or by a direct integration over 
the Brillouin zone.
Note that presently no interpolation formula is used at low temperatures
so \texttt{thermo\_pw} can not be used to obtain thermodynamic 
properties at very low temperatures. 
The extensive quantities plotted in the figures refer to an 
Avogadro number of unit 
cells. If you need them per mole you have to divide by the number 
of formula units in a unit cell. \\
The input variables that control this option are:

%\begin{tcolorbox}
\begin{verbatim}
freqmin_input : minimum frequency for phonon dos plot.
                Default: real determined from phonon frequencies
freqmax_input : maximum frequency for phonon dos plot.
                Default: real determined from phonon frequencies
deltafreq     : frequency interval for phonon dos plot.
                Default: real 1 cm^{-1}
ndos_input    : number of frequency points in the dos plot.
                Default: determined from previous data
nq1_d, nq2_d, nq3_d : thick mesh for phonon dos calculation.
                Default: integer 192, 192, 192
phdos_sigma   : the smearing used for phonon dos calculation 
                (in cm^-1).
                Default: real 2. cm^-1
idebye        : if 1, 2, or 3 the code computes the Debye 
                temperature as a function of temperature from 
                the free energy, the vibrational energy, or the 
                heat capacity respectively.
                Default: integer 0
after_disp    : if .TRUE. the dynamical matrices are supposed 
                to be already available in files in the current 
                directory. This option is needed to restart when 
                the outdir directory has been erased and ph.x 
                cannot be run without redoing the scf calculation. 
                The exact restart point depends on the files 
                already available on the current directory.
                Default: logical .FALSE.
fildyn        : the name of the dynamical matrix file, as 
                would be specified in the input of ph. To be used 
                when after_disp is .TRUE.. 
                Default: character ' '
zasr          : type of acoustic sum rule applied to the ifc.
                Default: character(len=*) 'Simple'
ltherm_dos    : if .TRUE. the thermal properties are calculated 
                from the phonon dos.
                Default: logical .TRUE.
ltherm_freq   : if .TRUE. the thermal properties are calculated 
                from the direct integration using the phonon 
                frequencies.
                Default: logical .TRUE.
flfrc         : file where the interatomic force constants are 
                written.
                Default: character(len=*) 'output_frc.dat.g1'
flfrq         : file where matdyn writes the interpolated 
                frequencies.
                Default: character(len=*) 'output_frq.dat.g1'
flvec         : file where the eigenvectors of the dynamical 
                matrix are written.
                Default: character(len=*) 'matdyn.modes'
fldosfrq      : file where the frequencies used to calculate 
                the phonon density of states are saved.
                Default: character(len=*) 'save_frequencies.dat'
fldos         : file where the phonon dos is written.
                Default: character(len=*) 'output_dos.dat.g1'
fltherm       : file where the harmonic thermodynamic
                quantities are written.
                Default: character(len=*) 'output_therm.dat.g1'
flpsdisp      : postscript file of the phonon dispersions.
                Default: character(len=*) 'output_disp'
flpsdos       : postscript file of the phonon dos.
                Default: character(len=*) 'output_dos'
flpstherm     : postscript file of the harmonic thermodynamic 
                quantities.
                Default: character(len=*) 'output_therm'
\end{verbatim}
%\end{tcolorbox}

This option requires \texttt{ldisp=.TRUE.} in the phonon input. \\
An example for this option can be found in \texttt{example04}. \\
Number of tasks for this option: number of parallelizable tasks of the 
phonon code (smaller but of the order of number of {\bf q} points times 
$3 N_{at}$, where $N_{at}$ is the number of atoms in the unit cell).

\newpage
{\color{coral}\section{what='scf\_elastic\_constants'}}
\color{black}
With this option the code calculates the elastic constants of the solid
at the geometry given in the \texttt{pw.x} input. 
There are four different algorithms that at convergence should give the
same results. In two of them, depending on the Laue class, the code 
calculates the nonzero components of the stress tensor for a set of strains
and obtains the elastic constants from the numerical first derivative
of the stress with respect to strain.
The two algorithms \texttt{standard} and \texttt{advanced} differ
only for the choice of the unit cell. In the \texttt{standard} method the
code applies the strain to the primitive vectors of the unstrained solid
and uses \texttt{ibrav=0} and the strained vectors to compute the stress
tensor.
The \texttt{advanced} method, available only for selected Bravais lattices, 
tries to optimize the calculation by choosing strains for which the number
of needed {\bf k}-points is reduced. Moreover it identifies the 
Bravais lattice of the strained solid and recalculates the primitive
vectors with the conventions of \qe. When available this should be the
most efficient method.
The other two algorithms are called \texttt{energy\_std} and \texttt{energy}.
Using the \texttt{energy\_std} or \texttt{energy} algorithm the elastic 
constants are calculated 
from a polynomial fit of the total energy as a function of strain 
without computing stress. This option usually requires more independent 
strains. It can be used when stress calculation is not implemented in \qe.
The difference between \texttt{energy\_std} and \texttt{energy} is 
the same between the algorithms that use the stress. With
\texttt{energy\_std} the code applies the strain using \texttt{ibrav=0},
while with \texttt{energy} an optimized cell is used. As the
\texttt{advanced} algorithm the \texttt{energy} algorithm is available
only for selected Bravais lattices.\\
For all methods the number of strains is \texttt{ngeo\_strain}
for each independent strain. 
For each strain, the code relaxes the ions to their equilibrium 
positions if \texttt{frozen\_ions=.FALSE.} or keeps them
in the strained positions if \texttt{frozen\_ions=.TRUE.}. 
Note that elastic constant calculations with \texttt{frozen}
\texttt{\_ions=.FALSE.}
might require smaller force convergence threshold than standard calculations. 
The default value of \texttt{forc\_conv\_thr} must be changed in the 
\texttt{pw.x} input.
At finite pressure all methods give the elastic constants that
relate linearly stress and strain. \\
The input variables that control this option are:

%\begin{tcolorbox}
\begin{verbatim}
frozen_ions: if .TRUE. the elastic constants are calculated 
             keeping the ions frozen in the strained positions. 
             Default: logical .FALSE.
ngeo_strain: the number of strained configurations used to 
             calculate each derivative. 
             Default: integer 4 ('standard' and 'advanced'), 
             6 ('energy')
elastic_algorithm: 'standard', 'advanced', 'energy_std' or 
             'energy'. See discussion above.
             Default: character 'standard' 
delta_epsilon: the interval of strain values between two 
             geometries. To avoid a zero strain geometry 
             that might have a different symmetry ngeo_strain 
             must be even.
             Default: real 0.005
epsilon_0:   a minimum strain. For small strains the ionic 
             relaxation routine requires a very small threshold 
             to give the correct internal relaxations and 
             sometimes fail to converge. In this case you 
             can increase delta_epsilon, but if delta_epsilon 
             becomes too large you can reach the nonlinear 
             regime. In this case you can use a small 
             delta_epsilon and a minimum strain (To be used 
             only for difficult systems).
             Default: real 0.0
poly_degree: degree of the polynomial used to interpolate 
             stress or energy. ngeo_strain must be larger 
             than poly_degree+1.
             Default: 3 ('standard', 'advanced', 2 
             if ngeo_strain < 6), 4 ('energy', 3 if 
             ngeo_strain < 6).
fl_el_cons:  the name of the file that contains the elastic 
             constants.
             Default: character(len=*) 'output_el_con.dat'
\end{verbatim}
%\end{tcolorbox}

The three algorithms are equivalent only at convergence both with
{\bf k}-point sampling and with the kinetic energy cut-off, but 
large differences between the elastic constants obtained with the 
\texttt{standard} and \texttt{advanced} algorithms might point to 
insufficient {\bf k}-point sampling. Large differences between the 
elastic constants obtained with the \texttt{energy\_std} or \texttt{energy}
algorithms with respect 
to the other two might point to insufficient kinetic energy cut-off. \\
Number of tasks for this option: \texttt{ngeo\_strain} times the number of
independent strains. \\

Using the elastic constants tensor the code can calculate and print
a few auxiliary quantities:
the bulk modulus, the poly-crystalline averages of the Young modulus,
of the shear modulus, and of the Poisson ratio. Both the Voigt and the
Reuss averages are printed together with the Hill average.
The Voigt-Reuss-Hill average of the shear modulus and of the bulk modulus are 
used to compute average sound velocities. The average of the Poisson ratio and
the bulk modulus allow the estimation of the Debye 
temperature. The Debye temperature is calculated also with the exact 
formula evaluating the average sound 
velocity from the angular average of the sound velocities calculated 
for each propagation direction solving the Christoffel wave equation.
The exact Debye temperature is used within the Debye model to calculate the
Debye's vibrational energy, free energy, entropy, and constant strain heat
capacity. These quantities are plotted in a postscript file as a function
of temperature.


%{\color{coral}\section{\sc what='scf\_piezoelectric\_tensor'}}
%\color{black}
%With this option the code calculates the piezoelectric tensor 
%($g_{\alpha,m}$ $\alpha=1,3,\ m=1,6$) of the solid.
%Depending on the point group, it calculates the polarization of the
%solid for a set of strains of the lattice. For each strain, it relaxes 
%the ions to their equilibrium positions when \texttt{frozen\_ions=.FALSE.} 
%or keep them in the strained positions when \texttt{frozen\_ions=.TRUE.}. 
%Finally it computes the piezoelectric tensor from the numerical derivatives 
%of the polarization with respect to strain.
%The number of strains used to compute each derivative 
%is \texttt{ngeo\_strain}.\\
%The input variables
%that control this option are the same as those used for the elastic constant:
%\begin{verbatim}
%frozen_ions: if .TRUE. the piezoelectric tensor is calculated 
%             keeping the ions frozen in the strained positions. 
%             Default: logical .FALSE.
%ngeo_strain: the number of strained configurations used. 
%             Default: integer 4
%nppl:        the number of k-points per line in Berry phase calculations
%             Default: integer 51
%delta_epsilon: the interval of strain values between two geometries.
%             To avoid a zero strain geometry that might have a
%             different symmetry ngeo_strain must be even.
%             Default: real 0.002
%\end{verbatim}
%If a file with the elastic constants is found on disk, the code computes also
%the direct piezoelectric tensor $d_{\alpha,m}$ which describes
%the polarization induced by an external stress, as $d_{\alpha,m}=
%\sum_n g_{\alpha,n} C_{nm}^{-1}$. All these quantities are calculated 
%for a vanishing electric field. \\
%Number of tasks for this option: \texttt{ngeo\_strain} times the number of
%independent strains. \\
%This feature is still incomplete and experimental.
%

\newpage
{\color{coral}\section{what='mur\_lc'}}
\color{black}
With this option the code runs several self-consistent calculations
at different geometries. The runs can be done in parallel when several images 
are available. This option has two working modes controlled by the 
logical variable \texttt{lmurn}. When \texttt{lmurn=.TRUE.} the total energy as 
a function of the volume is interpolated by an equation of state
(Murnaghan or Birch-Murnaghan) and 
a plot of the energy as a function of the volume, of the pressure 
as a function of the volume and of the enthalpy as a function of
pressure are produced. The volume is changed by 
changing only \texttt{celldm(1)}. \texttt{celldm(2)...celldm(6)} remain fixed
at the values given as input of \texttt{pw.x} or can be read from file
using the option \texttt{lgeo\_from\_file=.TRUE.}. 
When \texttt{lmurn=.FALSE.} the energy is calculated in a uniform
grid of parameters composed of \texttt{ngeo(1)} $\times$ \texttt{ngeo(2)...}
$\times$ \texttt{ngeo(6)} points.
The energies are fitted with a quadratic or quartic polynomial of
$N_k$ variables, where $N_k$ is the number of independent crystal
parameters for the given crystal system. A plot of the energy as 
a function of the lattice constant is produced for cubic systems.
For solids of the hexagonal, tetragonal, and trigonal systems
contour plots of the energy as a function of the two crystal parameters
($a$ and $c/a$ or $a$ and $\cos\alpha$)
are plotted. For orthorhombic systems contour plots of the energy as a function
of $a$ and $b/a$ are plotted for each value of $c/a$. 
Presently no graphical tool is implemented to plot the energy
for monoclinic and triclinic crystal systems. However in all cases
the enthalpy as a function of pressure is shown. Moreover
separate plots show the crystal parameters as well as the volume as 
a function of pressure. When in the directory \texttt{elastic\_constants}
there are the elastic constants for each geometry (calculated by the option
\texttt{what=elastic\_constants\_geo}), these are interpolated
at the pressure dependent crystal parameters and shown on output.
Using the input variable \texttt{lgeo\_to\_file=.TRUE.} the code writes
on file the crystal parameters that for each \texttt{celldm(1)} 
of the grid of crystal structures give a uniform pressure.
When \texttt{lmurn=.FALSE.} the bulk modulus
is not calculated. To obtain it, you can calculate the elastic constants at the
minimum geometry (see the option \texttt{what='mur\_lc\_elastic\_constants'}). 
With this option the pressure control is active. You can specify a 
finite pressure and the enthalpy is minimized instead of the
energy. Note however that if the minimum is distant from the starting
configuration its associated error can be large, larger for the
quadratic than for the Murnaghan interpolation. For this
reason the present option should be used starting from the minimum found by
\texttt{pw.x} using the \texttt{vc-relax} option and the pressure 
should not be too different from the pressure used for \texttt{vc-relax}.
Note that with this option the atomic coordinates are relaxed at each
geometry even if you specified \texttt{calculation='scf'} in the 
\texttt{pw.x} input. Use \texttt{frozen\_ion=.TRUE.} if you want to keep
them fixed. To increase the maximum number of ionic iterations use
\texttt{calculation='relax'} and give \texttt{nstep} (otherwise the default 
is $20$). Only the \texttt{bfgs} relaxation is supported by this
option.
When \texttt{lel\_free\_energy=.TRUE.} the code makes also an electronic bands
dos calculation at each geometry, computes the electronic thermodynamic
quantities as a function of temperature and writes them in separate files.
These files can be used to add the electronic contribution to the
anharmonic properties with the option \texttt{mur\_lc\_t}.\\
This option can be controlled by the following variables:

%\begin{tcolorbox}
\begin{verbatim}
ngeo(1),...,ngeo(6) : the number of geometries to use for each 
             celldm parameter. The lattice constant of these 
             geometries is calculated from the input of pw.x. 
             celldm(1),...,celldm(6) of this input is used 
             for the central geometry. For the others celldm(1),
             ...,celldm(6), are changed in steps of step_ngeo(1),
             ...,step_ngeo(6). ngeo(1) must be odd. Only the 
             values of celldm relevant for each Bravais lattice 
             are actually changed.
             Default: integer 1,1,1,1,1,1 for what=scf_*, 
             9,1,1,1,1,1 for what=mur_lc_* and lmurn=.TRUE. or 
             for cubic systems, 5 on all the relevant celldm 
             parameters when lmurn=.FALSE. and the system
             is not cubic.
step_ngeo(1),...,step_ngeo(6) : The step between the lattice 
             constants at different geometries. step_ngeo(1) is, 
             in atomic units, the change of a, step_ngeo(2), 
             step_ngeo(3) are dimensionless and are the changes 
             of the ratios b/a, c/a, step_ngeo(4), step_ngeo(5), 
             step_ngeo(6) are the changes in degree of the 
             angles alpha, beta, and gamma. The cosine of the 
             angle is calculated by the program.
             Default: real 0.05 a.u., 0.02, 0.02, 0.5, 0.5, 0.5
lmurn       : if .TRUE. the fit with an equation of state
             is done. Only ngeo(1) values of the energy are 
             fitted, the other values 
             of ngeo are not used. if .FALSE. use a quadratic 
             or quartic function to interpolate the energy as 
             a function of all celldm parameters. The number of 
             self-consistent calculations is ngeo(1) x ngeo(2) 
             x ngeo(3) x ngeo(4) x ngeo(5) x ngeo(6). In this 
             case only the minimum energy and the optimal celldm 
             are given in output. 
             Default: .TRUE. 
ieos        : choose the equation of state to use (only when 
              lmurn=.TRUE.):
              1 - Birch-Murnaghan third order
              2 - Birch-Murnaghan fourth order
              4 - Murnaghan
             Default: integer 4
show_fit   : if .TRUE. show the contour plot of the fitted 
             energy instead of the energy. Used by default 
             when reduced_grid is .TRUE..
             Default: logical .FALSE.
frozen_ions: if .TRUE. the atomic coordinates are obtained by 
             applying the strain to the coordinates given in the 
             pw.x input to the new cell parameters (equivalent to 
             keep the crystal coordinates fixed) and kept fixed. 
             If .FALSE. the atomic coordinates are relaxed at 
             each geometry.
             Default: logical .FALSE.
vmin_input : minimum volume for the plot of the energy as a 
             function of volume.
             Default: real 0.98 times the volume of the first 
             geometry.
vmax_input : maximum volume for the plot of the energy as a 
             function of volume.
             Default: real 1.02 times the volume of the last 
             geometry.
deltav     : distance between two volumes in the plot of the 
             energy as a function of the volume.
             Default: real calculated from nvol.
nvol       : number of volumes in equation of state plot.
             Default: integer 51
lquartic   : if .TRUE. fit the energy with a quartic polynomial.
             Default: logical .TRUE.
lsolve     : choose the algorithm used to fit the quartic 
             polynomial parameters.
             Allowed values:
             1 explicitly minimize chi^2 (usually less accurate 
             than the other two. Should be used only for tests).
             2 Use the QR algorithm to minimize chi^2 (lapack 
             routine dgels) 3 Use the SVD algorithm to minimize 
             chi^2 (lapack routine dgelss).
             Default: integer 2
flevdat    : file where the equation of state is written. The 
             results of the fit are then written in 
             flevdat.ev.out.
             Default: character(len=*) 'output_ev.dat'
flpsmur    : postscript file of the equation of state plot.
             Default: character(len=*) 'output_mur'
lel_free_energy : if .TRUE. computes the electronic thermodynamic 
             properties (energy, free energy, entropy, and constant 
             strain heat capacity) at each temperature and plots 
             them. See the scf_dos option for the parameters that 
             control the calculation.
             Default: .FALSE.
ncontours  : the number of contours in the energy plot. These 
             levels can be determined automatically by the code 
             or defined by the user. The energy levels can be 
             defined after the INPUT_THERMO namelist but before 
             the path, as a list:
             energy_level(1)      color(1)
             ...
             energy_level(ncontours)   color(ncontours) 
             Color is a string of the type color_red, color_green, 
             etc.
             The list of available colors is at the beginning of 
             each gnuplot script. energy_level is in Ry units.
             Default: integer 9
do_scf_relax : if .TRUE. the code makes a self-consistent relax 
             calculation at the equilibrium geometry to find 
             the optimized atomic coordinates. This step is 
             needed only for solids that have internal degrees 
             of freedom in the unstrained configuration. 
             If .FALSE. the coordinates of the input geometry 
             are strained uniformly to the equilibrium geometry.
             Default: logical .FALSE. 
lgeo_from_file : if .TRUE. the input geometries are read from file.
             ngeo(1) must have the total number of geometries 
             and lmurn must be .TRUE..
             Default : .FALSE.
lgeo_to_file : if .TRUE. at the end of the calculation the code
             writes in a file the geometries that correspond to
             the optimized crystal parameters for each value of 
             celldm(1) of the grid of geometries.
             Default : .FALSE.
flenergy   : name of the file that contains the energy in a 
             form that can be used by gnuplot to make contour 
             plots.
             Default: character(len=*) 'output_energy'
flgeom     : name of the file that contains the geometries  
             requested with the flags lgeo_to_file or 
             lgeo_from_file. The file is in the directory 
             energy files.
             Default: character(len=*) 'output_geometry'
flpsenergy : file with the contour plots of the energy as a 
             function of the crystal parameters.
             Default: character(len=*) 'output_energy'
\end{verbatim}
%\end{tcolorbox}

An example for this option can be found in \texttt{example05}.\\
Number of tasks for this option: \\
\texttt{ngeo(1)} when \texttt{lmurn=.TRUE.}, \\
\texttt{ngeo(1)}$\times$\texttt{ngeo(2)}$\times$\texttt{ngeo(3)}$\times$\texttt{ngeo(4)}$\times$\texttt{ngeo(5)}$\times$\texttt{ngeo(6)} when 
\texttt{lmurn=.FALSE.}. 

\newpage
{\color{coral}\section{what='mur\_lc\_bands'}}
\color{black}
With this option the code computes the band structure at the geometry 
that minimizes the energy (or the enthalpy at finite pressure). See  
\texttt{what='scf\_bands'} and \texttt{what='mur\_lc'} for a list of 
the variables that control these two options. 
An example for this option can be found in \texttt{example06}. \\
Number of tasks for this option: see \texttt{what='mur\_lc'} and
\texttt{what='scf\_bands'}.

\newpage
{\color{coral}\section{what='mur\_lc\_dos'}}
\color{black}
With this option the code computes the electronic dos at the geometry 
that minimizes the energy (or the enthalpy at finite pressure). See  
\texttt{what='scf\_dos'} and \texttt{what='mur\_lc'} for a list of 
the variables that control these two options.\\
Number of tasks for this option: see \texttt{what='mur\_lc'} and 
\texttt{what='scf\_dos'}.

\newpage
{\color{coral}\section{what='mur\_lc\_ph'}}
\color{black}
This option is similar to \texttt{what='scf\_ph'} but the phonon calculation
is made at the geometry that minimizes the energy (or the enthalpy
at finite pressure).
See \texttt{what='scf\_ph'} and \texttt{what='mur\_} \texttt{lc'} for a
list of the variables that control these options. 
An example for this option can be found in \texttt{example07}. \\
Number of tasks for this option: Maximum between the number of tasks 
needed by the \texttt{what='mur\_lc'} option and the number
of tasks of the phonon code (see above the option \texttt{what='scf\_ph'}).

\newpage
{\color{coral}\section{what='mur\_lc\_disp'}}
\color{black}
This option is similar to \texttt{what='scf\_disp'} but the phonon calculation
is made at the geometry that minimizes the energy (or the enthalpy
at finite pressure).
See \texttt{what='scf\_disp'} and \texttt{what='mur\_lc'} for a
list of the variables that control these options. 
An example for this option can be found in \texttt{example08}. \\
Number of tasks for this option: Maximum between the number of tasks  
needed by the \texttt{what='mur\_lc'} option and the number
of tasks of the phonon code (see above the option \texttt{what='scf\_ph'}).

\newpage
{\color{coral}\section{what='mur\_lc\_elastic\_constants'}}
\color{black}
As \texttt{what='scf\_elastic\_constants'} but the calculation is made at the
geometry that minimizes the energy (or the enthalpy at finite pressure). 
See \texttt{what='scf\_elastic\_constants'} and \texttt{what='mur\_lc'} for 
a list of the variables that control these two options.
An example for this option can be found in \texttt{example13}. \\
Number of tasks for this option: Maximum between the number of tasks
needed by the \texttt{what='mur\_lc'} option and the number of tasks
needed for the \texttt{what='scf\_elastic\_constants'} option.

%{\color{coral}\section{\sc what='mur\_lc\_piezoelectric\_tensor'}}
%\color{black}
%As \texttt{what='scf\_piezoelectric\_tensor'} but the calculation is made at the
%geometry that minimizes the energy. The
%energy minimization is made as described for \texttt{what='mur\_lc'}. \\
%This feature is still incomplete and experimental.

\newpage
{\color{coral}\section{what='mur\_lc\_t'}}
\color{black}
With this option the code calculates the anharmonic 
properties within the quasi-harmonic approximation. 
The outputs of the code are the values of crystal parameters 
(\texttt{celldm}) as a function of temperature. This calculation is done 
by computing the phonon dispersions on all the geometries specified as in
\texttt{what='mur\_lc'} (or on a subset of these geometries) and 
minimizing the Helmholtz free energy.
Separate plots of the phonon dispersions are obtained for all the 
calculated geometries.
For each geometry the code produces also plots of the phonon density
of states and of the harmonic thermodynamic quantities.
From \texttt{celldm} as a function of temperature the code computes the thermal
expansion tensor, the volume, and the volume thermal expansion as a 
function of temperature. 
The frequencies at all the calculated geometries are interpolated
by quadratic or quartic polynomials of the crystal parameters
and can be shown at crystal parameters given in input or at those that
minimize the free energy at a temperature given in input. The 
interpolated frequencies  
are shown on the same path used for the phonon dispersions.
In addition to the frequencies the code produces also several plots of
the derivatives of the frequencies with respect to the crystal parameters
multiplied by the crystal parameters. \\
When an equation of state is used to interpolate the
Helmholtz free energy (\texttt{lmurn=.TRUE.}), in addition to the volume, 
the bulk modulus and the pressure derivative of the bulk modulus are 
plotted as a function of 
temperature. Moreover the isobaric heat capacity, the isoentropic 
bulk modulus, and the average Gr\"uneisen parameter are calculated as 
a function of temperature.
The mode Gr\"uneisen parameters are calculated with cubic interpolations of the
phonon frequencies. Using the variable \texttt{with\_eigen} one can 
calculate these parameters as derivatives of the phonon frequencies 
(default) or as expectation values of the derivatives of the dynamical 
matrix on the central geometry eigenvectors (might require a lot of RAM). 
The mode Gr\"uneisen parameters are used to calculate the volume
thermal expansion and the result is compared with the volume thermal expansion
derived from the numerical derivative of the equilibrium volume obtained
from the minimization of the Helmholtz free energy. 
When the Helmholtz free energy is interpolated with a quadratic or cubic
polynomial (\texttt{lmurn=.FALSE.}), by default, the code computes only 
the temperature dependence of the lattice parameters and of the volume, the 
volume thermal expansion, and the thermal expansion tensor. However if a file 
with the elastic constants is found in the \texttt{elastic\_constants}
directory 
and \texttt{lb0\_t=.FALSE.} the bulk modulus is calculated and 
assumed independent from the temperature so that also the isobaric specific 
heat, the isoentropic bulk modulus, and the average Gr\"uneisen parameter 
are calculated as a function of temperature. The derivatives of the 
frequencies with respect to the crystal parameters are used to calculate 
the thermal expansion tensor which 
is compared with that obtained from the numerical derivatives of the 
crystal parameters obtained from the minimization of the Helmholtz free energy.
If many files with the elastic constants, one for each geometry, as
produced with the option \texttt{elastic\_constants\_geo}, are found
in the \texttt{elastic\_constants} directory and \texttt{lb0\_t=.TRUE.}, the bulk modulus and 
the elastic constants
are computed as a function of temperature within the ``quasi-static
approximation" and are used
to calculate the other thermodynamic properties. 
If one or many elastic constants files are found in the 
\texttt{anhar\_files} directory and \texttt{lb0\_t=.TRUE.} the bulk modulus 
and elastic constants are computed as a function of temperature 
within the ``quasi-harmonic approximation" and are used
to calculate the other thermodynamic properties (see a more detailed
discussion in the option \texttt{what=elastic\_constants\_geo}).
In addition to the
quantities plotted for cubic solids, the code plots also the elastic
constants and the bulk modulus as a function of the temperature and
the elastic compliances and the compressibility as a function of temperature.
The elastic constants are interpolated with a quadratic 
(\texttt{lquartic=.FALSE.}) or quartic (\texttt{lquartic=.TRUE.}) polynomial
of the crystal parameters.
Moreover the thermal stresses and the generalized average Gr\"uneisen
parameters are plotted.
These possibilities are implemented only for cubic, tetragonal, 
hexagonal, trigonal, and orthorhombic systems. \\
With this option the pressure control is active. You can specify a
finite pressure and the Gibbs energy is minimized instead of the
Helmholtz free energy. Note however that if the minimum is distant from 
the starting configuration its associated error can be large, larger for the
quadratic than for the equation of state interpolation. \\
By using the variables \texttt{ntemp\_plot} and \texttt{temp\_plot} (see
the Section Temperature and pressure) the code produces also plots
of the thermal expansion, of the bulk modulus, of the average 
Gruneisen parameter, and of the product of the thermal expansion
and the bulk modulus as a function of pressure. These plots contain
several lines one for each chosen temperature.
By using the variables \texttt{npress\_plot} and \texttt{press\_plot} (see
the Section Temperature and pressure) in addition to the plots of
the volume, the bulk modulus, the thermal expansion, the 
isobaric heat capacity, 
the isoentropic bulk modulus, and the average Gr\"uneisen parameter 
as a function of temperature calculated at the input pressure, the code
produces also a plot of the same quantities for all the \texttt{npress\_plot}
pressures. Note that these plots may be inaccurate if the chosen
pressures are at geometries distant from the set of geometries chosen
by the code for the anharmonic calculation, so these plots might
require values of \texttt{ngeo} larger than the default. Presently the
use of these variables is limited to cubic solids.
By using the variables \texttt{nvol\_plot} and \texttt{ivol\_plot} the
code produces a plot of the thermal pressure as a function of temperature
for several volumes. \\
The Helmholtz (or Gibbs at finite pressure) free-energy can be interpolated
in two ways depending on the variable \texttt{ltherm\_glob}. 
When \texttt{ltherm\_glob=.FALSE.} (default) the vibrational (plus 
the electronic) free energy is fitted separately by a polynomial, while
only the static energy is fitted by an equation of state. When
 \texttt{ltherm\_glob=.TRUE.} a different equation of state is used 
at each temperature and no polynomial interpolation is needed.
The electronic free energy is added if the flag 
\texttt{lel\_free\_energy=.TRUE.}. This electronic free energy can be 
calculated in two ways. When \texttt{hot\_electrons=.FALSE.} the code 
expects to find in the directory \texttt{therm\_files} the file with 
the electronic free energy computed with the option \texttt{what=mur\_lc}.
When \texttt{hot\_electrons=.TRUE.} the free energy due to excited
electrons is calculated from the total energy evaluated as a function
of sigma. The total energy for several values of \texttt{sigma} must
be put in directories called \texttt{restart2}, \texttt{restart3} ...
\texttt{restart\#nsigma}. \\
The input variables that control these plots are those described in the option
\texttt{what='mur\_lc'} and \texttt{what='mur\_lc\_disp'} in addition to the 
following:

%\begin{tcolorbox}
\begin{verbatim}
grunmin_input : minimum y coordinate for the Gruneisen 
                parameter plot.
                Default: real, calculated from the Gruneisen 
                parameters.
grunmax_input : maximum y coordinate for the Gruneisen 
                parameter plot.
                Default: real, calculated from the Gruneisen 
                parameters.
volume_ph     : The frequencies and Gruneisen parameters inter-
                polated at this volume are plotted on a postscript 
                file. When volume_ph=0.0 the volume is calculated 
                from temp_ph. This option is available only for 
                cubic solids. Otherwise use celldm_ph.
                Default: 0.0 (in (a.u.)**3)
celldm_ph     : The frequencies and Gruneisen parameters 
                interpolated at this crystal parameters are 
                plotted on a postscript file. 
                If this is 0.0 the celldm are calculated from 
                temp_ph. To have accurate Gruneisen parameters 
                and interpolated frequencies set the central 
                geometry as close as possible to celldm_ph. When 
                all nstep are odd, the central geometry is the one 
                given in the input of pw.x.
                Default: 0.0 (celldm(1) in a.u., celldm(2-6) 
                dimensionless)
temp_ph       : The frequencies and Gruneisen parameters inter-
                polated at the volume (cubic systems) or at 
                celldm (anisotropic systems) that minimize the 
                free energy at this temperature are plotted on a 
                postscript file (only when volume_ph=0.0 or 
                celldm_ph(1)=0.0).
                Default: real tmin (in K)
with_eigen    : if .TRUE. use the eigenvectors of the 
                dynamical matrix to calculate the Gruneisen 
                parameters used for anharmonic properties. 
                Could require a lot of RAM. Note however that 
                eigenvectors are always used to calculate
                the plotted Gruneisen bands (both in cubic 
                and anisotropic solids).
                Default: logical .FALSE. 
ltherm_glob   : when .FALSE. the vibrational (plus electronic) 
                free energy is fitted by a polynomial, while 
                only the static energy is fitted by an equation 
                of state. When .TRUE. a different equation of 
                state is fitted at each temperature.
                Default: logical .FALSE.
poly_degree_ph : degree of the polynomial used to interpolate 
                the vibrational free energy. Presently only the 
                values 1, 2, 3, or 4 are available for anisotropic 
                solids. 
                Default: integer 4
poly_degree_cv : degree of the polynomial used to interpolate 
                the heat capacity. Presently only the values 
                1, 2, 3, or 4 are available for anisotropic 
                solids.
                Default: integer 4
poly_degree_bfact : degree of the polynomial used to interpolate 
                the b factor. Presently only the values 1, 2, 3, 
                or 4 are available for anisotropic solids.
                Default: integer 4
poly_degree_elc : degree of the polynomial used to interpolate 
                the elastic constants. Presently only the values 
                1, 2, 3, or 4 are available for anisotropic solids.
                Default: integer 4
poly_degree_grun : degree of the polynomial used to interpolate 
                the frequencies (Used only when lmurn=.TRUE. 
                otherwise it is 2).
                Default: integer 4
lv0_t         : if .TRUE.  the calculation of the thermal 
                expansion with Gruneisen parameters uses the 
                equilibrium geometry as a function of temperature 
                computed from the free energy minimization, 
                otherwise the equilibrium geometry at T=0 K. 
                If reduced_grid=.TRUE. or both ltherm_freq=.FALSE. 
                and ltherm_dos=.FALSE. the input geometry is used 
                when lv0_t=.FALSE.
                Default: logical .TRUE.
lb0_t         : if .TRUE. the calculation of the thermal expansion 
                with Gruneisen parameters uses the bulk modulus as 
                a function of temperature computed from the free 
                energy minimization, otherwise the bulk modulus 
                computed at T=0 K (lmurn=.TRUE.). 
                For lmunrn=.FALSE. the code expects a single 
                elastic constant file when lb0_t=.FALSE. and an 
                elastic constants file for each geometry when 
                lb0_t=.TRUE.. Note that if lb0_t=.FALSE. and there 
                are many elastic constants files the code use a 
                constant bulk modulus computed with the elastic 
                constants found in the file of the central 
                geometry. If lb0_t=.TRUE. and there is a single 
                elastic constants file all the quantities that 
                depend on elastic properties are not computed.
                Default: logical .TRUE.
add_empirical : If .TRUE. adds to the free energy an empirical term
                that can represent the anharmonic contribution 
                or the electronic contribution.
                Default: .FALSE.
efe           : The type of empirical free energy
                1) (alpha1+alpha2 * V) T^2
                2) -2/3 k_B nat alpha1 (v/v0p)^alpha2 T^2
                Default: 0 (must be explicitly given)
alpha1        : parameter of the empirical free energy (see above)
                in eV/K^2 in 1), in 1/K in 2). 
                Default: 0.0
alpha2        : parameter of the empirical free energy (see above)
                in eV / K^2 / A^3 in 1), adimensional in 2)
                Default: 0.0
v0p           : parameter of the empirical free energy (equilibrium volume)
                in (a.u.)^3
                Default: 0.0
hot_electrons : If .TRUE. the electronic free energy is computed from the
                energy as a function of smearing, otherwise it is
                read from files computed from electron dos. Must be
                used together with lel_free_energy=.TRUE..
                Default: .FALSE.
nsigma        : The number of smearing values for which there are
                restart files. Note that nsigma includes the present
                restart, so the code expects to find restart2, 
                restart3, ..., restart#nsigma.
                Default: 0 (option not used). Minimum value 3 to
                make a quadratic fit of the energy.
sigma_ry(nsigma): the value of the smearing for each directory restart.
                In Ry units.
                Default: Must be set for each nsigma by the user.
lhugoniot     : If .TRUE. the code plots T(p) and V(p) along the 
                Hugoniot curve.
                Default: .FALSE.
flgrun        : file where the Gruneisen parameters are written. 
                Default: character(len=*) 'output_pgrun.dat'
flpgrun       : file where the Gruneisen parameters in a plotable 
                form are written.
                Default: character(len=*) 'output_grun.dat'
flpsgrun      : name of the postscript file with the Gruneisen 
                parameters plot. The frequencies are written in 
                a file with the same name plus the string _freq.
                Default: character(len=*) 'output_grun'
flanhar       : file where the anharmonic thermodynamic quantities 
                are written.
                Default: character(len=*) 'output_anhar.dat'
flpsanhar     : postscript file of the anharmonic quantities.
                Default: character(len=*) 'output_anhar'
fact_ngeo(1)...fact_ngeo(6) : With these factors the vibrational 
                free energy is interpolated using a smaller number 
                of geometries with respect to the total energy. The 
                phonons are always calculated at geometry 1, then 
                fact_ngeo(i)-1 geometries are not calculated and 
                so on. The last calculated geometry must be ngeo(i). 
                This happens when fact_ngeo(i) divides ngeo(i)-1. 
                For even ngeo(i), fact_ngeo(i) must be 1. For odd 
                ngeo(i) the following table gives a few examples
                ngeo   fact_ngeo      calculated geometries
                3         2            1,3
                5         2            1,3,5
                7         2            1,3,5,7
                7         3            1,4,7
                9         2            1,3,5,7,9
                9         4            1,5,9
                11        2            1,3,5,7,9,11
                11        5            1,6,11
                Default: integer 1,1,1,1,1,1
                This option is not active when one of the 
                ngeo_ph(i) is different from ngeo(i).
ngeo_ph(1),...,ngeo_ph(6) These variables are set to compute 
                the phonon dispersions in a subset of the 
                geometries used to compute the total energy. 
                All values must be smaller than the corresponding 
                ngeo and even or odd as the corresponding ngeo. 
                step_ngeo remains the same for the two meshes.
                The following table gives a few examples:
                ngeo  ngeo_ph     phonon calculated in geometries
                5        3            2,3,4
                6        2            3,4
                6        4            2,3,4,5
                7        3            3,4,5
                7        5            2,3,4,5,6
                9        3            4,5,6
                9        5            3,4,5,6,7
                Default: integer ngeo(1),...,ngeo(6)
reduced_grid: if .TRUE. the computed geometries are only along 
             one dimensional lines. So each parameter is varied 
             independently keeping the others fixed at the input 
             values. This option sets ltherm_freq=.FALSE., 
             ltherm_dos=.FALSE., lv0_t=.FALSE. and lb0_t=.FALSE..
             With this option the thermal expansion is 
             calculated only using the Gruneisen parameters 
             at the input geometry. The multidimensional fit 
             of the free energy is not done so this method 
             should be faster than the default one, but it
             is less precise. This option is used only with 
             lmurn=.FALSE. and requires a file with the 
             elastic constants at the input geometry.
             Default: logical .FALSE.
all_geometries_together: if .TRUE. all the phonon calculation 
             for all the geometries are used for the image 
             parallelization. To be used only if you have many 
             images (and CPUs) available.
             Default: logical .FALSE.
\end{verbatim}
%\end{tcolorbox}

The output files corresponding to different geometries can be identified
by the presence of the letters \texttt{g1}, \texttt{g2}, ... in the filename.
To exploit all the features of this option please write the dynamical matrices
in \texttt{.xml} format (using a \texttt{fildyn} with the \texttt{.xml}
extension).
An example for this option can be found in \texttt{example09}. \\
Number of tasks for this option: Maximum between the number of tasks  
needed by the \texttt{what='mur\_lc'} option and the number
of tasks of the phonon code (see above the option \texttt{what='scf\_ph'}). \\
When \texttt{all\_geometries\_together=.TRUE.}: number of tasks of the
phonon code times the number of geometries. \\

\newpage
{\color{coral}\section{what='elastic\_constants\_geo'}}
\color{black}

With this option the code can compute the elastic constants and elastic
compliances as a function of temperature using the quasi-harmonic
approximation. For the same geometries that are used to compute the
elastic constants with the \texttt{elastic\_algorithm='energy\_std'} or
\texttt{elastic\_algorithm=}
\texttt{'energy'}, the code can compute the phonon 
dispersions and compute the elastic constants at each temperature 
as the second derivatives of the Helmholtz free energy 
with respect to strain. The second derivatives are corrected so that
the stress-strain elastic constants are shown in the plots and in output. 
The temperature dependent elastic constants are calculated on a regular 
grid of unperturbed geometries, the same geometries chosen by the option
\texttt{what='mur\_lc'}, and written on separate files, one for
each unperturbed geometry, inside the directory \texttt{anhar\_files}.
In order to plot the elastic constants as a function of temperature 
within the `quasi-harmonic' approximation,
it is necessary to make another calculation with \texttt{what='mur\_lc\_t'} 
having on files the elastic constants calculated for each geometry 
with the present option. 
In this case \texttt{thermo\_pw} will be able to calculate the anharmonic 
properties using temperature dependent elastic constants and bulk moduli 
obtained by interpolating the ``fixed-geometry quasi-harmonic'' elastic 
constants computed by this option at the crystal parameters found at 
each temperature from the minimization of the free energy. 
The variables \texttt{fact\_ngeo} and \texttt{ngeo\_ph} are not 
available with this option. 
Using \texttt{start\_geometry\_qha} and \texttt{last\_geometry\_qha} it is
possible to compute the temperature dependent elastic constants for
selected or for a single unperturbed configuration. 
The use of \texttt{start\_geometry} and \texttt{last\_geometry} is
also allowed but it refers to the global number of geometries necessary 
to compute the elastic constants in all the grid. 

Since the calculation of the Helmholtz free energy derivatives is
quite heavy, it has to be requested explicitly using the flag 
\texttt{use\_free\_energy=.TRUE.}. \\
By default, the code computes only the elastic constants at $T=0$ K
as second derivatives of the energy and writes them on files in the 
directory \texttt{elastic\_} \texttt{constants}. A run of \texttt{thermo\_pw} using 
\texttt{what='mur\_lc\_t'} having on files the $T=0$ K elastic constants
in the directory \texttt{elastic\_constants}
allows the calculation of the anharmonic properties using temperature
dependent elastic constants and bulk moduli obtained by interpolating 
(within the ``quasi-static approximation") the elastic constants computed 
by this option at the crystal parameters that, at each temperature, 
minimize the free energy. 
When both the $T=0$ K and the temperature dependent elastic constants
are on file in the directory \texttt{elastic\_constants} and 
\texttt{anhar\_files} respectively, the latter are used.
When both \texttt{use\_free\_energy=.TRUE.} and
\texttt{lel\_free\_energy=.TRUE.} the electronic free energy is
added to the free energy before computing the elastic constants. In this case 
the code expects to find on file (in \texttt{therm\_files}) the electronic 
thermodynamic properties for each perturbed geometry. These files are
produced with this same option and the flags 
\texttt{use\_free\_energy=.FALSE.} and 
\texttt{lel\_free\_energy=} \texttt{.TRUE.}. In this case the code 
computes the electronic
thermodynamic properties at each perturbed geometry and writes them on 
file, without calculating the elastic constants.
Note that the user must be careful to use the same value for the
\texttt{lel\_free\_energy} flag with this option and in the following
\texttt{mur\_lc\_t} calculation that interpolates the elastic constants.
After computing the elastic constants at $T=0$ K with the present option
one can also run another \texttt{thermo\_pw} calculation with the option
\texttt{what='mur\_lc'} computing the crystal parameters for a range
of pressures (using \texttt{pmax} and \texttt{pmin} input variables). If
the elastic constants are found in the directory \texttt{elastic\_constants} 
they are interpolated at the pressure dependent crystal parameters and
plotted on output.
The variables that control this run are:

%\begin{tcolorbox}
\begin{verbatim}
use_free_energy : when .TRUE. computes the elastic constants 
              as second derivatives of the Helmholtz free 
              energy with respect to strain. When .FALSE. 
              the elastic constants are computed as second 
              derivatives of the energy or using the 
              stress-strain algorithms.
              Default: .FALSE.
start_geometry_qha : Among the geometries considered by the 
              option mur_lc_t the calculations of elastic 
              constants are done starting from this geometry.
              Default: integer 1
last_geometry_qha : Among the geometries considered by the 
              option mur_lc_t the calculations of elastic 
              constants are done only up to this geometry.
              Default: integer total number of geometries.
\end{verbatim}
%\end{tcolorbox}

An example for this option with \texttt{use\_free\_energy=.FALSE.}
can be found in \texttt{example22} while an example with
\texttt{use\_free\_energy=.TRUE.} can be found in \texttt{example23}. \\
Number of tasks for this option: The product of the number of tasks  
needed by the \texttt{what='scf\_elastic\_constants'} option 
and the number of geometries used with \texttt{what=mur\_lc\_t} when
\texttt{use\_free\_energy=.FALSE.}. When \texttt{use\_free\_}
\texttt{energy=.TRUE.}
and \texttt{all\_geometries\_together=.TRUE.} the number of tasks of the
previous case is further multiplied by the number of tasks needed to 
compute a phonon dispersion (see above the option \texttt{what='scf\_ph'}). \\
When \texttt{use\_free\_energy=.TRUE.} and 
\texttt{all\_geometries\_together=.FALSE.} the number of tasks of this
option is equal to the number of tasks needed to compute a phonon dispersion.


%{\color{coral}\section{\sc what='scf\_polarization'}}
%\color{black}
%With this option the code calculates the spontaneous polarization.
%The code makes a self-consistent calculation followed by three Berry
%phase calculations in which the polarization is calculated in the
%direction parallel to the three primitive reciprocal lattice vectors and 
%prints the spontaneous polarization in Cartesian coordinates.
%The input variable that controls this option is:
%\begin{verbatim}
%nppl: the number of k points per string. In the perpendicular plane the
%      number of k-point is that given as input of the pw.x code.
%      Default: integer 51
%\end{verbatim}
%This feature is still incomplete and experimental.
%
%{\color{coral}\section{\sc what='mur\_lc\_polarization'}}
%\color{black}
%As \texttt{what='scf\_polarization'} but the calculation is made at the
%geometry that corresponds to the minimum of the Murnaghan equation. The
%Murnaghan minimization is made as described for \texttt{what='mur\_lc'}.\\
%This feature is still incomplete and experimental.
%

\newpage
{\color{dark-blue}\chapter{Restarting an interrupted run}}
\color{black}

There are several situations that might require the restart of the \thermo\ 
code. We must distinguish two different cases:
\thermo\ stopped while running \qe\ routines, because the code reached
the maximum cpu time or because some external event stopped the run,
or \thermo\ stopped after doing some post-processing task. This second 
case comprises also the normal termination of
\thermo\ and the necessity to change some details of the plot rerunning
the post-processing tools without redoing the \qe\ calculations.

Support for the first case is based on the recover features provided by 
\qe\ routines and usually works when images are not used. This restarting 
method needs files in the \texttt{outdir}
directory. In this case \thermo\ behaves as \qe\ 
except for the fact that \texttt{max\_seconds} in the input of \texttt{pw.x} or
of \texttt{ph.x} is not active. To run \texttt{thermo\_pw}
for a fixed number of seconds \texttt{max\_seconds} must be set in the
\texttt{THERMO\_CONTROL} namelist. If the code stopped inside \texttt{pw.x},
\texttt{restart\_mode} must be set to \texttt{'restart'} in the input of 
\texttt{pw.x} while if the code stopped inside \texttt{ph.x} routines
\texttt{recover} must be set to \texttt{.TRUE.} in the input of \texttt{ph.x}.

When running \texttt{thermo\_pw} with several images and
calculating a phonon dispersion or using the \texttt{what='mur\_lc\_t'}
option, \texttt{max\_seconds} is controlled by the image driver of 
\texttt{thermo\_pw}. Presently, after \texttt{max\_seconds} a signal
is sent to the asynchronous driver and the master stops sending new works 
to the images. It stops the code when all the images have terminated their
current task. Recovering from this point is possible without loosing 
any previous work by keeping the \texttt{outdir} directory and by setting 
\texttt{recover=.TRUE.} in the \texttt{ph.x} input.
Note however that when \texttt{thermo\_pw} is stopped by the operating 
system in an unclean way this restart method could not work.

As a last resource you can remove the outdir directory, and \thermo\ 
will not recalculate the quantities contained in files that are already 
in the working directory. Completed phonon calculations at a given
geometry for which the dynamical matrices are available are
not redone if you put the \texttt{fildyn} name in the 
\texttt{thermo\_control} namelist. It is possible to stop 
\texttt{thermo\_pw} after the 
calculation of the phonon dispersions for a fixed number of geometries 
by setting the input variable \texttt{max\_geometries} in the 
\texttt{THERMO\_CONTROL} namelist or also specify exactly which geometries
to do in a given run using the variables \texttt{start\_geometry} and
\texttt{last\_geometry} or \texttt{start\_geometry\_qha} and 
\texttt{last\_geometry\_qha}. 

In general the restart of \texttt{thermo\_pw} from a post-processing task
is much easier.
Each routine checks if a file with the same name
as the file that it would produce is already in the working directory,
and if this happens, it reads its content and returns. This feature cannot be
disabled from input. In order to recalculate a given quantity, just remove
the file that contains it from the working directory.
For instance in an anharmonic calculation, if you have already all 
the dynamical 
matrices for all the geometries and you do not have any more the
\texttt{outdir} directory, it is possible to skip entirely
the phonon calculations and the reading of the files produced by
\texttt{pw.x} by setting the variable
\texttt{after\_disp=.TRUE.} and giving the name of the dynamical matrices file
using the variable \texttt{fildyn} in the \texttt{THERMO\_CONTROL} namelist. 
In this case \thermo\ can compute the anharmonic properties with a 
different set 
of temperatures, or with a different sampling on the phonon frequencies, 
etc.. You need to erase the output files that contain
the phonon dos, or the thermal properties from a previous calculation, keeping 
the dynamical matrices files and the \texttt{restart} directory and 
rerun \thermo.
Similarly if the files containing the bands energy eigenvalues are already
in the working directory, it is possible to set the input variable
\texttt{only\_bands\_plot} to change the bands plot without redoing the
bands calculation. Note however that in this case it is not possible 
to change the Brillouin zone path. \\
The following variables can be used to stop \thermo\ before it concludes all
the calculations:

%\begin{tcolorbox}
\begin{verbatim}
max_seconds: the code stops after max_seconds have elapsed. 
              Note that the check is not done continuously so 
              the clean stop might occur a few minutes after 
              max_seconds.
              Default: real 10E8

max_geometries: the code stops after computing the dispersions 
              in max_geometries.
              Default: integer 1000000

start_geometry: the code starts doing the phonons for 
              start_geometry.
              Default: integer 1

last_geometry: the code does only the phonons for geometries 
              with index lower than last_geometry.
              Default: integer total number of geometries.
\end{verbatim}
%\end{tcolorbox}

Note that the first time that you use \texttt{start\_geometry} and 
\texttt{last\_geometry} the codes makes a self-consistent \texttt{pw.x} 
run for all the geometries and saves the results in the \texttt{outdir}
directory. If for any reason the \texttt{outdir} directory is removed 
after computing some geometries, just remove also the
\texttt{restart} directory and the code will recreate the information
in \texttt{outdir} but will not recalculate the dynamical matrices 
already available. 

Finally we consider some typical runs of the most time consuming option
\texttt{what='mur\_lc\_t'}, but what we say is valid also for the computation
of the phonon dispersions at a single geometry.
With a single processor or with a personal computer with a small number of 
processors you can run the code without interruption.
In general in these cases it is not useful to use the image parallelization
before exploiting all the parallelization levels of \qe, since 
images have an overhead due to the necessity of reinitialize the phonon
calculation and recalculate the bands.
If you cannot complete the calculation in a single run 
you need to send several times the \thermo\ run. 
In this case you can set \texttt{start\_geometry=last\_geometry},
\texttt{max\_seconds} in the \texttt{thermo\_pw} input,
and \texttt{recover=.TRUE.} in the \texttt{ph.x} input. You need to
repeat this for all the geometries and finally you can collect the results
and compute the anharmonic properties. Note that when
\texttt{start\_geometry} or \texttt{last\_geometry} are set by the
user the anharmonic calculation is skipped.

\newpage
{\color{dark-blue}\chapter{Thermo\_pw on the GPU}}
\color{black}
\texttt{Thermo\_pw} uses the \qe\ routines which are GPU aware, so if 
you compile with the flag \texttt{-D\_\_CUDA}, GPU is active in
\texttt{thermo\_pw}. Starting from version \texttt{1.9.0} \thermo\ contains
also an experimental GPU version which improves on 
the \qe\ routines when the system is small and requires many
{\bf k}-points. This version which is activated by setting 
\texttt{many\_k=.TRUE.} in the \texttt{INPUT\_THERMO} namelist
uses several experimental routines presently available only 
in \texttt{thermo\_pw}. It has many limitations so it has to be used 
only for particular cases. It is implemented with \texttt{Davidson} 
diagonalization for \texttt{LDA} and \texttt{GGA} functionals. It
is working for norm conserving, ultrasoft, and PAW pseudopotentials,
both scalar and fully relativistic, but not for hybrid functionals or
for LDA+U schemes. When the response to a phonon
perturbation is calculated by \texttt{thermo\_pw} with the 
flag \texttt{many\_k=.TRUE.} the new routines are used, while
the old ones are used for the other perturbations. 
The method is found to be useful only when FFT sizes are smaller than
approximately ${48\times 48 \times 48}$. It is not compatible with
{\bf G}-vectors parallelization, but it can be used with \texttt{pools} 
parallelization. It must be used with a number of 
\texttt{MPI} processes equal to the number of GPUs and with a number
of \texttt{pools} equal to the number of \texttt{MPI} processes (one pool per
GPU). It uses \texttt{CUDA} fortran so it is active only
on NVIDIA GPUs. It has been tested on \texttt{marconi100} at CINECA
loading modules \texttt{hpc-sdk/2022--binary} or \texttt{hpc-sdk/2023--binary}.
The \texttt{many\_k} routines are active also on the CPU version of the code
for testing purposes, but there is no advantage to use them with CPU.\\

\begin{verbatim}
many_k: If .TRUE., the part of the code that loads 
        on the GPU several wavefunctions and Hamiltonians 
        is used. 
        Default: logical .FALSE.
memgpu: The available free memory in one GPU in GByte units.
        (If the code crashes due to memory allocation problems, 
        decrease this value). 
        Default: real 10.
\end{verbatim}

\newpage
{\color{dark-blue}\chapter{Tools}}
\color{black}

The directory \texttt{tools} contains a few tools that can be useful to build
structures of solids, surfaces, ribbons, and nanowires. Moreover it 
contains some miscellaneous codes that give additional information
on the internal conventions of \thermo\ or further process its output. 
Currently it contains the following programs:

\begin{itemize}

\item \texttt{average\_grun.x} reads the thermal expansion, 
the isothermal bulk modulus, the volume, the isochoric heat capacity,
and the temperature and computes the average Gr\"uneisen parameter,
the isobaric heat capacity, and the isoentropic bulk modulus.

\item \texttt{bravais\_lattices.x} tests the
module \texttt{lattices.f90} of the library. Pre- sently it can read three
primitive lattice vectors of a Bravais lattice and find the \texttt{ibrav}
code and the \texttt{celldm} parameters of the input lattice. It
can also read two sets of primitive vectors and decide if they describe
the same Bravais lattice. In the positive case it gives the orientation
of one lattice with respect to the other.
For an example of its use see \texttt{tools\_input/bravais\_lattice.in}.

\item \texttt{change\_dynmat\_name.x} A simple tool 
to change the number of the geometry to a set of dynamical matrix files. 
For an example of its use see \texttt{tools\_input/change\_name.in}.

\item \texttt{change\_e\_name.x} A simple tool to change the number of the 
geometry to a set of energy files inside the \texttt{restart} directory. 

\item \texttt{change\_tel\_name.x} A simple tool to change the number of the 
geometry to the electronic thermodynamic files inside the \texttt{therm\_file} 
directory. It can be used also to change the names of the elastic constant
files, both those in \texttt{elastic\_constants} and in \texttt{anhar\_files}
directories.

\item \texttt{crystal\_point\_group.x} is a crystal point group calculator.
It can give several information about the crystallographic point groups,
such as the list of symmetry operations, the product of two rotations,
the product table, the class structure, the character tables of 
the irreducible representations of the point group and of the double
point group and the projective representations. It
gives the list of subgroups and supergroups of a given group and 
the compatibility tables of a given group with its subgroups. It can 
also decompose the Kronecker product representations. Finally it
can list the conjugate groups and calculate the intersection of two
point groups.
For an example of its use see \texttt{tools\_input/crystal\_point\_}
\texttt{group.in}.

\item \texttt{debye.x} reads from input the Debye temperature and the
number of atoms per unit cell and writes in a file the thermodynamic
quantities (vibrational energy, free energy, entropy, and heat capacity) as
a function of temperature computed using the Debye model. When the
system has only one atomic type it writes the atomic $B$ factor
as a function of temperature computed using the Debye model.
For an example of its use see \texttt{tools\_input/debye.in}.

\item \texttt{density.x} reads the volume of a unit cell of a solid
and its mass (in a.m.u.) and computes the density, or reads the density 
of a solid and the mass (in a.m.u.) of a unit cell and computes the volume.

\item \texttt{elastic.x} reads the elastic constants of a
solid and computes the elastic compliances, the bulk modulus, and a few
poly-crystalline averages. It uses the \thermo\ library so the output is the
same. 
For an example of its use see \texttt{tools\_input/elastic.in}.

\item \texttt{emp\_f.x} reads a set of parameters for the construction of
the Helmholtz free energy of a solid as a function of volume and 
temperature and prints the thermodynamic quantities. Presently it 
supports the models of Dorogokupets, Litasov, Mie-Gruneisen-Debye, 
and high temperature Birch-Murnaghan equations (still in experimental form).

\item \texttt{emp\_g.x} reads a set of parameters for the construction of
the Gibbs free energy of a solid as a function of pressure and temperature and
prints the thermodynamic quantities. Presently it supports only the model
of Gustafson for tungsten.

\item \texttt{epsilon\_tpw.x} generalizes the routine
\texttt{epsilon.f90} of the \qe\ distribution. It calculates the 
complex dielectric
constant of a solid as a function of the frequency for independent electrons
using the LDA or GGA eigenvalues. It is limited to insulators, but supports
norm-conserving, ultrasoft, and PAW pseudopotentials. It supports both scalar
relativistic and fully relativistic pseudopotentials and it uses the point
group symmetry of the solid to reduce the number of {\bf k}-points.
For an example of its use see \texttt{example14}.

\item \texttt{gener\_nanowire.x} reads a two dimensional (2D)
Bravais lattice index and atomic coordinates and generates a sheet of type
$(m,n)$. A sheet contained between the two vectors 
{\bf C}$ = m$ {\bf a}$_1 + n ${\bf a}$_2$ and {\bf T} $= p ${\bf a}$_1 + q ${\bf a}$_2$
can be also generated and wrapped about {\bf C} in a nanotube form 
({\bf a}$_1$ and {\bf a}$_2$ are the primitive lattices of the 2D Bravais 
lattice). 
For lattices that allow it, $p$ and $q$ can be determined automatically so that 
{\bf T} is perpendicular to {\bf C}.
For an example of its use see \texttt{tools\_input/gener\_nanowire.in}.

\item \texttt{gener\_2d\_slab.x} reads a two dimensional 
Bravais lattice index and atomic coordinates and generates an infinite ribbon
perpendicular to {\bf G} $= m$ {\bf b}$_1 + n ${\bf b}$_2$, where 
{\bf b}$_1$ and {\bf b}$_2$ are the primitive reciprocal lattice vectors 
of the 2D Bravais lattice. The number of rows of the ribbon, and the number of
atoms per row are given as input variables.
For an example of its use see \texttt{tools\_input/gener\_2d\_slab.in}.

\item \texttt{gener\_3d\_slab.x} reads a three dimensional
Bravais lattice index and atomic coordinates and generates an infinite slab
perpendicular to {\bf G} $= m ${\bf b}$_1 + n ${\bf b}$_2 + o ${\bf b}$_3$, 
where {\bf b}$_1$, {\bf b}$_2$ and {\bf b}$_3$ are the primitive reciprocal 
lattice
vectors of the Bravais lattice. The number of layers of each slab, and the
size of the surface unit cell are given as input parameters.
For an example of its use see \texttt{tools\_input/gener\_3d\_slab.in}.

\item \texttt{hex\_trig.x} reads the values of $a$ and $c$
of the conventional hexagonal cell of a rhombohedral lattice (in \AA ngstrom),
and gives as output the size $a_r$ (in a.u.) and the cosine of the angle
$\alpha$ of the rhombohedral cell. This information can
be written in the input of \texttt{pw.x} for this type of cells. It
is used to convert the structural information contained in a CIF file
to the \texttt{pw.x} input.

\item \texttt{kovalev.x} writes the correspondence between point group
symmetry operations defined in the Kovalev tables and those used by \qe.

\item \texttt{mag\_point\_group.x} gives a few information on the magnetic point
groups.

\item \texttt{merge\_interp.x} This is a driver of the spline
interpolation routines of \texttt{QE}. It can read the meshes and the
functions to interpolate from two different files and provide the
first function on the mesh of the second. In the data files
lines that start with \texttt{\#} are considered comments.

\item \texttt{optical.x} contains a few utilities for
optical properties calculations. It transforms a complex dielectric 
constant into a complex index of refraction and computes the reflectivity 
or the absorption coefficient for cubic system. It converts also from 
energy of the photon in eV to the frequency in Hz or the wavelength in nm.

\item \texttt{pdec.x} reads the temperature dependent elastic constants files 
calculated for several pressures and makes a plot of the elastic constants as
a function of pressure at several temperatures.

\item \texttt{plot\_sur\_states.x} reads the dump file produced
by \thermo\ in a \texttt{what='scf\_2d\_bands'} calculation that
contains the planar averages of all the states, and plots
the states with the {\bf k} point and the band numbers requested in input.
For an example of its use see \texttt{tools\_input/plot\_sur\_}
\texttt{states.in}.

\item \texttt{rotate\_tensors.x} applies a rotation to a tensor
of rank 1, 2, 3, or 4 defined in a coordinate system 1 and finds the form
of the tensor in a new coordinate system 2. 

\item \texttt{space\_groups.x} gives several information on space groups.
It can give the names of the space group given the number reported in
the International Tables for Crystallography (ITA), or
the number given one of the names, translate the names between different
editions of the ITA tables or the Sh\"onflies name. It gives the list 
of coset representatives of each space group and the list of symmorphic 
space groups.
For an example of its use see \texttt{tools\_input/space\_groups.in}.

\item \texttt{supercell.x} reads a three dimensional
Bravais lattice index and the atomic coordinates of the atoms inside a unit
cell and produces a supercell with $n1 \times n2 \times n3$ cells of 
the original
unit cell or a supercell delimited by three arbitrary Bravais lattice
vectors given in crystal or cartesian coordinates. For centered cells 
there is the option to consider $n1$, $n2$,
and $n3$ for the primitive or for the centered Bravais lattices.
The input unit cell can be specified also by giving the space group and
the coordinates of the nonequivalents atoms.
It can be useful to study defects or to calculate all the 
atomic positions starting from the space-group and the 
nonequivalent positions.
It is also possible to give the input Bravais lattice by using
\texttt{ibrav=0} and the three principal vectors. In this case,
before generating the supercell, the code rotates the Bravais
lattice so that it has the same orientation of the vectors described
in the \texttt{thermo.pdf} guide.
For an example of its use see \texttt{tools\_input/supercell.in}.

\item \texttt{test\_colors.x} produces a postscript file with the 
\texttt{gnuplot} colors that can be used in the plots.   

\item \texttt{test\_eos.x} reads the parameters of an equation of state,
and optionally the coefficients of a polynomial. Produces in output
a file with the energy, the pressure, the bulk modulus, and its first
and possibly second derivative with respect to pressure.
It also checks the analytic results with those obtained by numerical
finite differences writing on file the relative errors.

\item \texttt{template.x} This is an example on how to write a tool
code that can interface with the \texttt{QE} routines and all the
library routines of \texttt{thermo\_pw}. The template activates 
the parallelization options of \texttt{QE} and can use images.

\item \texttt{translate.x} reads a set of atomic positions and a translation
vector and translates the atomic positions. It can read also a rotation matrix 
and roto-translate the atomic positions.

\item \texttt{units.x} writes on output the numerical constants used
to write the guide \texttt{units.pdf} and \texttt{equilibrium.pdf}. 
It computes also the error associated to each conversion factor.

\end{itemize}

For a detailed description of the input variables please look at the beginning 
of the \texttt{fortran} sources of each code.

\newpage
{\color{dark-blue}\chapter{Examples, examples\_qe, inputs, pseudo\_test, space\_groups, \\ tools\_inputs}}
\color{black}

The directories \texttt{examples}, \texttt{examples\_qe}, \texttt{inputs}, 
\texttt{pseudo\_tests}, \texttt{space\_groups} and \texttt{tools\_inputs} 
contain a set of examples that can be studied in order to learn how to 
use the \thermo\ package. The \texttt{examples} directory contains inputs 
that run quickly but do not give converged results. These examples can 
be studied 
to see how the \thermo\ code works in the different cases. The
reference directory of each example contains all the output
files produced by the run. A one-to-one comparison with the output
produced by running the example script is however not possible due to the
asynchronous nature of the runs. Only the plotted physical quantities 
should be the same. \\
The directory \texttt{examples\_qe} is used by developers. It
contains examples mostly imported from \qe\ that are used to check that
\qe\ functionalities are not spoiled by \texttt{thermo\_pw}. \\
The directory \texttt{inputs} contains a set
of realistic inputs and reasonably converged results. Not all
output files are reported in the reference directory of each run. 
The \texttt{inputs} examples are divided according to the structure type
and many material properties are calculated for each structure.
This directory can be seen as a gallery of the results that can be
obtained by the \thermo\ code, or as a source of information for the
construction of a particular input geometry. \\
The directory \texttt{pseudo\_test} contains a set of inputs that can
be used to test a pseudopotential library. It illustrates how to use \thermo\ 
for high-through-put calculations. \\
The directory \texttt{space\_groups} contains a collection of structures
ordered by the space group number. They are used to test the space
groups routines. These inputs are also examples for the
keyword \texttt{space\_group} and for the use of Wyckoff positions to give the
atomic coordinates in the \texttt{pw.x} input. You can also use these
structures for your calculations, but note that the cut-off energies and
the {\bf k}-point meshes are not converged. \\
The directory \texttt{tools\_inputs} give some examples of the inputs of
the auxiliary \texttt{tools} programs.



\newpage
{\color{dark-blue}\chapter{Color codes}}
\color{black}

In this section we briefly summarize the color codes of some of the figures
that can be obtained from \texttt{thermo\_pw}.

\begin{itemize} 
\item
Total energy versus kinetic energy. This is a figure of the total
energy versus wave-functions kinetic energy cut-offs. When the test
requires several charge density cut-offs there is a different curve
for each charge density cut-off. The curve corresponding to the lowest
charge density cut-off is \texttt{red}, the one corresponding to the
highest is \texttt{blue}, all the others are \texttt{green}.
Note that the total energy of the last configuration (highest wave function
and charge density cut offs) is subtracted from all energies.

\item
Total energy versus size of the {\bf k}-point mesh. This is a 
figure of the total energy as a function of the size of the {\bf k}-point
mesh. When the test requires several values of \texttt{degauss}, there
is a different curve for each \texttt{degauss}. The curve corresponding to the
first \texttt{degauss} is \texttt{red}, the one corresponding to the
last is \texttt{blue}, all the others are \texttt{green}.          
Note that the total energy of the last configuration (highest number of
points and lowest \texttt{degauss}) is subtracted from all energies.

\item
Total energy as a function of volume (\texttt{lmurn=.TRUE.}). 
This plot is composed of three figures: the total energy as a function of the
volume, the pressure as a function of the volume and the enthalpy
as a function of pressure. All curves are \texttt{red}.   
The points on the first curve are the energies calculated by 
\texttt{pw.x}, the continuous curve is the fit.

\item
Total energy as a function of one or two crystallographic parameters
(\texttt{lmurn=.FALSE.}). When there is a single parameter the curve is
\texttt{red} as in the case \texttt{lmurn=.TRUE.}. When there are two
parameters a contour plot of the energy as a function of two parameters
is shown. The contour levels, their number and their colors can be
given in input. By default the code shows nine levels with three
colors. From the lowest to the highest levels, the colors are \texttt{red},
\texttt{green}, and \texttt{blue}. The energy value of each level is 
written on output. When the user requests more levels without specifying
their colors, the code continues with three \texttt{yellow} levels, 
then \texttt{pink}, \texttt{cyan}, \texttt{orange}, \texttt{black}, and
when more than $24$ levels are requested the sequence of colors is repeated.
When \texttt{lgeo\_to\_file=.TRUE.} the path written on file is shown
in this plot with an \texttt{orange} points connected by a line.
For orthorhombic solids the code produces many postscript figures, one
for each value of $c/a$ on the grid. In each figure there is a contour
plot of the energy as a function of $a$ and $b/a$. The colors of the levels
follow the same conventions of the previous case. When the levels are chosen
by the code the entire energy range (for all $c/a$) is divided into nine
levels so each figure might have less that nine curves.
For crystal systems with more crystallographic parameters, this figure is
not available.

\item
Elastic constants (elastic compliances) as a function of pressure. 
The elastic constants (elastic compliances) are shown in different
plots in red. In a final plot all the elastic constants (elastic
compliances) are shown on the same figure in \texttt{red, green, blue, yellow, 
pink, cyan, orange and black} with same order of the previous plots.
When there are more than eight elastic constants the colors are repeated.

\item
The crystal parameters and the volume as a function of pressure.
When (\texttt{lmurn=.FALSE.}) the code plots the lattice parameters 
as a function of pressure, as well as the volume as a function of 
pressure (so far tested only for cubic cases). All plots are \texttt{red}.

\item
Energy bands. In this figure the bands have the color of
their irreducible representation. Each line of the path can have a different
point group and set of representations. See the \texttt{point\_groups.pdf} file
for the list of representations and their color code. 
When the symmetry analysis is not done all the bands are \texttt{red}.

\item
Energy bands with \texttt{enhance\_plot=.TRUE.}. In this case 
the background color of the panels with lines at the zone border are 
gray, yellow, or pink. Gray means that the point group (or double point
group) representations are used, yellow or pink means that a gauge 
transformation was applied and projective representations might have been used. 
A yellow background indicates that no switch 
from the point group to the double point group or viceversa was made, 
while a pink background means that such a switch was necessary.
                              
\item
Electron density of states. This is a plot composed of two figures,
the first contains the electron density of states, the second the integral
of the density of states up to that energy. The dos is \texttt{red}.
In the local spin density case, the dos for spin up is \texttt{red} the
one for spin down is \texttt{blue} and with a negative sign.          
The integrated density of states is \texttt{blue}. In the spin polarized
case, the curve shows the integral of the sum of the up and down density 
of states.

\item
Electronic energy, free energy, entropy, and isochoric heat capacity 
(metals only). This plot is composed of four pictures one for each
quantity. There is a single \texttt{blue} curve per plot.      

\item
Dielectric constant as a function of frequency (${\bf q}={\bf 0}$). There are 
two plots, one for the real part and one for the imaginary part. 
Other two plots contain the real and imaginary part of the complex index
of refraction. For cubic solids other two plots show the reflectivity
for normal incidence and the absorption coefficient.
All curves are in red. For hexagonal, trigonal, and tetragonal 
systems the $xx$ component is in red, while the $zz$ component is in green. 
For orthorombic systems the $xx$ component is in red, the $yy$ component in 
green and the $zz$ component in blue. For monoclinic and triclinic
systems the plot is not available.

\item
Inverse of the dielectric constant as a function of frequency 
(${\bf q}\neq {\bf 0}$). There are four plots: the real and imaginary part of 
$\varepsilon({\bf q},\omega)$ and the real and imaginary part of 
$1/\varepsilon({\bf q},\omega)$. They are all in red. Note that the latter
is really calculated, while the first is just its inverse. 

\item
Phonon dispersions. In this figure the phonon dispersions have the color 
of their irreducible representations. The same comments made for the plot of 
the band structure apply here.                       

\item
Phonon dos. There is one picture with a single \texttt{red} curve.  
   
\item
Vibrational energy, free energy, entropy, and isochoric heat capacity. This plot
is composed of four figures each one showing one quantity. In \texttt{red} the 
quantities obtained using the phonon density of states, 
in \texttt{blue} those obtained from integration over 
the Brillouin zone. In some cases the \texttt{red} curve is not visible 
because it is exactly below the \texttt{blue} one.

\item
Atomic B factors as a function of temperature. This plot is composed of 
one figure for each atom for cubic solids and of two figures for each atom
in the other cases. One figure contains $B_{xx}$ (red, pink), $B_{yy}$ 
(blue, light\_blue) and $B_{zz}$ (dark\_green, green) as a function of 
temperature. The first color refers to quantities calculated from 
generalized phonon density of states while the second refers to quantities 
calculated by Brillouin zone integration.
If the curves coincide, only the last one (green) will be visible. The second 
figure, when plotted shows $B_{xy}$ (red, pink), $B_{xz}$ (blue, light\_blue), 
and $B_{yz}$ (dark\_green, green).

\item
Debye vibrational energy, free energy, 
entropy, and isochoric heat capacity. This plot is composed of four 
figures, one for each quantity. The curves are in \texttt{blue}
and the word Debye appears in the $y$ axis label.

\item Crystal parameters as a function of pressure at several temperatures.
Volume as a function of pressure at several temperatures.
The number of plots depends on the crystal system.
In these plots the first temperature is red, the others follow in the 
order green, blue, yellow, pink, cyan, orange, black.
If there are more temperatures the sequence is 
repeated.

\item
Helmholtz free energy as a function of volume (\texttt{lmurn=.TRUE.}). 
The free energy calculated using the phonon dos (integral over
the Brillouin zone) is \texttt{red} (\texttt{blue}).
When required in input this figure contains also the
free energy as a function of volume for several temperatures.
The color sequence \texttt{red}, \texttt{green}, \texttt{blue},
\texttt{yellow}, \texttt{pink}, \texttt{cyan}, \texttt{orange},
\texttt{black} indicates the different temperatures.
In this case the same figure contains also the Gibbs energy as
a function of pressure for several temperatures,
the vibrational (plus electronic if available) free energy as a function 
of volume for several temperatures and the electronic free energy as 
a function of volume for several temperatures.

\item
The equilibrium volume as a function of temperature (\texttt{lmurn=.TRUE.}). 
The equilibrium volume obtained from the free energy calculated
using the phonon dos (integral over
the Brillouin zone) is \texttt{red} (\texttt{blue}).
When required in input this figure contains also the
volume as a function of temperature at several pressures.
The color sequence \texttt{red}, \texttt{green}, \texttt{blue},
\texttt{yellow}, \texttt{pink}, \texttt{cyan}, \texttt{orange},
\texttt{black} indicates the different pressures.
In this figure there is also the equilibrium volume divided by the 
equilibrium volume at $T= 300$ K (and the same pressure)
is plotted as a function of temperature.
When required in input this figure contains also the
equilibrium volume as a function of pressure at
several temperatures with the same color sequence.
In this case also the equilibrium volume divided by the 
equilibrium volume at $T= 300$ K and zero pressure
is plotted as a function of pressure for several temperatures.

\item
When requested in input, the pressure as a function of volume at 
several temperatures (\texttt{lmurn=.TRUE.}). 
The color sequence \texttt{red}, \texttt{green}, \texttt{blue},
\texttt{yellow}, \texttt{pink}, \texttt{cyan}, \texttt{orange},
\texttt{black}, indicates the different temperatures.
In the same figure there is also the thermal pressure as a function
of volume for several temperatures with the same color sequence
and the thermal pressure as a function
of temperature for several volumes with the same color sequence.

\item
The isothermal bulk modulus as a function of 
temperature (\texttt{lmurn=.TRUE.}). 
The isothermal bulk modulus obtained interpolating the free 
energy calculated using the phonon dos (integral over
the Brillouin zone) is \texttt{red} (\texttt{blue}).
When required in input this figure contains also the
isothermal bulk modulus as a function of temperature at 
several pressures.
The color sequence \texttt{red}, \texttt{green}, \texttt{blue},
\texttt{yellow}, \texttt{pink}, \texttt{cyan}, \texttt{orange},
\texttt{black}, indicates the different pressures.
When required in input this figure contains also the
isothermal bulk modulus as a function of pressure at
several temperatures with the same color sequence.
The same figure contains also the isoentropic bulk modulus
as a function of temperature and the difference between isothermal
and isoentropic bulk moduli. When required in input this figure 
contains also the isoentropic bulk modulus and the difference
isoentropic-isothermal bulk moduli as a function of temperature for
several pressures or as a function of pressure for several temperatures.

\item
Volume thermal expansion as a function of temperature (\texttt{lmurn=.TRUE.}). 
The thermal expansion obtained from the free energy
calculated using the phonon dos (integral over
the Brillouin zone) is \texttt{red} (\texttt{blue}).
The one obtained from the mode Gr\"uneisen parameters is 
\texttt{green}. When required in input this figure contains also the
thermal expansion as a function of temperature at several pressures.
The color sequence \texttt{red}, \texttt{green}, \texttt{blue},
\texttt{yellow}, \texttt{pink}, \texttt{cyan}, \texttt{orange}, 
\texttt{black}, indicates the different pressures.
When required in input this figure contains also the
thermal expansion as a function of pressure at several temperatures
with the same color sequence.

\item
The isochoric heat capacity as a function of temperature 
(\texttt{lmurn=.TRUE.}). 
The isochoric heat capacity calculated using the phonon dos (integral over
the Brillouin zone) is \texttt{red} (\texttt{blue}).
When required in input this figure contains also the
isochoric heat capacity as a function of temperature for
several pressures.
The color sequence \texttt{red}, \texttt{green}, \texttt{blue},
\texttt{yellow}, \texttt{pink}, \texttt{cyan}, \texttt{orange},
\texttt{black}, indicates the different pressures.
When required in input this figure contains also the
isochoric heat capacity as a function of pressure at
several temperatures with the same color sequence.
The same figure contains also the isobaric heat capacity
as a function of temperature and the difference 
isobaric-isochoric heat capacity. When required in input this figure 
contains also the isobaric heat capacity and the difference
isobaric-isochoric heat capacity as a function of temperature for
several pressures or as a function of pressure for several temperatures.

\item
Average Gr\"uneisen parameter as a function of temperature 
(\texttt{lmurn=.TRUE.}).
The parameter obtained from phonon dos (integral over
the Brillouin zone) is \texttt{red} (\texttt{blue}).
The one obtained from the mode Gr\"uneisen parameters is 
\texttt{green}. When required in input this figure contains also the
average Gr\"uneisen parameter as a function of temperature at 
several pressures.
The color sequence \texttt{red}, \texttt{green}, \texttt{blue},
\texttt{yellow}, \texttt{pink}, \texttt{cyan}, \texttt{orange}, 
\texttt{black}, indicates the different pressures.
When required in input this figure contains also the
average Gr\"uneisen parameter as a function of pressure at 
several temperatures with the same color sequence.

\item
Crystallographic parameters, volume, Helmholtz (or Gibbs at finite pressure)
free energy, thermal expansion tensor, volume thermal expansion, constant
strain heat capacity ($C_\epsilon$), isobaric heat capacity ($C_P$),
difference $C_P-C_V$ of isobaric and isochoric heat capacities,
difference $C_\sigma-C_\epsilon$ of constant stress and 
constant strain heat capacities (note that $C_\sigma=C_P$), 
difference $C_V-C_\epsilon$ of isochoric and constant strain heat capacities, 
difference $B_S-B_T$ of the isoentropic and isothermal bulk modulus,
and average Gr\"uneisen parameter as a function of temperature 
(\texttt{lmurn=.FALSE.}). The number of
figures in this plot depends on the crystal system and on the presence
of one or more files with the elastic constants. It shows $a$ as a function
of temperature for cubic solids, $a$, $c/a$, and $c$ for tetragonal and 
hexagonal 
solids. For orthorhombic solids it shows also $b/a$ and $b$ while for 
trigonal solids
it shows $a$ and $\cos\alpha$. For monoclinic
solids it shows $a$, $b/a$, $b$, $c/a$, $c$, and $\cos\alpha$ (c-unique) or 
$\cos\beta$ (b-unique). All the six crystallographic parameters 
as a function of temperature are shown for triclinic solids. 
All quantities calculated using the phonon 
density of states are in \texttt{red}, those calculated integrating
over the Brillouin zone are in \texttt{blue} with the exception of the
thermal expansion tensor. When this tensor is diagonal with all identical
components it follows the above rules while the tensor computed 
from mode Gr\"uneisen parameters is in \texttt{green}. For hexagonal,  
tetragonal and trigonal solids $\alpha_{xx}$ follows the above rules 
while $\alpha_{zz}$ is \texttt{pink}, \texttt{cyan}, and \texttt{orange}
when computed from phonon density of states, Brillouin zone integration, 
or mode Gr\"uneisen parameters, respectively. 
In the orthorhombic case $\alpha_{xx}$ and $\alpha_{zz}$ have the same
colors, while $\alpha_{yy}$ is \texttt{gold}, \texttt{olive}, and 
\texttt{light-blue} in the three cases, respectively. For the other
crystal systems the thermal expansion tensor is not given.
The thermal expansion tensor from the mode Gr\"uneisen parameters is 
calculated only when the \texttt{elastic\_constants} directory contains at 
least one file with the elastic constants. In this case also 
$C_P-C_V$, $C_\sigma-C_\epsilon$, $C_V-C_\epsilon$, and the 
average Gr\"uneisen parameters are calculated using this thermal 
expansion tensor and plotted in \texttt{green}.
The volume used in these calculations is the \texttt{blue} curve 
if \texttt{ltherm\_freq=.TRUE.}
or as in the \texttt{red} curve if \texttt{ltherm\_freq=.FALSE.}
and \texttt{ltherm\_dos=.TRUE.}.
When both \texttt{ltherm\_freq=.FALSE.} and \texttt{ltherm\_dos=.FALSE.}
the volume is kept fixed at the equilibrium volume at $T=0$ K.
The same applies for the bulk modulus calculated from a single elastic
constant file when the flag \texttt{lb0\_t=.FALSE.} or computed 
within the ``quasi-static'' approximation when \texttt{lb0\_t=.TRUE.}. 
The $C_P$, $C_\sigma-C_\epsilon$, $C_V-C_\epsilon$, and  
the average Gr\"uneisen parameter are plotted only in presence of 
one or more elastic constants file. 

\item
Thermal stresses as a function of temperature. This plot is composed of
one figure in cubic solids and of two figures in the other cases. 
One figure contains $b_{xx}$ (red, pink), $b_{yy}$ (blue, light\_blue)
and $b_{zz}$ (dark\_green, green) as a function of temperature. 
The first color refers to quantities calculated from phonon density of states
while the second refers to quantities calculated by Brillouin zone
integration. If the curves coincide, only the last one (green) will 
be visible. The second figure, when plotted, shows $b_{xy}$ (red, pink), 
$b_{xz}$ (blue, light\_blue), and $b_{yz}$ (dark\_green, green).

\item
Mode Gr\"uneisen parameters. In this plot the mode Gr\"uneisen parameters have
the color of the irreducible representation of the phonon dispersion curve
of which they are the derivative.
The same comments made for the band structure plot apply here.

\item
Generalized average Gr\"uneisen parameters as a function of temperature. 
This plot is composed of one figure in cubic solids and of two figures in 
the other cases. 
One figure contains $\gamma_{xx}$ (red, pink), $\gamma_{yy}$ (blue, light\_blue)
and $\gamma_{zz}$ (dark\_green, green) as a function of temperature. 
The first color refers to quantities calculated from phonon density of states
while the second color refers to quantities calculated by Brillouin zone
integration.
If the curves coincide, only the last one (green) will be visible. 
The second figure, 
when plotted shows $\gamma_{xy}$ (red, pink), $\gamma_{xz}$ 
(blue, light\_blue), and $\gamma_{yz}$ (dark\_green, green).

\item
Phonon dispersions at the geometry that corresponds to a given temperature. 
The colors are assigned on the basis of the irreducible representation of
each mode. The same comments made for the band structure plot apply here.

\item
Temperature dependence of the isothermal and isoentropic elastic constants 
within the ``quasi-static'', ``fixed geometry quasi-harmonic'' or
``quasi-harmonic'' approximation. There is a plot for each non-zero
elastic constant and a plot of the bulk modulus. The number of plots 
depends on the Laue class. 
Elastic constants 
interpolated at the geometry computed using the phonon density of states 
are in \texttt{red} (isothermal) and \texttt{green} (isoentropic), 
those calculated from integration over the 
Brillouin zone are in \texttt{blue} (isothermal) and \texttt{orange} 
(isoentropic).

\item
Temperature dependence of the isothermal and isoentropic elastic compliances
within the ``quasi-static'', ``fixed geometry quasi-harmonic'' or
``quasi-harmonic'' approximation. There is a plot for each non-zero
elastic compliance and a plot of the compressibility. The number of plots 
depends on the Laue class. 
Elastic compliances
interpolated at the geometry computed using the phonon density of states 
are in \texttt{red} (isothermal) and \texttt{green} (isoentropic), 
those calculated from integration over the 
Brillouin zone are in \texttt{blue} (isothermal) and \texttt{orange} 
(isoentropic).

\item
Temperature dependence of the isothermal elastic constants within the 
``fixed geometry quasi-harmonic'' approximation for all the geometries of
the mesh. There is a plot for each non-zero
elastic constant and a plot of the bulk modulus. The number of plots 
depends on the Laue class. 
Elastic constants of the different geometries are in the sequence 
\texttt{red}, \texttt{green}, \texttt{blue}, \texttt{yellow}, 
\texttt{pink}, \texttt{cyan}, \texttt{orange}, \texttt{black}. When there
are more than eight geometries the sequence is repeated. 
The same colors are used for the elastic constants obtained with the
phonon density of states or from the integration over the Brillouin zone.

\item
Temperature dependence of the isothermal elastic compliances within the 
``fixed geometry quasi-harmonic approximation'' for all the geometries of
the mesh. There is a plot for each non-zero
elastic compliance and a plot of the compressibility. The number of plots 
depends on the Laue class. 
Elastic compliances of the different geometries are in the sequence 
\texttt{red}, \texttt{green}, \texttt{blue}, \texttt{yellow}, 
\texttt{pink}, \texttt{cyan}, \texttt{orange}, \texttt{black}. When there
are more than eight geometries the sequence is repeated. 
The same colors are used for the elastic compliances obtained with the
phonon density of states or from the integration over the Brillouin zone.

\item
Anharmonic quantities as a function of temperature (pressure)
plotted for several pressures (temperatures) chosen using 
\texttt{npress\_plot} (\texttt{ntemp\_plot}). 
Each pressure (temperature) is
plotted with a different color, in the sequence \texttt{red}, 
\texttt{green}, \texttt{blue}, \texttt{yellow},
\texttt{pink}, \texttt{cyan}, \texttt{orange}, \texttt{black} from
\texttt{press\_plot}$(1)$ to \texttt{press\_plot}$($\texttt{npress\_plot}$)$.
If there are more than eight pressures (temperatures)
the sequence of colors is repeated.

\end{itemize}

\newpage
{\color{dark-blue}\chapter{Documentation}}
\color{black}

In addition to this user's guide, this directory contains the following
documents:

\begin{itemize}

\item
\texttt{tutorial.pdf}: a short guide that indicates where to find the
information needed to compute a given quantity. 

\item
\texttt{point\_groups.pdf}: a description of the crystallographic point 
groups, character tables of the irreducible representations of point
groups and of the double point groups and tables of the projective 
representations, for the interpretation of the color codes in the band 
and phonon dispersion plots.

\item
\texttt{thermo.pdf}: some notes on the thermodynamic expressions implemented
in \thermo.

\item
\texttt{developer\_guide.pdf}: some notes on the internal logic of
\thermo.

\item
\texttt{unit.pdf}: some notes on the atomic units used in 
\thermo.

\item
\texttt{equilibrium.pdf}: some notes on the atomic units used in 
\thermo\ for the equilibrium thermodynamic tensors.

\end{itemize}


\end{document}
