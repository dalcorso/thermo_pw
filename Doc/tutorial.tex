%!
%! Copyright (C) 2014-2023 Andrea Dal Corso 
%! This file is distributed under the terms of the
%! GNU General Public License. See the file `License'
%! in the root directory of the present distribution,
%! or http://www.gnu.org/copyleft/gpl.txt .
%!
\documentclass[12pt,a4paper,twoside]{report}
\def\version{1.9.0}

\usepackage[T1]{fontenc}
\usepackage{bookman}
\usepackage{html}
\usepackage{graphicx}
\usepackage{fancyhdr}
\usepackage[Lenny]{fncychap}
\usepackage{color}
\usepackage{geometry}

\pagestyle{fancy}
\lhead{Tutorial}
\rhead{}
\cfoot{\thepage}

\newgeometry{
      top=3cm,
      bottom=3cm,
      outer=3cm,
      inner=3.5cm,
}
\definecolor{web-blue}{rgb}{0,0.5,1.0}
\definecolor{coral}{rgb}{1.0,0.5,0.3}
\definecolor{red}{rgb}{1.0,0,0.0}
\definecolor{green}{rgb}{0.,1.0,0.0}
\definecolor{dark-blue}{rgb}{0.,0.0,0.6}

\def\qe{{\sc Quantum ESPRESSO}}
\def\pwx{\texttt{pw.x}}
\def\phx{\texttt{ph.x}}
\def\configure{\texttt{configure}}
\def\PWscf{\texttt{PWscf}}
\def\PHonon{\texttt{PHonon}}
\def\thermo{{\sc Thermo}\_{\sc pw}}
\def\make{\texttt{make}}

\begin{document} 

\author{Andrea Dal Corso (SISSA - Trieste)}
\date{}

\title{
  \includegraphics[width=5cm]{thermo_pw.jpg} \\
  \vspace{3truecm}
  % title
  \Huge \color{dark-blue} {\sc Thermo}\_{\sc pw} Tutorial \\ (v.\version)
}

\maketitle

\newpage

\tableofcontents

\newpage

{\color{dark-blue}\chapter*{Introduction}}
\color{black}

This guide gives a brief overview of the \thermo\ package. 
It gives you the minimum information needed to accomplish a given task.
It assumes that you have already downloaded and compiled \qe\ and \thermo\  
and you want to calculate some material property but you do not know where 
to start. 
The most complete reference for the use of \thermo\ is its user's guide.
However the capabilities of the code are rapidly expanding and the 
user's guide is getting long so, if you do not want
to read it entirely, here you find where to search the relevant information. 
There might be some limitations on the type of functionals that you can use 
or the electronic structure scheme available to compute a given quantity.
Please refer to the \qe\ documentation for this.

\newpage

{\color{coral}\section{People}}
\color{black}
This tutorial has been written by Andrea Dal Corso (SISSA - Trieste). 

\newpage

{\color{coral}\section{Overview}}
\color{black}

In order to make a calculation with \thermo\ you need  
an input file for the \texttt{pw.x} code of \qe. This input file
requires mainly five information:
\begin{itemize}
\item The Bravais lattice.

\item The position of the atoms inside the unit cell.

\item The type of atoms and the pseudopotentials files that you want to use.

\item The cut-off energies.

\item The {\bf k}-point mesh used to make integration over
the Brillouin zone. The smearing parameter for metals. 
\end{itemize}

The Bravais lattice is specified by an integer number \texttt{ibrav} and by the
crystal parameters \texttt{celldm} (up to six numbers).  
The \texttt{ibrav} codes and the required crystal parameters 
are described in the file \texttt{PW/Doc/INPUT\_PW} of the 
\qe\ distribution. 
In \qe\ you can use \texttt{ibrav=0} and give the primitive
lattice vectors of the Bravais lattice. Presently \thermo\ needs to
know the Bravais lattice number so this form of input is not recommended. 
If you use it, \thermo\ writes on output \texttt{ibrav}, 
\texttt{celldm} and the atomic coordinates needed to simulate
the same cell and stops.
You can just cut and paste these quantities in the input of \texttt{pw.x}
or you can set the flag \texttt{find\_ibrav=.TRUE.} in the \thermo\ input
and \thermo\ will make the conversion for you before runnig the job. 
After setting the correct
\texttt{ibrav} and \texttt{celldm}, \thermo\ might still tell you
that the Bravais lattice is not compatible with the point group. This
can happen, for instance, if you have isolated molecules, amorphous solids,
defects, or supercells. In these cases you can still continue but symmetry 
will not be 
used to reduce the number of components of the physical quantities tensors. 
In order to use the residual symmetry, you have to
use one of the suggested \texttt{ibrav}, adjusting the \texttt{celldm} to
the parameters of your cell. For instance if you have a cubic cell, but
the symmetry requires a tetragonal lattice, you have to use a tetragonal
lattice with \texttt{celldm(3)=1.0}.
In rare cases, with lattices such as the face-centered orthorhombic some
symmetry operations might be incompatible with the FFT grid found by 
\texttt{pw.x}. The choice made in \qe\ is to discard these symmetries making
the lattice incompatible with the point group. In these cases the code needs 
\texttt{nr1=nr2=nr3}. Set these three parameters in the \texttt{pw.x} input 
equal to the largest one. 

The positions of the atoms inside the unit cell are defined by an integer
number \texttt{nat} (the number of atoms) and by \texttt{nat} 
three-dimensional vectors as explained in the file \texttt{PW/Doc/INPUT\_PW}.
You can use several units, give the coordinates in the Cartesian or in the
crystal basis or you can give the space group number and the
crystal coordinates of the nonequivalent atoms. Note that in centered lattices
the crystal basis is the conventional one when using the space group number
and the primitive one when not using it.
These options are supported by \thermo. See the \texttt{pw.x} manual
for details. \\

The number of different types of atoms is given by an integer number 
\texttt{ntyp} and for each atomic type you need to specify a 
pseudopotential file. Pseudopotential files depend on the exchange and 
correlation functional and can be found in many 
different places. There is a pseudopotential page in the \qe\ website, or 
you can consider generating your pseudopotentials with the \texttt{pslibrary} 
inputs. You can consult the web page 
\begin{center}
\texttt{https://dalcorso.github.io/pslibrary/} 
\end{center}
for more information. \\

The kinetic energies cut-offs depend on the pseudopotentials
and on the accuracy of your calculation. You can 
find some hints about the required cut-offs inside the pseudopotentials files,
but you need to check the convergence of your results with the cut-off 
energies. Many tests of the kinetic energy cut-offs can be found also at 
\texttt{https://www.materials} \texttt{cloud.org/discover/sssp/}. \\

The {\bf k}-point mesh is given by three integer numbers and possible
shifts (0 or 1) in the three directions. The convergence of the results
with this mesh should be tested. For metals you have also to specify a
smearing method (for instance \texttt{occupations='smearing'}, 
\texttt{smearing='mp'}) and a value of the smearing parameter (in Ry) 
(see the variable \texttt{degauss} in the file \texttt{PW/Doc/INPUT\_PW}). 
Note that the convergence with respect to the {\bf k}-points depends on
the value of \texttt{degauss} and must be tested for each \texttt{degauss}. \\

Once you have an input for \texttt{pw.x}, in order to run \thermo\ you
have to write a file called \texttt{thermo\_control} that contains a
namelist called \texttt{INPUT\_THERMO}. This namelist contains a keyword
\texttt{what} that controls the calculation performed by 
\thermo. Ideally you need to set only \texttt{what} and call 
\texttt{thermo\_pw.x} instead of \texttt{pw.x}, giving as input the 
input prepared for \texttt{pw.x}. In practice each \texttt{what}
is controlled by several variables described in the user's guide.
These variables have default values that are usually sufficient to
give a first estimate of a given quantity but that must be fine tuned
to obtain publication quality results.

\newpage
{\color{dark-blue}\chapter{Howtos}}

{\color{coral}\section{How do I make a self-consistent calculation?}}
\color{black}
Use \texttt{what='scf'}. See \texttt{example01}. The calculation is
equivalent to a call to \texttt{pw.x} and is controlled by
its input. In particular in the input of \texttt{pw.x} you can choose
a single self-consistent calculation using \texttt{calculation='scf'}, 
an atomic relaxation using \texttt{calculation='relax'}, or a cell relaxation 
using \texttt{calculation='vc-relax'}.
The use of \texttt{calculation='nscf'} and \texttt{calculation}
\texttt{='bands'} is
not supported by \thermo\ and could give unpredictable results.
There is no advantage to use \thermo\ to do a molecular dynamic
calculation. 

\newpage

{\color{coral}\section{How do I plot the band structure?}}
\color{black}
Use \texttt{what='scf\_bands'}. See \texttt{example02}.
With this option \thermo\ calls 
\texttt{pw.x} twice, making first a self-consistent calculation with 
the parameters
given in the \texttt{pw.x} input and then a band calculation along a 
path chosen by \thermo, or along a path given by the user
after the \texttt{INPUT\_THERMO} namelist. In this case the path is given as
in the \texttt{pw.x} input (see the user's guide for additional details).
There are a few parameters that you can give in the \texttt{INPUT\_THERMO} namelist to control the band plot. The most useful are 
\texttt{emin\_input} and \texttt{emax\_input} that allow you to plot the 
bands in a given energy range. At the end of the run, the figure of 
the bands is in a file called by default \texttt{output\_band.ps}, a name
that can be changed in the \thermo\ input.
Check also the option \texttt{what='scf\_2d\_bands'} 
to plot the surface band structure.
\newpage

{\color{coral}\section{How do I plot the electronic density of states?}}
\color{black}
Use \texttt{what='scf\_dos'}. See \texttt{example18}. With this option
\thermo\ calls \texttt{pw.x} twice, making first a self-consistent calculation
followed by a non self-consistent calculation on a uniform {\bf k}-point mesh.
This mesh can be specified in the \thermo\ input (if none is given \thermo\ 
uses the default values). At the end of the run, the figure of the density of
states is in a file called by default \texttt{output\_eldos.ps}, a name
that can be changed in the \thermo\ input.
\newpage

{\color{coral}\section{How can I see the crystal structure?}}
\color{black}
\thermo\ is not a crystal structure viewer, but you can use the
\texttt{XCrySDen} code, which reads the \texttt{pw.x} input, to
see the crystal structure. If you use \texttt{what='plot\_bz'}, 
\thermo\ produces a \texttt{.xsf} file with the input structure 
that can be read by \texttt{XCrySDen}. This could
be useful when you give the space group number and the nonequivalent
atomic positions since this input is presently not readable by 
\texttt{XCrySDen}.
The generated \texttt{.xsf} file contains all the symmetry equivalent 
atomic positions. For the same purpose you could also use the output
of \texttt{pw.x}.

\newpage

{\color{coral}\section{How can I see the Brillouin zone?}}
\color{black}
Use \texttt{what='plot\_bz'}. See \texttt{example12}. With this option
\thermo\ does not call \texttt{pw.x}, it just produces a script for
the \texttt{asymptote} code with the instructions to plot the Brillouin
zone and the standard path (or the path that you have given in the \thermo\ 
input). Starting from version 1.5.0 \thermo\ produces also a
script to plot the Brillouin zone using the \texttt{freeCAD} software
(version 0.18). See more details in the user's guide.


\newpage

{\color{coral}\section{How can I plot the X-ray powder diffraction spectrum?}}
\color{black}
Use \texttt{what='plot\_bz'} to see the spectrum corresponding to
the geometry given in the \texttt{pw.x} input. You can also see the
spectrum corresponding to a relaxed structure using for instance
\texttt{what='scf'}, asking for an atomic (cell) relaxation in the \texttt{pw.x}
input and using \texttt{lxrdp=.TRUE.} variable in the \thermo\ input.
The X-ray powder diffraction spectrum is shown in a file called by default
\texttt{output\_xrdp.ps}, a name that can be changed
in the \thermo\ input. The scattering angles and intensities
are also written in a file called by default \texttt{output\_xrdp.dat}, 
which can also be changed in the \thermo\ input.

\newpage
{\color{coral}\section{How can I find the space group of my crystal?}}
\color{black}
Use \texttt{what='plot\_bz'} and look at the output. The space group is
identified. In the case you have a structure with \texttt{ibrav=0} and
the primitive lattice vectors use the option \texttt{find\_ibrav=.TRUE.} 
in the \thermo\ input (see the \thermo\ user's guide 
in the subsection
{\it Coordinates and structure}). This option has presently some limitations.
It does not work for noncollinear magnetic system, or for supercells, or
when the Bravais lattice and the point group are incompatible.

\newpage
{\color{coral}\section{How do I plot the phonon dispersions?}}
\color{black}
Use \texttt{what='scf\_disp'}. See \texttt{example04}. In this case you
have to prepare an input for the \texttt{ph.x} code that must be
called \texttt{ph\_control}. The required information in this input
is the {\bf q}-point mesh on which the dynamical matrices are computed
and the name of the files where the dynamical matrices are written.
See the \texttt{ph.x} guide if you need information on this point.
At the end of the run, the phonon dispersions are found in a file 
called by default \texttt{output\_disp.g1.ps},
a name that can be changed in the \thermo\ input.
The vibrational density of states is found in a file called by default
\texttt{output\_dos.g1.ps}, which can also be changed in the \thermo\ input.

\newpage
{\color{coral}\section{How do I calculate the vibrational energy, 
free energy, entropy, and heat capacity?}}
\color{black}
Use \texttt{what='scf\_disp'}. See \texttt{example04}. These quantities
are calculated after the phonon dispersion for the default 
temperature range ($1$ K - $800$ K) or for the range
given in the \thermo\ input. The figure of these quantities is in the file 
\texttt{output\_therm.g1.ps}.
Note that they are calculated at the geometry given in the input
of \texttt{pw.x}.

\newpage
{\color{coral}\section{How do I calculate the atomic B-factor of a solid?}}
\color{black}
Use \texttt{what='scf\_disp'} as in \texttt{example04} and add the
flag \texttt{with\_eigen=.TRUE.}. These quantities
are calculated after the phonon dispersions for the default 
temperature range ($1$ K - $800$ K) or for the range
given in the \thermo\ input. The figure of these quantities is in the file 
\texttt{output\_therm.g1\_dw.ps}.
Note that they are calculated at the geometry given in the input
of \texttt{pw.x}.

\newpage
{\color{coral}\section{How do I calculate the elastic constants?}}
\color{black}
Use \texttt{what='scf\_elastic\_constants'}. See \texttt{example13}. The
elastic constants appear in the output of \thermo\ and also in a file
called by default \texttt{output\_} \texttt{el\_cons.dat}, a name 
that can be changed in the \thermo\ input. This file can be read 
by \thermo\ for the options
that require the knowledge of the elastic constants.

\newpage
{\color{coral}\section{How do I calculate the Debye temperature?}}
\color{black}
Use \texttt{what='scf\_elastic\_constants'}. See \texttt{example13}. The
Debye temperature appears in the output of \thermo. A file called
\texttt{output\_therm\_debye.g1.ps} contains plots of the vibrational
energy, free energy, entropy, and heat capacity computed within the 
Debye model.

\newpage
{\color{coral}\section{How do I calculate the equilibrium structure?}}
\color{black}
Use \texttt{what='mur\_lc'}. See \texttt{example05} for the cubic case
and refer to the user's
guide for anisotropic solids. The crystal
parameters are written in the \thermo\ output file. Note that the structure is
searched interpolating with a polynomial or with the Murnaghan
equation the energy calculated for several geometries close to the geometry 
given in the input of \texttt{pw.x} so the closer this structure to the 
actual equilibrium structure the better the fit and the
closer the value found by \thermo\ to the real minimum. 
You can check on the file
\texttt{output\_mur.ps} (when \texttt{lmur=.true.}) or 
\texttt{output\_energy.ps} (when \texttt{lmur=.false.}) if the minimum
is within the range of calculated structures. If it is not, 
the calculated minimum is probably inaccurate and it is better to repeat the 
calculation starting from it.
Note also that almost all options can be specified using
\texttt{what='mur\_lc\_...'} instead of \texttt{what='scf\_...'}.
In this case the calculations are carried out at the equilibrium geometry 
instead of the geometry given in the \texttt{pw.x} input. 
Setting a finite pressure in the \thermo\ input, 
the equilibrium geometry is the one at the given pressure (see below)
and the calculations are carried out at this geometry.

\newpage
{\color{coral}\section{How do I calculate the equilibrium structure 
at a given pressure?}}
\color{black}
Use \texttt{what='mur\_lc'} and specify \texttt{pressure=value} in kbar in the
\thermo\ input. Note that in the input of \texttt{pw.x} you should 
give a geometry which is as close as possible to the equilibrium value
found at the given pressure (see above).

\newpage
{\color{coral}\section{How do I specify the temperature range}} 
\color{black}
See the subsection {\it Temperature and Pressure} in the 
\thermo\ user's guide. 

\newpage
{\color{coral}\section{How do I calculate the crystal parameters as a function
of temperature?}}
\color{black}
Use \texttt{what='mur\_lc\_t'}. See \texttt{example09}. Note that
for this option you need to give also the \texttt{ph.x} input.
For anisotropic solids using \texttt{lmurn=.TRUE.} you calculate 
the volume as a function of temperature varying \texttt{celldm(1)} 
but all the other crystal parameters are kept constant, while
using \texttt{lmurn=.FALSE.} you can calculate all the crystal
parameters as a function of temperature.
The crystal parameters are plotted as a function of temperature
in the standard range ($T=1$ K - $T=800$ K) or in the range  
requested in input, in the file 
\texttt{output\_anharm.ps}, a name that can be changed in the \thermo\ input.
Presently no temperature dependence is calculated 
for the atomic coordinates, so the present approach is applicable
to solids in which equilibrium atomic positions are fixed by 
symmetry, while it is an approximation in the other cases.

\newpage
{\color{coral}\section{How do I calculate the thermal expansion?}}
\color{black}
Use \texttt{what='mur\_lc\_t'}. See \texttt{example09}. The components
of the thermal expansion tensor are shown as a function of temperature 
in the file \texttt{output\_anharm.ps}, a name that can be changed in 
the \thermo\ input.

\newpage
{\color{coral}\section{How do I calculate the Helmholtz (or Gibbs) free energy
as a function of temperature keeping into account the thermal expansion?}}
\color{black}
Use \texttt{what='mur\_lc\_t'}. The Helmholtz (or Gibbs at finite pressure) 
free energy is shown as a function of temperature in the file 
\texttt{output\_anharm.ps}. Note that the absolute value of this energy
depends on the pseudopotentials as the total energy. You can however
compare the free energies for different crystal structures 
and predict if a phase transition occurs and at which temperature (also 
as a function of pressure).

\newpage
{\color{coral}\section{How do I calculate the bulk modulus as a function of 
temperature?}}
\color{black}
Use \texttt{what='mur\_lc\_t'} and the option \texttt{lmurn=.TRUE.}.
This approach is rigorously valid only for cubic solids, for anisotropic
solid it is an approximation in which only \texttt{celldm(1)} is
changed while the other crystal parameters are kept constant. 
For the general case, see the elastic constants as a function of 
temperature. This calculation gives also the bulk modulus as a function 
of temperature.

\newpage
{\color{coral}\section{How do I calculate the isobaric heat capacity?}}
\color{black}
Use \texttt{what='mur\_lc\_t'} and the option \texttt{lmurn=.TRUE.}.
This approach is rigorously valid only for cubic solids, for anisotropic
solid it is an approximation in which only \texttt{celldm(1)} is
changed while the other crystal parameters are kept constant.
For the general case, see the elastic constants
as a function of temperature. This calculation gives also the 
isobaric heat capacity as a function of temperature.

\newpage
{\color{coral}\section{How do I calculate the elastic constants 
as a function of temperature?}}
\color{black}
There are two ways. Both of them are a two step calculation. 
The fastest, but less accurate, method uses the ``quasi-static'' approximation.
First use 
\texttt{what='elastic\_constants\_geo'}. This option computes the elastic
constants at $T=0$ K for all the geometries used by 
\texttt{what='mur\_lc\_t'} and
saves them in the directory \texttt{elastic\_constants}.  
In the second step, using the same input, run again 
\thermo\ with \texttt{what='mur\_lc\_t'} after copying in your
working directory the directory \texttt{elastic\_constants} obtained 
in the previous step. Elastic constants are read only when
\texttt{lmurn=.FALSE.}.
The ``quasi-static'' approximation means that the code interpolates
the $T=0$ K elastic constants found in the first step at the geometry that
minimizes the Helmholtz (or Gibbs at finite pressure) free energy at
temperature $T$. \\
The second method uses the ``quasi-harmonic'' approximation and
requires many phonon calculations at many geometries.
First use \texttt{what='elastic\_} \texttt{constants\_geo'} and set
\texttt{use\_free\_energy=} \texttt{.TRUE.}. This option computes the 
temperature dependent elastic constants taking as unperturbed geometries 
those used by \texttt{what='mur\_lc\_t'} and saves them in the directory 
\texttt{anhar\_files}.  
In the second step, using the same input, run again 
\thermo\ with \texttt{what=} \texttt{'mur\_lc\_t'} after 
copying in your
working directory the directory \texttt{anhar\_files} obtained 
in the previous step. Elastic constants are read only when
\texttt{lmurn=} \texttt{.FALSE.}. 
The ``quasi-harmonic'' approximation means that the code interpolates
the temperature dependent elastic constants found in the first step as
second derivatives of the Helmholtz free energy (corrected to give the
stress-strain elastic constants), at the geometry that
minimizes the Helmholtz free energy. \\
The plot of the elastic constants will be found in a file whose default
name is \texttt{output\_anhar.el\_cons.ps}.
Note that when the elastic constant are available in a file, 
with the option \texttt{lmurn=.FALSE.} the thermodynamic properties 
are calculated also for non-cubic solids.
The main approximation of the present implementation is that the atomic 
coordinates are relaxed only at $T=0$ K minimizing the energy and
the free energy is 
not calculated as a function of the atomic positions.

\newpage
{\color{coral}\section{How do I calculate the electronic heat capacity
of a metal?}}
\color{black}
Use \texttt{what='scf\_dos'}. See \texttt{example18}. In the metallic
case in addition to a plot of the density of states this option produces
also a plot of the electronic energy, free energy, entropy, heat capacity,
and chemical potential in the standard temperature range 
($T=1$ K - $T=800$ K) or in the range requested in input.
These quantities are found in the file called by default
\texttt{output\_eltherm.ps}, a name that can be changed in the \thermo\ input.
Please be careful with the value of \texttt{deltae} or the electronic
thermodynamic properties could be wrong at low temperatures.

\newpage
{\color{coral}\section{How do I calculate the thermal expansion of a metal
accounting for the free energy due to the electronic excitations?}}
\color{black}
This is a two step calculation. First use \texttt{what='mur\_lc'} and
the flag \texttt{lel\_} \texttt{free\_energy=.TRUE.} to write on files the
electronic thermodynamic properties for each geometry used by
\texttt{what='mur\_lc'}. Then copy the files with
the electronic thermodynamic properties in the  
\texttt{therm\_files} directory and run \thermo\ with \texttt{what='mur\_lc\_t'}
and the flag \texttt{lel\_free\_energy=.TRUE.}. The electronic free energy
is read from files and added to the vibrational free energy before computing
the equilibrium crystal structure.
See \texttt{example24} for this case. 

\newpage
{\color{coral}\section{How do I calculate the elastic constants of a metal
accounting for the free energy due to the electronic excitations?}}
\color{black}
This is a four step calculation. First use \texttt{what='mur\_lc'} and
the flag \texttt{lel\_} \texttt{free\_energy=.TRUE.} as in the previous point
in order to get the electronic thermodynamic properties for all the 
unperturbed geometries. Then use \texttt{what=} \texttt{'elastic\_constant\_t'}
with the flag \texttt{lel\_free\_energy=.TRUE.} and the flag
\texttt{use\_free\_energy=.FALSE.} to write on file the electronic
thermodynamic properties for all strained geometries. Then use
\texttt{what='elastic\_constant\_t'} with \texttt{lel\_free\_energy=.TRUE.}
and \texttt{use\_free\_energy=.TRUE.} after copying the electronic
thermodynamic properties obtained at previous point in the 
\texttt{therm\_files} directory, so the 
free energy derived in this step to obtain the elastic constants 
has also the electronic contribution. 
Finally use \texttt{what='mur\_lc\_t'} with \texttt{lel\_free\_energy=.TRUE.}
and \texttt{use\_free\_energy=} \texttt{.TRUE.} after copying the elastic
constants obtained at previous step in the \texttt{anhar\_files} directory
and the electronic thermodynamic properties obtained in the first 
step in the \texttt{therm\_files} directory.

\newpage
{\color{coral}\section{How do I calculate the frequency dependent dielectric
constant of a material?}}
\color{black}
Use \texttt{what='scf\_ph'}. See \texttt{example16} and \texttt{example20}. 
Please note that this calculation is made by the extended version of the
phonon code available in \thermo, so you have to provide specific flags
to the phonon input. See the user's guide for details. 
The dielectric constant is in the file \texttt{output\_epsilon.ps}.
Note that for metals you need to specify a finite wave-vector {\bf q} and you 
cannot plot the dielectric constant when {\bf q} is the $\Gamma$ point.  
See \texttt{example17} and \texttt{example21} for this case. 

\newpage
{\color{coral}\section{How do I calculate the frequency dependent 
reflectivity of a material?}}
\color{black}
The calculation of this quantity is implemented only for isotropic 
(cubic) solids. It is plotted together with the frequency dependent 
dielectric constant in insulating cubic solids when {\bf q} is 
the $\Gamma$ point. In this case the code plots also the 
absorption coefficient. Both quantities can be plotted as a function of
the frequency or of the wavelength.

\newpage
{\color{coral}\section{Which is the meaning of the colors in
the electronic bands and phonon dispersions plots?}}
\color{black}
Different colors correspond to different irreducible representations of 
the point co-group of the {\bf k} or {\bf q} wavevector. To see the
correspondence color-representation see the \texttt{point\_groups.pdf} 
manual. The point group used for each {\bf k} or {\bf q} point is
written in the \thermo\ output and also in the plot if you set
\texttt{enhance\_plot=.TRUE.}. In the output you can also find,
close to each band energy or phonon frequency value, the name of the
irreducible representation. Relevant character tables are given in the
\texttt{point\_groups.pdf} manual, in the \thermo\ output, or by the
\texttt{crystal\_point\_group.x} tool.

\newpage
{\color{coral}\section{How do I specify a custom path for 
plotting the electronic bands and phonon dispersions?}}
\color{black}
You have to give an explicit path after the \texttt{INPUT\_THERMO}
namelist (see \texttt{what='scf\_bands'} and \texttt{what='scf\_2d\_bands'}). 
There are two possibilities to specify a path in the Brillouin zone.
You can use explicit coordinates of the first and last points of 
each line (in cartesian or crystal coordinates) and a weight that indicates
the number of points in the line, or you can use labels to 
indicate the first and/or the last points. The available labels depend on
the Brillouin zone and are illustrated in the file
\texttt{Doc/brillouin\_zones.pdf} in the main QE documentation.
Labels for points given by explicit coordinates can be added after 
the weights. 
Usually, the same label that indicates the coordinates of the point
is written in the dispersion plot, but it is also possible
to override this label, adding a letter after the weight as for 
the points given by explicit coordinates. 
Note that labels are given without quotes, while the additional letter
after the weight must be written between single quotes.
An example of a path for the fcc lattice:
\begin{verbatim}

7
gG   20   
X    20
W    20  'gS1'
1.0 1.0 0.0 20 'X'
gG    0
gG   20   
L     0
\end{verbatim}
This path has five lines, the first from $\Gamma$ to $X$ with $20$ points,
the second from $X$ to $W$ (that in the plot will have the label $\Sigma_1$)
with 20 points, the third from $W$ to the point of coordinates 
$(1.0, 1.0, 0.0)$ 
(that in the plot will have the label $X$), the fourth from $(1.0, 1.0, 0.0)$ 
to $\Gamma$ with $20$ points and a final line from $\Gamma$ to $L$ with 
$20$ points.
Until version \texttt{1.8.1} labels are read as three characters, so it 
is important not to write the weights too close to the labels.
Gr\"uneisen parameters are not analytic at the $\Gamma$ 
point so to obtain a correct plot, the $\Gamma$ label in the middle of a 
plot must be repeated twice, as in the example.
\end{document}
