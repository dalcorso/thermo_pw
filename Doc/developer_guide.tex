\documentclass[12pt,a4paper]{article}
\def\version{0.4.0}
\def\qe{{\sc Quantum ESPRESSO}}
\def\tpw{{\sc THERMO\_PW}}

\usepackage{html}

\usepackage{graphicx}

\textwidth = 17cm
\textheight = 24cm
\topmargin =-1 cm
\oddsidemargin = 0 cm

\def\pwx{\texttt{pw.x}}
\def\phx{\texttt{ph.x}}
\def\configure{\texttt{configure}}
\def\PWscf{\texttt{PWscf}}
\def\PHonon{\texttt{PHonon}}
\def\make{\texttt{make}}

\begin{document}
\author{Andrea Dal Corso (SISSA - Trieste)}
\date{}

\def\qeImage{../../Doc/quantum_espresso.pdf}
\def\democritosImage{../../Doc/democritos.pdf}
\def\SissaImage{./sissa.pdf}

\title{
%  \includegraphics[width=6cm]{\SissaImage}\\
  \vskip 1cm
  \Huge \tpw\ Developer's Guide \\
  \Large (version \version)
}
\maketitle

\tableofcontents

\newpage

\section{Introduction}
This guide discusses the internal organization of \tpw. It is not necessary
to read it to use \tpw. It contains some technical information not 
appropriate for the user's guide.
It assumes that you have read and understood the user's guide and that
you have already some familiarity with the input variables and 
the working modes of the \tpw\ code.

\section{People}
This guide has been written by Andrea Dal Corso (SISSA - Trieste). 

\section {Overview}
\tpw\ is logically divided into parts. In some parts the different
images run asynchronously doing different tasks and in some parts the
run is synchronous. Usually in this synchronous parts results are 
collected and only the first image is actually doing the work while
the other images are blocked waiting for the results.
Presently \tpw\ has two asynchronous runs each one followed
by one or more synchronous parts. The parts that are activated in
each run depend on the input variable \texttt{what}. After reading the input,
and writing some summary, the code calls the routine 
\texttt{initialize\_thermo\_work} that sets all the
logical flags that control the global run and the first asynchronous
part. The first asynchronous part (A) makes \texttt{nwork} \texttt{pw.x} 
calculations in parallel in each image. 
Then the following synchronous part makes one or more of
the following calculations:
\begin{itemize}

\item
B) \texttt{lconv\_ke\_test=.TRUE.} writes and plots the total
energy as a function of the kinetic energy.

\item
C) \texttt{lconv\_nk\_test=.TRUE.} writes and plots the total energy
as a function of the {\bf k}-points set.

\item
D) \texttt{lev\_syn\_1=.TRUE.} makes a Murnaghan fit of the 
total energy/enthalpy as a function of the unit cell volume or 
a quadratic fit of the total energy / enthalpy as a function of the crystal
parameters and determines the minimum energy geometry.

\item
E) \texttt{lpwscf\_syn\_1=.TRUE.} makes a self-consistent \texttt{pw.x} 
calculation.

\item
F) \texttt{lband\_syn\_1=.TRUE.} makes a non self-consistent \texttt{pw.x}
calculation to compute the band structure.

\end{itemize}

The second part can have an asynchronous part in which
many \texttt{pw.x} calculations are made (\texttt{lpart2\_pw=.TRUE.}) (G1)
or a loop of asynchronous phonon runs (G2), one for each geometry calculated
in A or at the minimum energy/enthalpy geometry
found in D whose self-consistent calculation has been done in E.
The G2 run is controlled by \texttt{lph=.TRUE.} and by 
\texttt{tot\_ngeo} which is the number of geometries. For each geometry 
one can calculate the phonon dispersions, the phonon dos and the 
harmonic thermodynamic quantities. The geometry to do is controlled only 
changing the \texttt{outdir} directory where the phonon run finds the
files produced by the \texttt{pw.x} calculation. The final collection
of the dynamical matrices, calculations of the phonon dispersions, phonon 
dos and harmonic thermodynamic quantities are done in a synchronous way 
within the loop over the geometries.

Following G1 or G2 there is another synchronous part. 
After G1 the synchronous part will do one of the following:

\begin{itemize}

\item
H) \texttt{lelastic\_const=.TRUE.} elastic constants calculations.

\item
I) \texttt{lpiezoelectric\_tensor=.TRUE.} piezoelectric tensor calculation.

\item
L) \texttt{lpolarization=.TRUE.} polarization calculation.

\end{itemize}

After G2, depending on \texttt{what} the synchronous part will do nothing, 
or if the phonons have been computed in several geometries it will do:

\begin{itemize}

\item
M) the interpolation and minimization of the 
Helmholtz or Gibbs free energy at any temperature.

\item
N) Calculation of the Gr\"uneisen parameters on a path over the Brillouin
zone and their plot.

\item
O) Calculation of the anharmonic quantities such as the thermal expansion
from the numerical differentiation of the crystal parameters. Calculation
of the isoentropic quantities when programmed.

\item
P) Calculation of the Gr\"uneisen parameters on a uniform mesh of {\bf q} 
points and calculation of the thermal expansion from the Gr\"uneisen parameters.
Calculation of the isoentropic quantities when programmed.

\end{itemize}

\section{Activated parts}

The different \texttt{what} activates the following parts of the code:

\begin{verbatim}
what         A   B   C   D   E    F     G1    H    I   L   G2   M   N   O   P

scf                          *
scf_ke       *   *
scf_nk       *       *
scf_band                     *    *
scf_ph                       *                              *
scf_disp                     *                              *
scf_elastic_constants                    *     *
scf_piezoelectric_tensor                 *          +
scf_polarization                         *              +
mur_lc       *           *
mur_lc_band  *           *   *    *
mur_lc_ph    *           *   *                              *
mur_lc_disp  *           *   *                              *
mur_lc_elastic_constants
             *           *               *      *
mur_lc_piezoelectric_tensor
             *           *               *          +
mur_lc_polarization
             *           *               *               +
mur_lc_t     *           *                                   *     *  *   *  *

+ not yet active or working.
\end{verbatim}


\section{Scripts and Files}

In almost all tasks, \tpw\ produces some data file containing the quantity 
to plot, some script to plot the data and, if the \texttt{gnuplot} code
is available, the postscript files with the plot of the data. So for each 
task there is an associated routine
that writes the output data and a routine that plots the output data
writing the \texttt{gnuplot} script. Usually the name of the routine that
writes the data starts with \texttt{write\_} and the routine that plots
the data is called \texttt{plot\_}. The following is a list of the routines
that write and plot the data and the name of the associated files
containing the data, the scripts and the postscript files.

\begin{verbatim}

what             routines         data           scripts            postscript

scf                 --             --               --                 --
scf_ke          write_e_ke       flkeconv
                plot_e_ke                       flgnuplot(_keconv)  flpskeconv
scf_nk          write_e_nk       flnkconv
                plot_e_nk                       flgnuplot(_nkconv)  flpsnkconv

scf_band        bands_sub
                write_bands      filband
                plotbands_sub    flpbands
                write_gnuplot_file              flgnuplot(_band)    flpsband     

scf_ph             --              --               --                  --
scf_disp                         fildyn
                 q2r_sub         flfrc
                 matdyn_sub      flfrq
                                 fldos
                 simple_plot                    flgnuplot(_dos)     flpsdos
                 write_thermo    fltherm
                 write_thermo_ph fltherm(_ph)
                 plot_thermo                    flgnuplot(_therm)   flpstherm

mur_lc           do_ev           flevdat        (input_ev)
                                 flevdat(.ev.out)
                 write_mur       flevdat(_mur)
                                 flevdat(_mur1)
                 plot_mur                       flgnuplot(_mur)     flpsmur
                 write_gnuplot_energy
                                 flenergy
                 plot_multi_energy              flgnuplot(_energy)  flpsenergy

mur_lc_t        
        For cubic solids: 
                 write_anharmonic
                                 flanhar
                                 flanhar(.aux)
                 write_anharmonic_ph
                                 flanhar(_ph)
                                 flanhar(.aux_ph)
                 write_anharmonic_grun
                                 flanhar(.aux_grun)
                 plot_anhar                     flgnuplot(_anhar)   flpsanhar
                 write_gruneisen_bands
                                 flgrun
                                 flgrun(_freq)
                 plotband_sub    flpgrun
                                 flpgrun(_freq)
                 write_gnuplot_file             flgnuplot(_grun)    flpsgrun
                                                flgnuplot(_grun_freq) 
                                                                 flpsgrun(_freq)
        For anisotropic solids:
                 write_anharm_anis
                                 flanhar(.celldm)
                 write_anharm_anis_ph
                                 flanhar(_ph.celldm)
                 plot_anhar_anis                flgnuplot(_anhar)   flpsanhar
                 write_gruneisen_bands_anis
                                 flgrun(_#)
                                 flgrun(_freq)
                 plot_gruneisen_band_anis
                                 flpgrun(_#)
                                 flpgrun(_freq)
                 write_gnuplot_file             flgnuplot(_grun_#)   flpsgrun_#
                                                flgnuplot(_grun_freq) 
                                                                 flpsgrun(_freq)
# is an integer from 1 to the number of crystal parameters
\end{verbatim} 

\noindent This is the list of the plots that can be obtained and 
the \texttt{what} to use:

\begin{verbatim}
Brillouin zone                                 plot_bz
Total energy versus kinetic energy             scf_ke
Total energy versus k points                   scf_nk
Total energy as a function of volume           mur_lc
Total energy as a function of two or three
crystallographic parameters                    mur_lc
Energy bands                                   scf_bands, mur_lc_bands
Phonon dispersions                             scf_disp, mur_lc_disp
Phonon dos                                     scf_disp, mur_lc_disp
Vibrational energy, free energy, entropy,      
isochoric heat capacity                        scf_disp, mur_lc_disp
Lattice parameters as a function of T          mur_lc_t
Thermal expansion as a function of T           mur_lc_t 
Gruneisen parameters dispersions               mur_lc_t
Phonon dispersions at the geometry 
that corresponds to a given temperature        mur_lc_t

For cubic systems also the following quantities are available:

Bulk modulus as a function of T                mur_lc_t
Difference isobaric-isochoric heat capacity    mur_lc_t
Difference isoentropic-isothermal bulk modulus mur_lc_t
Average Gruneisen parameter                    mur_lc_t

\end{verbatim}

\end{document}

