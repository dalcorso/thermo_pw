%!
%! Copyright (C) 2015 - Present Andrea Dal Corso 
%! This file is distributed under the terms of the
%! GNU General Public License. See the file `License'
%! in the root directory of the present distribution,
%! or http://www.gnu.org/copyleft/gpl.txt .
%!
\documentclass[12pt,a4paper]{article}
\def\version{0.6.0}
\def\qe{{\sc Quantum ESPRESSO}}
\def\tpw{{\sc THERMO\_PW}}

\usepackage{html}
\usepackage{color}

\usepackage{graphicx}

\definecolor{web-blue}{rgb}{0,0.5,1.0}
\definecolor{coral}{rgb}{1.0,0.5,0.3}
\definecolor{red}{rgb}{1.0,0,0.0}
\definecolor{green}{rgb}{0.,1.0,0.0}

\textwidth = 17cm
\textheight = 24cm
\topmargin =-1 cm
\oddsidemargin = 0 cm

\def\pwx{\texttt{pw.x}}
\def\phx{\texttt{ph.x}}
\def\configure{\texttt{configure}}
\def\PWscf{\texttt{PWscf}}
\def\PHonon{\texttt{PHonon}}
\def\make{\texttt{make}}

\begin{document}
\author{Andrea Dal Corso (SISSA - Trieste)}
\date{}

%\def\SissaImage{./sissa_on_white.png}

\title{
%  \includegraphics[width=6cm]{\SissaImage}\\
  \vskip 1cm
  {\color{red}\Huge \tpw\ Developer's Guide} \\
  \Large (version \version)
}
\maketitle

\tableofcontents

\newpage

\section{\color{coral}Introduction}
This guide discusses the internal organization of \tpw. It is not necessary
to read it to use \tpw. It contains some technical information not 
appropriate for the user's guide.
It assumes that you have read and understood the user's guide and that
you have already some familiarity with the input variables and 
the working modes of the \tpw\ code.

\section{\color{coral}People}
This guide has been written by Andrea Dal Corso (SISSA - Trieste). 

\newpage
\section{\color{coral}Overview}
\tpw\ is logically divided into parts. In some parts the different
images run asynchronously doing different tasks and in some parts the
run is synchronous. Usually in this synchronous parts results are 
collected and only the first image is actually doing the work while
the other images are blocked waiting for the results.
Presently \tpw\ has two asynchronous runs each one followed
by one or more synchronous parts. The parts that are activated in
each run depend on the input variable \texttt{what}. After reading the input,
and writing some summary, the code calls the routine 
\texttt{initialize\_thermo\_work} that sets all the
logical flags that control the global run and the first asynchronous
part. The first asynchronous part (A) makes \texttt{nwork} \texttt{pw.x} 
calculations in parallel in each image. 
Then the following synchronous part makes one or more of
the following calculations:
\begin{itemize}

\item
B) \texttt{lconv\_ke\_test=.TRUE.} writes and plots the total
energy as a function of the kinetic energy.

\item
C) \texttt{lconv\_nk\_test=.TRUE.} writes and plots the total energy
as a function of the {\bf k}-points set.

\item
D) \texttt{lev\_syn\_1=.TRUE.} makes a Murnaghan fit of the 
total energy/enthalpy as a function of the unit cell volume or 
a quadratic/quartic fit of the total energy / enthalpy as a function 
of the crystal parameters and determines the minimum energy geometry.

\item
E) \texttt{lpwscf\_syn\_1=.TRUE.} makes a self-consistent \texttt{pw.x} 
calculation.

\item
F) \texttt{lband\_syn\_1=.TRUE.} makes a non self-consistent \texttt{pw.x}
calculation to compute the band structure.

\end{itemize}

The second part can have an asynchronous part in which
many \texttt{pw.x} calculations are made (\texttt{lpart2\_pw=.TRUE.}) (G1)
or a loop of asynchronous phonon runs (G2), one for each geometry calculated
in A or at the minimum energy/enthalpy geometry
found in D whose self-consistent calculation has been done in E.
The G2 run is controlled by \texttt{lph=.TRUE.} and by 
\texttt{tot\_ngeo} which is the number of geometries. For each geometry 
one can calculate the phonon dispersions, the phonon dos and the 
harmonic thermodynamic quantities. The geometry to do is controlled only 
changing the \texttt{outdir} directory where the phonon run finds the
files produced by the \texttt{pw.x} calculation. The final collection
of the dynamical matrices, calculations of the phonon dispersions, phonon 
dos and harmonic thermodynamic quantities are done in a synchronous way 
within the loop over the geometries.

Following G1 or G2 there is another synchronous part. 
After G1 the synchronous part will do one of the following:

\begin{itemize}

\item
H) \texttt{lelastic\_const=.TRUE.} elastic constants calculations.

\item
I) \texttt{lpiezoelectric\_tensor=.TRUE.} piezoelectric tensor calculation.

\item
L) \texttt{lpolarization=.TRUE.} polarization calculation.

\end{itemize}

After G2, depending on \texttt{what} the synchronous part will do nothing, 
or if the phonons have been computed in several geometries it will do:

\begin{itemize}

\item
M) the interpolation and minimization of the 
Helmholtz or Gibbs free energy at any temperature.

\item
N) Calculation of the Gr\"uneisen parameters on a path over the Brillouin
zone and their plot.

\item
O) Calculation of the anharmonic quantities such as the thermal expansion
from the numerical differentiation of the crystal parameters. Calculation
of the isoentropic quantities when available.

\item
P) Calculation of the Gr\"uneisen parameters on a uniform mesh of {\bf q} 
points and calculation of the thermal expansion from the Gr\"uneisen parameters.
Calculation of the isoentropic quantities when available.

\end{itemize}

\newpage
\section{\color{coral}Activated parts}

The different \texttt{what} activates the following parts of the code:

\begin{verbatim}
what         A   B   C   D   E    F     G1    H    I   L   G2   M   N   O   P

scf                          *
scf_ke       *   *
scf_nk       *       *
scf_band                     *    *
scf_dos                      *    *
scf_ph                       *                              *
scf_disp                     *                              *
scf_elastic_constants                    *     *
scf_piezoelectric_tensor                 *          +
scf_polarization                         *              +
mur_lc       *           *
mur_lc_band  *           *   *    *
mur_lc_dos   *           *   *    *
mur_lc_ph    *           *   *                              *
mur_lc_disp  *           *   *                              *
mur_lc_elastic_constants
             *           *               *      *
mur_lc_piezoelectric_tensor
             *           *               *          +
mur_lc_polarization
             *           *               *               +
mur_lc_t     *           *                                   *     *  *   *  *
elastic_constants_t                      *      *  

+ not yet active or working.
\end{verbatim}

\newpage

\section{\color{coral}Plots}
This is the list of the plots that can be obtained and 
the \texttt{what} to use:

\begin{verbatim}
Brillouin zone                                 plot_bz
X-ray powder diffraction pattern               plot_bz
Total energy versus kinetic energy             scf_ke
Total energy versus k points                   scf_nk
Total energy as a function of volume           mur_lc
Total energy as a function of two or three
crystallographic parameters                    mur_lc
Energy bands                                   scf_bands, mur_lc_bands
Electronic density of states                     scf_dos, mur_lc_dos
Electronic energy, free energy, entropy,      
isochoric heat capacity (metals only)          scf_dos, mur_lc_dos
Phonon dispersions                             scf_disp, mur_lc_disp
Phonon dos                                     scf_disp, mur_lc_disp
Vibrational energy, free energy, entropy,      
isochoric heat capacity                        scf_disp, mur_lc_disp
Debye vibrational energy, free energy, 
entropy and isochoric heat capacity            scf_elastic_constants, 
                                               mur_lc_elastic_constants
Lattice parameters as a function of T          mur_lc_t
Thermal expansion as a function of T           mur_lc_t 
Mode Gruneisen parameters                      mur_lc_t
Phonon dispersions at the geometry 
that corresponds to a given temperature        mur_lc_t

For cubic systems also the following quantities are available:

Bulk modulus as a function of T                mur_lc_t
Difference isobaric-isochoric heat capacity    mur_lc_t
Difference isoentropic-isothermal bulk modulus mur_lc_t
Average Gruneisen parameter                    mur_lc_t

\end{verbatim}

\newpage
\section{\color{coral}Scripts and Files}

In almost all tasks, \tpw\ produces some data files containing the quantity 
to plot, some scripts to plot the data and, if the \texttt{gnuplot} code
is available, the postscript files with the plot of the data. So for each 
task there is an associated routine
that writes the output data and a routine that plots the output data
writing the \texttt{gnuplot} script. Usually the name of the routine that
writes the data starts with \texttt{write\_} and the routine that plots
the data is called \texttt{plot\_}. The following is a list of the routines
that write and plot the data and the name of the associated files
containing the data, the scripts and the postscript files.

\begin{verbatim}

what             routines         data           scripts            postscript

scf                 --             --               --                 --
scf_ke          write_e_ke       flkeconv
                plot_e_ke                       flgnuplot(_keconv)  flpskeconv
scf_nk          write_e_nk       flnkconv
                plot_e_nk                       flgnuplot(_nkconv)  flpsnkconv

scf_band        bands_sub
                write_bands      filband
                plotbands_sub    flpbands
                write_gnuplot_file              flgnuplot(_band)    flpsband     

scf_dos         dos_sub          fleldos
                plot_dos                        flgnuplot(_eldos)   flpseldos
                write_el_therm   fleltherm
                plot_el_therm                   flgnuplot(_eltherm) flpseltherm

scf_ph             --            fildyn               --                  --
scf_disp                         fildyn
                 q2r_sub         flfrc
                 write_ph_dispersions  flfrq
                 write_phdos     fldos
                 simple_plot                    flgnuplot(_dos)     flpsdos
                 write_thermo    fltherm
                 write_thermo_ph fltherm(_ph)
                 plot_thermo                    flgnuplot(_therm)   flpstherm

mur_lc           do_ev           flevdat        (input_ev)
                                 flevdat(.ev.out)
                 write_mur       flevdat(_mur)
                                 flevdat(_mur1)
                 plot_mur                       flgnuplot(_mur)     flpsmur
                 write_gnuplot_energy
                                 flenergy
                 plot_multi_energy              flgnuplot(_energy)  flpsenergy

mur_lc_t        
        For cubic solids: 
                 write_anharmonic
                                 flanhar
                                 flanhar(.aux)
                 write_anharmonic_ph
                                 flanhar(_ph)
                                 flanhar(.aux_ph)
                 write_anharmonic_grun
                                 flanhar(.aux_grun)
                 plot_anhar                     flgnuplot(_anhar)   flpsanhar
                 write_gruneisen_bands
                                 flgrun
                                 flgrun(_freq)
                 plotband_sub    flpgrun
                                 flpgrun(_freq)
                 write_gnuplot_file             flgnuplot(_grun)    flpsgrun
                                                flgnuplot(_grun_freq) 
                                                                 flpsgrun(_freq)
        For anisotropic solids:
                 write_anharm_anis
                                 flanhar(.celldm)
                 write_anharm_anis_ph
                                 flanhar(_ph.celldm)
                 plot_anhar_anis                flgnuplot(_anhar)   flpsanhar
                 write_gruneisen_bands_anis
                                 flgrun(_#)
                                 flgrun(_freq)
                 plot_gruneisen_band_anis
                                 flpgrun(_#)
                                 flpgrun(_freq)
                 write_gnuplot_file             flgnuplot(_grun_#)   flpsgrun_#
                                                flgnuplot(_grun_freq) 
                                                                 flpsgrun(_freq)
# is an integer from 1 to the number of crystal parameters
\end{verbatim} 

\newpage

\section{\color{coral}Files contents}

Many data files produced by \texttt{thermo\_pw} are in a form readable
by the \texttt{gnuplot} program and are therefore arranged in columns.
Usually the first column is the temperature and the other columns are
the physical quantities. Several other files are in a format exported
from the Quantum ESPRESSO package and are usually readable by its 
programs. In this section we report a short list of the files produced
by \texttt{thermo\_pw} and a short description of the quantities 
saved in each file. 

\begin{itemize}

\item 
\texttt{band\_files/filband(.rap)} (default name 
\texttt{output\_band.dat(.rap)}) contains the band eigenvalues for each 
{\bf k} point. The \texttt{.rap} file contains the number of the irreducible
representation for each band. The format of these files can be read
by the \texttt{plotband.f90} code of Quantum ESPRESSO.

\item 
\texttt{band\_files/flpband(.\#)(.\#)} 
(default name \texttt{output\_pband.dat(.\#)(.\#)})
These files contain the band eigenvalues along the path. They are in a 
format readable by \texttt{gnuplot} and are similar to the files written
by the \texttt{plotband.f90} code of Quantum ESPRESSO. The numbers indicate 
the number of the line along the path and the number of the irreducible 
representation.

\item 
\texttt{band\_files/info\_data} (default name \texttt{info\_data}) This
file contains a minimal number of information needed by \texttt{thermo\_pw}
to redo the plot when \texttt{only\_bands\_plot=.TRUE.}. It is written
in a form readable by \texttt{thermo\_pw}. Among other things it
contains the Fermi level.

\item 
\texttt{energy\_files/flevdat} (default name \texttt{output\_ev.dat})
This file contains the total energy as a function of the volume for the
points actually calculated by \texttt{thermo\_pw}. It has the format
\begin{verbatim}
Omega     energy
\end{verbatim}
where $\Omega$ is the volume and energy is the total energy (or enthalpy at
finite pressure).

\item 
\texttt{energy\_files/flevdat(\_mur)} 
(default name \texttt{output\_ev.dat\_mur})
This file contains the Murnaghan interpolation of the input data on a 
mesh of volumes and the pressure that corresponds to each volume. It has
the format:
\begin{verbatim}
# omega (a.u.)**3       energy (Ry)      pressure (kbar)
\end{verbatim}
where $\Omega$ is the volume, energy the total energy (or enthalpy at finite
pressure).

\item 
\texttt{energy\_files/flevdat(.ev.out)} 
(default name \texttt{output\_ev.dat.ev.out})
This file contains the output information on the Murnaghan interpolation 
in the same format as the output of the \texttt{ev.x} program of 
Quantum ESPRESSO package. This file is written in two formats.
There is also an \texttt{.xml} file where the same quantities are written 
with more digits and in the \texttt{XML} format.

\item 
\texttt{energy\_files/flenergy(\#)} 
(default name \texttt{output\_energy.dat(\#)}). This file is written only
when \texttt{lmurn=.FALSE.}. Its content depends on the crystal system.
For the cubic system it contains:
\begin{verbatim}
#   celldm(1)       energy
\end{verbatim}
For hexagonal and tetragonal systems it contains:
\begin{verbatim}
#   celldm(1)    celldm(3)   energy
\end{verbatim}
For trigonal systems it contains:
\begin{verbatim}
#   celldm(1)    celldm(4)   energy
\end{verbatim}
For orthorhombic system several of these files are produced, one for each
value of \texttt{celldm(3)} and they contain:
\begin{verbatim}
#   celldm(1)    celldm(2)   energy
\end{verbatim}
For all other systems a file is produced with the format:
\begin{verbatim}
#   celldm(1)    celldm(2)   celldm(3) celldm(4)  celldm(5) celldm(6) energy
\end{verbatim}

\item 
\texttt{therm\_files/fleldos} (default name \texttt{output\_eldos.dat})
This file contains the electron density of states. It has the format
\begin{verbatim}
#  E (eV)   dos(E)     Int dos(E)
\end{verbatim}
and contains the energy, the electron density of states per cell, and
its integral. In the local spin density case the format of the file is
\begin{verbatim}
#  E (eV)   dosup(E)  dosdn(E)    Int dos(E)
\end{verbatim}
with the density of states for spin up and for spin down. The integrated
density of states refers to the sum of spin up and spin down states.

\item 
\texttt{therm\_files/fleltherm} (default name \texttt{output\_eltherm.dat})
This file contains the electron thermodynamic properties as a function of
temperature. It has the format:
\begin{verbatim}
#   T     energy     free energy     entropy    C_v        chemical pot
\end{verbatim}
and contains the electron excitation energy, free energy, entropy, 
isochoric specific heat and chemical potential.

\item 
\texttt{phdisp\_files/flfrc.g\#)} (default name \texttt{output\_frc.dat.g\#})
For each geometry for which a phonon dispersion is calculated this file
contains the interatomic force constants. It is in a format readable by
the code \texttt{matdyn.f90} of the QE distribution. \texttt{g\#} indicates 
the number of the geometry.

\item 
\texttt{phdisp\_files/flvec.g\#)} (default name 
\texttt{matdyn.modes.g\#}). For each geometry for which a phonon dispersion
is calculated this file contains the vibrational modes and the frequencies
along the path. It is in a format defined by the \texttt{matdyn.f90} routine 
of the QE distribution.

\item 
\texttt{phdisp\_files/flfrq.g\#)} (default name \texttt{output\_frq.dat.g\#})
For each geometry for which a phonon dispersion is calculated this file
contains the frequencies along the path. It is in a format readable by
the code \texttt{plotband.f90} of the QE distribution.

\item 
\texttt{phdisp\_files/flpband.g\#)(.\#)(.\#)} 
(default name \texttt{output\_pband.dat.g\#(.\#)(.\#)})
For each geometry for which a phonon dispersion is calculated these files
contain the frequencies along the path. They are in a format readable by
\texttt{gnuplot} as produced by the \texttt{plotband.f90} code of Quantum
ESPRESSO. The numbers after the geometry indicate the number of the line 
along the path and the number of the irreducible representation.

\item 
\texttt{phdisp\_files/fldosfrq.g\#)} (default name 
\texttt{save\_frequencies.g\#}). For each geometry for which a phonon 
density of states is calculated this file contains the vibrational  
frequencies in all the {\bf q} vectors used for the phonon density of states.
The files contain, for each {\bf q} vector, its weight and the frequencies,
an information sufficient to recalculate the density of states.
The file is readable only by \texttt{thermo\_pw}.

\item 
\texttt{phdisp\_files/fldos.g\#)} (default name 
\texttt{output\_dos.dat.g\#}). For each geometry for which a phonon density of 
states is calculated (when \texttt{ltherm\_dos=.TRUE.}) this file contains:
\begin{verbatim}
#   Frequency       dos
\end{verbatim}
The frequency and the phonon density of states. 

\item 
\texttt{therm\_files/fltherm.g\#(\_ph)} (default name 
\texttt{output\_therm.dat.g\#(\_ph)}). For each geometry for which 
a complete phonon dispersion is calculated (and whose number is given
by \texttt{g\#}) the code produces one of these files that contain the harmonic 
thermodynamic quantities as a function of temperature. It contains:
\begin{verbatim}
#   T       energy       free energy       entropy      C_v 
\end{verbatim}
The temperature, the vibrational energy, free energy, entropy, and
isochoric heat capacity. The extension \texttt{(\_ph)} indicates 
quantities calculated from the direct integration over the Brillouin zone,
while no extension indicates quantities obtained from the phonon density 
of states.

\item 
\texttt{anhar\_files/flanhar(\_ph)} (default name \texttt{output\_anhar.dat}). The content of this file depends on the 
value of the flag \texttt{lmurn}. When \texttt{lmurn=.TRUE.} we have:
\begin{verbatim}
#   T      V(T)   B (T)    dB(T)/dP     beta 
\end{verbatim}
The temperature, the volume, the bulk modulus, the pressure derivative
of the bulk modulus derived from the Murnaghan equation and the volume
thermal expansion $\beta$ obtained from the numerical derivative of the
volume. When \texttt{lmurn=.FALSE.} the bulk modulus
and its pressure derivative are not available and the file contains
only the three quantities:
\begin{verbatim}
#   T        V(T)      beta 
\end{verbatim}
The presence (absence) of the \texttt{(\_ph)} extension indicates that
these quantities have been calculated from the free energy obtained from 
direct integration over the Brilluoin zone (computing the phonon density
of states). 

\item \texttt{anharm\_files/flanhar(.aux)(\_ph)} 
(default name \texttt{output\_anharm.dat.aux})
This file is present when \texttt{lmurn=.TRUE.} or when the elastic
constants are on file and \texttt{lb0\_t=.FALSE.}. It contains:
\begin{verbatim}
   T     gamma(T)    C_v (T)    (C_p - C_v)(T)  (B_S - B_T) (T) 
\end{verbatim}
The temperature, the average Gr\"uneisen parameter, the isochoric heat
capacity computed from the interpolation of the volume dependent isochoric heat
capacity at the volume corresponding to the temperature $T$, the difference
between isobaric and isochoric heat capacity, the difference between
isoentropic and isothermal bulk moduli.
The presence (absence) of the \texttt{(\_ph)} extension indicates that
these quantities have been calculated from the free energy obtained from 
direct integration over the Brilluoin zone (computing the phonon density
of states). 

\item 
\texttt{anhar\_files/flanhar(.aux\_grun)} 
(default name \texttt{output\_anhar.dat.aux\_grun}). This file is printed
when \texttt{lmurn=.TRUE.} or when the elastic
constants are on file and \texttt{lb0\_t=.FALSE.}. It contains
\begin{verbatim}
#   T        beta      gamma    (C_p - C_v)(T)     (B_S - B_T) (T)
\end{verbatim}
the temperature, the volume thermal expansion calculated from the mode 
Gr\"uneisen parameters, the average Gr\"uneisein parameter, the difference
between isobaric and isochoric heat capacities and the difference between
adiabatic and isothermal bulk moduli that results
from the new $\beta$ but keeping all the other quantities as calculated
from the free energy derivatives. 

\item 
\texttt{anhar\_files/flanhar(.anis)(\_ph)} 
(default name \texttt{output\_anhar.anis(\_ph)}). This file is printed
when \texttt{lmurn=.FALSE.}, the elastic constants are on file and 
\texttt{lb0\_t=.FALSE.}. It contains the anharmonic properties
calculated from the elastic constants. Presently it contains only
\begin{verbatim}
#   T         (C_p - C_v)(T)    
\end{verbatim}
the temperature, the difference between isostress and isostrain heat 
capacities that results from the thermal expansion tensor $\alpha$ and the
volume calculated from the crystal parameters obtained from the minimization of 
the free energy. No extension or the \texttt{(\_ph)} extension mean
that the free energy is calculated using the phonon density of states,
or from a three dimensional Brillouin zone integration.

\item 
\texttt{anhar\_files/flanhar.celldm(\_ph)(\_grun)} 
(default name \\ \texttt{output\_anhar.dat.celldm(\_ph)(\_grun)}). 
This file is printed
only when \texttt{lmurn=.FALSE.} and its content depends on the crystal 
system. For cubic solids it contains: 
\begin{verbatim}
#   T        celldm(1)        alpha_xx
\end{verbatim}
the temperature, the cubic lattice constant and the linear thermal expansion.
For hexagonal and tetragonal solids it contains:
\begin{verbatim}
#   T    celldm(1)   celldm(3)   alpha_xx   alpha_zz
\end{verbatim}
For trigonal solids it contains:
\begin{verbatim}
#   T    celldm(1)   celldm(4)   alpha_xx   alpha_zz
\end{verbatim}
For orthorhombic solids it contains:
\begin{verbatim}
#   T  celldm(1)  celldm(2)  celldm(3)  alpha_xx  alpha_yy  alpha_zz
\end{verbatim}
For monoclinic solids b-unique:
\begin{verbatim}
#   T    celldm(1)   celldm(2)   celldm(3)  celldm(5)
\end{verbatim}
For monoclinic solids c-unique:
\begin{verbatim}
#   T    celldm(1)   celldm(2)   celldm(3)  celldm(4)
\end{verbatim}
For triclinic solids:
\begin{verbatim}
#   T  celldm(1)  celldm(2)  celldm(3)  celldm(4)  celldm(5)  celldm(6)
\end{verbatim}
No extension or the extension \texttt{(\_ph)} 
indicate that the quantities have been calculated from the minimization of
the free energy obtained from the phonon density of states or from the
three dimensional integration over the Brillouin zone. The extension
\texttt{(\_grun)} indicates that the thermal expansions have been
calculated from the mode Gr\"uneisen parameters, while the \texttt{celldm}
parameters are still obtained from the free energy minimization. Which
one depend from the variable \texttt{ltherm\_dos} and \texttt{ltherm\_freq}.
If both are \texttt{.TRUE.} the second is used.

\item 
\texttt{anhar\_files/flgrun(\_\#)}(default name 
\texttt{output\_grun.dat(\_\#)}) This file contains the mode Gr\"uneisen
parameters. In anisotropic crystal systems with \texttt{lmurn=.FALSE.} this
file contains the derivative of the frequencies with respect to the
crystal parameters. The optional number in the filename indicates which
crystal parameter. The format of the file is readable by the program
\texttt{plotband.f90} of the Quantum ESPRESSO package.

\item 
\texttt{anhar\_files/flgrun(\_freq)} 
(default name \texttt{output\_grun.dat\_freq)}).
This file contains the phonon frequencies interpolated at the requested 
temperature or volume. The format of the file is readable by the program
\texttt{plotband.f90} of the Quantum ESPRESSO package.

\item 
\texttt{anhar\_files/flpgrun(.\#)(.\#)} 
(default name \texttt{output\_pgrun.dat(.\#)(.\#)})
These files contains the mode Gr\"uneisen parameters along the path.
They are in a format readable by \texttt{gnuplot} as produced by 
the \texttt{plotband.f90} code of Quantum
ESPRESSO. The numbers after the geometry indicate the number of the line 
along the path and the number of the irreducible representation.

\end{itemize}

\newpage
\section{\color{coral}Library}

\tpw\ offers a library of modules to deal with the properties of solids. The 
modules are independent from the \tpw\ variables, but might
depend on the \qe\ modules. They are used by the routines of \tpw\ 
but they can also be used by other programs such as those contained
in the \texttt{tools} directory. Presently the library contains the
following modules:

\begin{itemize}

\item \texttt{lattices.f90} contains routines to deal with Bravais lattices.
Presently there is a routine to calculate the primitive cell of
a given centered cell.

\item
\texttt{latgen\_2d.f90} contains routines to deal with two dimensional
Bravais lattices. Presently it contains the routines to generate the
primitive vectors of these lattices and their reciprocal lattice vectors.

\item
\texttt{bz\_2d\_form.f90} contains the information required to make a
plot of the two-dimensional Brillouin zones.

\item
\texttt{rotate.f90} contains routines to deal with rotations. Presently it
applies a given rotation to a vector or to a tensor.

\item
\texttt{space\_groups.f90} contains routines to detect the space group,
to set the FFT indeces compatible with a given space group, to find the
space group number given the name, or the name given the number etc.

\item
\texttt{point\_group.f90} contains routines to deal with crystallographic
point groups. Given a representation of a given group it can find 
the subgroups representations that correspond to it (also for
double groups). It has the same conventions as the routines 
in \texttt{find\_group.f90}, \texttt{divide\_class.f90}, and 
\texttt{divide\_class\_so.f90} of \qe.

\item
\texttt{eldos\_module.f90} contains routines to calculate the electronic
excitation energy, free energy, entropy, and constant strain heat capacity
from the electronic density of states, assuming a gas of independent 
electrons.   

\item
\texttt{ph\_freq\_module.f90} contains routines to calculate the vibrational
energy, free energy, entropy, and constant strain heat capacity from the
phonon frequencies.   

\item
\texttt{phdos\_module.f90} contains routines to calculate the vibrational
energy, free energy, entropy, and constant strain heat capacity from the
phonon density of states.

\item
\texttt{strain.f90} contains routines to apply a given strain to a given
Bravais lattice.

\item
\texttt{isoentropic.f90} contains routines to calculate the isoentropic
quantities or the constant stress heat capacity.

\item
\texttt{piezoelectric\_tensor.f90} contains routines to calculate the
piezoelectric tensor given the strain and the induced polarization.

\item
\texttt{elastic\_const.f90} contains routines to calculate the elastic
constants given the strain and the induced stress. It can also compute
quantities dependent on the elastic constants, such as the elastic 
compliances, the bulk moduli, the sound speeds or the Voigt-Reuss-Hill 
averages. It has also routines to transform from the Voigt to the 
tensorial notation and viceversa.

\item
\texttt{xrdp.f90} contains routines to calculate the atomic form factors 
and the X-rays powder diffraction spectra.

\item
\texttt{debye\_module.f90} contains routines to calculate the debye 
temperature, the vibrational energy, free-energy, and entropy according
to the Debye's model.

\item
\texttt{quadratic\_surfaces.f90} contains routines to fit a set of data
with a quadratic polynomial, to find the gradient or the hessian in 
a given point and to find the minimum of the quadratic function. 
The number of independent variables can vary from one to six.

\item
\texttt{quartic\_surfaces.f90} contains routines to fit a set of data
with a quartic polynomial, to find the gradient or the hessian in 
a given point and to find the minimum of the quartic function given a 
reasonably good guess of its position.
The number of independent variables can vary from one to six.

\item
\texttt{linear\_solvers.f90} contains routines that solve iteratively
a linear system. It contains also a few drivers that calls LAPACK routines
and do not require workspace information. 

\item
\texttt{mp\_asyn.f90} contains the routines needed to run many 
independent calculations in a asynchronous way. It supports a master-slave
approach in which a processor acts as a master and sends to the slaves
the information on the work to do when they are ready to do it.

\item
\texttt{asy.f90} contains the routines to write a script for the
\texttt{asymptote} code.

\item
\texttt{bz\_asy.f90} contains the informations needed to make the plot       
of the Brillouin zone using the \texttt{asymptote} code.      

\item
\texttt{gnuplot.f90} contains the routines to write a \texttt{gnuplot} script.

\item
\texttt{gnuplot\_color.f90} contains routines to insert the color names
inside the \texttt{gnuplot} scripts.

\end{itemize}


\end{document}

